\documentclass[a4paper,twoside,master.tex]{subfiles}
\begin{document}
\lecture{8}{Mon Sep 9 2019}{Review}

The general solutions, again, are:

\begin{equation}
   \nu\neq0\quad[a_\nu\rho^\nu+b_\nu\rho^{-\nu}]\sin(\nu\phi\alpha_\nu)
\end{equation}

\begin{equation}
   \nu = 0\quad[a_0+b_0\ln\rho][A_0+B_0\rho]
\end{equation}

In the case where the potential on both planes is $V_0$, $\alpha_\nu = 0$, and from the periodicity of the $\nu\neq 0$ condition, we can say $\nu = \frac{m\pi}{\beta}$. This discretizes $\nu$:

\begin{equation}
   \Phi = V_0 + \sum_{m=1}^\infty[a_m\rho^{\frac{m\pi}{\beta}}+b_m\rho^{-{\frac{m\pi}{\beta}}}]\sin({\frac{m\pi}{\beta}}\phi)
\end{equation}

There is another unspecified parameter which concerns what happens really far away and really up close. Let's assume we only want a solution which is finite at the vertex. Only the $a_m$ terms will remain finite here:

\begin{equation}
   \Phi = V_0 + \sum_{m=1}^\infty a_m\rho^{\frac{m\pi}{\beta}}\sin({\frac{m\pi}{\beta}}\phi)
\end{equation}

What does the vector field look like?

\begin{equation}
    \vec{E} = -\nabla\Phi = -\partial_\rho\Phi\hat{\rho}-\frac{1}{\rho}\partial_\rho\Phi\hat{\phi} = -a_m\rho^{\frac{m\pi}{\beta}-1}\sin(\frac{m\pi}{\beta}\phi)\hat{\rho} - \sum a_m \frac{m\pi}{\beta} \rho^{\frac{m\pi}{\beta}-1}\cos(\frac{m\pi}{\beta}\phi)\hat{\phi}
\end{equation}

Suppose $\beta > \pi$. This implies $E\propto\rho^{\frac{\pi}{\beta}-1}$ as $\rho\to 0^+$, so the field diverges in the corner if the corner is a sharp edge.

\section{Spherical Coordinates}%
\label{sec:spherical_coordinates}

\begin{equation}
   \nabla^2 = \frac{1}{r^2}\partial_r(r^2\partial_r)+\frac{1}{r^2\sin\theta}\partial_\theta(\sin\theta\partial_\theta)+\frac{1}{r^2\sin^2\theta}\partial_\phi^2
\end{equation}

Let us look at this from another perspective, an angular moment operator:

\begin{equation}
   \vec{\mathbb{L}} = \vec{x}\cross(-\imath\vec{\nabla})
\end{equation}

\begin{equation}
    \vec{x}\cdot\vec{\mathbb{L}} = 0
\end{equation}

\begin{equation}
    \mathbb{L}_l = (-\imath)\epsilon_{lmn}x_m\partial_n
\end{equation}

\begin{equation}
    -\mathbb{L}^2 = r^2\nabla^2 - \partial_r r^2 \partial_r
\end{equation}

Now we can see that

\begin{equation}
    \nabla^2 = \frac{1}{r^2}\partial_r r^2\partial_r - \frac{\mathbb{L}^2}{r^2}
\end{equation}

If we are dealing with completely spherical boundaries, we need the full range of $\phi$ and $\theta$.

If we say $\hbar = 1$, this is the same as the angular momentum operator from quantum:

\begin{equation}
    [\mathbb{L}^2, f(r)] = 0
\end{equation}

\begin{equation}
    [\mathbb{L}^2,\mathbb{L}_z] = 0
\end{equation}

\begin{equation}
    [\mathbb{L}_z, f(r)] = 0
\end{equation}

\begin{equation}
    \mathbb{L}^2|lm\rangle = l(l+1)|lm\rangle
\end{equation}

\begin{equation}
    \mathbb{L}_z|lm\rangle = m|lm\rangle
\end{equation}

where

\begin{equation}
    \langle\theta,\phi|lm\rangle = Y_{lm}(\theta,\phi)
\end{equation}
\end{document}
