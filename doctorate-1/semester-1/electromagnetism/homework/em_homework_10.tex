\documentclass[a4paper,twoside]{article}
% My LaTeX preamble file - by Nathaniel Dene Hoffman
% Credit for much of this goes to Olivier Pieters (https://olivierpieters.be/tags/latex)
% and Gilles Castel (https://castel.dev)
% There are still some things to be done:
% 1. Update math commands using mathtools package (remove ddfrac command and just override)
% 2. Maybe abbreviate \imath somehow?
% 3. Possibly format for margin notes and set new margin sizes
% First, some encoding packages and usefull formatting
%--------------------------------------------------------------------------------------------
\usepackage[l2tabu,orthodox]{nag}   % force newer (and safer) LaTeX commands
\usepackage[utf8]{inputenc}         % set character set to support some UTF-8
                                    %   (unicode). Do NOT use this with
                                    %   XeTeX/LuaTeX!
\usepackage[T1]{fontenc}
\usepackage[english]{babel}         % multi-language support
\usepackage{sectsty}                % allow redefinition of section command formatting
\usepackage{tabularx}               % more table options
\usepackage{booktabs}
\usepackage{titling}                % allow redefinition of title formatting
\usepackage{imakeidx}               % create and index of words
\usepackage{xcolor}                 % more colour options
\usepackage{enumitem}               % more list formatting options
\usepackage{tocloft}                % redefine table of contents, new list like objects
\usepackage{subfiles}               % allow for multifile documents

% Next, let's deal with the whitespaces and margins
%--------------------------------------------------------------------------------------------
\usepackage[centering,margin=1in]{geometry}
\setlength{\parindent}{0cm}
\setlength{\parskip}{2ex plus 0.5ex minus 0.2ex} % whitespace between paragraphs

% Redefine \maketitle command with nicer formatting
%--------------------------------------------------------------------------------------------
\pretitle{
  \begin{flushright}         % align text to right
    \fontsize{40}{60}        % set font size and whitespace
    \usefont{OT1}{phv}{b}{n} % change the font to bold (b), normally shaped (n)
                             %   Helvetica (phv)
    \selectfont              % force LaTeX to search for metric in its mapping
                             %   corresponding to the above font size definition
}
\posttitle{
  \par                       % end paragraph
  \end{flushright}           % end right align
  \vskip 0.5em               % add vertical spacing of 0.5em
}
\preauthor{
  \begin{flushright}
    \large                   % font size
    \lineskip 0.5em          % inter line spacing
    \usefont{OT1}{phv}{m}{n}
}
\postauthor{
  \par
  \end{flushright}
}
\predate{
  \begin{flushright}
  \large
  \lineskip 0.5em
  \usefont{OT1}{phv}{m}{n}
}
\postdate{
  \par
  \end{flushright}
}

% Mathematics Packages
\usepackage[Gray,squaren,thinqspace,cdot]{SIunits}      % elegant units
\usepackage{amsmath}                                    % extensive math options
\usepackage{amsfonts}                                   % special math fonts
\usepackage{mathtools}                                  % useful formatting commands
\usepackage{amsthm}                                     % useful commands for building theorem environments
\usepackage{amssymb}                                    % lots of special math symbols
\usepackage{mathrsfs}                                   % fancy scripts letters
\usepackage{cancel}                                     % cancel lines in math
\usepackage{esint}                                      % fancy integral symbols
\usepackage{relsize}                                    % make math things bigger or smaller
\usepackage{bm}                                         % bold math!

\newcommand\ddfrac[2]{\frac{\displaystyle #1}{\displaystyle #2}}    % elegant fraction formatting
\allowdisplaybreaks[1]                                              % allow align environments to break on pages

% Ensure numbering is section-specific
%--------------------------------------------------------------------------------------------
\numberwithin{equation}{section}
\numberwithin{figure}{section}
\numberwithin{table}{section}

% Citations, references, and annotations
%--------------------------------------------------------------------------------------------
\usepackage[small,bf,hang]{caption}        % captions
\usepackage{subcaption}                    % adds subfigure & subcaption
\usepackage{sidecap}                       % adds side captions
\usepackage{hyperref}                      % add hyperlinks to references
\usepackage[noabbrev,nameinlink]{cleveref} % better references than default \ref
\usepackage{autonum}                       % only number referenced equations
\usepackage{url}                           % urls
\usepackage{cite}                          % well formed numeric citations
% format hyperlinks
\colorlet{linkcolour}{black}
\colorlet{urlcolour}{blue}
\hypersetup{colorlinks=true,
            linkcolor=linkcolour,
            citecolor=linkcolour,
            urlcolor=urlcolour}

% Plotting and Figures
%--------------------------------------------------------------------------------------------
\usepackage{tikz}          % advanced vector graphics
\usepackage{pgfplots}      % data plotting
\usepackage{pgfplotstable} % table plotting
\usepackage{placeins}      % display floats in correct sections
\usepackage{graphicx}      % include external graphics
\usepackage{longtable}     % process long tables

% use most recent version of pgfplots
\pgfplotsset{compat=newest}

% Misc.
%--------------------------------------------------------------------------------------------
\usepackage{todonotes}  % add to do notes
\usepackage{epstopdf}   % process eps-images
\usepackage{float}      % floats
\usepackage{stmaryrd}   % some more nice symbols
\usepackage{emptypage}  % suppress page numbers on empty pages
\usepackage{multicol}   % use this for creating pages with multiple columns
\usepackage{etoolbox}   % adds tags for environment endings
\usepackage{tcolorbox}  % pretty colored boxes!


% Custom Commands
%--------------------------------------------------------------------------------------------
\newcommand\hr{\noindent\rule[0.5ex]{\linewidth}{0.5pt}}                % horizontal line
\newcommand\N{\ensuremath{\mathbb{N}}}                                  % blackboard set characters
\newcommand\R{\ensuremath{\mathbb{R}}}
\newcommand\Z{\ensuremath{\mathbb{Z}}}
\newcommand\Q{\ensuremath{\mathbb{Q}}}
\newcommand\C{\ensuremath{\mathbb{C}}}
\renewcommand{\arraystretch}{1.2}                                       % More space between table rows (could be 1.3)
\newcommand{\Cov}{\mathrm{Cov}}
\newcommand*{\dbar}{\ensuremath{\text{\dj}}}
% Custom Environments
%--------------------------------------------------------------------------------------------
\newcommand{\lecture}[3]{\hr\\{\centering{\large\textsc{Lecture #1: #3}}\\#2\\}\hr\markboth{Lecture #1: #3}{\rightmark}}   % command to title lectures
\usepackage{mdframed}
\theoremstyle{plain}
\newmdtheoremenv[nobreak]{theorem}{Theorem}[section]
\newtheorem{corollary}{Corollary}[theorem]
\newtheorem{lemma}[theorem]{Lemma}
\theoremstyle{definition}
\newtheorem*{ex}{Example}
\newmdtheoremenv[nobreak]{definition}{Definition}[section]
\theoremstyle{remark}
\newtheorem*{remark}{Remark}
\AtEndEnvironment{ex}{\null\hfill$\diamond$}%
% Note: A proof environment is already provided in the amsthm package
\tcbuselibrary{breakable}
\newenvironment{note}[1]{\begin{tcolorbox}[
    arc=0mm,
    colback=white,
    colframe=white!60!black,
    title=#1,
    fonttitle=\sffamily,
    breakable
]}{\end{tcolorbox}}
\newenvironment{problem}{\begin{tcolorbox}[
    arc=0mm,
    breakable,
    colback=white,
    colframe=black
]}{\end{tcolorbox}}

% Header and Footer
%--------------------------------------------------------------------------------------------
% set header and footer
\usepackage{fancyhdr}                       % header and footer
\pagestyle{fancy}                           % use package
\fancyhf{}
\fancyhead[LE,RO]{\textsl{\rightmark}}      % E for even (left pages), O for odd (right pages)
\fancyfoot[LE,RO]{\thepage}
\fancyfoot[LO,RE]{\textsl{\leftmark}}
\setlength{\headheight}{15pt}


% Physics
%--------------------------------------------------------------------------------------------
\usepackage[arrowdel]{physics}      % all the usual useful physics commands
%\usepackage{feyn}                   % for drawing Feynman diagrams
%\usepackage{bohr}                   % for drawing Bohr diagrams
\usepackage{elements}               % for quickly referencing information of various elements
\usepackage{tensor}                 % for writing tensors and chemical symbols

% Finishing touches
%--------------------------------------------------------------------------------------------
\author{Nathaniel D. Hoffman}

\title{33-761 Homework 10}
\date{\today}
\begin{document}
\maketitle

\section{Jackson 6.3; (a), (b), and (c)}

The homogeneous diffusion equation (5.160) for the vector potential for quasi-static fields in unbounded conducting media has a solution to the initial value problem of the form
\begin{equation}
    \va{A}( \va{x}, t) = \int \dd[3]{x'} G( \va{x} - \va{x}',t) \va{A}( \va{x}', 0)
\end{equation}
where $ \va{A}( \va{x}', 0) $ describes the initial field configuration and $ G $ is an appropriate kernel.

Equation 5.160:
\begin{equation}
    \laplacian{ \va{A}} = \mu \sigma \partial_t \va{A}
\end{equation}

\begin{itemize}
    \item[(a)] Solve the initial value problem by use of a three-dimensional Fourier transform in space for $ \va{A}( \va{x}, t) $. With the usual assumptions on interchange of orders of integration, show that the Green function has the Fourier representation,
        \begin{equation}
            G( \va{x}- \va{x}', t) = \frac{1}{(2 \pi)^3} \int \dd[3]{k} e^{-k^2 t/ \mu \sigma} e^{\imath \va{k} \vdot ( \va{x} - \va{x}')}
        \end{equation}
        and it is assumed that $ t>0 $.
        \begin{problem}
            Let's first do a three-dimensional Fourier transform on $ \va{A}( \va{x},t) $:
            \begin{equation}
                \va{A}( \va{x},t) = \frac{1}{(2 \pi)^3} \int \va{A}( \va{k},t) e^{- \imath \va{k} \vdot \va{x}} \dd[3]{k}
            \end{equation}
            Looking back at the diffusion equation, we now move the Laplacian inside this integral, which pulls out two factors of $ \imath \va{k} $, so the equation becomes
            \begin{equation}
                (\imath \va{k})^2 \va{A} = \mu \sigma \partial_t \va{A} 
            \end{equation}
            or
            \begin{equation}
                \partial_t \va{A} = - \frac{k^2}{\mu \sigma} \va{A}
            \end{equation}
            so
            \begin{equation}
                \va{A}( \va{k},t) = \va{A}( \va{k}, 0) e^{- \frac{k^2}{\mu \sigma} t}
            \end{equation}
            This is the Fourier transform of the solution, so
            \begin{equation}
                \va{A}( \va{x},t) = \frac{1}{(2 \pi)^3} \int \va{A}( \va{k}, 0) e^{- \frac{k^2}{\mu \sigma} t} e^{\imath \va{k} \vdot \va{x}} \dd[3]{k}
            \end{equation}
            and an inverse Fourier transform tells us that
            \begin{equation}
                \va{A}( \va{k},0) = \int \va{A}( \va{x}',0) e^{- \imath \va{k} \vdot \va{x}'} \dd[3]{x'}
            \end{equation}
            All together, we have
            \begin{equation}
                \va{A}( \va{x}, t) = \frac{1}{(2 \pi)^3} \iint \va{A}( \va{x}', 0) e^{- \frac{k^2}{\mu \sigma} t} e^{\imath \va{k} \vdot ( \va{x} - \va{x}')} \dd[3]{k} \dd[3]{x'}
            \end{equation}
            which is equal to
            \begin{equation}
                = \int G( \va{x} - \va{x}', t) \va{A}( \va{x}', 0) \dd[3]{x'}
            \end{equation}
            so
            \begin{equation}
                G( \va{x} - \va{x}', t) = \frac{1}{(2 \pi)^3} \int e^{- \frac{k^2}{\mu \sigma} t} e^{\imath \va{k} \vdot ( \va{x} - \va{x}')} \dd[3]{k}
            \end{equation}
        \end{problem}
    \item[(b)] By introducing a Fourier decomposition in both space and time, and performing the frequency integral in the complex $ \omega $ plane to recover the result of part a, show that $ G( \va{x} - \va{x}', t) $ is the diffusion Green function that satisfies the inhomogeneous equation,
        \begin{equation}
            \pdv{G}{t} - \frac{1}{\mu \sigma} \laplacian{G} = \delta^3( \va{x} - \va{x}') \delta(t)
        \end{equation}
        and vanishes for $ t<0 $.
        \begin{problem}
            The four-dimensional Fourier transform of a function of space and time is given by
            \begin{equation}
                G( \va{x},t) = \frac{1}{(2 \pi)^4} \int G( \va{k}, \omega) e^{\imath (\va{k} \vdot \va{x} - \omega t)} \dd[3]{k} \dd{\omega}
            \end{equation}
            Plugging this into the given inhomogeneous equation results in
            \begin{equation}
                \left[ (- \imath \omega) - \frac{\imath \va{k}^2}{\mu \sigma} \right]G = e^{- \imath \va{k} \vdot \va{x}'}
            \end{equation}
            or
            \begin{equation}
                G( \va{k}, \omega) = \frac{e^{- \imath \va{k} \vdot \va{x}'}}{\frac{k^2}{\mu \sigma} - \imath \omega}
            \end{equation}
            Let us first invert the transform in time (using $ \omega $):
            \begin{equation}
                G( \va{k},t) = \frac{1}{2 \pi} \int_{- \infty}^{\infty} \frac{e^{- \imath \va{k} \vdot \va{x}'}}{2 \pi} \frac{e^{- \imath \omega t}}{-\imath \omega + \frac{k^2}{\mu \sigma}}
            \end{equation}
            Our integrand is of the form $ e^{\imath a \omega} g(\omega) $, where $ a < 0 $ for $ t > 0 $, so we want to perform this integral using a closed contour in the lower-half plane, by Jordan's lemma:
            \begin{equation}
                \lim_{R \to \infty} \int_{C_R} e^{\imath a \omega} g(\omega) \dd{\omega} = 0
            \end{equation}
            over a contour in the upper-half plane for \textit{positive} $ a $, however, we have negative $ a $ so the semicircle part will integrate to zero as we expand the radius to zero and thus take the integral from $ - \infty $ to $ + \infty $. In doing this, we will go around the pole in the lower-half plane at $ \imath \omega = - \frac{k^2}{\mu \sigma} $, so the integral will be $ - 2 \pi \imath $ times the residue:
            \begin{equation}
                G( \va{k},t) = (- 2 \pi \imath) \frac{\imath e^{- \imath \va{k} \vdot \va{x}'}}{2 \pi} e^{- \frac{k^2}{\mu \sigma} t} = e^{- \frac{k^2}{\mu \sigma} t - \imath \va{k} \vdot \va{x}'}
            \end{equation}
            (I rearranged the equation to give me just $ \omega + \text{stuff} $ in the denominator)
            In the upper-half plane, there are no poles, so the integral evaluates to $ 0 $, or $ G = 0 $, which is the evaluation for $ t < 0 $.
            
            Finally, I have to transform back into position space to get the Green's function:
            \begin{equation}
                G( \va{x} - \va{x}', t) = \frac{\Theta(t)}{(2 \pi)^3} e^{- \frac{k^2}{\mu \sigma} t} e^{\imath \va{k} \vdot (\va{x} - \va{x}')} \dd[3]{k}
            \end{equation}
            where $ \Theta(t) $ is the Heaviside function, which gives us the vanishing function for $ t < 0 $ and evaluates to $ 1 $ for $ t > 0 $.
        \end{problem}
    \item[(c)] Show that if $ \sigma $ is uniform throughout all space, the Green function is
        \begin{equation}
            G( \va{x},t; \va{x}', 0) = \Theta(t) \left( \frac{\mu \sigma}{4 \pi t} \right)^{3/2} e^{\frac{- \mu \sigma \abs{\va{x}-\va{x}'}^2}{4t}}
        \end{equation}
        \begin{problem}
            If $\sigma$ is uniform and doesn't depend on $ x $, we can evaluate the integral by completing the square:
            \begin{equation}
                G( \va{x} - \va{x}', t) = \frac{\Theta(t)}{(2 \pi)^3} \int \exp(-\frac{t}{\mu \sigma} \abs{ \va{k} - \imath \mu \sigma( \va{x} - \va{x}')/2t}^2 - \frac{\mu \sigma}{4t} \abs{\va{x}-\va{x}'}^2) \dd[3]{k}
            \end{equation}
            so
            \begin{equation}
                G( \va{x} - \va{x}',t) = \frac{\Theta(t)}{(2 \pi)^3} \left( \frac{\pi \mu \sigma}{t} \right)^{3/2} e^{- \frac{\mu \sigma}{4t} \abs{\va{x}-\va{x}'}^2} = \Theta(t) \left( \frac{\mu \sigma}{4 \pi t} \right)^{3/2} e^{\frac{- \mu \sigma \abs{\va{x}-\va{x}'}^2}{4t}}
            \end{equation}
        \end{problem}
\end{itemize}

\section{Jackson 6.10 (modification)}
Assume we have free space, so only use $ \epsilon_0 $ and $ \mu_0 $. Maxwell stress tensor $ T_{ij} $ is the one given in Eq. (6.120) similarly $ \va{g} = \epsilon_0 \va{E} \cross \va{B} $ as in Eq. (6.118).

Equation 6.120:
\begin{equation}
    T_{\alpha \beta} = \epsilon_0 \left[ E_{\alpha} E_{\beta} + c^2 B_{\alpha} B_{\beta} - \frac{1}{2} \left( \va{E} \vdot \va{E} + c^2 \va{B} \vdot \va{B} \right) \delta_{\alpha \beta} \right]
\end{equation}
Equation 6.118:
\begin{equation}
    \va{g} = \frac{1}{c^2} (\va{E} \cross \va{H})
\end{equation}

Discuss the conservation of angular momentum. Show that the differential and integral forms of the conservation law are
\begin{equation}
    \pdv{t}\left( \mathscr{L}_{\text{mech}} + \mathscr{L}_{\text{field}} \right) + \div{\vb{M}} = 0
\end{equation}
and
\begin{equation}
    \dv{t} \int_{V} \left( \mathscr{L}_{\text{mech}} + \mathscr{L}_{\text{field}} \right) \dd[3]{x} + \int_{S} \va{n} \vdot \vb{M} \dd{a} = 0
\end{equation}
where the field angular-momentum density is
\begin{equation}
    \mathscr{L}_{\text{field}} = \va{x} \cross \va{g} = \mu \epsilon \cross ( \va{E} \cross \va{H})
\end{equation}
and the flux of angular momentum is described by the tensor
\begin{equation}
    \vb{M} = \vb{T} \cross \va{x}
\end{equation}
Note that we define the dyadic notation as $ [\va{n} \cdot \vb{M}]_j = \sum_i n_i M_{ij} $ and $ M_{ijk} = T_{ij} x_k - T_{ik} x_j $. This tensor is antisymmetric in the $ j $ and $ k $ indices, so it has three independent elements.

\begin{problem}
    I will begin by examining the force on a particle in free space under electric and magnetic fields. We write this force according to the Lorentz force law:
    \begin{equation}
        \va{F} = q \left( \va{E} + \va{v} \cross \va{B} \right)
    \end{equation}
    However, since everything in this problem is in terms of a density, we should rewrite this in terms of the force density:
    \begin{equation}
        \va{f} = \rho \va{E} + \va{J} \cross \va{B}
    \end{equation}
    Next, we use Maxwell's equations to rewrite the charges in terms of the fields:
    \begin{equation}
        \va{f} = \epsilon_0 (\div{ \va{E}}) \va{E} + \frac{1}{\mu_0} (\curl{ \va{B}}) \cross \va{B} - \epsilon_0 \partial_t \va{E} \cross \va{B}
    \end{equation}
    Using the product rule, we can rewrite the last term:
    \begin{equation}
        \va{f} = \epsilon_0 (\div{ \va{E}}) \va{E} + \frac{1}{\mu_0} (\curl{ \va{B}}) \cross \va{B} - \epsilon_0 \partial_t (\va{E} \cross \va{B}) - \epsilon_0 \va{E} \cross (\curl{ \va{E}})
    \end{equation}
    By inserting a zero (from $ \div{ \va{B}} = 0 $, we can make this equation look more symmetric:
        \begin{equation}
            \va{f} = \epsilon_0 \left[ (\div{ \va{E}}) \va{E} - \va{E} \cross (\curl{ \va{E}}) \right] + \frac{1}{\mu_0} \left[ (\div{ \va{B}}) \va{B} - \va{B} \cross (\curl{ \va{B}}) \right] - \epsilon_0 \partial_t (\va{E} \cross \va{B})
        \end{equation}
        Finally, by the identity
        \begin{equation}
            \frac{1}{2} \grad{ \va{A} \cdot \va{A}} = \va{A} \cross (\curl{ \va{A}}) + (\va{A} \vdot \grad) \va{A}
        \end{equation}
        we can rewrite this as
        \begin{equation}
            \va{f} = \epsilon_0 \left[ (\grad{ \va{E}}) \va{E} + (\va{E} \vdot \grad) \va{E} \right] + \frac{1}{\mu_0} \left[ (\div{ \va{B}}) \va{B} + (\va{B} \vdot \grad) \va{B} \right] - \frac{1}{2} \grad(\epsilon_0 E^2 + \frac{1}{\mu_0} B^2) - \partial_t \va{g}
        \end{equation}
        where $ \va{g} = \epsilon_0 (\va{E} \cross \va{B}) $ as defined in the problem.

        In this form, we can see that
        \begin{equation}
            \va{f} = \curl{ \vb{T}} - \partial_t \va{g}
        \end{equation}
        This force density can be related to the torque density in the system by
        \begin{equation}
            \va{x} \cross \va{f} = \va{\tau}
        \end{equation}
        which in turn can be related to the mechanical angular momentum density:
        \begin{equation}
            \va{x} \cross \va{f} = \va{\tau} = \partial_t \mathscr{L}_{\text{mech}}
        \end{equation}
        Therefore,
        \begin{align}
            \partial_t (\mathscr{L}_{\text{mech}} + \mathscr{L}_{\text{field}}) &= \va{x} \cross \va{f} + \partial_t \va{x} \cross \va{g} \\
            &= - \va{x} \cross \partial_t \va{g} + \va{x} \cross \div{ \vb{T}} + \va{x} \cross \partial_t \va{g} \\
            &= \va{x} \cross \div{ \vb{T}} = - \div{( \vb{T} \cross \va{x})} = - \div{ \vb{M}}
        \end{align}
        so
        \begin{equation}
            \partial_t (\mathscr{L}_{\text{mech}} + \mathscr{L}_{\text{field}}) + \div{ \vb{M}} = 0
        \end{equation}
        If we integrate over a volume, the right-hand side will still be zero, but we can then use divergence theorem to convert the volume integral over $ \div{ \vb{M}} $ into a surface integral over $ \va{n} \vdot \vb{M} $. This will yield the corresponding integral form of the conservation law.
\end{problem}

\section{Jackson 6.14; (a)}

An ideal circular parallel plate capacitor of radius $ a $ and plate separation $ d<<a $ is connected to a current source by axial leads. The current in the wire is $ I(t) = I_0 \cos(\omega t) $. Calculate the electric and magnetic fields between the plates to second order in powers of the frequency (or wave number), neglecting the effects of fringing fields.

\begin{problem}
    If we neglect edge effects, we know that the electric field between two charged plates will be oriented in the $ \vu{z} $ direction, and varying the charge in time will give us a magnetic field in the $ \vu{\varphi} $ direction, by the right-hand rule.

    First, let's expand the fields in $ \omega $. It is important to note that, for the electric field, leading order will be proportional to the charge on the plates, but this is proportional to an integral over $ I(t) \propto \frac{1}{\omega} $, so we need to insure the leading order term of the electric field is $ \omega^{-1} $. However, leading order in the electric field will be like an electrostatic solution, so there won't be a corresponding magnetic field. Additionally, we assume that the separation is small enough that the fields are uniform in $ z $ inside.
    \begin{equation}
        E_z = \sum_{n=-1}^{\infty} e_n(\rho) \omega^n
    \end{equation}
    and
    \begin{equation}
        B_{\varphi} = \sum_{n=0}^{\infty} b_n(\rho) \omega^n
    \end{equation}
    Since there are no currents between the plates, Ampere's law tells us
    \begin{equation}
        \frac{1}{\rho} \partial_{\rho} \rho B_{\varphi} = - \frac{\imath \omega}{c^2} E_z
    \end{equation}
    and Faraday's law tells us
    \begin{equation}
        \partial_{\rho} E_z = - \imath \omega B_{\varphi}
    \end{equation}
    We can substitute the second of these equations into the first (in two different ways) to give us second-order differential equations:
    \begin{equation}
        \frac{1}{\rho} \partial_{\rho} \rho \partial_{\rho} E_z = - \frac{\omega^2}{c^2} E_z
    \end{equation}
    and
    \begin{equation}
        \partial_{\rho} \frac{1}{\rho} \partial_{\rho} \rho B_{\varphi} = - \frac{\omega^2}{c^2} B_{\varphi}
    \end{equation}
    Expanding the derivatives on the left-hand sides of each of these equations, we find that they can be written as
    \begin{equation}
        \frac{\omega^2 \rho^2}{c^2} \partial^2_{\rho} E_z + \frac{\omega \rho}{c} \partial_{\rho} E_z + \frac{\omega^2 \rho^2}{c^2} E_z = 0
    \end{equation}
    and
    \begin{equation}
        \frac{\omega^2 \rho^2}{c^2} \partial^2_{\rho} B_{\varphi} + \frac{\omega \rho}{c} \partial B_{\varphi} + \left( \frac{\omega^2 \rho^2}{c^2} - 1 \right) B_{\varphi} = 0
    \end{equation}
    These are solutions of the Bessel equation:
    \begin{equation}
        x^2 \partial^2_x y + x \partial_x y + (x^2 - n^2)y = 0
    \end{equation}
    for $ x = \frac{\omega \rho}{c} $ and $ n = 0,1 $ for the $ E $ and $ B $-fields, respectively. This means we can write our solutions in terms of Bessel functions of the first kind:
    \begin{equation}
        E_z(\rho) = E_z^0 J_0 \left( \frac{\omega \rho}{c} \right)
    \end{equation}
    and
    \begin{equation}
        B_{\varphi}(\rho) = B_{\varphi}^0 J_1 \left( \frac{\omega \rho}{c} \right)
    \end{equation}
    where $ E^0 $ and $ B^0 $ are found from boundary conditions. In our case, the charge on the plates at time $ t $ is
    \begin{equation}
        q = \int I_0 e^{- \imath \omega t} = \frac{\imath I_0}{\omega} e^{-im \omega t}
    \end{equation}
    so the electric field will be $ \frac{\sigma}{\epsilon_0} = -\frac{\imath I_0 e^{- \imath \omega t}}{\epsilon_0 \pi a^2 \omega} $. Using the fact that $ J'_0 = - J_1 $, we find that
    \begin{equation}
        B_{\varphi} = \frac{\imath}{c} E^0_z J'_0 = - \frac{\imath}{c} E^0_z J_1
    \end{equation}
    we can conclude that
    \begin{equation}
        E_z = \frac{\imath I_0 e^{- \imath \omega t}}{\pi \epsilon_0 a^2 \omega} J_0 \left( \frac{\omega \rho}{c} \right)
    \end{equation}
    and
    \begin{equation}
        B_{\varphi} = - \frac{\mu_0 I_0 c e^{- \imath \omega t}}{\pi a^2 \omega} J_1 \left( \frac{\omega \rho}{c} \right)
    \end{equation}
    However, we only want the real parts of the fields, so we can change the exponential to a trig function. I will also expand the Bessel functions to second order:
    \begin{equation}
        E_z = - \frac{I_0 \sin(\omega t)}{\pi \epsilon_0 a^2 \omega} \left( 1 - \frac{\rho^2 \omega^2}{4 c^2} + \order{\omega^4} \right)
    \end{equation}
    and (factoring out a $\rho$ and factoring in a $ c $ and $ \omega $ to make this expansion look nicer):
    \begin{equation}
        B_{\varphi} = - \frac{\mu_0 I_0 \rho \cos(\omega t)}{2 \pi a^2} \left( 1 - \frac{\rho^2 \omega^2}{8 c^2} + \order{\omega^4} \right)
    \end{equation}
\end{problem}

\section{Propagation of Waves in Two Dimensions}
Suppose we study propagations of waves in two dimensions (this could be sound propagation in a medium for example) with velocity $ v $, and there are no boundaries, hence it is the full two-dimensional plane. The wave equation can be written as
\begin{equation}
    \left[ \pdv[2]{x} + \pdv[2]{y} \right] \psi( \va{x};t) - \frac{1}{v^2} \pdv[2]{\psi( \va{x};t)}{t} = 0. 
\end{equation}

Suppose that \textit{initial conditions are specified}, that is, $ \psi( \va{x};0) $ and $ \pdv{\psi( \va{x};0)}{t} $ are given functions. Using a Fourier transform, show that the solution is given by

\begin{equation}
    \psi( \va{x};t) = \frac{1}{2 \pi v} \pdv{t} \int_{\abs{\va{x}-\va{x}'}\leq vt} \frac{\dd[2]{x'}}{\sqrt{(vt)^2 - \abs{\va{x}-\va{x}'}^2}} \psi( \va{x}';0) + \frac{1}{2 \pi v} \int_{\abs{\va{x}-\va{x}'}\leq vt} \frac{\dd[2]{x'}}{\sqrt{(vt)^2 - \abs{\va{x}-\va{x}'}^2}} \pdv{\psi( \va{x}';0)}{t}
\end{equation}

\begin{problem}
    First, by a Fourier transform, we can write
    \begin{equation}
        \psi( \va{x};t) = \frac{1}{(2 \pi)^2} \int \psi( \va{k};t) e^{- \imath \va{k} \vdot \va{x}} \dd[2]{k}
    \end{equation}
    so the wave equation becomes
    \begin{equation}
        \left[ - k^2 - \frac{1}{v^2} \partial_t \right] \psi( \va{k};t) = 0
    \end{equation}
    We can solve this. It has solutions
    \begin{equation}
        \psi( \va{k};t) = A \cos(kvt) + B \sin(kvt)
    \end{equation}
    At $ t = 0 $, we find that the initial conditions give this solution the following form:
    \begin{equation}
        \psi( \va{k};t) = \psi( \va{k};0) \cos(kvt) + \partial_t \psi( \va{k};0) \sin(kvt)
    \end{equation}
    where
    \begin{equation}
        \psi( \va{k};0) = \int \dd[2]{x'} e^{\imath \va{k} \vdot \va{x}'} \psi( \va{x};0)
    \end{equation}
    and similar for the time derivative.

    Together, we have:
    \begin{align}
        \psi( \va{x};t) = &\frac{1}{(2 \pi)^2} \iint \cos(kvt) e^{- \imath \va{k} (\va{x} - \va{x}')} \psi( \va{x};0) \dd[2]{x'} \dd[2]{k} \\ &+ \frac{1}{(2 \pi)^2} \iint \sin(kvt) e^{- \imath \va{k} (\va{x} - \va{x}')} \partial_t \psi( \va{x};0) \dd[2]{x'} \dd[2]{k} 
    \end{align}
    
    If we rewrite $ \va{x} - \va{x}' = \abs{ \va{x} - \va{x}'} \cos(\theta) $ where $\theta$ is the angle between them, we see that $ e^{- \imath \va{k} \abs{ \va{x} - \va{x}'} \cos(\theta)} $ gives Bessel functions of the first kind (in particular we get $ \pi J_0(k\abs{ \va{x} - \va{x}'}) $ if we integrate over $ \theta $):
    \begin{equation}
        \int_0^{\pi} e^{\imath z \cos(\theta)} \dd{thet a} = \pi J_0(z)
    \end{equation}
    so
    \begin{align}
        \psi( \va{x};t) = &\frac{1}{(2 \pi)^2} \iint \cos(kvt) J_0(k\abs{ \va{x} - \va{x}'}) \psi( \va{x};0) \dd[2]{k} \dd[2]{x'} \\&+ \frac{1}{(2 \pi)^2} \iint \sin(kvt) J_0(k \abs{ \va{x} - \va{x}'}) \partial_t \psi( \va{x};0) \dd[2]{k} \dd[2]{x'}
    \end{align}
    These integrals are only nonzero in the region where $ \abs{ \va{x} - \va{x}'} \leq vt $, and they evaluate to $ \frac{1}{\sqrt{(vt)^2 - \abs{ \va{x} - \va{x}'}^2}} $, so our answer becomes
    \begin{equation}
        \psi( \va{x};t) = \frac{1}{(4 \pi} \int_{\abs{ \va{x} - \va{x}'} \leq vt} \frac{\dd[2]{x'}}{\sqrt{(vt)^2 - \abs{\va{x}-\va{x}'}^2}} \psi( \va{x}';0) + \frac{1}{4 \pi } \int_{\abs{\va{x}-\va{x}'}\leq vt} \frac{\dd[2]{x'}}{\sqrt{(vt)^2 - \abs{\va{x}-\va{x}'}^2}} \pdv{\psi( \va{x}';0)}{t}
    \end{equation}
    
    This isn't quite the correct answer, and I'm not sure where I missed the factor of $ \frac{2}{v} $ or the time derivative in the first half.

\end{problem}

\end{document}
