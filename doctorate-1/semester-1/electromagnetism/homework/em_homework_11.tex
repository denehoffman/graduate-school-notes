\documentclass[a4paper,twoside]{article}
% My LaTeX preamble file - by Nathaniel Dene Hoffman
% Credit for much of this goes to Olivier Pieters (https://olivierpieters.be/tags/latex)
% and Gilles Castel (https://castel.dev)
% There are still some things to be done:
% 1. Update math commands using mathtools package (remove ddfrac command and just override)
% 2. Maybe abbreviate \imath somehow?
% 3. Possibly format for margin notes and set new margin sizes
% First, some encoding packages and useful formatting
%--------------------------------------------------------------------------------------------
\usepackage{import}
\usepackage{pdfpages}
\usepackage{transparent}
\usepackage[l2tabu,orthodox]{nag}   % force newer (and safer) LaTeX commands
\usepackage[utf8]{inputenc}         % set character set to support some UTF-8
                                    %   (unicode). Do NOT use this with
                                    %   XeTeX/LuaTeX!
\usepackage[T1]{fontenc}
\usepackage[english]{babel}         % multi-language support
\usepackage{sectsty}                % allow redefinition of section command formatting
\usepackage{tabularx}               % more table options
\usepackage{booktabs}
\usepackage{titling}                % allow redefinition of title formatting
\usepackage{imakeidx}               % create and index of words
\usepackage{xcolor}                 % more colour options
\usepackage{enumitem}               % more list formatting options
\usepackage{tocloft}                % redefine table of contents, new list like objects
\usepackage{subfiles}               % allow for multifile documents

% Next, let's deal with the whitespaces and margins
%--------------------------------------------------------------------------------------------
\usepackage[centering,margin=1in]{geometry}
\setlength{\parindent}{0cm}
\setlength{\parskip}{2ex plus 0.5ex minus 0.2ex} % whitespace between paragraphs

% Redefine \maketitle command with nicer formatting
%--------------------------------------------------------------------------------------------
\pretitle{
  \begin{flushright}         % align text to right
    \fontsize{40}{60}        % set font size and whitespace
    \usefont{OT1}{phv}{b}{n} % change the font to bold (b), normally shaped (n)
                             %   Helvetica (phv)
    \selectfont              % force LaTeX to search for metric in its mapping
                             %   corresponding to the above font size definition
}
\posttitle{
  \par                       % end paragraph
  \end{flushright}           % end right align
  \vskip 0.5em               % add vertical spacing of 0.5em
}
\preauthor{
  \begin{flushright}
    \large                   % font size
    \lineskip 0.5em          % inter line spacing
    \usefont{OT1}{phv}{m}{n}
}
\postauthor{
  \par
  \end{flushright}
}
\predate{
  \begin{flushright}
  \large
  \lineskip 0.5em
  \usefont{OT1}{phv}{m}{n}
}
\postdate{
  \par
  \end{flushright}
}

% Mathematics Packages
\usepackage[Gray,squaren,thinqspace,cdot]{SIunits}      % elegant units
\usepackage{amsmath}                                    % extensive math options
\usepackage{amsfonts}                                   % special math fonts
\usepackage{mathtools}                                  % useful formatting commands
\usepackage{amsthm}                                     % useful commands for building theorem environments
\usepackage{amssymb}                                    % lots of special math symbols
\usepackage{mathrsfs}                                   % fancy scripts letters
\usepackage{cancel}                                     % cancel lines in math
\usepackage{esint}                                      % fancy integral symbols
\usepackage{relsize}                                    % make math things bigger or smaller
%\usepackage{bm}                                         % bold math!
\usepackage{slashed}

\newcommand\ddfrac[2]{\frac{\displaystyle #1}{\displaystyle #2}}    % elegant fraction formatting
\allowdisplaybreaks[1]                                              % allow align environments to break on pages

% Ensure numbering is section-specific
%--------------------------------------------------------------------------------------------
\numberwithin{equation}{section}
\numberwithin{figure}{section}
\numberwithin{table}{section}

% Citations, references, and annotations
%--------------------------------------------------------------------------------------------
\usepackage[small,bf,hang]{caption}        % captions
\usepackage{subcaption}                    % adds subfigure & subcaption
\usepackage{sidecap}                       % adds side captions
\usepackage{hyperref}                      % add hyperlinks to references
\usepackage[noabbrev,nameinlink]{cleveref} % better references than default \ref
\usepackage{autonum}                       % only number referenced equations
\usepackage{url}                           % urls
\usepackage{cite}                          % well formed numeric citations
% format hyperlinks
\colorlet{linkcolour}{black}
\colorlet{urlcolour}{blue}
\hypersetup{colorlinks=true,
            linkcolor=linkcolour,
            citecolor=linkcolour,
            urlcolor=urlcolour}

% Plotting and Figures
%--------------------------------------------------------------------------------------------
\usepackage{tikz}          % advanced vector graphics
\usepackage{pgfplots}      % data plotting
\usepackage{pgfplotstable} % table plotting
\usepackage{placeins}      % display floats in correct sections
\usepackage{graphicx}      % include external graphics
\usepackage{longtable}     % process long tables

% use most recent version of pgfplots
\pgfplotsset{compat=newest}

% Misc.
%--------------------------------------------------------------------------------------------
\usepackage{todonotes}  % add to do notes
\usepackage{epstopdf}   % process eps-images
\usepackage{float}      % floats
\usepackage{stmaryrd}   % some more nice symbols
\usepackage{emptypage}  % suppress page numbers on empty pages
\usepackage{multicol}   % use this for creating pages with multiple columns
\usepackage{etoolbox}   % adds tags for environment endings
\usepackage{tcolorbox}  % pretty colored boxes!


% Custom Commands
%--------------------------------------------------------------------------------------------
\newcommand\hr{\noindent\rule[0.5ex]{\linewidth}{0.5pt}}                % horizontal line
\newcommand\N{\ensuremath{\mathbb{N}}}                                  % blackboard set characters
\newcommand\R{\ensuremath{\mathbb{R}}}
\newcommand\Z{\ensuremath{\mathbb{Z}}}
\newcommand\Q{\ensuremath{\mathbb{Q}}}
%\newcommand\C{\ensuremath{\mathbb{C}}}
\renewcommand{\arraystretch}{1.2}                                       % More space between table rows (could be 1.3)
\newcommand{\Cov}{\mathrm{Cov}}
\newcommand\D{\mathrm{D}}
\newcommand*{\dbar}{\ensuremath{\text{\dj}}}

\newcommand{\incfig}[2][1]{%
    \def\svgwidth{#1\columnwidth}
    \import{./figures/}{#2.pdf_tex}
}

% Custom Environments
%--------------------------------------------------------------------------------------------
\newcommand{\lecture}[3]{\hr\\{\centering{\large\textsc{Lecture #1: #3}}\\#2\\}\hr\markboth{Lecture #1: #3}{\rightmark}}   % command to title lectures
\usepackage{mdframed}
\theoremstyle{plain}
\newmdtheoremenv[nobreak]{theorem}{Theorem}[section]
\newtheorem{corollary}{Corollary}[theorem]
\newtheorem{lemma}[theorem]{Lemma}
\theoremstyle{definition}
\newtheorem*{ex}{Example}
\newmdtheoremenv[nobreak]{definition}{Definition}[section]
\theoremstyle{remark}
\newtheorem*{remark}{Remark}
\newtheorem*{claim}{Claim}
\AtEndEnvironment{ex}{\null\hfill$\diamond$}%
% Note: A proof environment is already provided in the amsthm package
\tcbuselibrary{breakable}
\newenvironment{note}[1]{\begin{tcolorbox}[
    arc=0mm,
    colback=white,
    colframe=white!60!black,
    title=#1,
    fonttitle=\sffamily,
    breakable
]}{\end{tcolorbox}}
\newenvironment{problem}{\begin{tcolorbox}[
    arc=0mm,
    breakable,
    colback=white,
    colframe=black
]}{\end{tcolorbox}}

% Header and Footer
%--------------------------------------------------------------------------------------------
% set header and footer
\usepackage{fancyhdr}                       % header and footer
\pagestyle{fancy}                           % use package
\fancyhf{}
\fancyhead[LE,RO]{\textsl{\rightmark}}      % E for even (left pages), O for odd (right pages)
\fancyfoot[LE,RO]{\thepage}
\fancyfoot[LO,RE]{\textsl{\leftmark}}
\setlength{\headheight}{15pt}


% Physics
%--------------------------------------------------------------------------------------------
\usepackage[arrowdel]{physics}      % all the usual useful physics commands
\usepackage{feyn}                   % for drawing Feynman diagrams
%\usepackage{bohr}                   % for drawing Bohr diagrams
%\usepackage{tikz-feynman}
\usepackage{elements}               % for quickly referencing information of various elements
\usepackage{tensor}                 % for writing tensors and chemical symbols

% Finishing touches
%--------------------------------------------------------------------------------------------
\author{Nathaniel D. Hoffman}

\title{33-761 Homework 11}
\date{\today}
\begin{document}
\maketitle

\section*{1. Jackson Problem 7.16 (a) and (b)}
Plane waves propagate in a homogeneous, nonpermeable, but \textit{anisotropic} dielectric. The dielectric is characterized by a tensor $ \epsilon_{ij} $, but if coordinate axes are chosen as the principle axis, the components of displacement along these axes are related to the electric-field components by $ D_i = \epsilon_i E_i $ ($ i = 1,2,3 $), where $ \epsilon_i $ are the eigenvalues of the matrix $ \epsilon_{ij} $.
\begin{itemize}
    \item[(a)] Show that plane waves with frequency $ \omega $ and wave vector $ \va{k} $ must satisfy
        \begin{equation}
            \va{k} \cross ( \va{k} \cross \va{E}) + \mu_0 \omega^2 \va{D} = 0
        \end{equation}
        \begin{problem}
            A plane wave will have
            \begin{equation}
                \va{E} = \va{E}_0 e^{\imath \left( \va{k} \vdot \va{x} - \omega t \right)}
            \end{equation}
            and
            \begin{equation}
                \va{H} = \va{H}_0 e^{\imath \left( \va{k} \vdot \va{x} - \omega t \right)}
            \end{equation}
            Using the Maxwell curl equations in free space, we find that
            \begin{equation}
                \curl{ \va{E}} = \imath \va{k} \cross \va{E} = \imath \omega \va{B} = \imath \omega \mu_0 \va{H}
            \end{equation}
            and
            \begin{equation}
                \curl{ \va{H}} = \imath \va{k} \cross \va{H} = - \imath \omega \va{D}
            \end{equation}
            Taking the cross product of $ \imath \va{k} $ with the first equation, we find
            \begin{equation}
                \imath \va{k} \cross (\imath \va{k} \cross \va{E}) - \imath \omega \mu_0 (\imath \va{k} \cross \va{H}) = 0
            \end{equation}
            Substituting in the second equation, we have
            \begin{equation}
                \imath \va{k} \cross (\imath \va{k} \cross \va{E}) - \imath \mu_0 \omega ( \imath \omega \va{D})
            \end{equation}
            or
            \begin{equation}
                \va{k} \cross ( \va{k} \cross \va{E}) + \mu_0 \omega^2 \va{D} = 0
            \end{equation}
            once the $ \imath $s are canceled.
        \end{problem}
    \item[(b)] Show that for a given wave vector $ \va{k} = k \va{n} $ there are two distinct modes of propagation with different phase velocities $ v = \frac{\omega}{k} $ that satisfy the Fresnel equation
        \begin{equation}
            \sum_{i=1}^{3} \frac{n_i^2}{v^2 - v_i^2} = 0
        \end{equation}
        where $ v_i = \frac{1}{\sqrt{\mu_0 \epsilon_i}} $ is called a principal velocity, and $ n_i $ is the component of $ \va{n} $ along the $ i $th principal axis.
        \begin{problem}
            Substituting $ \va{k} = k \va{n} $ into the above equation and dividing out $ k $, we find
            \begin{equation}
                \va{n} \cross ( \va{n} \cross \va{E}) + \mu_0 v^2 \va{D} = 0
            \end{equation}
            Using the double cross-product identity, this is equal to
            \begin{equation}
                \va{n} ( \va{n} \vdot \va{E}) - \va{E} + \mu_0 v^2 \va{D} = 0
            \end{equation}
            Taking the dot product with $ \va{n} $:
            \begin{equation}
                (\va{n}^2 - 1)( \va{n} \vdot \va{E}) + \mu_0 v^2 \va{n} \vdot \va{D} = 0
            \end{equation}
            Next, we can write this in index notation:
            \begin{equation}
                (n_i^2 - 1)(n_i E_i) + v^2 \mu_0 \epsilon_i E_i n_i = 0
            \end{equation}
            We can recognize $ \mu_0 \epsilon_i = \frac{1}{v_i^2} $ and divide out $ E_i n_i $:
            \begin{equation}
                (n_i^2 - 1) + \frac{v^2}{v_i^2} = 0
            \end{equation}
            or
            \begin{align}
                v_i^2 (n_i^2 - 1) + v^2 &= 0 \\
                v_i^2 n_i^2 &= - (v^2 - v_i^2) \\
                - v_i^2 \frac{n_i^2}{v^2 - v_i^2} &= 0 \\
                \frac{n_i^2}{v^2 - v_i^2} &= 0
            \end{align}
            with the usual implied sum over $ i $.
        \end{problem}
\end{itemize}

\section*{2. Radiation Power}
For vacuum define $ \va{P} = \int \dd[3]{x} \epsilon_0 \va{E} \cross \va{B} $, and similarly define the total energy as $ U = \frac{1}{2}\int \dd[3]{x[\epsilon_0 \va{E}^2 + \frac{1}{\mu_0} \va{B}^2} $ (assuming localized fields to make integrals convergent).
\begin{itemize}
    \item[(a)] Show that we always have,
        \begin{equation}
            c \abs{ \va{P}} \leq U
        \end{equation}
        \begin{problem}
            First note that by definition,
            \begin{equation}
                \abs{\int f(x) \dd{x}} \leq \int \abs{f(x)} \dd{x}
            \end{equation}
            so
            \begin{equation}
                \abs{ \va{P}} \leq \int \dd[3]{x} \abs{\epsilon_0 \va{E} \cross \va{B}}
            \end{equation}
            Note that
            \begin{equation}
                \epsilon_0 = \frac{1}{c} \sqrt{\frac{\epsilon_0}{\mu_0}},
            \end{equation}
            so we can write this as
            \begin{equation}
                c \abs{ \va{P}} \leq \int \dd[3]{x} \sqrt{\frac{\epsilon_0}{\mu_0}} \abs{ \va{E} \cross \va{B}}
            \end{equation}
            By definition of the cross product,
            \begin{align}
                c \abs{ \va{P}} &\leq \int \dd[3]{x} \sqrt{\frac{\epsilon_0}{\mu_0}} \abs{ \va{E} \cross \va{B}} \\
                &= \int \dd[3]{x} \sqrt{\frac{\epsilon_0}{\mu_0}} \abs{ \va{E}} \abs{ \va{B}} \abs{\sin(\gamma)} \\
                &\leq \int \left[ \sqrt{\frac{\epsilon_0}{\mu_0}} \abs{ \va{E}} \abs{ \va{B}} + \frac{\epsilon_0}{2} \left( \va{E} - c \va{B} \right)^2  \right] \abs{\sin(\gamma)} \\ 
                &= \int \left[ \sqrt{\frac{\epsilon_0}{\mu_0}} \abs{ \va{E}} \abs{ \va{B}} + \frac{1}{2} \left( \sqrt{\epsilon_0} \va{E} - \frac{1}{\sqrt{\mu_0}}  \va{B} \right)^2  \right] \abs{\sin(\gamma)} \\
                &= \int \left[ \sqrt{\frac{\epsilon_0}{\mu_0}} \abs{ \va{E}} \abs{ \va{B}} + \frac{\epsilon_0}{2} \va{E}^2 + \frac{1}{2 \mu_0} \va{B}^2 - \frac{1}{2} 2 \sqrt{\frac{\epsilon_0}{\mu_0}} \abs{ \va{E}} \abs{ \va{B}} \right] \abs{\sin(\gamma)} \\
                &= \int \frac{1}{2} \left[ \epsilon_0 \va{E}^2 + \frac{1}{\mu_0} \va{B}^2 \right] \abs{\sin(\gamma)} \leq U
            \end{align}
            where $ \gamma $ is the angle between the field vectors and $ 0 \leq \abs{\sin(\gamma)} \leq 1 $ for the final inequality.
        \end{problem}
    \item[(b)] Show that if we demand the equality, this necessarily implies $ c \abs{ \va{B}} = \abs{ \va{E}} $ and $ \va{E} \vdot \va{B} = 0 $ as in plane wave solutions.
        \begin{problem}
            For the equality to hold, we demand that the quantity we added, $ \frac{\epsilon_0}{2} \left( \sqrt{\epsilon} \va{E} - \frac{1}{\sqrt{\mu_0}} \va{B} \right)^2 = \frac{\epsilon_0}{2} \left( \va{E} - c \va{B} \right)^2 = 0 $, which is true if $ \abs{ \va{E}} = c \abs{ \va{B}} $, and $ \sin(\gamma) = 1 $, which is true if $ \va{E} \vdot \va{B} = 0 $. 
        \end{problem}
\end{itemize}

\section*{3. Jackson Problem 9.3}
Note that here the dominant mode is an electric dipole.

Two halves of a spherical metallic shell of radius $ R $ and infinite conductivity are separated by a very small insulating gap. An alternating potential is applied between the two halves of the sphere so that the potentials are $ \pm V \cos(\omega t) $. In the long-wavelength limit, find the radiation fields, the angular distribution of radiated power, and the total radiated power from the sphere.
\begin{problem}
    First, we want to find an expansion for the potential. Due to the spherical and azimuthal symmetry, we can expand the potential as
    \begin{equation}
        \Phi = \sum_{l=0}^{\infty} \left[ A_l r^l + B_l r^{-(l+1)} \right] P_l(\cos(\theta))
    \end{equation}
    Outside the sphere, $ A_l = 0 $ and we can solve for $ B_l $:
    \begin{equation}
        V(\theta) = \sum_l B_l R^{-(l+1)} P_l(\cos(\theta))
    \end{equation}
    so
    \begin{equation}
        B_l = R^{l+1} \frac{2l+1}{2} \int_{-1}^{1} V(\cos(\theta)) P_l(\cos(\theta)) \dd{\cos(\theta)}
    \end{equation}
    Since the potentials are opposite in magnitude on either side of the sphere, this becomes
    \begin{equation}
        B_l = R^{l+1} \frac{2l+1}{2} V\left( \int_0^1 P_l(\cos(\theta)) \dd{\cos(\theta)} - \int_{-1}^{0} P_l(\cos(\theta)) \dd{\cos(\theta)} \right)
    \end{equation}
    For now, I am ignoring the time dependence in the potential and will just add it in at the end. Because the Legendre polynomials are even/odd if $ l $ is even/odd, the even $ l $ will cancel out:
    \begin{equation}
        B_l = R^{l+1} V (2l+1) \int_0^1 P_l(\cos(\theta)) \dd{\cos(\theta)} \qfor l \qodd
    \end{equation}
    We assume this system looks approximately like a dipole in the radiation zone, and the potential with respect to the dipole moment is
    \begin{equation}
        \Phi = \frac{1}{4 \pi \epsilon_0} \frac{p}{r^2} \cos(\theta) = \frac{1}{4 \pi \epsilon_0} \frac{p}{r^2} P_1(\cos(\theta))
    \end{equation}
    The $ l = 1 $ term in our first expansion is
    \begin{equation}
        V \frac{3}{2} \left( \frac{R}{r} \right)^2 P_1(\cos(\theta))
    \end{equation}
    and setting these equal to each other, we find that
    \begin{equation}
        \va{p} = 4 \pi \epsilon_0 V \frac{3}{2} R^2 \vu{z} = 6 \pi \epsilon_0 R^2 V \vu{z}
    \end{equation}
    assuming the symmetry is about the $ \vu{z} $-axis. We can now calculate the fields in the radiation zone using the dipole moment, adding in the time dependence:
    \begin{equation}
        \va{H} = \frac{c k^2}{4 \pi} \vu{n} \cross \va{p} \frac{e^{\imath kr}}{r} \mapsto - \frac{3V}{2Z_0} (kR)^2 \sin(\theta) \frac{e^{\imath (kr - \omega t)}}{r} \vu{\varphi} 
    \end{equation}
    \begin{equation}
        \va{E} = Z_0 \va{H} \cross \vu{n} \mapsto - \frac{3V}{2} (kR)^2 \sin(\theta) \frac{e^{\imath (kr - \omega t)}}{r} \vu{\theta}
    \end{equation}
    \begin{equation}
        \dv{P}{\Omega} = \frac{c^2 Z_0}{32 \pi^2} \abs{( \vu{n} \cross \va{p}) \cross \vu{n}}^2 k^4 = \frac{9}{8} (kR)^{4} \frac{V^2}{Z_0} \sin[2](\theta)
    \end{equation}
    and
    \begin{equation}
        P = \int \dv{P}{\Omega} \dd{\Omega} = 3 \pi (kR)^4 \frac{V^2}{Z_0}
    \end{equation}
\end{problem}

\section*{4. Jackson Problem 9.8 (a) and (c)}
\begin{itemize}
    \item[(a)] Show that a classical oscillating electric dipole $ \va{p} $ with fields given by (9.18) radiates electromagnetic angular momentum to infinity at the rate
        \begin{equation}
            \dv{ \va{L}}{t} = \frac{k^3}{12 \pi \epsilon_0} \Im[ \va{p}^* \cross \va{p}]
        \end{equation}
        Equation(s) 9.18:
        \begin{equation}
            \va{H} = \frac{ck^2}{4 \pi} ( \va{n} \cross \va{p}) \frac{e^{\imath kr}}{r} \left( 1 - \frac{1}{\imath kr} \right)
        \end{equation}
        \begin{equation}
            \va{E} = \frac{1}{4 \pi \epsilon_0} \left\{ k^2 \left( \va{n} \cross \va{p} \right) \cross \va{n} \frac{e^{\imath kr}}{r} + \left[ 3 \va{n} ( \va{n} \vdot \va{p}) - \va{p} \right]\left( \frac{1}{r^3} - \frac{\imath k}{r^2} \right) e^{\imath kr} \right\}
        \end{equation}
        \begin{problem}
            The linear momentum density is defined as
            \begin{equation}
                \va{g} = \frac{1}{2 c^2} \va{E} \cross \va{H}^*
            \end{equation}
            Using this, we can define angular momentum density as
            \begin{equation}
                \va{\ell} = \va{r} \cross \va{g} = \va{r} \cross ( \va{E} \cross \va{H}^*) = \frac{1}{2 c^2} \left[ \va{E} ( \va{r} \vdot \va{H}^*) - \va{H}^* ( \va{r} \vdot \va{E}) \right]
            \end{equation}
            Note that
            \begin{equation}
                n_i \epsilon_{ijk} n_j p_k = \epsilon_{ijk} \delta_{ij} p_k = 0
            \end{equation}
            so if $ \va{r} = r \va{n} $, $ \va{r} \vdot \va{H} = 0 $ and $ \va{r} \vdot \va{E} $ will only contain the second term in the large brackets:
            \begin{equation}
                \va{r} \vdot \va{E} = \frac{1}{4 \pi \epsilon_0} (2 \va{n} \vdot \va{p}) \left( \frac{1}{r^2} - \frac{\imath k}{r} \right) e^{\imath kr}
            \end{equation}
            so
            \begin{equation}
                \va{\ell} = \frac{1}{2c^2} \va{H}^* ( \va{r} \vdot \va{E}) = \frac{\imath k^3}{16 \pi^2 \epsilon_0 c r^2} \left( 1 + \frac{1}{k^2 r^2} \right) ( \va{n} \vdot \va{p}) ( \va{n} \cross \va{p}^*)
            \end{equation}
            We want to calculate the radiated angular momentum, so instead of integrating this over all space, we can just integrate it over a sphere of radius $ r $ and then take $ r \to \infty $. In other words, $ \dd{ \va{\ell}} = \va{\ell} \dd{a} \dd{r} = \va{\ell} r^2 \dd{r} \dd{\Omega} $ so
            \begin{equation}
                \dv{ \va{L}}{t} = \va{\ell} r^2 \dv{ \va{r}}{t} \dd{\Omega} = \va{\ell} r^2 c \dd{\Omega}
            \end{equation}
            so
            \begin{equation}
                \dv{ \va{L}}{t} = \frac{\imath k^3}{16 \pi^2 \epsilon_0} \left( 1 + \frac{1}{k^2 r^2} \right) \int ( \va{n} \vdot \va{p}) ( \va{n} \cross \va{p}^*) \dd{\Omega}
            \end{equation}
            To perform this integral, note that the integrand, in index notation, is
            \begin{equation}
                n_i p_i \epsilon_{ijk} n_j p_k^* = n_i n_j \epsilon_{ijk} p_k^* p_i \mapsto - n_i n_j (\va{p}^* \cross \va{p})
            \end{equation}
            Finally, the integral over $ \int n_i n_j \dd{\Omega} = \frac{4 \pi}{3} \delta_{ij} $, so, taking the limit as $ r \to \infty $,
            \begin{equation}
                \dv{ \va{L}}{t} = -\frac{\imath k^3}{16 \pi^2 \epsilon_0} \frac{4 \pi}{3} ( \va{p}^* \cross \va{p}) = \frac{k^3}{12 \pi \epsilon_0} \Im[ \va{p}^* \cross \va{p}]
            \end{equation}
        \end{problem}
        \item[(c)] For a charge $ e $ rotating in the $ x$-$y $ plane at radius $ a $ and angular speed $ \omega $, show that there is only a $ z $ component of radiated angular momentum with magnitude $ \dv{L_z}{t} = \frac{e^2 k^3 a^2}{6 \pi \epsilon_0} $. What about a charge oscillating along the $ z $ axis?
        \begin{problem}
            The charge distribution for such a system is given by
            \begin{equation}
                \rho = e \delta(x-a \cos(\omega t)) \delta(y - a \sin(\omega t)) \delta(z)
            \end{equation}
            The dipole moment is therefore
            \begin{equation}
                \va{p} = \int \va{x} \rho \dd[3]{x} = ea ( \vu{x} \cos(\omega t) + \vu{y} \sin(\omega t)) = \Re[ea( \vu{x} + \imath \vu{y}) e^{- \imath \omega t}]
            \end{equation}
            If we just look at
            \begin{equation}
                \va{p}( \va{x}) = ea ( \vu{x} + \imath \vu{y}),
            \end{equation}
            we see that
            \begin{equation}
                \va{p}^* \cross \va{p} = e^2 a^2 \imath( \vu{x} \cross \vu{y} - \vu{y} \cross \vu{x}) = 2 e^2 a^2 \imath \vu{z}
            \end{equation}
            so
            \begin{equation}
                \dv{ \va{L}}{t} = \frac{k^3 e^2 a^2}{6 \pi \epsilon_0} \vu{z}
            \end{equation}
            from the formula found in part (a).
        \end{problem}
\end{itemize}



\end{document}


