\documentclass[a4paper,twoside]{article}
% My LaTeX preamble file - by Nathaniel Dene Hoffman
% Credit for much of this goes to Olivier Pieters (https://olivierpieters.be/tags/latex)
% and Gilles Castel (https://castel.dev)
% There are still some things to be done:
% 1. Update math commands using mathtools package (remove ddfrac command and just override)
% 2. Maybe abbreviate \imath somehow?
% 3. Possibly format for margin notes and set new margin sizes
% First, some encoding packages and usefull formatting
%--------------------------------------------------------------------------------------------
\usepackage[l2tabu,orthodox]{nag}   % force newer (and safer) LaTeX commands
\usepackage[utf8]{inputenc}         % set character set to support some UTF-8
                                    %   (unicode). Do NOT use this with
                                    %   XeTeX/LuaTeX!
\usepackage[T1]{fontenc}
\usepackage[english]{babel}         % multi-language support
\usepackage{sectsty}                % allow redefinition of section command formatting
\usepackage{tabularx}               % more table options
\usepackage{booktabs}
\usepackage{titling}                % allow redefinition of title formatting
\usepackage{imakeidx}               % create and index of words
\usepackage{xcolor}                 % more colour options
\usepackage{enumitem}               % more list formatting options
\usepackage{tocloft}                % redefine table of contents, new list like objects
\usepackage{subfiles}               % allow for multifile documents

% Next, let's deal with the whitespaces and margins
%--------------------------------------------------------------------------------------------
\usepackage[centering,margin=1in]{geometry}
\setlength{\parindent}{0cm}
\setlength{\parskip}{2ex plus 0.5ex minus 0.2ex} % whitespace between paragraphs

% Redefine \maketitle command with nicer formatting
%--------------------------------------------------------------------------------------------
\pretitle{
  \begin{flushright}         % align text to right
    \fontsize{40}{60}        % set font size and whitespace
    \usefont{OT1}{phv}{b}{n} % change the font to bold (b), normally shaped (n)
                             %   Helvetica (phv)
    \selectfont              % force LaTeX to search for metric in its mapping
                             %   corresponding to the above font size definition
}
\posttitle{
  \par                       % end paragraph
  \end{flushright}           % end right align
  \vskip 0.5em               % add vertical spacing of 0.5em
}
\preauthor{
  \begin{flushright}
    \large                   % font size
    \lineskip 0.5em          % inter line spacing
    \usefont{OT1}{phv}{m}{n}
}
\postauthor{
  \par
  \end{flushright}
}
\predate{
  \begin{flushright}
  \large
  \lineskip 0.5em
  \usefont{OT1}{phv}{m}{n}
}
\postdate{
  \par
  \end{flushright}
}

% Mathematics Packages
\usepackage[Gray,squaren,thinqspace,cdot]{SIunits}      % elegant units
\usepackage{amsmath}                                    % extensive math options
\usepackage{amsfonts}                                   % special math fonts
\usepackage{mathtools}                                  % useful formatting commands
\usepackage{amsthm}                                     % useful commands for building theorem environments
\usepackage{amssymb}                                    % lots of special math symbols
\usepackage{mathrsfs}                                   % fancy scripts letters
\usepackage{cancel}                                     % cancel lines in math
\usepackage{esint}                                      % fancy integral symbols
\usepackage{relsize}                                    % make math things bigger or smaller
\usepackage{bm}                                         % bold math!

\newcommand\ddfrac[2]{\frac{\displaystyle #1}{\displaystyle #2}}    % elegant fraction formatting
\allowdisplaybreaks[1]                                              % allow align environments to break on pages

% Ensure numbering is section-specific
%--------------------------------------------------------------------------------------------
\numberwithin{equation}{section}
\numberwithin{figure}{section}
\numberwithin{table}{section}

% Citations, references, and annotations
%--------------------------------------------------------------------------------------------
\usepackage[small,bf,hang]{caption}        % captions
\usepackage{subcaption}                    % adds subfigure & subcaption
\usepackage{sidecap}                       % adds side captions
\usepackage{hyperref}                      % add hyperlinks to references
\usepackage[noabbrev,nameinlink]{cleveref} % better references than default \ref
\usepackage{autonum}                       % only number referenced equations
\usepackage{url}                           % urls
\usepackage{cite}                          % well formed numeric citations
% format hyperlinks
\colorlet{linkcolour}{black}
\colorlet{urlcolour}{blue}
\hypersetup{colorlinks=true,
            linkcolor=linkcolour,
            citecolor=linkcolour,
            urlcolor=urlcolour}

% Plotting and Figures
%--------------------------------------------------------------------------------------------
\usepackage{tikz}          % advanced vector graphics
\usepackage{pgfplots}      % data plotting
\usepackage{pgfplotstable} % table plotting
\usepackage{placeins}      % display floats in correct sections
\usepackage{graphicx}      % include external graphics
\usepackage{longtable}     % process long tables

% use most recent version of pgfplots
\pgfplotsset{compat=newest}

% Misc.
%--------------------------------------------------------------------------------------------
\usepackage{todonotes}  % add to do notes
\usepackage{epstopdf}   % process eps-images
\usepackage{float}      % floats
\usepackage{stmaryrd}   % some more nice symbols
\usepackage{emptypage}  % suppress page numbers on empty pages
\usepackage{multicol}   % use this for creating pages with multiple columns
\usepackage{etoolbox}   % adds tags for environment endings
\usepackage{tcolorbox}  % pretty colored boxes!


% Custom Commands
%--------------------------------------------------------------------------------------------
\newcommand\hr{\noindent\rule[0.5ex]{\linewidth}{0.5pt}}                % horizontal line
\newcommand\N{\ensuremath{\mathbb{N}}}                                  % blackboard set characters
\newcommand\R{\ensuremath{\mathbb{R}}}
\newcommand\Z{\ensuremath{\mathbb{Z}}}
\newcommand\Q{\ensuremath{\mathbb{Q}}}
\newcommand\C{\ensuremath{\mathbb{C}}}
\renewcommand{\arraystretch}{1.2}                                       % More space between table rows (could be 1.3)
\newcommand{\Cov}{\mathrm{Cov}}
\newcommand*{\dbar}{\ensuremath{\text{\dj}}}
% Custom Environments
%--------------------------------------------------------------------------------------------
\newcommand{\lecture}[3]{\hr\\{\centering{\large\textsc{Lecture #1: #3}}\\#2\\}\hr\markboth{Lecture #1: #3}{\rightmark}}   % command to title lectures
\usepackage{mdframed}
\theoremstyle{plain}
\newmdtheoremenv[nobreak]{theorem}{Theorem}[section]
\newtheorem{corollary}{Corollary}[theorem]
\newtheorem{lemma}[theorem]{Lemma}
\theoremstyle{definition}
\newtheorem*{ex}{Example}
\newmdtheoremenv[nobreak]{definition}{Definition}[section]
\theoremstyle{remark}
\newtheorem*{remark}{Remark}
\AtEndEnvironment{ex}{\null\hfill$\diamond$}%
% Note: A proof environment is already provided in the amsthm package
\tcbuselibrary{breakable}
\newenvironment{note}[1]{\begin{tcolorbox}[
    arc=0mm,
    colback=white,
    colframe=white!60!black,
    title=#1,
    fonttitle=\sffamily,
    breakable
]}{\end{tcolorbox}}
\newenvironment{problem}{\begin{tcolorbox}[
    arc=0mm,
    breakable,
    colback=white,
    colframe=black
]}{\end{tcolorbox}}

% Header and Footer
%--------------------------------------------------------------------------------------------
% set header and footer
\usepackage{fancyhdr}                       % header and footer
\pagestyle{fancy}                           % use package
\fancyhf{}
\fancyhead[LE,RO]{\textsl{\rightmark}}      % E for even (left pages), O for odd (right pages)
\fancyfoot[LE,RO]{\thepage}
\fancyfoot[LO,RE]{\textsl{\leftmark}}
\setlength{\headheight}{15pt}


% Physics
%--------------------------------------------------------------------------------------------
\usepackage[arrowdel]{physics}      % all the usual useful physics commands
%\usepackage{feyn}                   % for drawing Feynman diagrams
%\usepackage{bohr}                   % for drawing Bohr diagrams
\usepackage{elements}               % for quickly referencing information of various elements
\usepackage{tensor}                 % for writing tensors and chemical symbols

% Finishing touches
%--------------------------------------------------------------------------------------------
\author{Nathaniel D. Hoffman}

\title{33-761 Homework 6}
\date{\today}
\begin{document}
\maketitle


\section*{1. Uniqueness of General Linear Dielectrics}
Let us consider general linear dielectrics. Recall that in principle we could have $ D_i(x) = \epsilon_{ij} (x) E_j(x) $ where summation over repeated indices are implied. If the medium has some specified free charge distribution inside as well as some boundary conditions, we then need to solve the following equation for the potential $ \phi $
\begin{equation}
    \dv{x^i}\left( \epsilon_{ij} \dv{x^j}\phi \right)= -\rho.
\end{equation}
Assume that we have a bunch of conductors immersed into the medium with fixed charges on them as well as some other boundary surfaces with potentials specified on them. Using these boundary conditions show that the above equation has a unique solution (assuming there is a solution). For this, recall the Green identity in the following form
\begin{equation}
    \int_{\Omega} \dd[3]{x} \left[ \partial_i(\epsilon_{ij}\partial_j U)U + \epsilon_{ij}\partial_j U\partial_i U \right] = \sum_{a} \oint_{S_a} U \epsilon_{ij} \partial_j U \dd{a_i}
\end{equation}
where $\Omega$ is the domain of interest, union of $ S_a $’s is the boundary of the domain $ \Omega $, and $ U $ is a scalar field.
\begin{tcolorbox}[breakable]
    First, let's suppose we have two different solutions, $ \phi_1 $ and $ \phi_2 $ and define $ U = \phi_1 - \phi_2 $. The first thing we should note is that the right-hand side of the equation for the Green identity is now zero, since
    \begin{equation}
        \sum_{a} \oint_{S_a} U \epsilon_{ij} \partial_j U \dd{a_i} = \sum_{a} \oint_{S_a} (\phi_1 - \phi_2) \epsilon_{ij} \partial_j U \dd{a_i} = 0
    \end{equation}
    and the potential on any given surface should be the same under both solutions. Next, we can examine the first part of the identity:
    \begin{equation}
        \partial_i(\epsilon_{ij}\partial_j U) = 0
    \end{equation}
    since the Laplacian of the potentials should give the charge in the domain, but this charge should be the same for any given conductor, so the difference in the charge, which would be given by the Laplacian over $ U $, should vanish. Finally, we are left with
    \begin{equation}
        \int_{\Omega} \dd[3]{x} \epsilon_{ij} \partial_j U \partial_i U = 0
    \end{equation}
    This means that either the dielectric tensor is zero (which would imply a rather boring problem), or the gradient of $ U $ is zero, which means that $ \phi $ is unique up to a constant.
\end{tcolorbox}

\section*{2. Solve Jackson Problem 4.5 part (a) only.}
A localized charge density $\rho$ is placed in an external electrostatic field described by potential $ \Phi^{(0)} $. The external potential varies slowly in space over the region where the charge density is different from zero.

From first principles calculate the total force acting on the charge distribution as an expansion in multipole moments times derivatives of the electric field, up to and including the quadrupole moments. Compare this to the expansion of the energy W. What is its connection to $ F $?
\begin{tcolorbox}[breakable]
    We can begin with the statement that
    \begin{equation}
        F = \int \dd[3]{x} \rho(x) E^{(0)}(x) = \int \dd[3]{x} \rho(x) \sum_{i} E_i^{(0)} \hat{x}_i
    \end{equation}
    Now I will Taylor expand around $ x = 0 $ in $ E^{(0)} $:
    \begin{equation}
        E_i(x) = \left[ E_i (x') + \sum_j x_j\partial_j' E_i(x') + \frac{1}{2} \sum_{jk} x_j x_k\partial_j'\partial_k' E_i(x') \right]\eval_{x'\to 0}
    \end{equation}
    Next, we plug this into the force equation (I am leaving off the superscript $ E^{(0)} $):
    \begin{align}
        F =& \sum_i \hat{x}_i E_i(x') \int \dd[3]{x} \rho(x)\\
        &+ \sum_{ij} \hat{x}_i \partial_j' E_i(x') \int \dd[3]{x} x_j \rho(x)\\
        &+ \frac{1}{2} \sum_{ijk} \hat{x}_i \partial_j'\partial_k' E_i(x') \int \dd[3]{x} x_j x_k \rho(x)
    \end{align}
    Next, we can use the fact that $ \curl{ \vec{E}} = 0 \implies \partial_j E_i = \partial_i E_j $:
    \begin{align}
        F =& qE(x')\\
        &+ \sum_{ij} \hat{x}_i \partial_i' E_j(x') \int \dd[3]{x} x_j \rho(x)\\
        &+ \frac{1}{2} \sum_{ijk} \hat{x}_i \partial_i'\partial_k' E_j(x') \int \dd[3]{x} x_j x_k \rho(x)
    \end{align}
    Now that we are summing over the same derivative components, we can simplify the summations as $ \nabla $'s. Additionally, we can recognize the integrals as the dipole and quadrupole moments. The quadrupole moment requires us to take the fact that the field is changing slowly, which implies $ \div{ \vec{E}} \approx 0 $ so we can subtract $ \frac{1}{6} r^2 \div{ \vec{E} (x')} $ to get the proper form:
    \begin{align}
        F =& qE^{(0)}(0)\\
        &+ \grad{[p \cdot E^{(0)}(x')]}\\
        &+ \frac{1}{6} \grad{[\sum_{jk}\partial_k' E_j^{(0)}(x')Q_jk]}
    \end{align}
    where $ x' \to 0 $.

    Since $ E^{(0)} = \grad{\Phi^{(0)}} $, we can rewrite this whole expression, factoring the gradient:
    \begin{equation}
        F = \grad{q\Phi(0) - p \cdot E(0) - \frac{1}{6} \sum_{ij} Q_{ij} \partial_i E_j(0)}
    \end{equation}
    or simply
    \begin{equation}
        \int F dx = W
    \end{equation}

\end{tcolorbox}

\section*{3. Nonisotropic Medium}
Suppose we have a nonisotropic but otherwise uniform dielectric medium of infinite extend with dielectric tensor $ \epsilon_{ij} $. Let us put a point charge in it, show that the resulting Coulomb potential can be written as
\begin{equation}
    \phi( \vec{x} ) = \frac{q}{4 \pi \sqrt{\det(\epsilon)(\epsilon_{ij}^{-1}x_ix_j)}}.
\end{equation}
Hint: Go to a coordinate system in which $ \epsilon_{ij} $ becomes diagonal, since this tensor is symmetric and positive definite, there are three positive eigenvalues $ \epsilon^1 $, $ \epsilon^2 $, $ \epsilon^3 $ and the associated mutually orthogonal directions.
\begin{tcolorbox}[breakable]
    We can diagonalize $ \epsilon_{ij} $ by writing it as
    \begin{equation}
        \epsilon_{ij} = (P^T)_{ik}\Lambda_{ke}P_{ej} = P_{ki}\Lambda_{ke}P_{ej}
    \end{equation}
    where $ \Lambda $ is a diagonal matrix of the eigenvalues of $ \epsilon $. Our charge density is at a point, so
    \begin{equation}
        \partial_i \epsilon_{ij} \partial_j \Phi = -\delta(\vec{x})q
    \end{equation}
    We will transform
    \begin{equation}
        \partial_i = (P^T)_{im}\partial'_m
    \end{equation}
    such that
    \begin{equation}
        \epsilon_{ij}\partial_i\partial_j = \Lambda_{ij}\partial_i'\partial_j' = \sum_i\epsilon_i\partial_i\partial_i
    \end{equation}
    We now make a substitution $ y = \frac{1}{\sqrt{\epsilon_i}} x_i' $ such that
    \begin{equation}
        \pdv{y_i} \pdv{y_i}\Phi(y) = \delta(\vec{x})q
    \end{equation}
    We can equivalently scale the $ x $ on the left-hand side:
    \begin{equation}
        \delta(\vec{x}) = \abs{\epsilon} \delta(y)
    \end{equation}
    Now we can solve it as if it were a regular point charge in free space and then transform back by the scaling factor and then by the rotation. I'm not quite sure how to do this, but the scaling factor would give the determinant, since $ \dd{y} = \frac{\dd{x}}{\det(\epsilon)} $ since $ \det(\epsilon) = \epsilon_1\epsilon_2\epsilon_3 $.
\end{tcolorbox}

\section*{4. Solve Jackson Problem 4.7 part (a) only.}
\begin{equation}
    \rho(r) = \frac{1}{64 \pi} r^2 e^{-r} \sin^2 \theta 
\end{equation}
Make a multipole expansion of the potential due to this charge density and determine all the nonvanishing multipole moments. Write down the potential at large distances as a finite expansion in Legendre polynomials.
\begin{tcolorbox}[breakable]
    We first need to calculate the multipole moments:
    \begin{equation}
        q_{lm} = \int Y^*_{lm}( \theta', \varphi') r'^l \rho(x') \dd[3]{x'}
    \end{equation}
    By inserting the charge density into this equation, we find
    \begin{equation}
        q_{lm} = \frac{1}{64 \pi} \int Y^*_{lm}( \theta', \varphi' ) r'^{l+4} e^{-r} \sin^3\theta' \dd{r'} \dd{\theta'} \dd{\varphi'}
    \end{equation}
    We know from the charge distribution that the system is azimuthally symmetric, so only the $ m = 0 $ terms survive. Next, we can complete the $\varphi$ integral and reduce the spherical harmonics to Legendre polynomials:
    \begin{equation}
        q_{l0} = \frac{1}{32} \sqrt{\frac{2l+1}{4 \pi}} \int_0^{2 \pi} \dd{\theta'} P_l(\cos\theta')\sin^3\theta' \int0^{\infty} r'^{l+4} e^{-r'} \dd{r'}
    \end{equation}
    Next, I will focus on the first integral over $ \theta' $. If we make the substitution $ x = \cos(\theta') $, $ \dd{x} = - \sin(\theta') \dd{\theta'} $, and $ \sin[2](\theta') = 1 - x^2 $, we can reduce it to the form
    \begin{equation}
        \int_{-1}^{1} P_l(x)(1-x^2) \dd{x}
    \end{equation}
    We can then rewrite $ 1 - x^2 $ in terms of two of the first few Legendre polynomials:
    \begin{equation}
        1 - x^2 = \frac{2}{3} (P_0(x) - P_2(x))
    \end{equation}
    Now we have
    \begin{equation}
        \int_{-1}^{1} \dd{x} P_l(x)\frac{2}{3} (P_0(x) - P_2(x))
    \end{equation}
    By orthogonality of the Legendre polynomials, only the $ l = 0 $ and $ l = 2 $ terms are nonzero. Additionally,
    \begin{equation}
        \int_{-1}^{1} P_l(x)^2\dd{x} = \frac{2}{2l + 1}
    \end{equation}
    so
    \begin{equation}
        q_{00} = \frac{2}{3} \frac{1}{32} \sqrt{\frac{1}{\pi}} \int_0^{\infty} \dd{r'} r'^{4} e^{-r'} = \sqrt{\frac{1}{4 \pi}}
    \end{equation}
    and
    \begin{equation}
        q_{20} = - \frac{2}{3} \frac{1}{32} \sqrt{\frac{1}{5 \pi}} \int_0^{\infty} \dd{r'} r'^6 e^{-r'} = -6 \sqrt{\frac{5}{4 \pi}}
    \end{equation}

    The multipole expansion can be given as
    \begin{equation}
        \Phi(x) = \frac{1}{4 \pi \epsilon_0} \sum_{l=0}^{infty} \sum_{m=-l}^{l} \frac{4 \pi}{2 l + 1} q_{lm} \frac{Y_{lm}(\theta, \varphi)}{r^{l+1}}
    \end{equation}
    Now, all but two of these terms are zero, and what remains is
    \begin{equation}
        \frac{1}{4 \pi \epsilon_0} \left( 4 \pi \sqrt{\frac{1}{4 \pi}} \frac{Y_{00}}{r} - \frac{4 \pi}{5} 6 \sqrt{\frac{5}{4 \pi}} \frac{Y_{20}}{r^3}\right)
    \end{equation}
    Using the closed form of these two spherical harmonics, the problem can be reduced to
    \begin{equation}
        \Phi(x) = \frac{1}{4 \pi \epsilon_0} \left[ \frac{1}{r} - \frac{3}{r^3} \left( 3\cos^2\theta - 1 \right) \right]
    \end{equation}
\end{tcolorbox}
\end{document}
