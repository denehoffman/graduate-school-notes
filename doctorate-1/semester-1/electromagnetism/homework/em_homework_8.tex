\documentclass[a4paper,twoside]{article}
% My LaTeX preamble file - by Nathaniel Dene Hoffman
% Credit for much of this goes to Olivier Pieters (https://olivierpieters.be/tags/latex)
% and Gilles Castel (https://castel.dev)
% There are still some things to be done:
% 1. Update math commands using mathtools package (remove ddfrac command and just override)
% 2. Maybe abbreviate \imath somehow?
% 3. Possibly format for margin notes and set new margin sizes
% First, some encoding packages and useful formatting
%--------------------------------------------------------------------------------------------
\usepackage{import}
\usepackage{pdfpages}
\usepackage{transparent}
\usepackage[l2tabu,orthodox]{nag}   % force newer (and safer) LaTeX commands
\usepackage[utf8]{inputenc}         % set character set to support some UTF-8
                                    %   (unicode). Do NOT use this with
                                    %   XeTeX/LuaTeX!
\usepackage[T1]{fontenc}
\usepackage[english]{babel}         % multi-language support
\usepackage{sectsty}                % allow redefinition of section command formatting
\usepackage{tabularx}               % more table options
\usepackage{booktabs}
\usepackage{titling}                % allow redefinition of title formatting
\usepackage{imakeidx}               % create and index of words
\usepackage{xcolor}                 % more colour options
\usepackage{enumitem}               % more list formatting options
\usepackage{tocloft}                % redefine table of contents, new list like objects
\usepackage{subfiles}               % allow for multifile documents

% Next, let's deal with the whitespaces and margins
%--------------------------------------------------------------------------------------------
\usepackage[centering,margin=1in]{geometry}
\setlength{\parindent}{0cm}
\setlength{\parskip}{2ex plus 0.5ex minus 0.2ex} % whitespace between paragraphs

% Redefine \maketitle command with nicer formatting
%--------------------------------------------------------------------------------------------
\pretitle{
  \begin{flushright}         % align text to right
    \fontsize{40}{60}        % set font size and whitespace
    \usefont{OT1}{phv}{b}{n} % change the font to bold (b), normally shaped (n)
                             %   Helvetica (phv)
    \selectfont              % force LaTeX to search for metric in its mapping
                             %   corresponding to the above font size definition
}
\posttitle{
  \par                       % end paragraph
  \end{flushright}           % end right align
  \vskip 0.5em               % add vertical spacing of 0.5em
}
\preauthor{
  \begin{flushright}
    \large                   % font size
    \lineskip 0.5em          % inter line spacing
    \usefont{OT1}{phv}{m}{n}
}
\postauthor{
  \par
  \end{flushright}
}
\predate{
  \begin{flushright}
  \large
  \lineskip 0.5em
  \usefont{OT1}{phv}{m}{n}
}
\postdate{
  \par
  \end{flushright}
}

% Mathematics Packages
\usepackage[Gray,squaren,thinqspace,cdot]{SIunits}      % elegant units
\usepackage{amsmath}                                    % extensive math options
\usepackage{amsfonts}                                   % special math fonts
\usepackage{mathtools}                                  % useful formatting commands
\usepackage{amsthm}                                     % useful commands for building theorem environments
\usepackage{amssymb}                                    % lots of special math symbols
\usepackage{mathrsfs}                                   % fancy scripts letters
\usepackage{cancel}                                     % cancel lines in math
\usepackage{esint}                                      % fancy integral symbols
\usepackage{relsize}                                    % make math things bigger or smaller
%\usepackage{bm}                                         % bold math!
\usepackage{slashed}

\newcommand\ddfrac[2]{\frac{\displaystyle #1}{\displaystyle #2}}    % elegant fraction formatting
\allowdisplaybreaks[1]                                              % allow align environments to break on pages

% Ensure numbering is section-specific
%--------------------------------------------------------------------------------------------
\numberwithin{equation}{section}
\numberwithin{figure}{section}
\numberwithin{table}{section}

% Citations, references, and annotations
%--------------------------------------------------------------------------------------------
\usepackage[small,bf,hang]{caption}        % captions
\usepackage{subcaption}                    % adds subfigure & subcaption
\usepackage{sidecap}                       % adds side captions
\usepackage{hyperref}                      % add hyperlinks to references
\usepackage[noabbrev,nameinlink]{cleveref} % better references than default \ref
\usepackage{autonum}                       % only number referenced equations
\usepackage{url}                           % urls
\usepackage{cite}                          % well formed numeric citations
% format hyperlinks
\colorlet{linkcolour}{black}
\colorlet{urlcolour}{blue}
\hypersetup{colorlinks=true,
            linkcolor=linkcolour,
            citecolor=linkcolour,
            urlcolor=urlcolour}

% Plotting and Figures
%--------------------------------------------------------------------------------------------
\usepackage{tikz}          % advanced vector graphics
\usepackage{pgfplots}      % data plotting
\usepackage{pgfplotstable} % table plotting
\usepackage{placeins}      % display floats in correct sections
\usepackage{graphicx}      % include external graphics
\usepackage{longtable}     % process long tables

% use most recent version of pgfplots
\pgfplotsset{compat=newest}

% Misc.
%--------------------------------------------------------------------------------------------
\usepackage{todonotes}  % add to do notes
\usepackage{epstopdf}   % process eps-images
\usepackage{float}      % floats
\usepackage{stmaryrd}   % some more nice symbols
\usepackage{emptypage}  % suppress page numbers on empty pages
\usepackage{multicol}   % use this for creating pages with multiple columns
\usepackage{etoolbox}   % adds tags for environment endings
\usepackage{tcolorbox}  % pretty colored boxes!


% Custom Commands
%--------------------------------------------------------------------------------------------
\newcommand\hr{\noindent\rule[0.5ex]{\linewidth}{0.5pt}}                % horizontal line
\newcommand\N{\ensuremath{\mathbb{N}}}                                  % blackboard set characters
\newcommand\R{\ensuremath{\mathbb{R}}}
\newcommand\Z{\ensuremath{\mathbb{Z}}}
\newcommand\Q{\ensuremath{\mathbb{Q}}}
%\newcommand\C{\ensuremath{\mathbb{C}}}
\renewcommand{\arraystretch}{1.2}                                       % More space between table rows (could be 1.3)
\newcommand{\Cov}{\mathrm{Cov}}
\newcommand\D{\mathrm{D}}
\newcommand*{\dbar}{\ensuremath{\text{\dj}}}

\newcommand{\incfig}[2][1]{%
    \def\svgwidth{#1\columnwidth}
    \import{./figures/}{#2.pdf_tex}
}

% Custom Environments
%--------------------------------------------------------------------------------------------
\newcommand{\lecture}[3]{\hr\\{\centering{\large\textsc{Lecture #1: #3}}\\#2\\}\hr\markboth{Lecture #1: #3}{\rightmark}}   % command to title lectures
\usepackage{mdframed}
\theoremstyle{plain}
\newmdtheoremenv[nobreak]{theorem}{Theorem}[section]
\newtheorem{corollary}{Corollary}[theorem]
\newtheorem{lemma}[theorem]{Lemma}
\theoremstyle{definition}
\newtheorem*{ex}{Example}
\newmdtheoremenv[nobreak]{definition}{Definition}[section]
\theoremstyle{remark}
\newtheorem*{remark}{Remark}
\newtheorem*{claim}{Claim}
\AtEndEnvironment{ex}{\null\hfill$\diamond$}%
% Note: A proof environment is already provided in the amsthm package
\tcbuselibrary{breakable}
\newenvironment{note}[1]{\begin{tcolorbox}[
    arc=0mm,
    colback=white,
    colframe=white!60!black,
    title=#1,
    fonttitle=\sffamily,
    breakable
]}{\end{tcolorbox}}
\newenvironment{problem}{\begin{tcolorbox}[
    arc=0mm,
    breakable,
    colback=white,
    colframe=black
]}{\end{tcolorbox}}

% Header and Footer
%--------------------------------------------------------------------------------------------
% set header and footer
\usepackage{fancyhdr}                       % header and footer
\pagestyle{fancy}                           % use package
\fancyhf{}
\fancyhead[LE,RO]{\textsl{\rightmark}}      % E for even (left pages), O for odd (right pages)
\fancyfoot[LE,RO]{\thepage}
\fancyfoot[LO,RE]{\textsl{\leftmark}}
\setlength{\headheight}{15pt}


% Physics
%--------------------------------------------------------------------------------------------
\usepackage[arrowdel]{physics}      % all the usual useful physics commands
\usepackage{feyn}                   % for drawing Feynman diagrams
%\usepackage{bohr}                   % for drawing Bohr diagrams
%\usepackage{tikz-feynman}
\usepackage{elements}               % for quickly referencing information of various elements
\usepackage{tensor}                 % for writing tensors and chemical symbols

% Finishing touches
%--------------------------------------------------------------------------------------------
\author{Nathaniel D. Hoffman}

\title{33-761 Homework 8}
\date{\today}
\begin{document}
\maketitle
\section{Verify the Following Identity}
\begin{equation}
    \curl{ \vec{L}} = - \imath \vec{x} \laplacian + \imath \grad{\left[ 1 + \vec{x} \cdot \nabla \right]}
\end{equation}
\begin{problem}
    \begin{equation}
        \vec{L} -\imath \vec{x} \times \nabla
    \end{equation}
    so we want to show that
    \begin{equation}
        \curl{ x \times \nabla } = x \laplacian - \grad{(1 + \vec{x} \cdot \nabla)}
    \end{equation}
    The $ i $th component of this is
    \begin{equation}
        \epsilon^{ijk}\partial_j L_k = \epsilon^{ijk}\epsilon^{klm}\partial_j x_l \partial_m
    \end{equation}
    We can expand this using the contracted epsilon identity:
    \begin{align}
        \epsilon^{ijk}\epsilon^{klm}\partial_j x_l \partial_m &= (\delta_{il} \delta_{jm} - \delta_{jl} \delta_{im})\partial_j x_l \partial_m\\
        &= \partial_j x_i \partial_j - \partial_j x_j \partial_i\\
        &= x_i \partial_j^2 + \delta_{ij} \partial_j - \partial_j x_j \partial_i
    \end{align}
    Since $ \curl{ \vec{x}} = 0 $, we can write $ \partial_i x_j = \partial_j x_i $. However, $ x_j\partial_i = 2\delta_{ij} + x_i\partial_j $, so
    \begin{equation}
        x_i \partial_j^2 + \delta_{ij} \partial_j - \partial_j x_j \partial_i = x_i\partial_j^2 + \partial_i - 2\partial_i - \partial_i x_j \partial_j = \vec{x} \laplacian - \grad{[1 + \vec{x} \cdot \nabla]}
    \end{equation}
\end{problem}

\section{Magnetic Dipole Moment}
Show that for a current circuit represented by a (regular non-self-intersecting) curve in space, the magnetic dipole moment is given by
\begin{equation}
    \vec{m} = I \int_{\Sigma} \dd{ \vec{a}}
\end{equation}
where $ I $ is the current in the circuit and $ \Sigma $ is any (regular) surface admitting the circuit as its boundary (in the plane this becomes the familiar $ m = I \cdot A $ formula).
\begin{problem}
    Jackson says that in general, for a current confined to a path,
    \begin{equation}
        \vec{m} = \frac{I}{2} \int_{\Gamma} x \times \dd{ \vec{l}}
    \end{equation}
    In our first homework, we showed that
    \begin{equation}
        \int_{\Gamma} \lambda \dd{ \vec{l}} = -\int_{\Sigma} (\grad{\lambda}) \times \dd{ \vec{a}}
    \end{equation}
    Writing this in index notation, we find that the $ k $th element is:
    \begin{equation}
        \int_\Gamma \lambda \dd{l_k} = -\int_\Sigma \epsilon^{klm}\partial_l\lambda \dd{a_m}
    \end{equation}
    If we write the magnetic moment in index notation, we find that
    \begin{equation}
        m_i = \frac{I}{2} \int_\Gamma \epsilon^{ijk}x_j \dd{l_k}
    \end{equation}
    so if we set $ \lambda = \epsilon^{ijk}x_j $, we find
    \begin{align}
        m_i &= - \frac{I}{2} \int_\Sigma \epsilon^{ijk} \epsilon^{klm} \partial_l x_j \dd{a_m}\\
        &= - \frac{I}{2} \int_\Sigma (\delta_{il} \delta_{jm} - \delta_{\imath} \delta_{jl}) \partial_l x_j \dd{a_m}\\
        &= - \frac{I}{2} \left(\int_\Sigma \partial_i x_j \dd{a_j} - \int_\Sigma \partial_j x_j \dd{a_i} \right)\\
        &= \frac{I}{2} \left( -\int_\Sigma \delta_{ij}\partial_i x_j \dd{a_j} + \int_\Sigma \partial_j x_j \dd{a_i} \right)\\
        \rightarrow \vec{m} &= \frac{I}{2} \int_\Sigma (3-1) \dd{ \vec{a}} = I \int_\Sigma \dd{ \vec{a}}
    \end{align}
\end{problem}

\section{General Force between Current Distributions}
Show that the force acting on a localized current distribution in a region $ \Omega_1 $ due to a localized current distribution in a region $ \Omega_2 $ in the magnetostatic approximation is given by
\begin{equation}
    \vec{F}_{12} = \lim_{\abs{ \vec{a}_1} \to 0} \frac{\mu_0}{4 \pi} \nabla_{ \vec{a}_1} \int_{\Omega_1} \dd[3]{x_1} \int_{\Omega_2} \dd[3]{x_2} \frac{ \vec{J}_1( \vec{x}_1) \cdot \vec{J}_2( \vec{x}_2)}{\abs{ \vec{x}_1 + \vec{a}_1 - \vec{x}_2}}
\end{equation}
\begin{problem}
   We start with
   \begin{equation}
       \vec{F}_{12} = \int \vec{J}^{(1)}( \vec{x}_1) \times \vec{B}^{(2)}( \vec{x_1}) \dd[3]{x_1}
   \end{equation}
   We are imagining the force acting on the currents $ J^{(1)} $ from external currents which make a magnetic field. We can write this field in terms of the curl of its vector potential, which has an associated current density integral:
   \begin{align}
       F^{(12)}_i &= \int_{\Omega_1} \epsilon^{ijk} J^{(1)}_j B^{(2)}_k \dd{x^{(1)}_i}\\
       &= \int_{\Omega_1} \epsilon^{ijk} J^{(1)}_j \epsilon^{klm} \partial_l \frac{\mu_0}{4 \pi} \int_{\Omega} J^{(2)}_m \frac{1}{\abs{ \vec{x}_1 - \vec{x}_2}} \dd{x^{(2)}_m}\\
       &= \frac{\mu_0}{4 \pi} \left( \delta_{il} \delta_{jm} - \delta_{im} \delta_{jl} \right) \int_{\Omega_i}J^{(1)}_j \partial_l \dd{x^{(1)}_j} \int_{\Omega_2} J^{(2)}_m \frac{1}{\abs{ \vec{x}_1 - \vec{x}_2}} \dd{x^{(2)}_m}\\
       &= \frac{\mu_0}{4 \pi} \left[ \int_{\Omega_i} J^{(1)}_j \partial_i \dd{x^{(1)}_i} \int_{\Omega_2} J^{(2)}_j \frac{1}{\abs{ \vec{x}_1 - \vec{x}_2}} \dd{x^{(2)}_j} - \int_{\Omega_i} J^{(1)}_j \partial_j \dd{x^{(1)}_i} \int_{\Omega_2} J^{(2)}_i \frac{1}{\abs{ \vec{x}_1 - \vec{x}_2}} \dd{x^{(2)}_i} \right]
   \end{align}
    Acting the differential over the current in the first term cancels the second term, so we are left with
    \begin{equation}
        \frac{\mu_0}{4 \pi} \int_{\Omega_i} \dd{x^{(1)}_i} \int_{\Omega_2} \dd{x^{(2)}_j} J^{(1)}_j J^{(2)}_j \partial_i \frac{1}{\abs{ \vec{x}_1 - \vec{x}_2}}
    \end{equation}
    Because
    \begin{equation}
        \dv{f}{x}= \lim_{a \to 0} \dv{a}f(x+a)
    \end{equation}
    we can write this as
    \begin{equation}
        F_{12} = \lim_{\abs{ \vec{a}_1} \to 0} \frac{\mu_0}{4 \pi} \nabla_{ \vec{a}_1} \int_{\Omega_1} \dd[3]{x_1} \int_{\Omega_2} \dd[3]{x_2} \frac{ \vec{J}_1( \vec{x}_1) \cdot \vec{J}_2( \vec{x}_2)}{\abs{ \vec{x}_1 + \vec{a}_1 - \vec{x}_2}}
    \end{equation}
\end{problem}

\section{Complete the discussion presented in Section 5.12}
\begin{problem}
    While I'm not entirely sure what there is to complete about the discussion presented in Jackson, I guess I will just fill in the intermediate calculations. If we suppose
    \begin{equation}
        \Phi_M =
        \begin{cases}
            -H_0 r \cos(\theta) + \sum_{l=0}^{\infty} \frac{\alpha_l}{r^{l+1}} P_l(\cos(\theta)) & b < r\\
            \sum_{l=0}^{\infty} \left( \beta_l r^l \gamma_l \frac{1}{r^{l+1}} \right) P_l(\cos(\theta)) & a < r < b\\
            \sum_{l=0}^{\infty} \delta_l r^l P_l(\cos(\theta)) & r < a
        \end{cases}
    \end{equation}
    we can now look at the boundary conditions. Jackson outlines them as the $ H $ field in the $ \theta $ direction (tangent) is continuous across the boundary, and the $ B $ field in the radial direction (perpendicular) must also be continuous. By orthogonality of the $ P_l(\cos(\theta)) $ we only have the $ l=1 $ terms because the field outside has some proportionality to $ \cos(\theta) $ which is $ P_1(\cos(\theta)) $. If we look at the boundary at $ r = a $, we have
    \begin{align}
        \partial_\theta \Phi(a_+) &= \partial_\theta \Phi(a_-)\\
        - \delta_1 a &= -\beta_1 a - \frac{\gamma_1}{a^2}\\
        -\delta_1 a^3 + \beta_1 a^3 + \gamma_1 = 0
    \end{align}
    Using the $ B $-field condition,
    \begin{align}
        \mu_0\partial_r\Phi(b_+) &= \mu\partial_r\Phi(b_-)\\
        \mu_0 (-H_0 + -2 \frac{\alpha_1}{b^3}) &= \mu(\beta_1 - 2 \frac{\gamma_1}{b^3})\\
        2 \alpha_1 + \mu' b^3 \beta_1 - 2 \mu' \gamma &= -b^3 H_0
    \end{align}
    Similarly, at the boundary $ r = b $, we have 
    \begin{align}
        \partial_\theta \Phi(b_+) &= \partial_\theta \Phi(b_-)\\
        H_0 b \sin(\theta) - \frac{\alpha_1}{b^2} \sin(\theta) &= \left( - \frac{\gamma_1}{b^2} - \beta_1 b \right) \sin(\theta)\\
        H_0 b^3 - \alpha_1 + \gamma_1 + \beta_1 b^3 &= 0
        \end{align}
    and finally,
    \begin{equation}
        2 \alpha_1 + \mu' b^3 \beta_1 - 2 \mu' \gamma_1 + b^3 H_0 = 0
    \end{equation}
    comes from continuity of $ B $ at $ r = b $.
    
    Jackson solves for $ \alpha_1 $ and $ \delta_1 $, and while I'm pretty good at Mathematica, I could not quite get it in the same form, so I'll just assume he's correct and take the limits as $ \mu >> \mu_0 $ or $ \mu' \to \infty $.
    \begin{equation}
        \alpha_1 = \left[ \frac{(2 \mu' + 1)(\mu' - 1)}{(2 \mu' + 1)(\mu' +2)- 2 \frac{a^3}{b^3} (\mu' - 1)^2} \right](b^3 - a^3)H_0
    \end{equation}
    \begin{equation}
        \delta_1 = 0 \left[ \frac{9 \mu'}{(2 \mu' + 1)(\mu' +2)- 2 \frac{a^3}{b^3} (\mu' - 1)^2} \right] H_0
    \end{equation}
    We can see that the equation for $ \alpha_1 $ is of leading order $ 2 $ in $ \mu' $ in both the numerator and denominator, so the limit as $ \mu' \to \infty $ will be proportional to the leading order terms:
    \begin{equation}
        \lim_{\mu' \to \infty} \alpha_1 = \frac{2 \mu'^2}{2 \mu'^2 - 2 \frac{a^3}{b^3} \mu'^2 } (b^3 - a^3)H_0 = b^3 H_0
    \end{equation}
    For $ \delta_1 $, the numerator is only of order $ 1 $, so the resulting limit will yield
    \begin{equation}
        \delta_1 - \frac{9 H_0}{2 \mu'^2 - 2 \frac{a^3}{b^3} \mu'^2} = \frac{0 \mu_0}{2 \mu \left( 1 - \frac{a^3}{b^3} \right)} H_0
    \end{equation}
    Because the inner field is inversely proportional to $ \mu $, having a large $ \mu $ means the inner field is small. The field inside is proportional to $ \grad{\sum_{l=0}^{\infty} \delta_l r^l P_l(\cos(\theta))} $.
\end{problem}

\section{Jackson 5.20 (a)}
Starting from the force equation (5.12) and the fact that a magnetization $ \vec{M} $ inside a volume $ V $ bounded by a surface $ S $ is equivalent to a volume current density $ \vec{J}_M = \curl{ \vec{M}} $ and a surface current density $ \vec{M} \times \hat{n} $, show that in the absence of macroscopic conduction currents the total magnetic force on the body can be written
\begin{equation}
    \vec{F} = - \int_{V} ( \div{ \vec{M}}) \vec{B}_e \dd[3]{x} + \int_S ( \vec{M} \cdot \hat{n}) \vec{B}_e \dd{a}
\end{equation}
where $ \vec{B}_e $ is the applied magnetic induction (not including that of the body in question). The force is now expressed in terms of the effective charge densities $ \rho_{M} $ and $ \sigma_{m} $. If the distribution of magnetization is not discontinuous, the surface can be at infinity and the force given by just the volume integral.
\begin{problem}
    We begin with equation (5.12) which states
    \begin{equation}
        \vec{F} = \int_V \vec{J} \times \vec{B} \dd[3]{x}
    \end{equation}
    In our case, we split up the current into a surface term and a volume term:
    \begin{equation}
        \vec{F} = \int_V \vec{J}_M \times \vec{B}_e \dd[3]{x} + \int_S \vec{K}_M \times \vec{B}_e \dd{a}
    \end{equation}
    We know what these are in terms of the magnetization:
    \begin{equation}
        \vec{F} = \int_V (\curl{ \vec{M}}) \times \vec{B}_e \dd[3]{x} + \int_S ( \vec{M} \times \hat{n}) \times \vec{B}_e \dd{a}
    \end{equation}
    In the volume integral, we apply the identity
    \begin{equation}
        (\curl{ \vec{M}}) \times \vec{B}_e = - \vec{B}_e \times (\curl{ \vec{M}}) = ( \vec{M} \cdot \nabla) \vec{B}_e + ( \vec{B}_e \cdot \nabla) \vec{M} + \vec{M} (\underbrace{ \curl{\vec{B}_e}}_{J_\text{cond} =0}) - \grad{(\vec{M} \cdot \vec{B}_e)}
    \end{equation}
    By divergence theorem, the final term $ \int_V \grad{( \vec{M} \cdot \vec{B}_e)} \dd[3]{x} = \int_S ( \vec{M} \cdot \vec{B}_e) \hat{n} \dd{a} $, so we now have
    \begin{equation}
        \vec{F} = \int_V ( \vec{M} \cdot \nabla) \vec{B}_e + ( \vec{B}_e \cdot \nabla) \vec{M} \dd[3]{x} + \int_S ( \vec{M} \times \hat{n}) \times \vec{B}_e - ( \vec{M} \cdot \vec{B}_e) \hat{n} \dd{a}
    \end{equation}
    Next, we use the identity
    \begin{equation}
        ( \vec{M} \times \hat{n}) \times \vec{B}_e = - \vec{B}_e \times ( \vec{M} \times \hat{n}) = ( \vec{B}_e \cdot \hat{n}) \vec{M} - ( \vec{B}_e \cdot \vec{M}) \hat{n}
    \end{equation}
    so
    \begin{equation}
        \vec{F} = \int_V ( \vec{M} \cdot \nabla) \vec{B}_e + ( \vec{B}_e \cdot \nabla) \vec{M} \dd[3]{x} - \int_S ( \vec{B}_e \cdot \hat{n}) \vec{M} \dd{a} 
    \end{equation}
    Next, we use the identity
    \begin{equation}
        \int_V ( \vec{B}_e \cdot \nabla) \vec{M} \dd[3]{x} = - \int_V (\div{ \vec{B}_e}) \vec{M} + \int_S ( \hat{n} \cdot \vec{B}_e) \vec{M}\dd{a}
    \end{equation}
    We now have
    \begin{equation}
        \vec{F} = \int_V ( \vec{M} \cdot \nabla) \vec{B}_e - (\underbrace{\div{ \vec{B}_e}}_{0}) \vec{M} \dd[3]{x}
    \end{equation}
    Applying the identity once more, we have
    \begin{equation}
        \vec{F} = \int_V ( \vec{M} \cdot \nabla) \vec{B}_e = -\int_V (\div{ \vec{M}}) \vec{B}_e \dd[3]{x} + \int_S ( \vec{M} \cdot \hat{n}) \vec{B}\dd{a} 
    \end{equation}

\end{problem}
\end{document}
