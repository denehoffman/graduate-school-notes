\documentclass[a4paper,twoside]{article}
% My LaTeX preamble file - by Nathaniel Dene Hoffman
% Credit for much of this goes to Olivier Pieters (https://olivierpieters.be/tags/latex)
% and Gilles Castel (https://castel.dev)
% There are still some things to be done:
% 1. Update math commands using mathtools package (remove ddfrac command and just override)
% 2. Maybe abbreviate \imath somehow?
% 3. Possibly format for margin notes and set new margin sizes
% First, some encoding packages and usefull formatting
%--------------------------------------------------------------------------------------------
\usepackage[l2tabu,orthodox]{nag}   % force newer (and safer) LaTeX commands
\usepackage[utf8]{inputenc}         % set character set to support some UTF-8
                                    %   (unicode). Do NOT use this with
                                    %   XeTeX/LuaTeX!
\usepackage[T1]{fontenc}
\usepackage[english]{babel}         % multi-language support
\usepackage{sectsty}                % allow redefinition of section command formatting
\usepackage{tabularx}               % more table options
\usepackage{booktabs}
\usepackage{titling}                % allow redefinition of title formatting
\usepackage{imakeidx}               % create and index of words
\usepackage{xcolor}                 % more colour options
\usepackage{enumitem}               % more list formatting options
\usepackage{tocloft}                % redefine table of contents, new list like objects
\usepackage{subfiles}               % allow for multifile documents

% Next, let's deal with the whitespaces and margins
%--------------------------------------------------------------------------------------------
\usepackage[centering,margin=1in]{geometry}
\setlength{\parindent}{0cm}
\setlength{\parskip}{2ex plus 0.5ex minus 0.2ex} % whitespace between paragraphs

% Redefine \maketitle command with nicer formatting
%--------------------------------------------------------------------------------------------
\pretitle{
  \begin{flushright}         % align text to right
    \fontsize{40}{60}        % set font size and whitespace
    \usefont{OT1}{phv}{b}{n} % change the font to bold (b), normally shaped (n)
                             %   Helvetica (phv)
    \selectfont              % force LaTeX to search for metric in its mapping
                             %   corresponding to the above font size definition
}
\posttitle{
  \par                       % end paragraph
  \end{flushright}           % end right align
  \vskip 0.5em               % add vertical spacing of 0.5em
}
\preauthor{
  \begin{flushright}
    \large                   % font size
    \lineskip 0.5em          % inter line spacing
    \usefont{OT1}{phv}{m}{n}
}
\postauthor{
  \par
  \end{flushright}
}
\predate{
  \begin{flushright}
  \large
  \lineskip 0.5em
  \usefont{OT1}{phv}{m}{n}
}
\postdate{
  \par
  \end{flushright}
}

% Mathematics Packages
\usepackage[Gray,squaren,thinqspace,cdot]{SIunits}      % elegant units
\usepackage{amsmath}                                    % extensive math options
\usepackage{amsfonts}                                   % special math fonts
\usepackage{mathtools}                                  % useful formatting commands
\usepackage{amsthm}                                     % useful commands for building theorem environments
\usepackage{amssymb}                                    % lots of special math symbols
\usepackage{mathrsfs}                                   % fancy scripts letters
\usepackage{cancel}                                     % cancel lines in math
\usepackage{esint}                                      % fancy integral symbols
\usepackage{relsize}                                    % make math things bigger or smaller
\usepackage{bm}                                         % bold math!

\newcommand\ddfrac[2]{\frac{\displaystyle #1}{\displaystyle #2}}    % elegant fraction formatting
\allowdisplaybreaks[1]                                              % allow align environments to break on pages

% Ensure numbering is section-specific
%--------------------------------------------------------------------------------------------
\numberwithin{equation}{section}
\numberwithin{figure}{section}
\numberwithin{table}{section}

% Citations, references, and annotations
%--------------------------------------------------------------------------------------------
\usepackage[small,bf,hang]{caption}        % captions
\usepackage{subcaption}                    % adds subfigure & subcaption
\usepackage{sidecap}                       % adds side captions
\usepackage{hyperref}                      % add hyperlinks to references
\usepackage[noabbrev,nameinlink]{cleveref} % better references than default \ref
\usepackage{autonum}                       % only number referenced equations
\usepackage{url}                           % urls
\usepackage{cite}                          % well formed numeric citations
% format hyperlinks
\colorlet{linkcolour}{black}
\colorlet{urlcolour}{blue}
\hypersetup{colorlinks=true,
            linkcolor=linkcolour,
            citecolor=linkcolour,
            urlcolor=urlcolour}

% Plotting and Figures
%--------------------------------------------------------------------------------------------
\usepackage{tikz}          % advanced vector graphics
\usepackage{pgfplots}      % data plotting
\usepackage{pgfplotstable} % table plotting
\usepackage{placeins}      % display floats in correct sections
\usepackage{graphicx}      % include external graphics
\usepackage{longtable}     % process long tables

% use most recent version of pgfplots
\pgfplotsset{compat=newest}

% Misc.
%--------------------------------------------------------------------------------------------
\usepackage{todonotes}  % add to do notes
\usepackage{epstopdf}   % process eps-images
\usepackage{float}      % floats
\usepackage{stmaryrd}   % some more nice symbols
\usepackage{emptypage}  % suppress page numbers on empty pages
\usepackage{multicol}   % use this for creating pages with multiple columns
\usepackage{etoolbox}   % adds tags for environment endings
\usepackage{tcolorbox}  % pretty colored boxes!


% Custom Commands
%--------------------------------------------------------------------------------------------
\newcommand\hr{\noindent\rule[0.5ex]{\linewidth}{0.5pt}}                % horizontal line
\newcommand\N{\ensuremath{\mathbb{N}}}                                  % blackboard set characters
\newcommand\R{\ensuremath{\mathbb{R}}}
\newcommand\Z{\ensuremath{\mathbb{Z}}}
\newcommand\Q{\ensuremath{\mathbb{Q}}}
\newcommand\C{\ensuremath{\mathbb{C}}}
\renewcommand{\arraystretch}{1.2}                                       % More space between table rows (could be 1.3)
\newcommand{\Cov}{\mathrm{Cov}}
\newcommand*{\dbar}{\ensuremath{\text{\dj}}}
% Custom Environments
%--------------------------------------------------------------------------------------------
\newcommand{\lecture}[3]{\hr\\{\centering{\large\textsc{Lecture #1: #3}}\\#2\\}\hr\markboth{Lecture #1: #3}{\rightmark}}   % command to title lectures
\usepackage{mdframed}
\theoremstyle{plain}
\newmdtheoremenv[nobreak]{theorem}{Theorem}[section]
\newtheorem{corollary}{Corollary}[theorem]
\newtheorem{lemma}[theorem]{Lemma}
\theoremstyle{definition}
\newtheorem*{ex}{Example}
\newmdtheoremenv[nobreak]{definition}{Definition}[section]
\theoremstyle{remark}
\newtheorem*{remark}{Remark}
\AtEndEnvironment{ex}{\null\hfill$\diamond$}%
% Note: A proof environment is already provided in the amsthm package
\tcbuselibrary{breakable}
\newenvironment{note}[1]{\begin{tcolorbox}[
    arc=0mm,
    colback=white,
    colframe=white!60!black,
    title=#1,
    fonttitle=\sffamily,
    breakable
]}{\end{tcolorbox}}
\newenvironment{problem}{\begin{tcolorbox}[
    arc=0mm,
    breakable,
    colback=white,
    colframe=black
]}{\end{tcolorbox}}

% Header and Footer
%--------------------------------------------------------------------------------------------
% set header and footer
\usepackage{fancyhdr}                       % header and footer
\pagestyle{fancy}                           % use package
\fancyhf{}
\fancyhead[LE,RO]{\textsl{\rightmark}}      % E for even (left pages), O for odd (right pages)
\fancyfoot[LE,RO]{\thepage}
\fancyfoot[LO,RE]{\textsl{\leftmark}}
\setlength{\headheight}{15pt}


% Physics
%--------------------------------------------------------------------------------------------
\usepackage[arrowdel]{physics}      % all the usual useful physics commands
%\usepackage{feyn}                   % for drawing Feynman diagrams
%\usepackage{bohr}                   % for drawing Bohr diagrams
\usepackage{elements}               % for quickly referencing information of various elements
\usepackage{tensor}                 % for writing tensors and chemical symbols

% Finishing touches
%--------------------------------------------------------------------------------------------
\author{Nathaniel D. Hoffman}

\title{33-761 Homework 9}
\date{\today}
\begin{document}
\maketitle

\section{Coaxial Cable}
\begin{itemize}
    \item[(a)] Consider a coaxial cable with uniform cylindrical cross section. Assume that the inner thin cylinder has radius $ a $ and the outer one has radius $ b $. Current $ I $ goes through one and returns from the other. Calculate the self-inductance $ L $ per unit length of this cable. In a similar way we can calculate the capacitance $ C $ per unit length, then verify the formula $ CL = \epsilon_0 \mu_0 $.
        \begin{problem}
            Using an Ampereian loop, we can find the magnetic field inside the coaxial cable to be $ \vec{B} = \frac{\mu_0 I}{2 \pi r} \hat{\varphi} $. Now let's imagine taking a rectangle which goes from the inside to the outside conductors with length $ \dd{l} $ in the direction of the cable. The magnetic flux through such a rectangle would give us the flux per unit length. Because the flux is given by
            \begin{equation}
                \Phi = \int \vec{B} \cdot \dd{\vec{a}}
            \end{equation}
            and $ a = r \dd{l} $ (which we integrate from $ a $ to $ b $ to get the full flux), the integral becomes
            \begin{equation}
                \Phi = \int_{a}^{b} B(r) \dd{r} = \frac{\mu_0 I}{2 \pi} \ln(\frac{b}{a})
            \end{equation}
            The inductance is defined as $ \frac{\Phi}{I} $ so
            \begin{equation}
                L = \frac{\mu_0}{2 \pi} \ln(\frac{b}{a})
            \end{equation}
            For the capacitance, we first find the electric field between the two conductors, which is just $ \vec{E} = \frac{\lambda}{2 \pi \epsilon_0 r} \hat{r} $ from the usual Gauss's law, assuming the Gaussian surface contains charge $ \lambda $ per unit length. Therefore, the change in voltage is just the integral
            \begin{equation}
                \Delta V = \int_{a}^{b} E(r) \dd{r} = \frac{\lambda}{2 \pi \epsilon_0} \ln(\frac{b}{a})
            \end{equation}
            The capacitance is defined as $ \frac{\lambda}{\Delta V} $ so
            \begin{equation}
                C = \frac{2 \pi \epsilon_0}{\ln(\frac{b}{a})}
            \end{equation}
            so
            \begin{equation}
                CL = \mu_0 \epsilon_0
            \end{equation}
        \end{problem}
    \item[(b)] Assume that the coaxial cable has an arbitrary cross section, show that we can verify the relation $ CL = \epsilon_0 \mu_0 $ in this case as well, even though we cannot compute them explicitly.
        \begin{problem}
            I'll work backwards from the last problem:
            \begin{equation}
                CL = \frac{\lambda}{\Delta V} \frac{\Phi}{I} = \frac{\lambda}{I} \frac{\int \vec{B} \cdot \dd{ \vec{a}_1}}{\int \vec{E} \cdot \dd{ \vec{a}_2}}
            \end{equation}
            Where $ a_1 $ is a surface which is perpendicular to the conductors and $ a_2 $ is a surface which is parallel to the conductors. We can no longer use cylindrical symmetry to pull out factors of $ 2 \pi $, but in general an integral over the $ B $-field should be proportional to the enclosed current times $ \mu_0 $  and the integrating over the same path over the $ E $-field should give the enclosed charge (per unit length) divided by $ \epsilon_0 $:
            \begin{equation}
                CL = \frac{\lambda}{I} \frac{I}{\lambda} \mu_0 \epsilon_0 \frac{\iint \vec{b} \dd{\gamma} \cdot \dd{ \vec{a}_1}}{\iint \vec{e} \dd{\gamma } \cdot \dd{ \vec{a}_2}} = \mu_0 \epsilon_0
            \end{equation}
            Here $ \vec{b} $ and $ \vec{e} $ are functions of position and should depend on the path $ \gamma $ in the same way, making this fraction equal to $ 1 $.
        \end{problem}
\end{itemize}

\section{Jackson 5.21}
Note that the terms of the form $ \int \dd[3]{x} \vec{M} \cdot \vec{M} $ are constant and can be ignored.

A magnetostatic field is due entirely to a localized distribution of permanent magnetization.
\begin{itemize}
    \item[(a)] Show that
        \begin{equation}
            \int \vec{B} \cdot \vec{H} \dd[3]{x} = 0
        \end{equation}
        provided the integral is taken over all space.
        \begin{problem}
            Because of the first condition, we can safely assume there are no free currents, so $ \curl{ \vec{H}} = 0 $. Next, we expand the magnetic field in terms of the vector potential:
            \begin{align}
                \int \vec{B} \cdot \vec{H} \dd[3]{x} &= \int (\curl{ \vec{A}}) \cdot \vec{H} \dd[3]{x} \\
                &= \int \div( \vec{A} \times \vec{H}) \dd[3]{x} + \int A \cdot \underbrace{(\curl{ \vec{H}})}_{= 0} \dd[3]{x} \\
                &= \int_{S(\infty)} ( \vec{A} \times \vec{H}) \dd{ \vec{a}} = 0
            \end{align}
            by divergence theorem. If the first integral was over all space, the surface is a surface at infinity, and since the magnetization is local, $ H $ must vanish at infinity.
        \end{problem}
    \item[(b)] From the potential energy (5.72) of a dipole in an external field, show that for a continuous distribution of permanent magnetization the magnetostatic energy can be written
        \begin{equation}
            W = \frac{\mu_0}{2} \int \vec{H} \cdot \vec{H} \dd[3]{x} = - \frac{\mu_0}{2} \int \vec{M} \cdot \vec{H} \dd[3]{x}
        \end{equation}
        apart from an additive constant, which is independent of the orientation or position of the various constituent magnetized bodies.
        \begin{problem}
            The energy from a single dipole in an external magnetic field is $ W = - \vec{m} \cdot \vec{B} $. If one dipole is brought into the presence of another, we know the dipole moments will interact with the magnetic fields generated by the other dipoles, so for finite dipoles, $ W = - \frac{1}{2} \sum_{i \neq j} \vec{m}_i \cdot \vec{B}_j $, where the $ \frac{1}{2} $ avoids double counting the energy of dipole $ a $ in magnetic field $ b $ and dipole $ b $ in magnetic field $ a $. For a continuous distribution of magnetization, this becomes an integral over infinitesimal magnetic moments:
            \begin{equation}
                W = - \frac{1}{2} \int \vec{B} \cdot \dd{ \vec{m}}
            \end{equation}
            Integrating over these dipoles is the same as integrating the magnetization over space:
            \begin{equation}
                W = - \frac{1}{2} \int \vec{M} \cdot \vec{B} \dd[3]{x}
            \end{equation}
            We can expand the magnetic field as $ \vec{B} = \mu_0 ( \vec{M} + \vec{H}) $:
            \begin{equation}
                W = - \frac{\mu_0}{2} \int \vec{M} \cdot \vec{M} \dd[3]{x} - \frac{\mu_0}{2} \int \vec{M} \cdot \vec{H} \dd[3]{x}
            \end{equation}
            We ignore the first term since it is just an additive constant and does not depend on the distribution:
            \begin{equation}
                W = - \frac{\mu_0}{2} \int \vec{M} \cdot \vec{H} \dd[3]{x}
            \end{equation}
            Expanding $ \vec{M} $ as $ \vec{M} = \frac{1}{\mu_0} \vec{B} - \vec{H} $ we find
            \begin{equation}
                W = - \frac{1}{2} \int \vec{B} \cdot \vec{H} \dd[3]{x} + \frac{\mu_0}{2} \int \vec{H} \cdot \vec{H} \dd[3]{x}
            \end{equation}
            We showed in (a) that the first term is zero so
            \begin{equation}
                W = \frac{\mu_0}{2} \int \vec{H} \cdot \vec{H} \dd[3]{x}
            \end{equation}
            as long as there are no free currents.
        \end{problem}
\end{itemize}

\section{Jackson 5.23 (a) and (b) only}
Two identical circular loops of radius $ a $ are initially located a distance $ R $ apart on a common axis perpendicular to their planes.
\begin{itemize}
    \item[(a)] From the expression $ W_{12} = \int \dd[3]{x} \vec{J}_1 \cdot \vec{A}_2 $ and the result for $ A_{\phi} $ from Problem 5.10b, show that the mutual inductance of the loops is
        \begin{equation}
            M_{12} = \mu_0 \pi a^2 \int_{0}^{\infty} \dd{k} e^{-kR} J_1^2(ka)
        \end{equation}
        \begin{problem}
           We are given the potential of a current loop of radius $ a $ as
           \begin{equation}
               A_{\phi}(\rho, z) = \frac{\mu_0 I a}{2} \int_{0}^{\infty} \dd{k} e^{-k \abs{z}} J_1(ka)J_1(k \rho)
           \end{equation}
           We also know that
           \begin{equation}
               W = \frac{1}{2} \sum_{i=1}^{N} L_i I_i^2 + \sum_{i=1}^{N} \sum_{j>i}^{N} M_{ij} I_i I_j
           \end{equation}
           so
           \begin{equation}
               W_{12} = M_{12} I_1 I_2
           \end{equation}
           or
           \begin{equation}
               M_{12} = \frac{W_{12}}{I_1 I_2}
           \end{equation}
           Using our formula for $ W_{12} $, we know that the current is proportional to $ I $ and is always in the radial direction, so
           \begin{equation}
               W_{12} = \int_{0}^{2 \pi} a \dd{\phi} I_1 A_{\phi}(a, R) = \mu_0 \pi a^2 \int_{0}^{\infty} \dd{k} e^{-kR} J_1^2(ka)
           \end{equation}
        \end{problem}
    \item[(b)] Show that for $ R > 2a $, $ M_{12} $ has the expansion,
        \begin{equation}
            M_{12} = \frac{\mu_0 \pi a}{2} \left[ \left( \frac{a}{R} \right)^3 - 3 \left( \frac{a}{R} \right)^5 + \frac{75}{8} \left( \frac{a}{R} \right)^7 + \cdots \right]
        \end{equation}
        \begin{problem}
            We are basically just assuming $ a $ is small, so we can expand around $ ka $ in the integral. If we expand $ J_1^2(ka) $ as a Taylor series about $ 0 $, we find that the first few terms are
            \begin{equation}
                J_1^2(ka) = \frac{(ka)^2}{4} - \frac{(ka)^4}{16} + \frac{5(ka)^6}{768} - \cdots
            \end{equation}
            so the integral becomes
            \begin{equation}
                M_{12} = \mu_0 \pi a^2 \left[ \int_{0}^{\infty} \dd{k} e^{-kR} \left( \frac{a^2}{4} k^2 - \frac{a^4}{16} k^4 + \frac{5 a^6}{768} k^6 - \cdots  \right) \right]
            \end{equation}
            Integrals from $ 0 $ to $ \infty $ of exponentials multiplied by polynomials are well defined, and evaluating this expression gives
            \begin{equation}
                M_{12} = \mu_0 \pi a^2 \frac{1}{2} \left[\frac{a^2}{R^3} - \frac{3 a^4}{R^5} + \frac{75}{8} \frac{a^6}{R^7} - \cdots  \right]
            \end{equation}
            Distributing an $ a $ gives the desired answer.
        \end{problem}
\end{itemize}

\section{Jackson Section 5.18 Part B}
Go through the details of Section 5.18 part B of Jackson and verify the answer given at equation (5.176). This is a self-study exercise, it is nice to work it out and see how the field gradually decreases in the sample.
\begin{problem}
    We begin by defining the current density to be $ J_y = H_0[\delta(z+a) - \delta(z-a)] $. Suppose $ H_x(z,t) = \int_{0}^{\infty} e^{-pt} \overline{h}(p,z) \dd{p} $. Plugging this into the diffusion equation, $ \laplacian{ \vec{H}} = \mu \sigma \partial_t \vec{H} $ gives us just the $ z $-derivative, so we have
    \begin{equation}
        \int_{0}^{\infty} \partial^2_z e^{-pt} \overline{h}(p,z) \dd{p} + \int_{0}^{\infty} \underbrace{\mu \sigma p}_{k^2} e^{-pt} \overline{h}(p,z) \dd{p} = 0
    \end{equation}
    This means $ \overline{h} $ satisfies the equation
    \begin{equation}
        \partial_z^2 \overline{h} + k^2 \overline{h} = 0
    \end{equation}
    since $ t=0 \implies e^{-pt} = 1 $ and the integrands follow the above equation.

    Symmetry apparently suggests $ \overline{h} \propto \cos(kz) $, so
    \begin{equation}
        H_x(z, t) = \int_{0}^{\infty} e^{- \frac{k^2}{\mu \sigma} t} h(k) \cos(kz) \dd{k} 
    \end{equation}
    By the Laplace transform of the initial current distribution, we have
    \begin{equation}
        \int_{0}^{\infty} h(k) \cos(kz) \dd{k} = H_0(\Theta(z+a) - \Theta(z-a))
    \end{equation}
    We can split the cosine on the left side into two exponentials:
    \begin{equation}
        \int_{0}^{\infty} h(k) \cos(kz) \dd{k} = \int_{0}^{\infty} h(k) \frac{1}{2} e^{-\imath kz} \dd{k} + \int_{0}^{\infty} h(k) \frac{1}{2} e^{\imath kz} \dd{k} = \int_{- \infty}^{\infty} h(k) e^{\imath kz} \dd{k}
    \end{equation}
    This only works assuming $ h(k) $ is even about zero. It probably is, since Jackson says so and also because we defined $ k $ to be $ k^2 = \mu \sigma p $ so changing the sign of $ k $ shouldn't mess with the constant determined by initial conditions.

    Next, let's look at the right side. By definition, the Heaviside functions really look like
    \begin{equation}
        H_0(\Theta(z+a) - \Theta(z-a)) = H_0 \left[ \int_{- \infty}^{z+a} \delta(s) \dd{s} - \int_{- \infty}^{z-a} \delta(s) \dd{s} \right] = H_0 \int_{-a}^{a} \delta(s) \dd{s}
    \end{equation}
    so we can invert the Fourier transform (don't forget the $ 2 \pi $ factor) to get
    \begin{equation}
        h(k) = \frac{2 H_0}{2 \pi} \int_{-a}^{a} e^{-\imath k z} \dd{z} = \frac{2 H_0}{\pi k} \sin(ka)
    \end{equation}
    Plugging this back into the original equation for $ H_x $ we get
    \begin{equation}
        H_x(z,t) = \int_{0}^{\infty} e^{- \frac{k^2}{\mu \sigma} t} \left( \frac{2 H_0}{\pi k} \right) \sin(ka) \cos(ka) \dd{k} = \frac{2 H_0}{\pi} \int_{0}^{\infty} e^{- \nu t \kappa^2} \frac{\sin(\kappa)}{\kappa} \cos(\frac{z}{a} \kappa) \dd{\kappa}
    \end{equation}
    making the substitutions $ \kappa = ka $, $ \dd{\kappa} = a \dd{k} $, and $ \nu = \frac{1}{\mu \sigma a^2} $.

    Next, we are told to evaluate this integral using
    \begin{equation}
        \Phi(\xi) = \frac{2}{\pi} \int_{0}^{\infty} e^{- \frac{x^2}{4 \xi^2}} \frac{\sin(x)}{x} \dd{x}
    \end{equation}
    If I split up the cosine as previously, it gives the nice factor of $ \frac{1}{2} $ that is required to get this problem in the correct form, but unfortunately puts an interesting exponent into the problem. Now I have
    \begin{equation}
        e^{- \nu t \kappa^2 \pm \imath \frac{z}{a} \kappa}
    \end{equation}
    inside the integral but working backwards from the answer, I can't see how this is supposed to be equal to
    \begin{equation}
        e^{- \frac{1}{2} \left( \nu t \left( 1 \pm \frac{z}{a} \right)^{-2} \right) \kappa^2}
    \end{equation}
    It must be true because Jackson deems it so, but unfortunately I can't figure out why.
\end{problem}

\end{document}
