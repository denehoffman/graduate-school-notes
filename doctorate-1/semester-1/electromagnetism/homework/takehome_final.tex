\documentclass[a4paper,twoside]{article}
% My LaTeX preamble file - by Nathaniel Dene Hoffman
% Credit for much of this goes to Olivier Pieters (https://olivierpieters.be/tags/latex)
% and Gilles Castel (https://castel.dev)
% There are still some things to be done:
% 1. Update math commands using mathtools package (remove ddfrac command and just override)
% 2. Maybe abbreviate \imath somehow?
% 3. Possibly format for margin notes and set new margin sizes
% First, some encoding packages and useful formatting
%--------------------------------------------------------------------------------------------
\usepackage{import}
\usepackage{pdfpages}
\usepackage{transparent}
\usepackage[l2tabu,orthodox]{nag}   % force newer (and safer) LaTeX commands
\usepackage[utf8]{inputenc}         % set character set to support some UTF-8
                                    %   (unicode). Do NOT use this with
                                    %   XeTeX/LuaTeX!
\usepackage[T1]{fontenc}
\usepackage[english]{babel}         % multi-language support
\usepackage{sectsty}                % allow redefinition of section command formatting
\usepackage{tabularx}               % more table options
\usepackage{booktabs}
\usepackage{titling}                % allow redefinition of title formatting
\usepackage{imakeidx}               % create and index of words
\usepackage{xcolor}                 % more colour options
\usepackage{enumitem}               % more list formatting options
\usepackage{tocloft}                % redefine table of contents, new list like objects
\usepackage{subfiles}               % allow for multifile documents

% Next, let's deal with the whitespaces and margins
%--------------------------------------------------------------------------------------------
\usepackage[centering,margin=1in]{geometry}
\setlength{\parindent}{0cm}
\setlength{\parskip}{2ex plus 0.5ex minus 0.2ex} % whitespace between paragraphs

% Redefine \maketitle command with nicer formatting
%--------------------------------------------------------------------------------------------
\pretitle{
  \begin{flushright}         % align text to right
    \fontsize{40}{60}        % set font size and whitespace
    \usefont{OT1}{phv}{b}{n} % change the font to bold (b), normally shaped (n)
                             %   Helvetica (phv)
    \selectfont              % force LaTeX to search for metric in its mapping
                             %   corresponding to the above font size definition
}
\posttitle{
  \par                       % end paragraph
  \end{flushright}           % end right align
  \vskip 0.5em               % add vertical spacing of 0.5em
}
\preauthor{
  \begin{flushright}
    \large                   % font size
    \lineskip 0.5em          % inter line spacing
    \usefont{OT1}{phv}{m}{n}
}
\postauthor{
  \par
  \end{flushright}
}
\predate{
  \begin{flushright}
  \large
  \lineskip 0.5em
  \usefont{OT1}{phv}{m}{n}
}
\postdate{
  \par
  \end{flushright}
}

% Mathematics Packages
\usepackage[Gray,squaren,thinqspace,cdot]{SIunits}      % elegant units
\usepackage{amsmath}                                    % extensive math options
\usepackage{amsfonts}                                   % special math fonts
\usepackage{mathtools}                                  % useful formatting commands
\usepackage{amsthm}                                     % useful commands for building theorem environments
\usepackage{amssymb}                                    % lots of special math symbols
\usepackage{mathrsfs}                                   % fancy scripts letters
\usepackage{cancel}                                     % cancel lines in math
\usepackage{esint}                                      % fancy integral symbols
\usepackage{relsize}                                    % make math things bigger or smaller
%\usepackage{bm}                                         % bold math!
\usepackage{slashed}

\newcommand\ddfrac[2]{\frac{\displaystyle #1}{\displaystyle #2}}    % elegant fraction formatting
\allowdisplaybreaks[1]                                              % allow align environments to break on pages

% Ensure numbering is section-specific
%--------------------------------------------------------------------------------------------
\numberwithin{equation}{section}
\numberwithin{figure}{section}
\numberwithin{table}{section}

% Citations, references, and annotations
%--------------------------------------------------------------------------------------------
\usepackage[small,bf,hang]{caption}        % captions
\usepackage{subcaption}                    % adds subfigure & subcaption
\usepackage{sidecap}                       % adds side captions
\usepackage{hyperref}                      % add hyperlinks to references
\usepackage[noabbrev,nameinlink]{cleveref} % better references than default \ref
\usepackage{autonum}                       % only number referenced equations
\usepackage{url}                           % urls
\usepackage{cite}                          % well formed numeric citations
% format hyperlinks
\colorlet{linkcolour}{black}
\colorlet{urlcolour}{blue}
\hypersetup{colorlinks=true,
            linkcolor=linkcolour,
            citecolor=linkcolour,
            urlcolor=urlcolour}

% Plotting and Figures
%--------------------------------------------------------------------------------------------
\usepackage{tikz}          % advanced vector graphics
\usepackage{pgfplots}      % data plotting
\usepackage{pgfplotstable} % table plotting
\usepackage{placeins}      % display floats in correct sections
\usepackage{graphicx}      % include external graphics
\usepackage{longtable}     % process long tables

% use most recent version of pgfplots
\pgfplotsset{compat=newest}

% Misc.
%--------------------------------------------------------------------------------------------
\usepackage{todonotes}  % add to do notes
\usepackage{epstopdf}   % process eps-images
\usepackage{float}      % floats
\usepackage{stmaryrd}   % some more nice symbols
\usepackage{emptypage}  % suppress page numbers on empty pages
\usepackage{multicol}   % use this for creating pages with multiple columns
\usepackage{etoolbox}   % adds tags for environment endings
\usepackage{tcolorbox}  % pretty colored boxes!


% Custom Commands
%--------------------------------------------------------------------------------------------
\newcommand\hr{\noindent\rule[0.5ex]{\linewidth}{0.5pt}}                % horizontal line
\newcommand\N{\ensuremath{\mathbb{N}}}                                  % blackboard set characters
\newcommand\R{\ensuremath{\mathbb{R}}}
\newcommand\Z{\ensuremath{\mathbb{Z}}}
\newcommand\Q{\ensuremath{\mathbb{Q}}}
%\newcommand\C{\ensuremath{\mathbb{C}}}
\renewcommand{\arraystretch}{1.2}                                       % More space between table rows (could be 1.3)
\newcommand{\Cov}{\mathrm{Cov}}
\newcommand\D{\mathrm{D}}
\newcommand*{\dbar}{\ensuremath{\text{\dj}}}

\newcommand{\incfig}[2][1]{%
    \def\svgwidth{#1\columnwidth}
    \import{./figures/}{#2.pdf_tex}
}

% Custom Environments
%--------------------------------------------------------------------------------------------
\newcommand{\lecture}[3]{\hr\\{\centering{\large\textsc{Lecture #1: #3}}\\#2\\}\hr\markboth{Lecture #1: #3}{\rightmark}}   % command to title lectures
\usepackage{mdframed}
\theoremstyle{plain}
\newmdtheoremenv[nobreak]{theorem}{Theorem}[section]
\newtheorem{corollary}{Corollary}[theorem]
\newtheorem{lemma}[theorem]{Lemma}
\theoremstyle{definition}
\newtheorem*{ex}{Example}
\newmdtheoremenv[nobreak]{definition}{Definition}[section]
\theoremstyle{remark}
\newtheorem*{remark}{Remark}
\newtheorem*{claim}{Claim}
\AtEndEnvironment{ex}{\null\hfill$\diamond$}%
% Note: A proof environment is already provided in the amsthm package
\tcbuselibrary{breakable}
\newenvironment{note}[1]{\begin{tcolorbox}[
    arc=0mm,
    colback=white,
    colframe=white!60!black,
    title=#1,
    fonttitle=\sffamily,
    breakable
]}{\end{tcolorbox}}
\newenvironment{problem}{\begin{tcolorbox}[
    arc=0mm,
    breakable,
    colback=white,
    colframe=black
]}{\end{tcolorbox}}

% Header and Footer
%--------------------------------------------------------------------------------------------
% set header and footer
\usepackage{fancyhdr}                       % header and footer
\pagestyle{fancy}                           % use package
\fancyhf{}
\fancyhead[LE,RO]{\textsl{\rightmark}}      % E for even (left pages), O for odd (right pages)
\fancyfoot[LE,RO]{\thepage}
\fancyfoot[LO,RE]{\textsl{\leftmark}}
\setlength{\headheight}{15pt}


% Physics
%--------------------------------------------------------------------------------------------
\usepackage[arrowdel]{physics}      % all the usual useful physics commands
\usepackage{feyn}                   % for drawing Feynman diagrams
%\usepackage{bohr}                   % for drawing Bohr diagrams
%\usepackage{tikz-feynman}
\usepackage{elements}               % for quickly referencing information of various elements
\usepackage{tensor}                 % for writing tensors and chemical symbols

% Finishing touches
%--------------------------------------------------------------------------------------------
\author{Nathaniel D. Hoffman}

\title{33-761 Take-Home Final}
\date{\today}
\begin{document}
\maketitle

\section*{1.}
\begin{itemize}
    \item[(a)] Recall total angular momentum conservation we worked out in problem 6.10 of Jackson, show that the integral version can be recast into the following form,
        \begin{equation}
            \dv{ \va{L}_{\text{total}}}{t} = \int_{\Sigma} \left[ \dd{ \va{a}} \vdot \va{E} ( \va{x} \cross \epsilon_0 \va{E}) + \dd{ \va{a}} \vdot \va{B} \left( \va{x} \cross \frac{1}{\mu_0} \va{B} \right) \right] + \frac{1}{2} \int_{\Sigma} ( \dd{ \va{a}} \cross \va{x}) \left[ \epsilon \va{E}^2 + \frac{1}{\mu_0} \va{B}^2 \right]
        \end{equation}
        \begin{problem}
            From problem 6.10, we found that
            \begin{equation}
                \dv{ \va{L}}{t} = -\int_{\Sigma} \vu{n} \vdot \bar{M} \dd{a}
            \end{equation}
            with the following definitions:
            \begin{equation}
                \bar{M} = \bar{T} \cross \va{x}
            \end{equation}
            and
            \begin{align}
                T_{ij} &= \left[ \epsilon E_i E_j + \mu H_i H_j - \frac{1}{2} \delta_{ij} \left( \epsilon E^2 + \mu H^2 \right) \right] \\
                &= \left[ \epsilon_0 E_i E_j + \frac{1}{\mu_0} B_i B_j - \frac{1}{2} \delta_{ij} \left( \epsilon_0 E^2 + \frac{1}{\mu_0} B^2 \right) \right]
            \end{align}
            since $ \va{H} = \frac{1}{\mu_0} \va{B} $ and we assume we are working in a space with vacuum permittivity and permeability. Next, I will write this all in index notation, using
            \begin{equation}
                M_{il} = \epsilon_{ijk} T_{jl} x_k
            \end{equation}
            to denote the dyadic cross product.
            \begin{align}
                \dv{L_l}{t} &= -\int_{\Sigma} n_l M_{il} \dd{a} \\
                &= - \int_{\Sigma} n_l \epsilon_{ijk} T_{jl} x_k \dd{a} \\
                &= - \int_{\Sigma} n_l \epsilon_{ijk} \left[ \epsilon_0 E_j E_l + \frac{1}{\mu_0} B_j B_l - \frac{1}{2} \delta_{jl} \left( \epsilon_0 E^2 + \frac{1}{\mu_0} B^2 \right) \right] x_k \dd{a} \\
                &= - \int_{\Sigma} \dd{a_l} E_l (\epsilon_{ijk} \epsilon_0 E_j x_k) + \dd{a_l} B_l \left( \epsilon_{ijk} \frac{1}{\mu_0} B_j x_k \right) - \frac{1}{2} \dd{a_l} \epsilon_{ijk} \delta_{jl} x_k \left( \epsilon_0 E^2 + \frac{1}{\mu_0} B^2 \right)  \\
                &= \int_{\Sigma} \dd{a_l} E_l \left( \epsilon_{ikj} x_k E_j \right) + \dd{a_l} B_l \left( \epsilon_{ikj} \frac{1}{\mu_0} x_k B_j  \right) + \frac{1}{2} \int_{\Sigma}  \epsilon_{ijk} \dd{a_j}  x_k \left( \epsilon_0 E^2 + \frac{1}{\mu_0} B^2 \right) \\
                \dv{ \va{L}}{t} &= \int_{\Sigma} \left[ \dd{ \va{a}} \vdot \va{E} ( \va{x} \cross \epsilon_0 \va{E}) + \dd{ \va{a}} \vdot \va{B} \left( \va{x} \cross \frac{1}{\mu_0} \va{B} \right) \right] + \frac{1}{2} \int_{\Sigma} ( \dd{ \va{a}} \cross \va{x}) \left[ \epsilon \va{E}^2 + \frac{1}{\mu_0} \va{B}^2 \right]
            \end{align}
        \end{problem}
    \item[(b)] Consider a local current distribution which has \textit{no electric dipole and electric quadrupole moments} but the current distribution generates to the leading order a magnetic dipole $ \va{m} $ which oscillates in time with frequency $ \omega $. Find the radiated angular momentum for this case using the expression in part (a) and taking $ \Sigma $ as a sphere far away. NOTE that to find a nonzero result we should keep \textit{next to leading order terms in} $ \frac{1}{r} $, so the  non-radiation part contributes to the radiated angular momentum. (It is common to average $ \dv{ \va{L}}{t} $ over a period, it does not vanish).
        \begin{problem}
            Such a current distribution will give rise to fields described by equations 9.35 and 9.36 from Jackson:
            \begin{equation}
                \va{B} = \frac{\mu_0}{4 \pi} \left\{ k^2 ( \vu{n} \cross \va{m}) \cross \vu{n} \frac{e^{\imath kr}}{r} + \left[ 3 \vu{n} (\vu{n} \vdot \va{m}) - \va{m} \right] \left( \frac{1}{r^3} - \frac{\imath k}{r^2} \right) e^{\imath kr} \right\}
            \end{equation}
            \begin{equation}
                \va{E} = - \frac{Z_0}{4 \pi} k^2 ( \vu{n} \cross \va{m}) \frac{e^{\imath kr}}{r} \left( 1 - \frac{1}{\imath kr} \right)
            \end{equation}
            The final integral in the formula from (a) vanishes on the sphere, since $ \dd{ \va{a}} \cross \va{x} $ will be zero because those vectors will always be parallel. I will take the remaining terms one at a time. Note that we must use the complex conjugate of the field and divide by two in order to obtain the real part:
            \begin{align}
                \frac{1}{2} \dd{ \va{a}} \vdot \va{E} ( \va{x} \cross \epsilon_0 \va{E}^*) &= \underbrace{\frac{Z^2 \epsilon_0 k^4}{32 \pi^2} \left( \frac{1}{r} \left( 1 + \frac{1}{k^2 r^2} \right) \right)}_{A} \dd{ \va{a}} \vdot ( \vu{n} \cross \va{m}) ( \va{x} \cross ( \vu{n} \cross \va{m}^*)) \\
                &= A \dd{a} \vu{n} \vdot ( \vu{n} \cross \va{m})( \va{x} \cross ( \vu{n} \cross \va{m}^*)) \\
                &= 0
            \end{align}
            since $ \vu{n} \vdot ( \vu{n} \cross \va{m}) = \vu{n} \vdot ( \va{m} \cross \vu{n}) - \va{m} ( \vu{n} \cross \vu{n}) = \vu{n} \vdot (\va{m} \cross \vu{n}) = - \vu{n} \vdot ( \vu{n} \cross \va{m}) = 0 $

            Now let's examine the term which doesn't vanish. Note that on the sphere, $ \va{x} = \va{r} = r \vu{n} $:
            \begin{align}
                \frac{1}{2}\dd{ \va{a}} \vdot \va{B} \left( \va{x} \cross \frac{1}{\mu_0} \va{B}^* \right) &= \dd{a}\left\{ \left( \frac{\mu_0 k^2}{8 \pi} \frac{e^{\imath kr}}{r} \right) \overbrace{\vu{n} \vdot (\vu{n} \cross \va{m}) \cross \vu{n}}^{\ast} \right. \\
                &+ \left. \left( \frac{\mu_0}{8 \pi} e^{\imath kr} \left( \frac{1}{r^3} - \frac{\imath k}{r^2} \right) \right) \overbrace{(3 \vu{n} \vdot \vu{n} (\vu{n} \vdot \va{m}) - \vu{n} \vdot \va{m})}^{\ast\ast}\right\} \left( \va{x} \cross \frac{1}{\mu_0} \va{B}^*\right)
            \end{align}
            \begin{align}
                \mathlarger{\ast} &= \vu{n} \vdot (\vu{n} \cross \va{m}) \cross \vu{n} \\
                &= - \vu{n} \vdot (\vu{n} \cross (\vu{n} \cross \va{m})) \\
                &= - \vu{n} \vdot ( (\vu{n} \vdot \va{m}) \vu{n} - (\vu{n} \vdot \vu{n}) \va{m}) \\
                &= - ( \vu{n} \vdot \vu{n})( \vu{n} \vdot \va{m}) + ( \vu{n} \vdot \vu{n}) ( \vu{n} \vdot \va{m}) \\
                &= 0\\\\
                \mathlarger{\ast\ast} &= 3 (\vu{n} \vdot \va{m}) - \vu{n} \vdot \va{m} \\
                &= 2( \vu{n} \vdot \va{m})
            \end{align}
            since $ \vu{n} \vdot \vu{n} = 1 $.
            
            The final term is
            \begin{align}
                \va{x} \cross \frac{1}{\mu_0} \va{B}^* &= \left(\frac{k^2}{4 \pi} \frac{e^{-\imath kr}}{r} \right) \overbrace{( \va{x} \cross ( (\vu{n} \cross \va{m}^*) \cross \vu{n}))}^{\dagger} \\
                &+ \left( \frac{e^{-\imath kr}}{4 \pi} \left( \frac{1}{r^3} + \frac{\imath k}{r^2} \right) \right) \overbrace{( \va{x} \cross ( 3\vu{n} ( \vu{n} \vdot \va{m}^*) - \va{m}^*))}^{\dagger\dagger}
            \end{align}
            \begin{align}
                \mathlarger{\dagger} &= r \left( \vu{n} \cross ( (\vu{n} \cross \va{m}^*) \cross \vu{n}) \right) \\
                &= r \left( - \vu{n} \cross ( \vu{n} \cross ( \vu{n} \cross \va{m}^*)) \right) \\
                &= - r \left( \vu{n} \cross ((\vu{n} \vdot \va{m}^*) \vu{n} - (\vu{n} \vdot \vu{n}) \va{m}^*) \right) \\
                &= - r \left( (\vu{n} \vdot \va{m}^*)( \vu{n} \cross \vu{n}) - (\vu{n} \vdot \vu{n})( \vu{n} \cross \va{m}^*) \right) \\
                &= r( \vu{n} \cross \va{m}^*) \\\\
                \mathlarger{\dagger\dagger} &= r \left( \vu{n} \cross (3 \vu{n} ( \vu{n} \vdot \va{m}^*)) - \vu{n} \cross \va{m}^* \right) \\
                &= r \left(3 (\vu{n} \cross \vu{n}) (\vu{n} \vdot \va{m}^*) - \vu{n} \cross \va{m}^* \right) \\
                &= -r (\vu{n} \cross \va{m}^*)
            \end{align}

            All together, we now have
            \begin{align}
                \mathscr{I} &= \frac{\mu_0}{8 \pi} e^{\imath kr} \left( \frac{1}{r^3} - \frac{\imath k}{r^2} \right) 2 (\vu{n} \vdot \va{m})\left( \frac{e^{- \imath kr}}{4 \pi} \left( k^2 (\vu{n} \cross \va{m}^*) - \left( \frac{1}{r^2} + \frac{\imath k}{r} \right) (\vu{n} \cross \va{m}^*) \right) \right) \\
                &= - \left( \frac{\mu_0}{16 \pi^2 r^5} + \frac{\mu_0 \imath k^3}{16 \pi^2 r^2} \right) (\vu{n} \cross \va{m}^*)( \vu{n} \vdot \va{m}) 
            \end{align}
            Now we must integrate this factor over a sphere of radius $ r $ and take the limit as $ r \to \infty $:
            \begin{align}
                \dv{ \va{L}}{t} &= - \left( \frac{\mu_0}{16 \pi^2 r^5} + \frac{\mu_0 \imath k^3}{16 \pi^2 r^2} \right) \int_{\Sigma} (\vu{n} \vdot \va{m})( \vu{n} \cross \va{m}^*) \dd{a} \\
                &= \left( \frac{\mu_0}{16 \pi^2 r^5} + \frac{\mu_0 \imath k^3}{16 \pi^2 r^2} \right) \left( \frac{4 \pi}{3} r^2 ( \va{m}^* \cross \va{m}) \right) \\
                &= \left( \frac{\mu_0}{12 \pi r^3} + \frac{\imath k^3 \mu_0}{12 \pi}  \right) ( \va{m}^* \cross \va{m})
            \end{align}
            The integral is essentially the same one we had in the homework. The $ r^2 $ term comes from the spherical Jacobian, and we can use the fact that $ (\vu{n} \vdot \va{m})( \vu{n} \cross \va{m}^*) \mapsto n_i m_i \epsilon_{ijk} n_j m_k^* = n_i n_j \epsilon_{ijk} m_k^* m_i \mapsto - n_i n_j ( \va{m}^* \cross \va{m}) $ and $ \int n_i n_j \dd{\Omega} = \frac{4 \pi}{3} \delta_{ij} $.

            Taking the limit as $ r \to \infty $, we have
            \begin{equation}
                \dv{ \va{L}}{t} = \frac{\imath k^3 \mu_0}{12 \pi} \left( \va{m}^* \cross \va{m} \right) = \frac{k^3 \mu_0}{12 \pi} \Im[ \va{m} \cross \va{m}^* ]
            \end{equation}
            (note that I switched the cross product in the last step to get rid of the negative sign)
        \end{problem}
\end{itemize}


\section*{2.}
Consider a very thin conductor of length $ d $ placed along the $ z $-axis with its midpoint at the origin. Suppose that we have a current running in this conductor:
\begin{equation}
    I = I_0 \sin(kz) e^{- \imath \omega t}
\end{equation}
with $ k = \frac{\omega}{c} = \frac{4 \pi}{d} $. It is more convenient to express this current density in spherical coordinates for a multipole calculation.
\begin{itemize}
    \item[(a)] Obtain the \textit{exact multipoles} for this current distribution. Note that we cannot use the approximation $ kd << 1 $ here (since $ kd = 4 \pi $).
        \begin{problem}
            To begin, we need to restate the current density in spherical coordinates. I will propose the following density and then justify each term:
            \begin{equation}
                \va{J}_{\omega} = \frac{I_0 \sin(kr \cos(\theta))}{r^2} \delta(\varphi) \left[ \delta(\cos(\theta) - 1) + \delta(\cos(\theta) + 1) \right] \Theta\left( \frac{d}{2} - r \right) \left( \cos(\theta) \vu{r} - \sin(\theta) \vu{\theta} \right)
            \end{equation}
            where $ \va{J} = \va{J_0} e^{- \imath \omega t} $.

            First, the $ I_0 \sin(kr \cos(\theta)) $ term comes directly from the formula, since $ z = r \cos(\theta) $. Next, we have to confine this density to the thin conductor. Since it is running along the $ z $-axis, we want the $ \delta(\cos(\theta) \pm 1) $ terms, which set $ \theta $ on the $ z $-axis. The direction of this current must also be along the $ z $-axis, which is where the $ \cos(\theta) \vu{r} - \sin(\theta) \vu{\theta} = \vu{z} $ term comes from. The Heaviside function $ \Theta\left( \frac{d}{2} - r \right) $ ensures the current density is only nonzero on the conductor (which is centered at $ 0 $ so $ r = d/2 $ at each end). Finally, for simplicity, we set $ \varphi = 0 $ ($ \delta(\varphi) $) and normalize by $ \frac{1}{r^2} $ so that when we integrate over the $\delta$ functions in spherical coordinates, the Jacobian in spherical coordinates cancels the $ \frac{1}{r^2} $ term to give us the current we want in the problem.

            We can further simplify this current density by noticing that the $\delta$ functions set $ \cos(\theta) = \pm 1 $ which will set $ \sin(\theta) = 0 $:
            \begin{equation}
                \va{J}_{\omega} = \frac{I_0 \sin(kr)}{r^2} \delta(\varphi) \left[ \delta(\cos(\theta) - 1) + \delta(\cos(\theta) + 1) \right] \Theta\left( \frac{d}{2} - r \right) \vu{r}
            \end{equation}
            We can then find the associated charge density:
            \begin{equation}
                \rho_{\omega} = \frac{\div{ \va{J}_{\omega}}}{\imath \omega} = \frac{I_0}{\imath \omega} \left( \div{ \hat{r}} \right) \left[ \frac{\sin(kr)}{r^2} \left[ \delta(\cos(\theta) - 1) + \delta(\cos(\theta) + 1) \right] \delta(\varphi) \Theta\left( \frac{d}{2} - r \right) \right]
            \end{equation}
            The negative sign from $ \sin(-kr) = - \sin(kr) $ when the $ \delta(\cos(\theta) - 1) $ is used cancels with the negative sign from the $ \cos(\theta) \vu{r} $. In spherical coordinates, $ \div{ \vu{r}} = \frac{1}{r^2} \partial_r r^2 $, so this becomes
            \begin{equation}
                \rho_{\omega} = \frac{I_0}{\imath \omega r^2} k \cos(kr) \delta(\varphi) \left[ \delta(\cos(\theta) - 1) + \delta(\cos(\theta) + 1 ) \right] \Theta\left( \frac{d}{2} - r \right) + \cancelto{0}{\sin(kr) \left( \cdots \right)}
            \end{equation}
            The second term will be zero here because the derivative of the Heaviside function is a $\delta$ function and $ \delta \left( \frac{d}{2} - r \right) $ makes $ kr = 2 \pi $ and $ \sin(2 \pi) = 0 $.

            Now we can go about calculating the multipoles:
            \begin{equation}
                a_M = \frac{k^2}{\imath \sqrt{l(l+1)}} \int Y_{lm}^* \left( \div{( \va{r} \cross \va{J}_{\omega})} j_l(kr) \right) \dd[3]{x} = 0
            \end{equation}
            since $ \va{r} \cross \va{J}_{\omega} = \va{r} \cross J_{\omega} \hat{r} = 0 $.

            The other multipole is nonzero:
            \begin{equation}
                a_E = \frac{k^2}{\imath \sqrt{l(l+1)}} \int \left( \overbrace{Y_{lm}^*c \rho \partial_r \left[ r j_l(kr) \right]}^{\ast} + \overbrace{Y_{lm}^*\imath k \left( \va{r} \vdot \va{J} \right) j_l(kr)}^{\ast\ast} \right) \dd[3]{x}
            \end{equation}

            \begin{align}
                \mathlarger{\ast} &= \int \underbrace{[Y_{lm}^*(\pi,0) + Y_{lm}^*(0,0)]  c \rho(r) \dd[3]{x}}_{u} \underbrace{\partial_r \left[ r j_l(kr) \right]}_{\dd{v}} \dd{r} \\
                &= \underbrace{\eval{[Y_{lm}^*(\pi,0) + Y_{lm}^*(0,0)] c \rho(r) r j_l(kr) r^2}_{r = 0}^{d/2}}_{uv} \\ 
                &- \int \underbrace{ [Y^*_{lm}(\pi,0) + Y^*_{lm}(0,0)] c \partial_r[\rho r^2] r j_l(kr) \dd{r}}_{v \dd{u}} \\
                &= uv - \int_{0}^{d/2} [Y_{lm}^*(\pi,0) + Y_{lm}^*(0,0)] c \partial_r \left[ \frac{I_0 k}{\imath \omega} \cos(kr)  \right] r j_l(kr) \dd{r}
            \end{align}
            \begin{align}
                \partial_r \left[ \rho r^2 \right] &= -\frac{I_0k}{\imath \omega} k\sin(kr)  \\
                &= -\frac{I_0k^2}{\imath \omega} \sin(kr)
            \end{align}
            so the second term becomes
            \begin{equation}
                -\int_{0}^{d/2} [Y_{lm}^*(\pi,0) + Y_{lm}^*(0,0)] \imath k I_0 \sin(kr) r j_l(kr) \dd{r}
            \end{equation}
            since $ c = \frac{\omega}{k} $.

            Next we will look at the other term:
            \begin{align}
                \mathlarger{\ast\ast} &= \int Y_{lm}^* \imath k \left( \va{r} \vdot \va{J}_{\omega} \right) j_l(kr) \dd[3]{x} \\
                &= \int Y_{lm}^* \imath k r J_{\omega} j_l(kr) \dd[3]{x} \\
                &= \int_{0}^{d/2} [Y_{lm}^*(\pi,0) + Y_{lm}^*(0,0)] \imath k I_0 \sin(kr) r j_l(kr) \dd{r}
            \end{align}
            This exactly cancels the integral term we found above. Now we have only one term (the $ v u $ term from integration by parts):
            \begin{align}
                a_E &= \eval{\frac{k^2}{\imath \sqrt{l(l+1)}} [Y_{lm}^*(\pi,0) + Y_{lm}^*(0,0)] c \rho(r) r^3 j_l(kr)}_{r = 0}^{d/2} \\
                &= \frac{ I_0 k^2}{\imath^2\sqrt{l(l+1)}} [Y_{lm}^*(\pi,0) + Y_{lm}^*(0,0)]\eval{[\cos(kr) r j_l(kr)]}_{r=0}^{d/2} \\
                &= \frac{- I_0 k^2}{\sqrt{l(l+1)}} [Y_{lm}^*(\pi,0) + Y_{lm}^*(0,0)] \frac{d}{2} j_l(2 \pi)
            \end{align}
            We can further simplify this by writing the spherical harmonics in terms of Legendre polynomials. Since there is azimuthal symmetry about the $ z $-axis, we know that $ m = 0 $, so
            \begin{equation}
                a_E = - \frac{d I_0 k^2}{2\sqrt{l(l+1)}} \left[ \sqrt{\frac{2 l + 1}{4 \pi}}( P_{l}(-1) + P_{l}(+1)) \right] j_l(2 \pi)
            \end{equation}
            Additionally, $ P_l(1) = (-1)^l P_l(-1) $, so
            \begin{equation}
                a_E = - \frac{d I_0 k^2}{2\sqrt{l(l+1)}} \sqrt{\frac{2l+1}{4 \pi}} \left( (-1)^l + 1 \right) P_l(1) j_l(2 \pi)
            \end{equation}
            We know that
            \begin{equation}
                (-1)^l + 1 = \begin{cases} 2 \qif l \qeven \\ 0 \qif l \qodd \end{cases}
            \end{equation}
            and $ P_l(1) = 1 $, so
            \begin{equation}
                a_E = - \frac{d I_0 k^2}{\sqrt{l(l+1)}} \sqrt{\frac{2l+1}{4 \pi}} j_l(2 \pi) = - I_0 k^2 d \sqrt{\frac{(2l+1)}{4 \pi l(l+1)}} j_l(2 \pi) \qc l \qeven 
            \end{equation}
            and, lest we forget,
            \begin{equation}
                a_M = 0
            \end{equation}
        \end{problem}
    \item[(b)] Find the angular distribution of radiated power as well as the total power radiated in terms of the multipoles.
        \begin{problem}
            Since $ a_M = 0 $, we can write the angular distribution of radiated power as
            \begin{equation}
                \dv{P}{\Omega} = \frac{Z_0}{2 k^2} \abs{\sum_l \underbrace{(- \imath)^{l+1}}_{\imath (-1)^{l/2}} (a_E(l) \va{\mathbb{X}}_{l0} \cross \vu{n})}^2 \qc l \qeven
            \end{equation}
            The where $ \va{\mathbb{X}}_{l0} $ are the vector spherical harmonics:
            \begin{equation}
                \va{\mathbb{X}}_{l0} = \frac{1}{\sqrt{l(l+1)}} \va{\mathbb{L}} Y_{l0}
            \end{equation}
            where
            \begin{equation}
                \va{\mathbb{L}} = - \imath ( \va{x} \cross \grad)
            \end{equation}
            Next, we want to take the cross product with $ \vu{n} $:
            \begin{align}
                \va{\mathbb{L}} Y_{l0} \cross \vu{n} &= - \imath \left( \va{x} \cross \grad{Y_{l0}} \right) \cross \vu{n} \\
                &= \imath \vu{n} \cross \left( \va{x} \cross \grad{Y_{l0}} \right) \\
                &= \imath \left[ (\vu{n} \vdot \grad{Y_{l0}}) \va{r} - (\vu{n} \vdot \va{r}) \grad{Y_{l0}} \right] \\
                &= - \imath \left[ r \grad{Y_{l0}} \right]
            \end{align}
            since the first term would be derivatives of $ Y_{lm}(\Omega) $ with respect to $ r $, which are zero. Therefore, we are left with
            \begin{align}
                \dv{P}{\Omega} &= \frac{Z_0}{2 k^2} \abs{ \sum_l \frac{\imath(-1)^{l/2}}{\sqrt{l(l+1)}} a_E (- \imath r \grad{Y_{l0}})}^2 \qc l \qeven \\
                &= \frac{Z_0}{2 k^2} \abs{ \frac{(-1)^{l/2}}{\sqrt{l(l+1)}} a_E \sqrt{\frac{2l+1}{4 \pi}} r \grad{P_l(\cos(\theta))}}^2 \qc l \qeven\\
            \end{align}
            We can compute the gradient of the Legendre polynomials as:
            \begin{align}
                \grad{P_l(\cos(\theta))} &= \frac{1}{r} \partial_{\theta} P_l(\cos(\theta)) \vu{\theta} \\
                &= \sin(\theta) \frac{l}{\cos[2](\theta) - 1} \left( \cos(\theta) P_l(\cos(\theta)) - P_{l-1}(\cos(\theta)) \right) \\
                &= \frac{l}{\sin(\theta)} \left[ \cos(\theta) P_l(\cos(\theta)) - P_{l-1}(\cos(\theta)) \right]
            \end{align}
            so
            \begin{equation}
                \dv{P}{\Omega} = \frac{Z_0}{2 k^2} \abs{\sum_l \frac{(-1)^{l/2}}{\sqrt{l(l+1)}} a_E \sqrt{\frac{2l+1}{4 \pi}} \frac{l r}{\sin(\theta)} \left[ \cos(\theta) P_l(\cos(\theta)) - P_{l-1}(\cos(\theta)) \right]}^2 \qc l \qeven
            \end{equation}
            Inserting the equation we had for $ a_E $, we find:
            \begin{equation}
                \dv{P}{\Omega} = \frac{Z_0 d^2 k^2 I_0^2}{32 \pi^2} \abs{ \sum_l(\imath^l) \frac{2l+1}{ (l+1)} \frac{j_l(2 \pi)}{\sin(\theta)} \left[ \cos(\theta) P_l(\cos(\theta)) - P_{l-1}(\cos(\theta)) \right]}^2 \qc l \qeven
            \end{equation}
            Using the fact that $ kd = 4 \pi $, we can write this as
            \begin{equation}
                \dv{P}{\Omega} = \frac{Z_0 I_0^2}{2} \abs{\sum_l (\imath^l) \frac{2l+1}{l+1} \frac{j_l(2 \pi)}{\sin(\theta)} \left[ \cos(\theta) P_l(\cos(\theta)) - P_{l-1}(\cos(\theta)) \right]}^2 \qc l \qeven
            \end{equation}
            For the total power, we can use
            \begin{align}
                P = \frac{Z_0}{2k^2} \sum_l \abs{a_E(l)}^2 &= \frac{Z_0 I_0^2 k^2 d^2}{8 \pi} \sum_l \abs{\sqrt{\frac{2l+1}{l(l+1)}} j_l(2 \pi)}^2 \qc l \qeven \\
                &= \frac{Z_0 I_0^2 k^2 d^2}{8 \pi} \sum_l \frac{2l+1}{l(l+1)} \abs{j_l(2 \pi)}^2 \qc l \qeven \\
                &= 2 Z_0 I_0^2 \sum_l \frac{2 l + 1}{l(l + 1)} \abs{j_l(2 \pi)}^2 \qc l \qeven
            \end{align}
        \end{problem}
\end{itemize}


\end{document}
