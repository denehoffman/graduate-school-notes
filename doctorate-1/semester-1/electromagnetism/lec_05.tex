\documentclass[a4paper,twoside,master.tex]{subfiles}
\begin{document}
\lecture{5}{Wed Sep 4 2019}{Green's Functions for Special Geometries}

\section*{The Method of Images}%
\label{sec:the_method_of_images}

Suppose we have a grounded conductor along the xy-plane and a charge $q$ at position $(x',y',z')$ off of the plane. We claim the Green's function involves a charge at $-z'$ with opposite charge:

\begin{equation}
   G_D(\vec{x},\vec{x}') = \frac{1}{\sqrt(x-x')^2+(y-y')^2+(z-z')^2} - \frac{1}{\sqrt(x-x')^2+(y-y')^2+(z+z')^2}
\end{equation}

$\nabla'^2\left(\frac{1}{\sqrt(x-x')^2+(y-y')^2+(z+z')^2}\right) = 0$ so we still have $-\nabla'^2G_D=4\pi\delta(\vec{x}-\vec{x}')$.

\begin{equation}
   G_D(\vec{x},\vec{x'})\bigg|_{\Sigma = \{z'=0\mid(x,y,0)\in\mathbb{R}^3\}} = 0
\end{equation}

\section*{Solving for Potential}%
\label{sec:solving_for_potential}

When we put an actual charge $q$ at $(x',y',z')$,

\begin{equation}
   \Phi = \frac{1}{4\pi\epsilon_0}\left[\frac{q}{\sqrt(x-x')^2+(y-y')^2+(z-z')^2} - \frac{q}{\sqrt(x-x')^2+(y-y')^2+(z+z')^2}\right]
\end{equation}

Surface charge:
\begin{equation}
   \sigma = -\epsilon_0\frac{\partial\Phi}{\partial z}\bigg|_{z=0}
\end{equation}

\begin{note}{Note:}
The method of images is a bit contrived, since the solution is assumed and the Green's function is then derived.
\end{note}

\section*{Grounded Sphere}%
\label{sec:grounded_sphere}

Suppose we have a grounded sphere of radius $a$ with a point charge $q$ outside.

Suppose the distance from the center of the sphere to the charge is $x'$. Introduce a new point along the line joining the origin and the charge:

\begin{equation}
   y' = \frac{a^2}{x'}\hat{n}'
\end{equation} (pointing away from the origin).

Geometrically, you take the tangent from the circle to the point and project the point on the sphere where it touches onto the radial axis. We can use this as a 1-to-1 map between the inside and outside of the sphere. The sphere itself is invariant under this transformation. Alternative mappings would only work if they obey the Laplace equation!

\begin{equation}
   \Phi = \frac{q}{4\pi\epsilon_0|\vec{x}-\vec{x}'|} + \frac{q'}{4\pi\epsilon_0|\vec{x}-\frac{a^2}{x'}\hat{n}'|}.
\end{equation}

The second part follows the Laplace equation for $\vec{x}$, but it is not trivial that it also is a solution for $\vec{x}'$ for particular choice of $q$.

\begin{equation}
    \nabla^2\Phi=\frac{q}{\epsilon_0}
\end{equation} (outside, so $|\vec{x}|\geq a$)

\begin{equation}
    \Phi\bigg|_{\Sigma = S^2\text{ with radius }a} = \frac{q}{4\pi\epsilon_0|a\hat{n}-x'\hat{n}'|}+\frac{q'}{4\pi\epsilon_0|a\hat{n}-x'\hat{n}'|}
\end{equation}
\begin{equation}
    = \frac{q}{4\pi\epsilon_0\sqrt{a^2+x'^2-2ax'\hat{n}\cdot\hat{n}'}}+\frac{q'}{4\pi\epsilon_0\sqrt{a^2+\frac{a^4}{x'^2}-2a\frac{a^2}{x'}\hat{n}\cdot\hat{n}''}}
\end{equation}
\begin{equation}
    = \frac{1}{4\pi\epsilon_0}\left[\frac{q}{\sqrt{a^2+x'^2-2ax'\hat{n}\cdot\hat{n}'}}+\frac{q'}{\frac{a}{x'}\sqrt{a^2+\frac{a^4}{x'^2}-2a\hat{n}\cdot\hat{n}'}}\right]
\end{equation}

On the surface,
\begin{equation}
    \Phi\bigg|_\Sigma = \frac{1}{4\pi\epsilon_0}\left[\frac{q}{\sqrt{\dots}} + \frac{q'\frac{x'}{a}}{\sqrt{\dots}}\right]
\end{equation}

If $q' = -q\frac{a}{x'}$, $\Phi\bigg|_\Sigma = 0$, so

\begin{equation}
    \Phi = \frac{1}{4\pi\epsilon_0}\left[\frac{q}{|\vec{x}-\vec{x}'|} -\frac{q\frac{a}{x'}}{\left|\vec{x}-\frac{a^2}{x'^2}\vec{x}'\right|}\right]
\end{equation}

\begin{equation}
    \sigma\bigg|_\Sigma = -\epsilon\frac{\partial\Phi}{\partial r}\bigg|_{r=a}.
\end{equation}
If we integrate this, $\oint_{S^2}\sigma da = q'$.

\begin{remark}
We actually construct the Green's function for this problem thanks to $-\nabla'^2 G_D = 4\pi\delta(\vec{x}-\vec{x}')$, where $G_D = \frac{1}{|\vec{x}-\vec{x}'|} - \frac{\frac{a}{x'}}{\left|\vec{x}-\frac{a^2}{x'^2}\vec{x}'\right|}$
\end{remark}

\section*{Force of Attraction on Charges}%
\label{sec:force_of_attraction_on_charges}

What is the force between the sphere and the charge?

\begin{equation}
   \oint_{S^2}\frac{1}{4\pi\epsilon_0}\frac{q\sigma (a^2 d\Omega)}{|\vec{x}-\vec{x}'|^2}\frac{(\vec{x}'-\vec{x})}{|\vec{x}-\vec{x}'|} = \frac{1}{4\pi\epsilon_0}\frac{q\cdot -q\frac{a}{x'}}{|\frac{a^2}{x'^2}\vec{x'} -\vec{x'}|^2}\hat{n}'
\end{equation}

Basically the force between the surface charge $\sigma$ and point
charge $q$ is the same as the force on $q$ due to $q'$, the image
charge.

What about a charged sphere? Suppose we have a charge $Q$ on the
sphere. Find the grounded solution ($Q = 0$) and superimpose the
effect. The grounded sphere has charge $q' = -\frac{q}{x'}a$, so the
potential is now

\begin{equation}
   \frac{1}{4\pi\epsilon_0}\frac{Q+\frac{q}{x'}a}{|\vec{x}|}+\frac{1}{4\pi\epsilon_0}\left[\frac{q}{|\vec{x}-\vec{x}'|} -\frac{q\frac{a}{x'}}{\left|\vec{x}-\frac{a^2}{x'^2}\vec{x}'\right|}\right]
\end{equation}

\section*{Energy}%
\label{sec:energy}

What is the electrostatic energy of this configuration? Inside, $E=0$, but so $W=0$, but outside,

\begin{equation}
   W = \frac{1}{2}\epsilon_0\int E^2dx = \frac{1}{2}\int\rho\Phi d^3x
\end{equation}

We have to remove the infinity which comes from the $E$ field from the point charge itself. The self-energy of this charge would work if it's a continuous distribution, but the discreteness of the charge messes with the integral.

\begin{note}{N.B.}
For a charged sphere, $W=\frac{2}{5}\frac{Q^2}{4\pi\epsilon_0 R}$. When $R\to 0$, we have an obvious problem. This is inherent in classical electrodynamics. It is ``cured'' in QM, but in a complicated way.
\end{note}

\begin{equation}
   W = \frac{1}{2}\int q\delta(\vec{x}-\vec{x}')\Phi^{\text{reduced}}(\vec{x})d^3x
\end{equation}

where

\begin{equation}
   \Phi^\text{reduced} = \frac{1}{4\pi\epsilon_0}\frac{-\frac{q}{x'}a}{\left|\vec{x}-\frac{a^2}{x'^2}\vec{x}'\right|}.
\end{equation}

\end{document}
