\documentclass[a4paper,twoside,master.tex]{subfiles}
\begin{document}
\lecture{9}{Mon Sep 9 2019}{More on $Y_{lm}$ Functions}

\begin{note}{N.B.}
I will be using $L$ for $\mathbb{L}$ from here onward.
\end{note}

If we don't have the full range of the spherical angles, we actually have to solve the original $\nabla^2$ differential equations and can't use $L$ and $L^2$ or the $Y_{lm}$ functions.

\begin{equation}
    Y_{lm}(\theta,\phi) = \sqrt{\frac{2l+1}{4\pi}\frac{(l-m)!}{(l+m)!}}P_l^m(\cos\theta)e^{\imath m\phi}
\end{equation}
where
\begin{equation}
    P_l^m(x) = \frac{(-1)^m}{2^l l!}(1-x^2)^{m/2}\frac{d^{l+m}}{dx^{l+m}}(x^2-1)^l.
\end{equation}

Orthogonality tells us:
\begin{equation}
    \int Y_{lm}^*(\theta,\phi)Y_{l'm'}(\theta,\phi)d\Omega = \delta_{ll'}\delta_{mm'}
\end{equation}

The spectral decomposition tells us:
\begin{equation}
    \sum_{lm} Y_{lm}(\theta,\phi)Y_{lm}^*(\theta',\phi') = \delta(\phi-\phi')\delta(\cos\theta-\cos\theta')
\end{equation}

since $\delta(f(x)) = \frac{1}{f'(x_0)}\delta(x-x_0)$.

\begin{note}{N.B.}
In EM, we write $Y_{lm}$ such that
$Y_{lm}(\theta,\phi) = (-1)^m Y_{l,-m}^*(\theta,\phi)$.
\end{note}

For general spherical solutions,

$\Phi = \sum g_{lm}(r)Y_{lm}(\theta,\phi)$ or
\begin{equation}
    \frac{1}{r^2}\partial_r r^2\partial_r g - \frac{l(l+1)}{r^2} = 0
\end{equation}

so $r^2\partial_r^2 g + 2r\partial_r g - l(l+1)g = 0$.

Suppose $g = r^\lambda$:
\begin{equation}
    [\lambda(\lambda-1)+2\lambda - l(l+1)]r^\lambda = 0
\end{equation}
so $\lambda = l\text{ or } -(l+1)$

Therefore, the general solution in spherical systems (which use the periodicity in both $\phi$ and $\theta$) is:

\begin{equation}
    \Phi = \sum_{l=0}^\infty \sum_{m=-l}^l [A_l r^l + B_l r^{-(l+1)}]Y_{lm}(\theta,\phi).
\end{equation}

\subsection{Systems with $\phi$-independence}%
\label{sub:systems_with_phi_independence}

If we have an axis of symmetry, set the z-axis as the axis of symmetry. Therefore solutions should be independent of the angle around the z-axis ($\phi$).

This means $Y_{lm}\to Y_{l0}$ so $P_l^m\to P_l(x) = \frac{1}{2^l l!}\frac{d^l}{dx^l}(x^2-1)^l$, which are not normalized for ``historical reasons''. The differential equation then becomes:

\begin{equation}
    \frac{d}{dx}(1-x^2)\frac{d}{dx}P_l + l(l+1)P_l = 0,\ x\in[-1,1].
\end{equation}

\begin{remark}
There are other solutions
$Q_l(x)=\frac{1}{2}P_l(x)\ln\left[\frac{1-x}{1+x}\right]+R_l(x)$ where $R_l$ is a polynomial of degree $l-1$.
\end{remark}

Additionally $\int_{-1}^1 P_l(x)P_{l'}(x)dx = \frac{2l+1}{2}d_{ll'}$

Jackson notes some ``easy'' relations from Rodriguez's formula:

\begin{enumerate}
\item $\frac{d}{dx}P_{l+1}-\frac{d}{dx}P_{l-1}-(2l+1)P_l = 0$
\item $(l+1)P_{l+1} - (2l+1)xP_l + lP_{l-1} = 0$
\item $P_{2k}(0) = \frac{(2k-1)!!}{2^k k!}(-1)^k$
\end{enumerate}

It can be shown that:

\begin{equation}
    \frac{1}{|\vec{x}-\vec{x}'|} = \begin{cases}\sum \frac{1}{r^{l+1}} A_l P_l(\cos\gamma)& r>r'\\\sum r^l B_l P_l(\cos\gamma)& r<r'\\\end{cases}
\end{equation}

where $\gamma$ is the angle between the vectors.

Equivalently,

\begin{equation}
    \frac{1}{|\vec{x}-\vec{x}'|} = \sum_{l=0}^\infty \frac{r_<^l}{r_>^{l+1}}P_l(\cos\gamma)
\end{equation}
where $r_<$ and $r_>$ correspond to the smaller and larger vector.
\end{document}
