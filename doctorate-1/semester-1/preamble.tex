% My LaTeX preamble file - by Nathaniel Dene Hoffman
% Credit for much of this goes to Olivier Pieters (https://olivierpieters.be/tags/latex)
% and Gilles Castel (https://castel.dev)
% There are still some things to be done:
% 1. Update math commands using mathtools package (remove ddfrac command and just override)
% 2. Maybe abbreviate \imath somehow?
% 3. Possibly format for margin notes and set new margin sizes
% First, some encoding packages and usefull formatting
%--------------------------------------------------------------------------------------------
\usepackage[l2tabu,orthodox]{nag}   % force newer (and safer) LaTeX commands
\usepackage[utf8]{inputenc}         % set character set to support some UTF-8
                                    %   (unicode). Do NOT use this with
                                    %   XeTeX/LuaTeX!
\usepackage[english]{babel}         % multi-language support
\usepackage{sectsty}                % allow redefinition of section command formatting
\usepackage{tabularx}               % more table options
\usepackage{booktabs}
\usepackage{titling}                % allow redefinition of title formatting
\usepackage{imakeidx}               % create and index of words
\usepackage{xcolor}                 % more colour options
\usepackage{enumitem}               % more list formatting options
\usepackage{tocloft}                % redefine table of contents, new list like objects
\usepackage{subfiles}               % allow for multifile documents

% Next, let's deal with the whitespaces and margins
%--------------------------------------------------------------------------------------------
\usepackage[centering,margin=1in]{geometry}
\setlength{\parindent}{0cm}
\setlength{\parskip}{2ex plus 0.5ex minus 0.2ex} % whitespace between paragraphs

% Redefine \maketitle command with nicer formatting
%--------------------------------------------------------------------------------------------
\pretitle{
  \begin{flushright}         % align text to right
    \fontsize{40}{60}        % set font size and whitespace
    \usefont{OT1}{phv}{b}{n} % change the font to bold (b), normally shaped (n)
                             %   Helvetica (phv)
    \selectfont              % force LaTeX to search for metric in its mapping
                             %   corresponding to the above font size definition
}
\posttitle{
  \par                       % end paragraph
  \end{flushright}           % end right align
  \vskip 0.5em               % add vertical spacing of 0.5em
}
\preauthor{
  \begin{flushright}
    \large                   % font size
    \lineskip 0.5em          % inter line spacing
    \usefont{OT1}{phv}{m}{n}
}
\postauthor{
  \par
  \end{flushright}
}
\predate{
  \begin{flushright}
  \large
  \lineskip 0.5em
  \usefont{OT1}{phv}{m}{n}
}
\postdate{
  \par
  \end{flushright}
}

% Mathematics Packages
\usepackage[Gray,squaren,thinqspace,cdot]{SIunits}      % elegant units
\usepackage{amsmath}                                    % extensive math options
\usepackage{amsfonts}                                   % special math fonts
\usepackage{mathtools}                                  % useful formatting commands
\usepackage{amsthm}                                     % useful commands for building theorem environments
\usepackage{amssymb}                                    % lots of special math symbols
\usepackage{mathrsfs}                                   % fancy scripts letters
\usepackage{bm}                                         % bold math!
\usepackage{cancel}                                     % cancel lines in math
\usepackage{esint}                                      % fancy integral symbols
\usepackage{relsize}                                    % make math things bigger or smaller

\newcommand\ddfrac[2]{\frac{\displaystyle #1}{\displaystyle #2}}    % elegant fraction formatting
\allowdisplaybreaks[1]                                              % allow align environments to break on pages

% Ensure numbering is section-specific
%--------------------------------------------------------------------------------------------
\numberwithin{equation}{section}
\numberwithin{figure}{section}
\numberwithin{table}{section}

% Citations, references, and annotations
%--------------------------------------------------------------------------------------------
\usepackage[small,bf,hang]{caption}        % captions
\usepackage{subcaption}                    % adds subfigure & subcaption
\usepackage{sidecap}                       % adds side captions
\usepackage{hyperref}                      % add hyperlinks to references
\usepackage[noabbrev,nameinlink]{cleveref} % better references than default \ref
\usepackage{autonum}                       % only number referenced equations
\usepackage{url}                           % urls
\usepackage{cite}                          % well formed numeric citations
% format hyperlinks
\colorlet{linkcolour}{black}
\colorlet{urlcolour}{blue}
\hypersetup{colorlinks=true,
            linkcolor=linkcolour,
            citecolor=linkcolour,
            urlcolor=urlcolour}

% Plotting and Figures
%--------------------------------------------------------------------------------------------
\usepackage{tikz}          % advanced vector graphics
\usepackage{pgfplots}      % data plotting
\usepackage{pgfplotstable} % table plotting
\usepackage{placeins}      % display floats in correct sections
\usepackage{graphicx}      % include external graphics
\usepackage{longtable}     % process long tables

% use most recent version of pgfplots
\pgfplotsset{compat=newest}

% Misc.
%--------------------------------------------------------------------------------------------
\usepackage{todonotes}  % add to do notes
\usepackage{epstopdf}   % process eps-images
\usepackage{float}      % floats
\usepackage{stmaryrd}   % some more nice symbols
\usepackage{emptypage}  % suppress page numbers on empty pages
\usepackage{multicol}   % use this for creating pages with multiple columns
\usepackage{etoolbox}   % adds tags for environment endings
\usepackage{tcolorbox}  % pretty colored boxes!
\usepackage{halloweenmath}  % for use near halloween


% Custom Commands
%--------------------------------------------------------------------------------------------
\newcommand\hr{\noindent\rule[0.5ex]{\linewidth}{0.5pt}}                % horizontal line
\newcommand\N{\ensuremath{\mathbb{N}}}                                  % blackboard set characters
\newcommand\R{\ensuremath{\mathbb{R}}}
\newcommand\Z{\ensuremath{\mathbb{Z}}}
\newcommand\Q{\ensuremath{\mathbb{Q}}}
\newcommand\C{\ensuremath{\mathbb{C}}}
\renewcommand{\arraystretch}{1.2}                                       % More space between table rows (could be 1.3)

% Custom Environments
%--------------------------------------------------------------------------------------------
\newcommand{\lecture}[3]{\hr\\{\centering{\large\textsc{Lecture #1: #3}}\\#2\\}\hr\markboth{Lecture #1: #3}{\rightmark}}   % command to title lectures
\usepackage{mdframed}
\theoremstyle{plain}
\newmdtheoremenv[nobreak]{theorem}{Theorem}[section]
\newtheorem{corollary}{Corollary}[theorem]
\newtheorem{lemma}[theorem]{Lemma}
\theoremstyle{definition}
\newtheorem*{ex}{Example}
\newmdtheoremenv[nobreak]{definition}{Definition}[section]
\theoremstyle{remark}
\newtheorem*{remark}{Remark}
\AtEndEnvironment{ex}{\null\hfill$\diamond$}%
% Note: A proof environment is already provided in the amsthm package
\tcbuselibrary{breakable}
\newenvironment{note}[1]{\begin{tcolorbox}[
    arc=0mm,
    colback=white,
    colframe=white!60!black,
    title=#1,
    fonttitle=\sffamily,
    breakable
]}{\end{tcolorbox}}
\newenvironment{problem}{\begin{tcolorbox}[
    arc=0mm,
    breakable,
    colback=white,
    colframe=black
]}{\end{tcolorbox}}

% Header and Footer
%--------------------------------------------------------------------------------------------
% set header and footer
\usepackage{fancyhdr}                       % header and footer
\pagestyle{fancy}                           % use package
\fancyhf{}
\fancyhead[LE,RO]{\textsl{\rightmark}}      % E for even (left pages), O for odd (right pages)
\fancyfoot[LE,RO]{\thepage}
\fancyfoot[LO,RE]{\textsl{\leftmark}}
\setlength{\headheight}{15pt}


% Physics
%--------------------------------------------------------------------------------------------
\usepackage[arrowdel]{physics}      % all the usual useful physics commands
\usepackage{feyn}                   % for drawing Feynman diagrams
\usepackage{bohr}                   % for drawing Bohr diagrams
\usepackage{elements}               % for quickly referencing information of various elements
\usepackage{tensor}                 % for writing tensors and chemical symbols

% Finishing touches
%--------------------------------------------------------------------------------------------
\author{Nathaniel D. Hoffman}
