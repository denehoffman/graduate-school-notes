\documentclass[a4paper,twoside,master.tex]{subfiles}
\begin{document}
\lecture{42}{Friday, November 22, 2019}{Rotations of Quantum States}

\section{Rotations of Quantum States}
\label{sec:rotations_of_quantum_states}

Let's denote a rotation as a vector $ \va{\alpha} = \alpha \vu{u} $. We think of this rotation as an action $ \mathscr{R}_{ \vu{u}}(\alpha)\colon \va{v} \to \va{v}' = \mathscr{R} \va{v} $. This is what we call an active rotation; The coordinate system is fixed and there is a vector which is being actively transformed. We can combine rotations, and the order of operation doesn't matter as long as we rotate around the same axis:
\begin{equation}
    \mathscr{R}_u(\alpha) \mathscr{R}_u(\alpha') = \mathscr{R}_u(\alpha') \mathscr{R}_u(\alpha)
\end{equation}
However, this is generally not true for rotations by different axes:
\begin{equation}
    \mathscr{R}_u(\alpha) \mathscr{R}_{u'}(\alpha') \neq \mathscr{R}_{u'}(\alpha') \mathscr{R}_u(\alpha)
\end{equation}

How do rotations transform quantum states? First, let's consider a spinless particle. Suppose $ \psi( \va{r}) =\bra{ \va{r}}\ket{\psi} $. Let's define a rotation operator $ \mathscr{R}\colon \va{r} \to \va{r}' = \mathscr{R} \va{r} $. We are looking for a state such that $ \psi'( \va{r}') =\bra{ \va{r}'}\ket{\psi'} = \psi( \va{r}) $.
\begin{equation}
    \psi'( \mathscr{R} \va{r}) = \psi( \va{r}) \implies \psi'( \va{r}) = \psi( \mathscr{R}^{-1} \va{r})
\end{equation}
by replacing $ \mathscr{R} \va{r} \equiv \va{s} $ and then $ \va{s} \to \va{r} $. These are dummy variables, $ \va{r} $ can take any value since the coordinate system is standing still. What does the rotation do to the ket vector?
\begin{equation}
    \mathscr{R}\colon\ket{\psi} \to\ket{\psi'} = \vu{R}\ket{\psi}
\end{equation}
Therefore
\begin{equation}
    \bra{ \va{r}}( \vu{R}\ket{\psi}) =\bra{\mathscr{R}^{-1} \va{r}}\ket{\psi}
\end{equation}
This is a unitary operator, it is intended to maintain orthogonality of Classical vectors, but we can see it also preserves orthogonality of ket vectors.

How do we connect these operators with angular momentum? Let's think of an infinitesimal rotation, $ \mathscr{R}_{ \vu{u}}(\dd{\alpha})\colon \va{r} \to \va{r}' \approx \va{r} + \dd{\alpha} \vu{u} \cross \va{r} + \cdots $. We can also look at what happens to the wave function:
\begin{equation}
    \psi \to \psi'( \va{r}) = \psi( \va{r} - \dd{\alpha} \vu{u} \cross \va{r}) 
\end{equation}

For example, lets take $ \vu{u} = \vu{e}_2 $:
\begin{align}
    \psi'(x,y,z) &= \psi(x + y \dd{\alpha}, y - x \dd{\alpha}, z) = \psi(x,y,z) + \dd{\alpha} \left[ y \partial_x \psi - x \partial_y \psi \right] \\
    &= \left\{ \left( 1 - \frac{\imath}{\hbar} \dd{\alpha} \vu{L}_z \right) \psi \right\}(x,y,z) \\
    &= \psi'(x,y,z)
\end{align}
so
\begin{equation}
    \vu{R}_{ \vu{u}}(\dd{\alpha}) = \vu{1} - \frac{\imath}{\hbar} \dd{\alpha} \vu{u} \vdot \vu{L}
\end{equation}
We can divide by $ \dd{\alpha} $ (Leibnitz hates this):
\begin{equation}
    \dv{ \vu{R}_{ \vu{u}}(\alpha)}{\alpha} = - \frac{\imath}{\hbar} \vu{u} \vdot \vu{L}
\end{equation}
so
\begin{equation}
    \vu{R}_{ \vu{u}}(\alpha) = e^{- \frac{\imath}{\hbar} \alpha \vu{u} \vdot \vu{L}}
\end{equation}

What if we weren't working with a spinless particle? In this case, we are talking about a spinor, a tensor product with the spin state. For spin-$ 1/2 $, $ \va{S} = \frac{\hbar}{2} \va{\sigma} $, in terms of the Pauli matrices.

Define $ \vu{R}_{ \vu{u}}(\theta) = e^{- \frac{\imath}{\hbar} \theta \vu{u} \vdot \vu{S}} $. Therefore
\begin{equation}
    \vu{R}_{ \vu{u}}(\theta) = \cos(\frac{\theta}{2}) \vu{I} - \imath \vu{u} \vdot \va{\sigma} \sin(\frac{\theta}{2})
\end{equation}

What does this operator do to a spinor $\ket{\chi} $?

\begin{equation}
    \ket{\chi'} \equiv \vu{R}_{ \vu{u}}(\theta)\ket{\chi}
\end{equation}
where $\ket{\chi} $ could be $\ket{+}_z $, for example.

Suppose we want to rotate this spinor into the x/y-plane. For such a rotation, the end-state will be defined by the angle $ \varphi $ in the plane, so $ \vu{u} = ( \cos(\varphi + \frac{\pi}{2}), \sin(\varphi + \frac{\pi}{2}), 0) $, so
\begin{equation}
    \vu{R} = \cos(\frac{\theta}{2}) \vu{I} + \frac{1}{2} \left( \sigma_+ e^{- \imath \varphi} - \sigma_- e^{\imath \varphi} \right) \sin(\frac{\theta}{2})
\end{equation}
so
\begin{equation}
    \vu{R}\ket{\chi} = \cos(\frac{\theta}{2})\ket{+}_z + e^{\imath \varphi} \sin(\frac{\theta}{2})\ket{-}_z
\end{equation}
Homework 3 told us how to interpret such a state: The end state has the property $ \va{S} = + \frac{\hbar}{2} $ in the direction $ \va{v}' $.

Let's now look at the total angular momentum, $ \va{J} = \va{L} + \va{S} $ which acts in the Hilbert space $ \mathcal{H}_{ \va{r}} \otimes \mathcal{H}_S $. We now want introduce a rotation operator which is a combination of the spatial operator (from the spinless example) and the spin operator: $ \vu{R} = \vu{R}^{(r)} \otimes \vu{R}^{(s)} $, so
\begin{equation}
    \vu{R} = e^{- \frac{\imath}{\hbar} \theta \vu{u} \vdot \va{J}}
\end{equation}

One application of this is that if we have a angularly-invariant potential, the total angular momentum is conserved, since
\begin{equation}
    \vu{H} = \frac{ \vu{P}^2}{2m} + \vu{V}(\abs{ \va{R}})
\end{equation}
is scalar so
\begin{equation}
    \comm{ \va{J} \vdot \vu{u}}{ \vu{H}} = 0 \implies \pdv{t} \ev{ \va{J}} = 0
\end{equation}

Also, because $ \comm{ \vu{J}_{\pm}}{ \vu{H}} = 0 $, if $\ket{kjm} $ is an eigenstate of $ \vu{H} $,
\begin{equation}
    \vu{J}_{\pm}( \vu{H}\ket{kjm} = E_{kjm}\ket{kjm}) \implies \vu{H}( \vu{J}_{\pm}\ket{kjm}) = E_{kjm} ( \vu{J}_{\pm}\ket{kjm})
\end{equation}
so
\begin{equation}
    E_{kjm} = E_{kj}
\end{equation}
so there is a degeneracy in $ 2j + 1 $.

Finally, for some scalar observable $ \vu{A} $,
\begin{equation}
    \bra{k'j'm'} \vu{A} \ket{kjm} =\bra{k'_{jj}} \vu{A}\ket{k_{jj}} \delta_{jj'} \delta_{mm'}
\end{equation}
is independent of $ m $.


\end{document}
