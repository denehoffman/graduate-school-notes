\documentclass[a4paper,twoside]{article}
% My LaTeX preamble file - by Nathaniel Dene Hoffman
% Credit for much of this goes to Olivier Pieters (https://olivierpieters.be/tags/latex)
% and Gilles Castel (https://castel.dev)
% There are still some things to be done:
% 1. Update math commands using mathtools package (remove ddfrac command and just override)
% 2. Maybe abbreviate \imath somehow?
% 3. Possibly format for margin notes and set new margin sizes
% First, some encoding packages and usefull formatting
%--------------------------------------------------------------------------------------------
\usepackage[l2tabu,orthodox]{nag}   % force newer (and safer) LaTeX commands
\usepackage[utf8]{inputenc}         % set character set to support some UTF-8
                                    %   (unicode). Do NOT use this with
                                    %   XeTeX/LuaTeX!
\usepackage[T1]{fontenc}
\usepackage[english]{babel}         % multi-language support
\usepackage{sectsty}                % allow redefinition of section command formatting
\usepackage{tabularx}               % more table options
\usepackage{booktabs}
\usepackage{titling}                % allow redefinition of title formatting
\usepackage{imakeidx}               % create and index of words
\usepackage{xcolor}                 % more colour options
\usepackage{enumitem}               % more list formatting options
\usepackage{tocloft}                % redefine table of contents, new list like objects
\usepackage{subfiles}               % allow for multifile documents

% Next, let's deal with the whitespaces and margins
%--------------------------------------------------------------------------------------------
\usepackage[centering,margin=1in]{geometry}
\setlength{\parindent}{0cm}
\setlength{\parskip}{2ex plus 0.5ex minus 0.2ex} % whitespace between paragraphs

% Redefine \maketitle command with nicer formatting
%--------------------------------------------------------------------------------------------
\pretitle{
  \begin{flushright}         % align text to right
    \fontsize{40}{60}        % set font size and whitespace
    \usefont{OT1}{phv}{b}{n} % change the font to bold (b), normally shaped (n)
                             %   Helvetica (phv)
    \selectfont              % force LaTeX to search for metric in its mapping
                             %   corresponding to the above font size definition
}
\posttitle{
  \par                       % end paragraph
  \end{flushright}           % end right align
  \vskip 0.5em               % add vertical spacing of 0.5em
}
\preauthor{
  \begin{flushright}
    \large                   % font size
    \lineskip 0.5em          % inter line spacing
    \usefont{OT1}{phv}{m}{n}
}
\postauthor{
  \par
  \end{flushright}
}
\predate{
  \begin{flushright}
  \large
  \lineskip 0.5em
  \usefont{OT1}{phv}{m}{n}
}
\postdate{
  \par
  \end{flushright}
}

% Mathematics Packages
\usepackage[Gray,squaren,thinqspace,cdot]{SIunits}      % elegant units
\usepackage{amsmath}                                    % extensive math options
\usepackage{amsfonts}                                   % special math fonts
\usepackage{mathtools}                                  % useful formatting commands
\usepackage{amsthm}                                     % useful commands for building theorem environments
\usepackage{amssymb}                                    % lots of special math symbols
\usepackage{mathrsfs}                                   % fancy scripts letters
\usepackage{cancel}                                     % cancel lines in math
\usepackage{esint}                                      % fancy integral symbols
\usepackage{relsize}                                    % make math things bigger or smaller
\usepackage{bm}                                         % bold math!

\newcommand\ddfrac[2]{\frac{\displaystyle #1}{\displaystyle #2}}    % elegant fraction formatting
\allowdisplaybreaks[1]                                              % allow align environments to break on pages

% Ensure numbering is section-specific
%--------------------------------------------------------------------------------------------
\numberwithin{equation}{section}
\numberwithin{figure}{section}
\numberwithin{table}{section}

% Citations, references, and annotations
%--------------------------------------------------------------------------------------------
\usepackage[small,bf,hang]{caption}        % captions
\usepackage{subcaption}                    % adds subfigure & subcaption
\usepackage{sidecap}                       % adds side captions
\usepackage{hyperref}                      % add hyperlinks to references
\usepackage[noabbrev,nameinlink]{cleveref} % better references than default \ref
\usepackage{autonum}                       % only number referenced equations
\usepackage{url}                           % urls
\usepackage{cite}                          % well formed numeric citations
% format hyperlinks
\colorlet{linkcolour}{black}
\colorlet{urlcolour}{blue}
\hypersetup{colorlinks=true,
            linkcolor=linkcolour,
            citecolor=linkcolour,
            urlcolor=urlcolour}

% Plotting and Figures
%--------------------------------------------------------------------------------------------
\usepackage{tikz}          % advanced vector graphics
\usepackage{pgfplots}      % data plotting
\usepackage{pgfplotstable} % table plotting
\usepackage{placeins}      % display floats in correct sections
\usepackage{graphicx}      % include external graphics
\usepackage{longtable}     % process long tables

% use most recent version of pgfplots
\pgfplotsset{compat=newest}

% Misc.
%--------------------------------------------------------------------------------------------
\usepackage{todonotes}  % add to do notes
\usepackage{epstopdf}   % process eps-images
\usepackage{float}      % floats
\usepackage{stmaryrd}   % some more nice symbols
\usepackage{emptypage}  % suppress page numbers on empty pages
\usepackage{multicol}   % use this for creating pages with multiple columns
\usepackage{etoolbox}   % adds tags for environment endings
\usepackage{tcolorbox}  % pretty colored boxes!


% Custom Commands
%--------------------------------------------------------------------------------------------
\newcommand\hr{\noindent\rule[0.5ex]{\linewidth}{0.5pt}}                % horizontal line
\newcommand\N{\ensuremath{\mathbb{N}}}                                  % blackboard set characters
\newcommand\R{\ensuremath{\mathbb{R}}}
\newcommand\Z{\ensuremath{\mathbb{Z}}}
\newcommand\Q{\ensuremath{\mathbb{Q}}}
\newcommand\C{\ensuremath{\mathbb{C}}}
\renewcommand{\arraystretch}{1.2}                                       % More space between table rows (could be 1.3)
\newcommand{\Cov}{\mathrm{Cov}}
\newcommand*{\dbar}{\ensuremath{\text{\dj}}}
% Custom Environments
%--------------------------------------------------------------------------------------------
\newcommand{\lecture}[3]{\hr\\{\centering{\large\textsc{Lecture #1: #3}}\\#2\\}\hr\markboth{Lecture #1: #3}{\rightmark}}   % command to title lectures
\usepackage{mdframed}
\theoremstyle{plain}
\newmdtheoremenv[nobreak]{theorem}{Theorem}[section]
\newtheorem{corollary}{Corollary}[theorem]
\newtheorem{lemma}[theorem]{Lemma}
\theoremstyle{definition}
\newtheorem*{ex}{Example}
\newmdtheoremenv[nobreak]{definition}{Definition}[section]
\theoremstyle{remark}
\newtheorem*{remark}{Remark}
\AtEndEnvironment{ex}{\null\hfill$\diamond$}%
% Note: A proof environment is already provided in the amsthm package
\tcbuselibrary{breakable}
\newenvironment{note}[1]{\begin{tcolorbox}[
    arc=0mm,
    colback=white,
    colframe=white!60!black,
    title=#1,
    fonttitle=\sffamily,
    breakable
]}{\end{tcolorbox}}
\newenvironment{problem}{\begin{tcolorbox}[
    arc=0mm,
    breakable,
    colback=white,
    colframe=black
]}{\end{tcolorbox}}

% Header and Footer
%--------------------------------------------------------------------------------------------
% set header and footer
\usepackage{fancyhdr}                       % header and footer
\pagestyle{fancy}                           % use package
\fancyhf{}
\fancyhead[LE,RO]{\textsl{\rightmark}}      % E for even (left pages), O for odd (right pages)
\fancyfoot[LE,RO]{\thepage}
\fancyfoot[LO,RE]{\textsl{\leftmark}}
\setlength{\headheight}{15pt}


% Physics
%--------------------------------------------------------------------------------------------
\usepackage[arrowdel]{physics}      % all the usual useful physics commands
%\usepackage{feyn}                   % for drawing Feynman diagrams
%\usepackage{bohr}                   % for drawing Bohr diagrams
\usepackage{elements}               % for quickly referencing information of various elements
\usepackage{tensor}                 % for writing tensors and chemical symbols

% Finishing touches
%--------------------------------------------------------------------------------------------
\author{Nathaniel D. Hoffman}

\title{33-755 Homework 10}
\date{\today}
\begin{document}
\maketitle

\section*{5.2: Anisotropic Three-Dimensional Harmonic Oscillator}
\label{sec:anisotropic_three-dimensional_harmonic_oscillator}

In a three-dimensional problem, consider a particle of mass $ m $ and of potential energy:
\begin{equation}
    \vu{V}(\vu{X},\vu{Y},\vu{Z}) = \frac{m \omega^2}{2} \left[ \left( 1 + \frac{2 \lambda}{3} \right)(\vu{X}^2 + \vu{Y}^2) + \left( 1 - \frac{4 \lambda}{3} \right)\vu{Z}^2 \right]
\end{equation}
where $ \omega $ and $\lambda$ are constants which satisfy:
\begin{equation}
    \omega \geq 0,\quad 0 \leq \lambda \leq \frac{3}{4}
\end{equation}

\begin{itemize}
    \item[a.] What are the eigenstates of the Hamiltonian and the corresponding energies?
        \begin{problem}
            We can separate the Hamiltonian as $ \vu{H} = \vu{H}_x + \vu{H}_y + \vu{h}_z $ where
            \begin{equation}
                \vu{H}_x = \frac{ \vu{P}^2}{2m} + \frac{m \omega_x^2}{2} \vu{X}^2
            \end{equation}
            and similar for $ y $ and $ z $. Here, I am defining
            \begin{equation}
                \omega_x = \omega_y = \omega \sqrt{1 + \frac{2 \lambda}{3}}
            \end{equation}
            and
            \begin{equation}
                \omega_z = \omega \sqrt{1 - \frac{4 \lambda}{3}}
            \end{equation}
            We know that each of these component Hamiltonians will have eigenstates which are similar to 1-D harmonic oscillators:
            \begin{equation}
                \vu{H}_x\ket{\varphi_a} = \left( a + \frac{1}{2} \right) \hbar \omega_x\ket{\varphi_a} \longleftrightarrow [a \leftrightarrow b,c], [x \leftrightarrow y,z]
            \end{equation}
            These eigenvectors each belong to a Hilbert space, and the product of these spaces is the Hilbert space of the 3-D system:
            \begin{equation}
                \ket{\varphi_a} \in \mathcal{H}_x\quad\ket{\varphi_b} \in \mathcal{H}_y\quad\ket{\varphi_c} \in \mathcal{H}_z
            \end{equation}
            \begin{equation}
                \mathcal{H}_{ \va{r}} = \mathcal{H}_x \otimes \mathcal{H}_y \otimes \mathcal{H}_z
            \end{equation}
            so the eigenvectors of the system can be written as products of the eigenvectors of the subspaces:
            \begin{equation}
                \ket{\psi_{abc}} =\ket{\varphi_a}\ket{\varphi_b}\ket{\varphi_c}
            \end{equation}
            or
            \begin{equation}
                \psi_{abc}( \va{r}) = \left[ \frac{1}{\sqrt{2^a a!}} \left( \frac{m \omega_x}{\pi \hbar} \right)^{1/4} e^{- \frac{m \omega_x x^2}{2 \hbar}} H_a\left( \sqrt{\frac{m \omega_x}{\hbar}} x \right) \right] \times \longleftrightarrow [a \leftrightarrow b,c], [x \leftrightarrow y,z]
            \end{equation}
            where $ H_i(x) $ are the Hermite polynomials.
            The energy of each eigenstate can be written as a sum of the energies from each individual 1-D state, since $ \vu{H} = \sum_{i=x,y,z} \vu{H}_i $:
            \begin{equation}
                \vu{H}\ket{\psi_{abc}} = E\ket{\psi_{abc}} = \left[ \left( a + \frac{1}{2} \right) \hbar \omega_x + \left( b + \frac{1}{2} \right) \hbar \omega_y + \left( c + \frac{1}{2} \right) \hbar \omega_z \right]\ket{\psi_{abc}}
            \end{equation}
        \end{problem}
    \item[b.] Calculate and discuss, as functions of $ \lambda $, the variation of the energy, the parity and degree of degeneracy of the ground state and the first two excited states.
        \begin{problem}
            As $ \lambda \to 0 $, the system becomes isotropic since $ \omega_z \to \omega_x = \omega_y $. Additionally, as $ \lambda \to\frac{3}{4} $, $ \omega_z \to 0 $. In this case, the Hamiltonian in $ z $ becomes that of a free particle, so the problem is reduced to a quantum harmonic oscillator in the $ x $ and $ y $ dimensions only.

            The wording is a bit confusing as to what the first two excited states are (there are two degenerate states with higher energy than the first excited state). If we look at the ground state, $ \psi_{000} $, its clear that the energy will be $ \frac{\hbar}{2}(\omega_x + \omega_y + \omega_z) = \frac{\hbar}{2}\left( 2 \sqrt{\frac{2 \lambda }{3}+1} \omega +\sqrt{1-\frac{4 \lambda }{3}} \omega\right) \propto \sqrt{\lambda} $. The wave function for this state is even, since $ H_0(x) = 1 $ and the rest of the equation is even in $ x $ (because $ e^{x^2} $ is even since $ x^2 $ is even). In fact, for any state, the parity depends on the Hermite polynomials, which are even if their subscript is even and odd if it isn't. Multiplying even and odd functions acts like multiplying even and odd numbers, so $ H_a \times H_b \times H_c $ will be even iff $ a + b + c $ is an even number and odd otherwise.
            
            For the excited states, note that $ \omega_z < \omega_x = \omega_y $, so $ E_z < E_x = E_y $. Therefore, the state $ \psi_{001} $ is the next lowest energy state to the ground state, and because of the results of the above discussion, it will have an odd wave function. Next, because $ E_x = E_y $, the states $ \psi_{100} $ and $ \psi_{010} $ are degenerate, although their wave function is still odd under parity.

            However, these degenerate states are only the next excited state in the range $ \lambda < \frac{1}{2} $. In the range $ \frac{1}{2} \lambda \frac{3}{4} $, the state $ \psi_{002} $ actually has a lower energy than $ \psi_{100} = \psi_{010} $, and it has an even wave function. At $ \lambda = \frac{1}{2} $, the states $ \psi_{100} $, $ \psi_{010} $, and $ \psi_{002} $ are degenerate.
        \end{problem}
\end{itemize}

\section*{5.6: Charged Harmonic Oscillator in a Variable Electric Field}
\label{sec:5.6:_charged_harmonic_oscillator_in_a_variable_electric_field}

A one-dimensional harmonic oscillator is composed of a particle of mass $ m $, charge $ q $, and potential energy $ \vu{V}(\vu{X}) = \frac{1}{2} m \omega^2 \vu{X}^2 $. We assume in this exercise that the particle is placed in an electric field $ \mathscr{E}(t) $ parallel to $ \vu{x} $ and time-dependent, so that to $ \vu{V}(\vu{X}) $ must be added the potential energy:
\begin{equation}
    \vu{W}(t) = - q \mathscr{E}(t) \vu{X}
\end{equation}
\begin{itemize}
    \item[a.] Write the Hamiltonian $ \vu{H}(t) $ of the particle in terms of the operators $ \vu{a} $ and $ \vu{a}^\dagger $. Calculate the commutators of $ \vu{a} $ and $ \vu{a}^\dagger $ with $ \vu{H}(t) $.
        \begin{problem}
            We can rewrite $ \vu{X}^2 = \frac{\hbar}{2m \omega} \left( (\vu{a}^\dagger)^2 + 2\vu{N} + 1 + \vu{a}^2 \right) $ and $ \vu{P}^2 = - \frac{\hbar m \omega}{2} \left( (\vu{a}^\dagger)^2 - 2 \vu{N} - 1 + \vu{a}^2 \right) $, since $ \vu{X} = \sqrt{\frac{\hbar}{2 m \omega}} \left( \vu{a}^\dagger + \vu{a} \right) $. Therefore, the Hamiltonian is
            \begin{equation}
                \vu{H} = \hbar \omega \vu{a}^\dagger \vu{a} + \frac{\hbar \omega}{2} - q \mathscr{E}(t) \sqrt{\frac{\hbar}{2m \omega}} \left( \vu{a}^\dagger + \vu{a} \right)
            \end{equation}
            \begin{equation}
                \comm{ \vu{a}}{ \vu{H}} = \hbar \omega \comm{ \vu{a}}{ \vu{N}} - q \mathscr{E}(t) \sqrt{\frac{\hbar}{2m \omega}} \comm{ \vu{a}}{ \vu{a}^\dagger} = \hbar \omega \vu{a} - q \mathscr{E}(t) \sqrt{\frac{\hbar}{2m \omega}}
            \end{equation}
            and
            \begin{equation}
                \comm{ \vu{a}^\dagger}{ \vu{H}} = \hbar \omega \comm{ \vu{a}^\dagger}{ \vu{N}} - q \mathscr{E}(t) \sqrt{\frac{\hbar}{2m \omega}} \comm{ \vu{a}}{ \vu{a}^\dagger} = - \hbar \omega \vu{a}^\dagger + q \mathscr{E}(t) \sqrt{\frac{\hbar}{2m \omega}} 
            \end{equation}
        \end{problem}
    \item[b.] Let $ \alpha(t) $ be the number defined by:
        \begin{equation}
            \alpha(t) = \ev{\vu{a}}{\psi(t)}
        \end{equation}
        where $\ket{\psi(t)} $ is the normalized state vector of the particle under study. Show from the results of the preceding question that $ \alpha(t) $ satisfies the differential equation:
        \begin{equation}
            \dv{t} \alpha(t) = - \imath \omega \alpha(t) + \imath \lambda(t) 
        \end{equation}
        where $ \lambda(t) $ is defined by:
        \begin{equation}
            \lambda(t) = \frac{q}{\sqrt{2m \hbar \omega}} \mathscr{E}(t)
        \end{equation}
        Integrate this differential equation. At time $ t $, what are the mean values of the position and momentum of the particle?
        \begin{problem}
            We are saying that $ \alpha(t) = \ev{ \vu{a}}_{\psi(t)} $, so by Ehrenfest's theorem:
            \begin{align}
                \dv{t} \alpha(t) &= \frac{1}{\imath \hbar} \ev{\comm{ \vu{a}}{ \vu{H}}}_{\psi(t)} \\
                &= \frac{1}{\imath \hbar} \left( \hbar \omega \alpha(t) - q \mathscr{E}(t) \sqrt{\frac{\hbar}{2m \omega}} \right) \\
                &= - \imath \omega \alpha(t) + \imath \lambda(t)
            \end{align}
            Next, we are told to integrate this differential equation:
            \begin{align}
                \int_0^t \dv{t} \alpha(t') \dd{t'} = \alpha(t) &= \int_0^t -\imath \omega \alpha(t') \dd{t'} + \int_0^t \imath \lambda(t') \dd{t'} \\
                &= e^{- \imath \omega t} \left( \alpha(0) + \imath\int_0^t e^{\imath \omega t'} \lambda(t') \dd{t'} \right)
            \end{align}
            Finally, we can find the mean values of the position and momentum of the particle using the fact that $ \ev{ \vu{a}^\dagger} = \ev{ \vu{a}}^\dagger $, so
            \begin{equation}
                \ev{ \vu{X}} = \sqrt{\frac{\hbar}{2m \omega}} \left( \ev{ \vu{a}^\dagger} + \ev{ \vu{a}} \right) = \sqrt{\frac{\hbar}{2m \omega}} \left( \alpha^*(t) + \alpha(t) \right)
            \end{equation}
            and similarly
            \begin{equation}
                \ev{ \vu{P}} = \sqrt{\frac{\hbar \omega m}{2}} \imath \left( \alpha^*(t) - \alpha(t) \right)
            \end{equation}
        \end{problem}
    \item[c.] The ket $\ket{\varphi(t)} $ is defined by:
        \begin{equation}
            \ket{\varphi(t)} = [\vu{a} - \alpha(t)]\ket{\psi(t)}
        \end{equation}
        where $ \alpha(t) $ has the value calculated in $ b $. Using the results of questions $ a $ and $ b $, show that the evolution of $\ket{\varphi(t)} $ is given by:
        \begin{equation}
            \imath \hbar \dv{t}\ket{\varphi(t)} = [\vu{H}(t) + \hbar \omega]\ket{\varphi(t)}
        \end{equation}
        How does the norm of $\ket{\varphi(t)} $ vary with time?
        \begin{problem}
            First, it is important to recall that
            \begin{equation}
                \dv{t} \alpha(t) = \alpha'(t) = - \imath \omega \alpha(t) + \imath \lambda(t)
            \end{equation}
            Additionally, $\ket{\psi(t)} $ follows the time-dependent Schr\"odinger equation:
            \begin{equation}
                \imath \hbar \dv{t}\ket{\psi(t)} = \vu{H}\ket{\psi(t)}
            \end{equation}
            Therefore,
            \begin{align}
                \imath \hbar \dv{t}\ket{\varphi(t)} &= \imath \hbar \dv{t} (\vu{a} - \alpha(t))\ket{\psi(t)} \\
                &= \imath \hbar \dv{t} \vu{a}\ket{\psi(t)} - \imath \hbar \dv{t} \alpha(t)\ket{\psi(t)} \\
                &= \imath \hbar \vu{a} \dv{t}\ket{\psi(t)} - \imath \hbar \alpha'(t)\ket{\psi(t)} - \imath \hbar \alpha(t) \dv{t}\ket{\psi(t)} \\
                &= \vu{a} \vu{H}(t)\ket{\psi(t)} - \imath \hbar \alpha'(t)\ket{\psi(t)} - \alpha(t) \vu{H}\ket{\psi(t)} \\
                &= ( \vu{H}(t) \vu{a} + \hbar \omega \vu{a} - \hbar \lambda(t))\ket{\psi(t)} - \imath \hbar \alpha'(t)\ket{\psi(t)} - \alpha(t) \vu{H}(t)\ket{\psi(t)} \\
                &= \vu{H}(t)\ket{\varphi(t)} + \hbar \omega \vu{a}\ket{\psi(t)} - \hbar \lambda(t)\ket{\psi(t)} - \imath \hbar \alpha'(t)\ket{\psi(t)} \\
                &= \vu{H}(t)\ket{\varphi(t)} + \hbar \omega \vu{a}\ket{\psi(t)} - \hbar \lambda(t)\ket{\psi(t)} - \alpha(t)\ket{\psi(t)} + \hbar \lambda(t)\ket{\psi(t)} \\
                &= \left( \vu{H}(t) + \hbar \omega \right)\ket{\varphi(t)}
            \end{align}
            As for the norm of $\ket{\varphi(t)} $, I'm not sure how to show it, but I'm guessing it doesn't change in time, since in the next question we are asked to find the eigenvalues of $ \vu{a} $ and it would be nice if they were $ \alpha(t) $.
        \end{problem}
    \item[d.] Assuming that $\ket{\psi(0)} $ is an eigenvector of $ \vu{a} $ with the eigenvalue $ \alpha(0) $, show that $\ket{\psi(t)} $ is also an eigenvector of $ \vu{a} $, and calculate its eigenvalue.

        Find at time $ t $ the mean value of the unperturbed Hamiltonian
        \begin{equation}
            \vu{H}_0 = \vu{H}(t) - \vu{W}(t)
        \end{equation}
        as a function of $ \alpha(t) $. Give the root-mean-square deviations $ \Delta \vu{X} $, $ \Delta \vu{P} $, and $ \Delta \vu{H}_0 $; How do they vary with time?
        \begin{problem}
            \begin{equation}
                \ket{\varphi(0)} = \vu{a}\ket{\psi(0)} - \alpha(0)\ket{\psi(0)} = \alpha(0)\ket{\psi(0)} - \alpha(0)\ket{\psi(0)} = 0
            \end{equation}
            From the previous section, I'm guessing the norm of $\ket{\varphi} $  doesn't change in time, so $\ket{\varphi(t)} = 0 $. Therefore
            \begin{equation}
                \ket{\varphi(t)} = 0 = \vu{a}\ket{\psi(t)} - \alpha(t)\ket{\psi(t)}
            \end{equation}
            so
            \begin{equation}
                \vu{a}\ket{\psi(t)} = \alpha(t)\ket{\psi(t)} 
            \end{equation}

            Next, it will be helpful to define the following equivalences:
            \begin{align}
                \ev{ \vu{a}} &= \alpha(t) \\
                \ev{ \vu{a}^\dagger} &= \alpha^*(t) \\
                \ev{ \vu{a}^\dagger \vu{a}} &= \alpha^*(t) \alpha(t) = \norm{\alpha(t)}^2 = \ev{ \vu{N}} \\
                \ev{ \vu{a} \vu{a}^\dagger} &= \ev{ \vu{a}^\dagger \vu{a} + 1} = \norm{\alpha(t)}^2 + 1 \\
                \ev{ \vu{N}^2} &= \ev{ \vu{a}^\dagger \vu{a} \vu{a}^\dagger \vu{a}} = \ev{ \vu{a}^\dagger ( \vu{a}^\dagger \vu{a} + 1) \vu{a}} = \norm{\alpha}^4 + \norm{\alpha}^2
            \end{align}
            Next,
            \begin{equation}
                \ev{ \vu{H}_0} = \hbar \omega \ev{ \vu{N}} + \frac{\hbar \omega}{2} = \hbar \omega \left( \norm{\alpha(t)}^2 + \frac{1}{2} \right)
            \end{equation}
            To calculate the root-mean-square deviation of $ \vu{X} $, we begin with
            \begin{equation}
                \ev{ \vu{X}} = \sqrt{\frac{\hbar}{2 m \omega}} (\ev{ \vu{a}^\dagger} + \ev{ \vu{a}}) = \sqrt{\frac{\hbar}{2m \omega}} (\alpha^*(t) + \alpha(t))
            \end{equation}
            so
            \begin{equation}
                \ev{ \vu{X}}^2 = \frac{\hbar}{2m \omega} \left( (\alpha^*(t))^2 + 2 \norm{\alpha(t)}^2 + (\alpha(t))^2 \right)
            \end{equation}
            and
            \begin{equation}
                \ev{ \vu{X}^2} = \frac{\hbar}{2m \omega} \left( \ev{ \vu{a}^\dagger \vu{a}^\dagger} + \ev{ \vu{N}} + \ev{ \vu{N} + 1} + \ev{ \vu{a} \vu{a}} \right) = \frac{\hbar}{2m \omega} \left( (\alpha^*(t))^2 + 2 \norm{\alpha}^2 + (\alpha(t))^2 + 1 \right)
            \end{equation}
            so
            \begin{equation}
                \Delta \vu{X} = \sqrt{\ev{ \vu{X}^2} - \ev{ \vu{X}}^2} = \sqrt{\frac{\hbar}{2m \omega}}
            \end{equation}
            I could repeat this with $ \vu{P} $, but the derivation is exactly the same, except there is an extra factor of $ -1 $:
            \begin{equation}
                \Delta \vu{P} = \sqrt{\frac{\hbar m \omega}{2}}
            \end{equation}
            Note that $ \Delta \vu{X} \Delta \vu{P} = \frac{\hbar}{2} $, which agrees with the uncertainty principle.
            As for the unperturbed Hamiltonian,
            \begin{align}
                \ev{ \vu{H}_0^2} &= \hbar^2 \omega^2 \ev{ \vu{N}^2} + \frac{\hbar^2 \omega^2}{2} \ev{ \vu{N}} + \frac{\hbar^2 \omega^2}{2} \ev{ \vu{N} + 1} + \frac{\hbar^2 \omega^2}{4} \\
                &= \hbar^2 \omega^2 \left( \norm{\alpha(t)^4} + 2\norm{\alpha(t)}^2 + \frac{1}{2} + \frac{1}{4} \right) 
            \end{align}
            and
            \begin{equation}
                \ev{ \vu{H}_0}^2 = \hbar^2 \omega^2 \left( \norm{\alpha(t)}^4 + \norm{\alpha(t)}^2 + \frac{1}{4} \right)
            \end{equation}
            so
            \begin{equation}
                \Delta \vu{H}_0 = \hbar \omega \sqrt{\norm{\alpha(t)}^2 + \frac{1}{2}}
            \end{equation}
        \end{problem}
    \item[e.] Assume that at $ t = 0 $, the oscillator is in the ground state $\ket{\varphi_0} $. The electric field acts between times $ 0 $ and $ T $ and then falls to zero. When $ t > T $, what is the evolution of the mean values $ \ev{\vu{X}}(t) $ and $ \ev{\vu{P}}(t) $? Application: Assume that between $ 0 $ and $ T $, the field $ \mathscr{E}(t) $ is given by $ \mathscr{E}(t) = \mathscr{E}_0 \cos(\omega' t) $; Discuss the phenomena observed (resonance) in terms of $ \Delta \omega = \omega' - \omega $. If, at $ t > T $, the energy is measured, what results can be found, and with what probabilities?
        \begin{problem}
            In general, in this situation we can divide up the integral in $ \alpha(t) $ into two parts, one before $ T $ and one after $ T $, the second of which is zero because $ \lambda(t>T) = 0 $ since the field is turned off:
            \begin{equation}
                \alpha(t>T) = e^{- \imath \omega t} \left( \alpha(0) + \int_0^T \imath e^{\imath \omega t'} \lambda(t') \dd{t'} \right)
            \end{equation}
            Similarly,
            \begin{equation}
                \alpha(t>T) = e^{\imath \omega t} \left( \alpha^*(0) - \int_0^T \imath e^{-\imath \omega t'} \lambda(t') \dd{t'} \right)
            \end{equation}
            Therefore,
            \begin{align}
                \ev{ \vu{X}}(t) &=  \sqrt{\frac{\hbar}{2m \omega}} (\ev{ \vu{a}^\dagger} + \ev{ \vu{a}}) \\
                &= \sqrt{\frac{\hbar}{2m \omega}} \left( e^{- \imath \omega t} \left( \alpha(0) + \int_0^T \imath e^{\imath \omega t'} \lambda(t') \dd{t'} \right) + e^{\imath \omega t} \left( \alpha^*(0) - \int_0^T \imath e^{-\imath \omega t'} \lambda(t') \dd{t'} \right) \right)
            \end{align}
            and
            \begin{equation}
                \ev{ \vu{P}}(t) = \sqrt{\frac{\hbar m \omega}{2}} \left( e^{- \imath \omega t} \left( \alpha(0) + \int_0^T \imath e^{\imath \omega t'} \lambda(t') \dd{t'} \right) - e^{\imath \omega t} \left( \alpha^*(0) - \int_0^T \imath e^{-\imath \omega t'} \lambda(t') \dd{t'} \right) \right)
            \end{equation}

            If we now look at the example, we assume that before $ T $ we have $ \mathscr{E}(t) = \mathscr{E}_0 \cos(\omega't) $, so the integral in $ \alpha(t) $ evaluates to
            \begin{equation}
                \frac{q\mathscr{E}_0 \left( e^{\imath \omega T} (\omega \cos(\omega'T) - \imath \omega' \sin(\omega'T)) - \omega \right)}{(\omega^2 - \omega'^2) \sqrt{2m \omega \hbar}}
            \end{equation}
            or
            \begin{equation}
                \frac{q\mathscr{E}_0 \left( e^{-\imath \omega T} (\omega \cos(\omega'T) + \imath \omega' \sin(\omega'T)) - \omega \right)}{(\omega^2 - \omega'^2) \sqrt{2m \omega \hbar}}
            \end{equation}
            in $ \alpha^*(t) $. It's easy to see here that something significant will happen in the denominator when $ \omega = \omega' $. Notably, when we measure at time $ t > T$, we can use the time-independent Schr\"odinger equation to find the energy expectation value:
            \begin{equation}
                \ev{ \vu{H}} = E
            \end{equation}
            Note that here, the Hamiltonian is just
            \begin{equation}
                \vu{H} = \hbar \omega \left( \vu{N} + \frac{1}{2} \right)
            \end{equation}
            since the electric field has been turned off. Therefore, we expect the energy to be $ E = \hbar \omega \left( \norm{\alpha}^2 + \frac{1}{2} \right) $:
            \begin{align}
                E = \frac{e^{-\imath T (\omega+2 \omega')}}{(16 m \omega \hbar (\omega^2-\omega'^2)^2)} (&\sqrt{2} \mathscr{E}_0 q (-2 \omega e^{\imath T \omega'}+(\Delta\omega) e^{\imath T (\omega+2 \omega')}+e^{\imath T \omega} (\omega+\omega'))\\
                &+4 \alpha(0) e^{\imath T \omega'} (\omega'^2-\omega^2) \sqrt{m \omega \hbar }) (\sqrt{2} \mathscr{E} q (\omega (-2 e^{\imath T (\omega+\omega')}+e^{2 \imath T \omega'}+1)\\
                &+\omega' (e^{2 \imath T \omega'}-1))\\
                &+4 \alpha(0) (\omega^2-\omega'^2) \sqrt{m \omega \hbar } e^{\imath T (\omega+\omega')})
            \end{align}
            This is as far as I can figure it out. I'm sure there's a way to do this that's a million times simpler, but I don't know it.
        \end{problem}
\end{itemize}

\end{document}
