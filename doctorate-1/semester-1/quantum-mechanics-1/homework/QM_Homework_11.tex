\documentclass[a4paper,twoside]{article}
% My LaTeX preamble file - by Nathaniel Dene Hoffman
% Credit for much of this goes to Olivier Pieters (https://olivierpieters.be/tags/latex)
% and Gilles Castel (https://castel.dev)
% There are still some things to be done:
% 1. Update math commands using mathtools package (remove ddfrac command and just override)
% 2. Maybe abbreviate \imath somehow?
% 3. Possibly format for margin notes and set new margin sizes
% First, some encoding packages and usefull formatting
%--------------------------------------------------------------------------------------------
\usepackage[l2tabu,orthodox]{nag}   % force newer (and safer) LaTeX commands
\usepackage[utf8]{inputenc}         % set character set to support some UTF-8
                                    %   (unicode). Do NOT use this with
                                    %   XeTeX/LuaTeX!
\usepackage[T1]{fontenc}
\usepackage[english]{babel}         % multi-language support
\usepackage{sectsty}                % allow redefinition of section command formatting
\usepackage{tabularx}               % more table options
\usepackage{booktabs}
\usepackage{titling}                % allow redefinition of title formatting
\usepackage{imakeidx}               % create and index of words
\usepackage{xcolor}                 % more colour options
\usepackage{enumitem}               % more list formatting options
\usepackage{tocloft}                % redefine table of contents, new list like objects
\usepackage{subfiles}               % allow for multifile documents

% Next, let's deal with the whitespaces and margins
%--------------------------------------------------------------------------------------------
\usepackage[centering,margin=1in]{geometry}
\setlength{\parindent}{0cm}
\setlength{\parskip}{2ex plus 0.5ex minus 0.2ex} % whitespace between paragraphs

% Redefine \maketitle command with nicer formatting
%--------------------------------------------------------------------------------------------
\pretitle{
  \begin{flushright}         % align text to right
    \fontsize{40}{60}        % set font size and whitespace
    \usefont{OT1}{phv}{b}{n} % change the font to bold (b), normally shaped (n)
                             %   Helvetica (phv)
    \selectfont              % force LaTeX to search for metric in its mapping
                             %   corresponding to the above font size definition
}
\posttitle{
  \par                       % end paragraph
  \end{flushright}           % end right align
  \vskip 0.5em               % add vertical spacing of 0.5em
}
\preauthor{
  \begin{flushright}
    \large                   % font size
    \lineskip 0.5em          % inter line spacing
    \usefont{OT1}{phv}{m}{n}
}
\postauthor{
  \par
  \end{flushright}
}
\predate{
  \begin{flushright}
  \large
  \lineskip 0.5em
  \usefont{OT1}{phv}{m}{n}
}
\postdate{
  \par
  \end{flushright}
}

% Mathematics Packages
\usepackage[Gray,squaren,thinqspace,cdot]{SIunits}      % elegant units
\usepackage{amsmath}                                    % extensive math options
\usepackage{amsfonts}                                   % special math fonts
\usepackage{mathtools}                                  % useful formatting commands
\usepackage{amsthm}                                     % useful commands for building theorem environments
\usepackage{amssymb}                                    % lots of special math symbols
\usepackage{mathrsfs}                                   % fancy scripts letters
\usepackage{cancel}                                     % cancel lines in math
\usepackage{esint}                                      % fancy integral symbols
\usepackage{relsize}                                    % make math things bigger or smaller
\usepackage{bm}                                         % bold math!

\newcommand\ddfrac[2]{\frac{\displaystyle #1}{\displaystyle #2}}    % elegant fraction formatting
\allowdisplaybreaks[1]                                              % allow align environments to break on pages

% Ensure numbering is section-specific
%--------------------------------------------------------------------------------------------
\numberwithin{equation}{section}
\numberwithin{figure}{section}
\numberwithin{table}{section}

% Citations, references, and annotations
%--------------------------------------------------------------------------------------------
\usepackage[small,bf,hang]{caption}        % captions
\usepackage{subcaption}                    % adds subfigure & subcaption
\usepackage{sidecap}                       % adds side captions
\usepackage{hyperref}                      % add hyperlinks to references
\usepackage[noabbrev,nameinlink]{cleveref} % better references than default \ref
\usepackage{autonum}                       % only number referenced equations
\usepackage{url}                           % urls
\usepackage{cite}                          % well formed numeric citations
% format hyperlinks
\colorlet{linkcolour}{black}
\colorlet{urlcolour}{blue}
\hypersetup{colorlinks=true,
            linkcolor=linkcolour,
            citecolor=linkcolour,
            urlcolor=urlcolour}

% Plotting and Figures
%--------------------------------------------------------------------------------------------
\usepackage{tikz}          % advanced vector graphics
\usepackage{pgfplots}      % data plotting
\usepackage{pgfplotstable} % table plotting
\usepackage{placeins}      % display floats in correct sections
\usepackage{graphicx}      % include external graphics
\usepackage{longtable}     % process long tables

% use most recent version of pgfplots
\pgfplotsset{compat=newest}

% Misc.
%--------------------------------------------------------------------------------------------
\usepackage{todonotes}  % add to do notes
\usepackage{epstopdf}   % process eps-images
\usepackage{float}      % floats
\usepackage{stmaryrd}   % some more nice symbols
\usepackage{emptypage}  % suppress page numbers on empty pages
\usepackage{multicol}   % use this for creating pages with multiple columns
\usepackage{etoolbox}   % adds tags for environment endings
\usepackage{tcolorbox}  % pretty colored boxes!


% Custom Commands
%--------------------------------------------------------------------------------------------
\newcommand\hr{\noindent\rule[0.5ex]{\linewidth}{0.5pt}}                % horizontal line
\newcommand\N{\ensuremath{\mathbb{N}}}                                  % blackboard set characters
\newcommand\R{\ensuremath{\mathbb{R}}}
\newcommand\Z{\ensuremath{\mathbb{Z}}}
\newcommand\Q{\ensuremath{\mathbb{Q}}}
\newcommand\C{\ensuremath{\mathbb{C}}}
\renewcommand{\arraystretch}{1.2}                                       % More space between table rows (could be 1.3)
\newcommand{\Cov}{\mathrm{Cov}}
\newcommand*{\dbar}{\ensuremath{\text{\dj}}}
% Custom Environments
%--------------------------------------------------------------------------------------------
\newcommand{\lecture}[3]{\hr\\{\centering{\large\textsc{Lecture #1: #3}}\\#2\\}\hr\markboth{Lecture #1: #3}{\rightmark}}   % command to title lectures
\usepackage{mdframed}
\theoremstyle{plain}
\newmdtheoremenv[nobreak]{theorem}{Theorem}[section]
\newtheorem{corollary}{Corollary}[theorem]
\newtheorem{lemma}[theorem]{Lemma}
\theoremstyle{definition}
\newtheorem*{ex}{Example}
\newmdtheoremenv[nobreak]{definition}{Definition}[section]
\theoremstyle{remark}
\newtheorem*{remark}{Remark}
\AtEndEnvironment{ex}{\null\hfill$\diamond$}%
% Note: A proof environment is already provided in the amsthm package
\tcbuselibrary{breakable}
\newenvironment{note}[1]{\begin{tcolorbox}[
    arc=0mm,
    colback=white,
    colframe=white!60!black,
    title=#1,
    fonttitle=\sffamily,
    breakable
]}{\end{tcolorbox}}
\newenvironment{problem}{\begin{tcolorbox}[
    arc=0mm,
    breakable,
    colback=white,
    colframe=black
]}{\end{tcolorbox}}

% Header and Footer
%--------------------------------------------------------------------------------------------
% set header and footer
\usepackage{fancyhdr}                       % header and footer
\pagestyle{fancy}                           % use package
\fancyhf{}
\fancyhead[LE,RO]{\textsl{\rightmark}}      % E for even (left pages), O for odd (right pages)
\fancyfoot[LE,RO]{\thepage}
\fancyfoot[LO,RE]{\textsl{\leftmark}}
\setlength{\headheight}{15pt}


% Physics
%--------------------------------------------------------------------------------------------
\usepackage[arrowdel]{physics}      % all the usual useful physics commands
%\usepackage{feyn}                   % for drawing Feynman diagrams
%\usepackage{bohr}                   % for drawing Bohr diagrams
\usepackage{elements}               % for quickly referencing information of various elements
\usepackage{tensor}                 % for writing tensors and chemical symbols

% Finishing touches
%--------------------------------------------------------------------------------------------
\author{Nathaniel D. Hoffman}

\title{33-755 Homework 11}
\date{Thursday, November 21, 2019}
\begin{document}
\maketitle

\section*{Harmonic Oscillator in Thermal Equilibrium}
The density operator for a harmonic oscillator in thermal equilibrium is
\begin{equation}
    \vu{\rho} = \frac{1}{Z} e^{- \frac{\vu{H}}{k_B T}},
\end{equation}
where
\begin{equation}
    \vu{H} = \frac{ \vu{P}^2}{2m} + \frac{1}{2} m \omega^2 \vu{X}^2 = \left( \vu{N} + \frac{1}{2} \right) \hbar \omega
\end{equation}
with $ \vu{N} = \vu{a}^\dagger \vu{a} $.
\begin{itemize}
    \item[(a)] Show that $ e^{\frac{ \vu{H}}{k_B T}} a e^{- \frac{ \vu{H}}{k_B T}} = a e^{- \frac{\hbar \omega}{k_B T}} $, and hence $ \ev{ \vu{a}^\dagger \vu{a}} \equiv \Tr[\vu{\rho}\vu{N}] = \ev{ \vu{a} \vu{a}^\dagger} e^{- \frac{\hbar \omega}{k_B T}} $.
        \begin{problem}
            Using Baker-Campbell-Hausdorff, we know that
            \begin{equation}
                e^{A} B e^{-A} = B + \comm{A}{B} + \frac{1}{2} \comm{A}{\comm{A}{B}} + \cdots + \frac{1}{n!} \comm{A}{\overbrace{\cdots}^{n}}
            \end{equation}
            so
            \begin{equation}
                e^{ \frac{\vu{H}}{k_B T}} \vu{a} e^{- \frac{\vu{H}}{k_B T}} = \vu{a} + \frac{1}{k_B T} \comm{ \vu{H}}{ \vu{a}} + \ldots
            \end{equation}
            The commutator of $ \vu{H} $ and $ \vu{a} $ is $ - \hbar \omega \vu{a} $, so each additional commutation just adds an extra power of $ - \frac{\hbar \omega}{k_BT} $:
            \begin{equation}
                = \vu{a} + - \frac{\hbar \omega}{k_BT} \vu{a} + \frac{1}{2} (- \frac{\hbar \omega}{k_BT})^2 \vu{a} + \cdots + \frac{1}{n!} (- \frac{\hbar \omega}{k_BT})^n \vu{a}
            \end{equation}
            We can factor out the operator $ \vu{a} $ and the rest is a series which sums to an exponential
            \begin{equation}
                = \vu{a} \sum_n \frac{(- \frac{\hbar \omega}{k_B T})^n}{n!} = \vu{a} e^{- \frac{\hbar \omega}{k_B T}}
            \end{equation}
        \end{problem}
    \item[(b)] Use the commutation relation $ \comm{ \vu{a}}{ \vu{a}^\dagger} = 1 $ to show that $ \ev{ \vu{N}} = \frac{1}{e^{\frac{\hbar \omega}{k_B T}} - 1} $.
        \begin{problem}
            \begin{align}
                \ev{ \vu{N}} &= \Tr( \vu{\rho} \vu{N}) = \sum_n\bra{n} \vu{\rho} \vu{N}\ket{n} = \sum_n n\bra{n} \vu{\rho}\ket{n} = \sum_n \frac{n}{Z} e^{- \frac{\hbar \omega n}{k_BT}} = \frac{1}{Z} \left( \frac{1}{1 - e^{\frac{\hbar \omega}{k_BT}}} \right) \\ &= \frac{1}{e^{\frac{\hbar \omega}{k_BT}} -1}
            \end{align}
        \end{problem}
\end{itemize}

\section*{Zero-Point Motion of a Harmonic Chain}
\begin{itemize}
    \item[(a)] Show that the expectation values of kinetic and potential energy of the harmonic oscillator obey the virial relation for quadratic potentials, $ \ev{ \vu{V}} = \ev{ \vu{K}} $ when in an energy eigenstate. Use this result to calculate the mean square displacement $ \ev{ \vu{X}^2} $ in the ground state. Show, further, that the time average expectation value $ \overline{\ev{ \vu{V}}} = \overline{\ev{ \vu{K}}} $ regardless of the quantum state.
        \begin{problem}
            \begin{equation}
                \ev{ \vu{K}} = \ev{\frac{ \vu{P}^2}{2m}} = \frac{1}{2m} \left( - \frac{\hbar m \omega}{2} \right) \ev{( \vu{a}^\dagger - \vu{a} )^2}
            \end{equation}
            If we expand the operators in the last statement, we get two squared terms which will have no expectation value, since the eigenstates of energy are orthonormal. However, the terms $ \vu{a}^\dagger \vu{a} = \vu{N} $ and $ \vu{a} \vu{a}^\dagger = \vu{N} + 1 $ will have expectation values, so the stuff in the expectation brackets evaluates to $ - 2n $ (because of the minus sign between them in the square). Therefore
            \begin{equation}
                \ev{ \vu{K}} = \frac{\hbar \omega}{2} n
            \end{equation}

            Similarly,
            \begin{equation}
                \ev{ \vu{V}} = \frac{m \omega^2}{2} \ev{ \vu{X}^2} = \frac{m \omega^2}{2} \left( \frac{\hbar}{2 m \omega} \right) \ev{2 \vu{N}} = \frac{\hbar \omega}{2} n = \ev{ \vu{K}}
            \end{equation}

            Next, we can evaluate the mean square displacement in the ground state by noting that $ \ev{ \vu{X}^2} = \frac{2}{m \omega} \ev{ \vu{V}} = \frac{\hbar}{m \omega} n $. In the ground state, $ n = 0 $, so $ \ev{ \vu{X}^2}_{0} = 0 $.

            I'm not quite sure what the time average expectation value is, so I don't know what to do for the last part of the question.
        \end{problem}
    \item[(b)] Consider the infinite periodic chain of coupled oscillators discussed by Cohen-Tannoudji (complement $ J_V $). Express the position of the $ j $th oscillator, $ \vu{X}_j $ in terms of the normal mode coordinates $ \vu{\Xi}(k) $, with $ k \in (- \pi / l, \pi / l) $, where $ l $ is the period of the chain.
        \begin{problem}
            The conversion between $ \vu{X} $ and $ \vu{\Xi} $ is like a discrete Fourier transform in one direction and a continuous transform in the other direction. In the classical example, we wrote these transforms as
            \begin{equation}
                x_j(t) = \frac{l}{2 \pi} \int_{- \frac{\pi}{l}}^{ \frac{\pi}{l}} \xi(k, t) e^{\imath k j l}
            \end{equation}
            so, following this logic, we should be able to just promote each side to a quantum operator:
            \begin{equation}
                \vu{X}_q = \frac{l}{2 \pi} \int_{- \frac{\pi}{l}}^{ \frac{\pi}{l}} \vu{\Xi}(k) e^{\imath q k l}
            \end{equation}
        \end{problem}
    \item[(c)] Evaluate the expectation value $ \ev{\Xi(k) \Xi^\dagger(k')} $ in the ground state.
        \begin{problem}
            First, we can define
            \begin{equation}
                \vu{\Xi}(k) = \sqrt{\frac{\hbar}{2 m \Omega(k)}} ( \vu{a}^\dagger(k) + \vu{a}(k))
            \end{equation}
            similar to the position operator in $ x $-space. With this in mind, the expectation value should be similar to that of $ \vu{X}^2 $:
            \begin{equation}
                \ev{ \vu{\Xi}(k) \vu{\Xi}^\dagger(k')} = \frac{\hbar}{2m \sqrt{\Omega(k) \Omega(k')}} \ev{ \vu{a}^\dagger(k) \vu{a}(k') + \vu{a}(k) \vu{a}^\dagger(k')}
            \end{equation}
            In the ground state, I'm guessing $ k = k' = 0 $, so we just get two number operator equivalents which pull out $ k $ values from $ k $ energy states, so
            \begin{equation}
                \ev{ \vu{\Xi}(k) \vu{\Xi}^\dagger(k')}_{k=k'=0} = \frac{\hbar}{m \Omega(0)} \eval{k}_{0} = 0
            \end{equation}
        \end{problem}
    \item[(d)] Let the potential $ U $ vanish but keep $ V $ nonzero (i.e. $ \omega = 0 $ but $ \omega_1 \neq 0 $ in Cohen-Tannoudji's notation). Show that the mean square displacement $ \ev{ \vu{X}_j^2} $ of each mass $ j $ diverges in the ground state, but the mean square separation of neighboring masses $ \ev{( \vu{X}_{j+1} - \vu{X}_{j})^2} $ remains finite.
        \begin{problem}
            Using the results from the previous two problems, we can see that
            \begin{equation}
                \ev{ \vu{X}_j^2} = \frac{l^2}{4 \pi^2} \int_{- \frac{\pi}{l}}^{ \frac{\pi}{l}} \frac{\hbar^2}{m^2 \Omega^2(k)} k^2 e^{2\imath j k l}
            \end{equation}
            Here, $ \Omega^2(k) = 4 \omega_1^2 \sin[2](\frac{kl}{2}) $. This integral is best done with Mathematica, and it diverges.

            Next, we want to look at the separation:
            \begin{equation}
                ( \vu{X}_{j+1} - \vu{X}_{j})^2 = \left( \frac{l}{2 \pi} \int_{- \frac{\pi}{l}}^{ \frac{\pi}{l}} \vu{\Xi}(k) \left( e^{\imath (j+1) k l} - e^{\imath j k l} \right) \right)^2
            \end{equation}
            I couldn't figure out how to get Mathematica to perform this integral (the only difference is the exponential term, and it outputs a whole bunch of hypergeometric-type functions, so I'd assume that means the answer is finite).
        \end{problem}
\end{itemize}
\end{document}
