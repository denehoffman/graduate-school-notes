\documentclass[a4paper,twoside]{article}
% My LaTeX preamble file - by Nathaniel Dene Hoffman
% Credit for much of this goes to Olivier Pieters (https://olivierpieters.be/tags/latex)
% and Gilles Castel (https://castel.dev)
% There are still some things to be done:
% 1. Update math commands using mathtools package (remove ddfrac command and just override)
% 2. Maybe abbreviate \imath somehow?
% 3. Possibly format for margin notes and set new margin sizes
% First, some encoding packages and usefull formatting
%--------------------------------------------------------------------------------------------
\usepackage[l2tabu,orthodox]{nag}   % force newer (and safer) LaTeX commands
\usepackage[utf8]{inputenc}         % set character set to support some UTF-8
                                    %   (unicode). Do NOT use this with
                                    %   XeTeX/LuaTeX!
\usepackage[T1]{fontenc}
\usepackage[english]{babel}         % multi-language support
\usepackage{sectsty}                % allow redefinition of section command formatting
\usepackage{tabularx}               % more table options
\usepackage{booktabs}
\usepackage{titling}                % allow redefinition of title formatting
\usepackage{imakeidx}               % create and index of words
\usepackage{xcolor}                 % more colour options
\usepackage{enumitem}               % more list formatting options
\usepackage{tocloft}                % redefine table of contents, new list like objects
\usepackage{subfiles}               % allow for multifile documents

% Next, let's deal with the whitespaces and margins
%--------------------------------------------------------------------------------------------
\usepackage[centering,margin=1in]{geometry}
\setlength{\parindent}{0cm}
\setlength{\parskip}{2ex plus 0.5ex minus 0.2ex} % whitespace between paragraphs

% Redefine \maketitle command with nicer formatting
%--------------------------------------------------------------------------------------------
\pretitle{
  \begin{flushright}         % align text to right
    \fontsize{40}{60}        % set font size and whitespace
    \usefont{OT1}{phv}{b}{n} % change the font to bold (b), normally shaped (n)
                             %   Helvetica (phv)
    \selectfont              % force LaTeX to search for metric in its mapping
                             %   corresponding to the above font size definition
}
\posttitle{
  \par                       % end paragraph
  \end{flushright}           % end right align
  \vskip 0.5em               % add vertical spacing of 0.5em
}
\preauthor{
  \begin{flushright}
    \large                   % font size
    \lineskip 0.5em          % inter line spacing
    \usefont{OT1}{phv}{m}{n}
}
\postauthor{
  \par
  \end{flushright}
}
\predate{
  \begin{flushright}
  \large
  \lineskip 0.5em
  \usefont{OT1}{phv}{m}{n}
}
\postdate{
  \par
  \end{flushright}
}

% Mathematics Packages
\usepackage[Gray,squaren,thinqspace,cdot]{SIunits}      % elegant units
\usepackage{amsmath}                                    % extensive math options
\usepackage{amsfonts}                                   % special math fonts
\usepackage{mathtools}                                  % useful formatting commands
\usepackage{amsthm}                                     % useful commands for building theorem environments
\usepackage{amssymb}                                    % lots of special math symbols
\usepackage{mathrsfs}                                   % fancy scripts letters
\usepackage{cancel}                                     % cancel lines in math
\usepackage{esint}                                      % fancy integral symbols
\usepackage{relsize}                                    % make math things bigger or smaller
\usepackage{bm}                                         % bold math!

\newcommand\ddfrac[2]{\frac{\displaystyle #1}{\displaystyle #2}}    % elegant fraction formatting
\allowdisplaybreaks[1]                                              % allow align environments to break on pages

% Ensure numbering is section-specific
%--------------------------------------------------------------------------------------------
\numberwithin{equation}{section}
\numberwithin{figure}{section}
\numberwithin{table}{section}

% Citations, references, and annotations
%--------------------------------------------------------------------------------------------
\usepackage[small,bf,hang]{caption}        % captions
\usepackage{subcaption}                    % adds subfigure & subcaption
\usepackage{sidecap}                       % adds side captions
\usepackage{hyperref}                      % add hyperlinks to references
\usepackage[noabbrev,nameinlink]{cleveref} % better references than default \ref
\usepackage{autonum}                       % only number referenced equations
\usepackage{url}                           % urls
\usepackage{cite}                          % well formed numeric citations
% format hyperlinks
\colorlet{linkcolour}{black}
\colorlet{urlcolour}{blue}
\hypersetup{colorlinks=true,
            linkcolor=linkcolour,
            citecolor=linkcolour,
            urlcolor=urlcolour}

% Plotting and Figures
%--------------------------------------------------------------------------------------------
\usepackage{tikz}          % advanced vector graphics
\usepackage{pgfplots}      % data plotting
\usepackage{pgfplotstable} % table plotting
\usepackage{placeins}      % display floats in correct sections
\usepackage{graphicx}      % include external graphics
\usepackage{longtable}     % process long tables

% use most recent version of pgfplots
\pgfplotsset{compat=newest}

% Misc.
%--------------------------------------------------------------------------------------------
\usepackage{todonotes}  % add to do notes
\usepackage{epstopdf}   % process eps-images
\usepackage{float}      % floats
\usepackage{stmaryrd}   % some more nice symbols
\usepackage{emptypage}  % suppress page numbers on empty pages
\usepackage{multicol}   % use this for creating pages with multiple columns
\usepackage{etoolbox}   % adds tags for environment endings
\usepackage{tcolorbox}  % pretty colored boxes!


% Custom Commands
%--------------------------------------------------------------------------------------------
\newcommand\hr{\noindent\rule[0.5ex]{\linewidth}{0.5pt}}                % horizontal line
\newcommand\N{\ensuremath{\mathbb{N}}}                                  % blackboard set characters
\newcommand\R{\ensuremath{\mathbb{R}}}
\newcommand\Z{\ensuremath{\mathbb{Z}}}
\newcommand\Q{\ensuremath{\mathbb{Q}}}
\newcommand\C{\ensuremath{\mathbb{C}}}
\renewcommand{\arraystretch}{1.2}                                       % More space between table rows (could be 1.3)
\newcommand{\Cov}{\mathrm{Cov}}
\newcommand*{\dbar}{\ensuremath{\text{\dj}}}
% Custom Environments
%--------------------------------------------------------------------------------------------
\newcommand{\lecture}[3]{\hr\\{\centering{\large\textsc{Lecture #1: #3}}\\#2\\}\hr\markboth{Lecture #1: #3}{\rightmark}}   % command to title lectures
\usepackage{mdframed}
\theoremstyle{plain}
\newmdtheoremenv[nobreak]{theorem}{Theorem}[section]
\newtheorem{corollary}{Corollary}[theorem]
\newtheorem{lemma}[theorem]{Lemma}
\theoremstyle{definition}
\newtheorem*{ex}{Example}
\newmdtheoremenv[nobreak]{definition}{Definition}[section]
\theoremstyle{remark}
\newtheorem*{remark}{Remark}
\AtEndEnvironment{ex}{\null\hfill$\diamond$}%
% Note: A proof environment is already provided in the amsthm package
\tcbuselibrary{breakable}
\newenvironment{note}[1]{\begin{tcolorbox}[
    arc=0mm,
    colback=white,
    colframe=white!60!black,
    title=#1,
    fonttitle=\sffamily,
    breakable
]}{\end{tcolorbox}}
\newenvironment{problem}{\begin{tcolorbox}[
    arc=0mm,
    breakable,
    colback=white,
    colframe=black
]}{\end{tcolorbox}}

% Header and Footer
%--------------------------------------------------------------------------------------------
% set header and footer
\usepackage{fancyhdr}                       % header and footer
\pagestyle{fancy}                           % use package
\fancyhf{}
\fancyhead[LE,RO]{\textsl{\rightmark}}      % E for even (left pages), O for odd (right pages)
\fancyfoot[LE,RO]{\thepage}
\fancyfoot[LO,RE]{\textsl{\leftmark}}
\setlength{\headheight}{15pt}


% Physics
%--------------------------------------------------------------------------------------------
\usepackage[arrowdel]{physics}      % all the usual useful physics commands
%\usepackage{feyn}                   % for drawing Feynman diagrams
%\usepackage{bohr}                   % for drawing Bohr diagrams
\usepackage{elements}               % for quickly referencing information of various elements
\usepackage{tensor}                 % for writing tensors and chemical symbols

% Finishing touches
%--------------------------------------------------------------------------------------------
\author{Nathaniel D. Hoffman}

\title{33-755 Homework 12}
\date{\today}
\begin{document}
\maketitle

\section*{Cohen-Tannoudji 6.2: Observables in Angular Momentum Superposition States}
Consider an arbitrary physical system whose four-dimensional state space is spanned by a basis of four eigenvectors $\ket{j,m_z} $ common to $ \va{J}^2 $ and $ J_z $ ($ j = 0 $ or $ 1 $; $ -j \leq m_z \leq +j $), of eigenvalues $ j(j+1) \hbar^2 $ and $ m_z \hbar $ such that:
\begin{equation}
    J_{\pm}\ket{j, m_z} = \hbar \sqrt{j(j+1) - m_z (m_z \pm 1)}\ket{j, m_z \pm 1}
\end{equation}
\begin{equation}
    J_+\ket{j,j} = J_-\ket{j,-j} = 0
\end{equation}
\begin{itemize}
    \item[a.] Express in terms of the kets $\ket{j, m_z} $, the eigenstates common to $ \va{J}^2 $ and $ J_x $, to be denoted by $\ket{j, m_x} $.
        \begin{problem}
            First, both bases share the ground state $\ket{0,0} $ because this is the only eigenstate of $ \va{J}^2 $ with eigenvalue $ 0 $. Next, we can express $ J_x $ in terms of the raising and lowering operators: $ J_x = \frac{1}{2} (J_+ + J_-) $.
            
            If we act $ J_x $ on an arbitrary vector in the $ J_z $ basis, $\ket{\psi} = \alpha\ket{1,-1} + \beta\ket{1,0} + \gamma\ket{1,1} $, we find:
            \begin{equation}
                J_x\ket{\psi} = \frac{\sqrt{2}}{2} \left[ (\alpha + \gamma)\ket{1,0} + \beta (\ket{1,-1} +\ket{1,1}) \right]
            \end{equation}
            Therefore, the eigenstate of $ J_x $ with eigenvalue $ + \hbar $,
            \begin{equation}
                J_x\ket{1,m_x = +1} = \hbar\ket{1, m_x = +1}
            \end{equation}
            will be the state when
            \begin{equation}
                (\alpha + \gamma) \frac{\sqrt{2}}{2} = \beta \qand \beta \frac{\sqrt{2}}{2} = \alpha = \gamma
            \end{equation}
            or
            \begin{equation}
                \beta = \sqrt{2}, \alpha = \gamma = 1
            \end{equation}
            so
            \begin{equation}
                \ket{1, m_x = +1} = \frac{1}{2} \left[\ket{1,-1} + \sqrt{2}\ket{1,0} +\ket{1,-1}\right]
            \end{equation}
            accounting for normalization.

            Similarly,
            \begin{equation}
                \ket{1, m_x = -1} = \frac{1}{2} \left[\ket{1,1} - \sqrt{2}\ket{1,0} +\ket{1,-1} \right]
            \end{equation}
            and
            \begin{equation}
                \ket{1, m_x = 0} = \frac{1}{\sqrt{2}} \left[\ket{1,1} -\ket{1,-1} \right]
            \end{equation}
        \end{problem}
    \item[b.] Consider a system in the normalized state:
        \begin{equation}
            \ket{\psi} = \alpha\ket{1,1} + \beta\ket{1,0} + \gamma\ket{1,-1} + \delta\ket{0,0}
        \end{equation}
        \subitem(i) What is the probability of finding $ 2 \hbar^2 $ and $ \hbar $ if $ \va{J}^2 $ and $ J_x $ are measured simultaneously?
        \begin{problem}
            \begin{equation}
                \Pr( \va{J}^2 = 2 \hbar^2, J_z = \hbar) = \abs{\bra{1,1}\ket{\psi}}^2 = \abs{\alpha}^2
            \end{equation}
        \end{problem}
        \subitem(ii) Calculate the mean value of $ J_z $ when the system is in the state $\ket{\psi} $, and the probabilities of the various possible results of a measurement bearing only on this observable.
        \begin{problem}
            Only the $ m_z \neq 0 $ terms will contribute in the expectation value since the eigenvalue of $ J_z $ on a state with $ m_z = 0 $ is $ 0 $: 
            \begin{equation}
                \ev{ J_z} = (\abs{\alpha}^2 - \abs{\gamma}^2) \hbar
            \end{equation}
            \begin{equation}
                \Pr(J_z = \hbar) = \abs{\alpha}^2
            \end{equation}
            \begin{equation}
                \Pr(J_z = - \hbar) = \abs{\gamma}^2
            \end{equation}
            \begin{equation}
                \Pr(J_z = 0) = \abs{\beta}^2 + \abs{\delta}^2
            \end{equation}
        \end{problem}
        \subitem(iii) Same questions for the observable $ \va{J}^2 $ and for $ J_x $.
        \begin{problem}
            \begin{align}
                \Pr(J^2 = 2 \hbar^2, J_x = \hbar) &= \abs{\bra{1, m_x = 1}\ket{\psi}}^2 \\
                &= \frac{1}{4} \abs{\bra{1,1}\ket{\psi} + \sqrt{2}\bra{1,0}\ket{\psi} +\bra{1,-1}\ket{\psi}}^2 \\
                &= \frac{1}{4} \abs{\alpha + \sqrt{2} \beta + \gamma}^2
            \end{align}
            \begin{equation}
                \Pr(J_x = \hbar) = \abs{\bra{1,m_x = 1}\ket{\psi}}^2 = \frac{1}{4} \abs{\alpha + \sqrt{2} \beta + \gamma}^2
            \end{equation}
            \begin{equation}
                \Pr(J_x = - \hbar) = \abs{\bra{1,m_x = 1}\ket{\psi}}^2 = \frac{1}{4} \abs{\alpha - \sqrt{2} \beta + \gamma}^2
            \end{equation}
            \begin{equation}
                \Pr(J_x = 0) = \abs{\bra{1,m_x = 0}\ket{\psi}}^2 + \abs{\bra{0,m_x = 0}\ket{\psi}}^2 = \abs{\alpha - \gamma}^2 + \abs{\delta}^2
            \end{equation}
        \end{problem}
        \subitem(iv) $ J_z^2 $ is now measured; What are the possible results, their probabilities, and their mean value?
        \begin{problem}
            The two possible outcomes are $ 0 $ and $ \hbar^2 $.
            \begin{equation}
                \Pr(J^2_z = 0) = \Pr(J_z = 0) = \abs{\beta}^2 + \abs{\delta}^2
            \end{equation}
            \begin{equation}
                \Pr(J^2_z = \hbar^2) = \Pr(J_z = \hbar) + \Pr(J_z = - \hbar) = \abs{\alpha}^2 + \abs{\gamma}^2
            \end{equation}
            \begin{equation}
                \ev{J_z^2} = \hbar \ev{J_z} = \hbar^2 (\abs{\alpha}^2 + \abs{\gamma}^2)
            \end{equation}
        \end{problem}
\end{itemize}



\section*{Cohen-Tannoudji 6.6: Electric Quadrupole Hamiltonian}
Consider a system of angular momentum $ l=1 $. A basis of its state space is formed by the three eigenvectors of $ L_z $: $\ket{+1} $, $\ket{0} $, $\ket{-1} $, whose eigenvalues are, respectively, $ + \hbar $, $ 0 $, and $ - \hbar $, and which satisfy:
\begin{equation}
    L_{\pm}\ket{m} = \hbar \sqrt{2}\ket{m \pm 1}
\end{equation}
\begin{equation}
    L_+\ket{1} = L_-\ket{-1} = 0
\end{equation}
This system, which possesses an electric quadrupole moment, is placed in an electric field gradient, so that its Hamiltonian can be written:
\begin{equation}
    H = \frac{\omega_0}{\hbar} (L_u^2 - L_v^2)
\end{equation}
where $ L_u $ and $ L_v $ are the components of $ \va{L} $ along the two directions $ Ou $ and $ Ov $ of the $ xOz $ plane which form angles of $ 45^\circ $ with $ Ox $ and $ Oz $; $ \omega_0 $ is a real constant.

\begin{itemize}
    \item[a.] Write the matrix which represents $ H $ in the $ \{\ket{+1},\ket{0},\ket{-1}\} $ basis. What are the stationary states of the system and what are their energies? (These states are to be written $\ket{E_1},\ket{E_2},\ket{E_3} $, in order of decreasing energies.)
        \begin{problem}
            We can write
            \begin{equation}
                L_u = \frac{1}{\sqrt{2}} (L_x + L_z)
            \end{equation}
            and
            \begin{equation}
                L_v = \frac{1}{\sqrt{2}} (L_x - L_z)
            \end{equation}
            so that
            \begin{equation}
                H = \frac{\omega_0}{\hbar} (L_x L_z + L_z L_x)
            \end{equation}
            With
            \begin{equation}
                L_z = \hbar \mqty(\dmat{1,0,-1})
            \end{equation}
            and
            \begin{equation}
                L_x = \frac{\hbar}{\sqrt{2}} \mqty(0 & 1 & 0 \\ 1 & 0 & 1 \\ 0 & 1 & 0)
            \end{equation}
            we can write
            \begin{equation}
                H = \frac{\omega_0 \hbar}{\sqrt{2}} \mqty(0 & 1 & 0 \\ 1 & 0 & 1 \\ 0 & 1 & 0)
            \end{equation}
            The eigenstates of this matrix are
            \begin{equation}
                \ket{E_1} = \frac{1}{2}\left(-\ket{+1} - \sqrt{2}\ket{0} +\ket{-1} \right) \qc E_1 = \hbar \omega_0
            \end{equation}
            \begin{equation}
                \ket{E_2} = \frac{1}{2} \left(\ket{+1} +\ket{-1} \right) \qc E_2 = 0
            \end{equation}
            \begin{equation}
                \ket{E_3} = \frac{1}{2}\left(-\ket{+1} + \sqrt{2}\ket{0} +\ket{-1} \right) \qc E_1 = -\hbar \omega_0
            \end{equation}
        \end{problem}
    \item[b.] At time $ t = 0 $, the system is in the state:
        \begin{equation}
            \ket{\psi(0)} = \frac{1}{\sqrt{2}} \left[\ket{+1} -\ket{-1} \right]
        \end{equation}
        What is the state vector $\ket{\psi(t)} $ at time $ t $? At $ t $, $ L_z $ is measured; What are the probabilities of the various possible results?
        \begin{problem}
            We can write the vector as a superposition of energy eigenstates:
            \begin{equation}
                \ket{\psi(0)} = - \frac{1}{\sqrt{2}} (\ket{E_1} +\ket{E_3})
            \end{equation}
            so
            \begin{equation}
                \ket{\psi(t)} = \frac{1}{2 \sqrt{2}}\left[ \left( e^{- \imath \omega t} + e^{\imath \omega t} \right)(\ket{+1} -\ket{-1}) + \left( e^{- \imath \omega t} - e^{\imath \omega t} \right) \sqrt{2} \ket{0}  \right]
            \end{equation}
            since
            \begin{equation}
                \ket{E_1(t)} = e^{- \imath \omega \hbar t / \hbar}
            \end{equation}
            and
            \begin{equation}
                \ket{E_3(t)} = e^{\imath \omega \hbar t / \hbar}
            \end{equation}
            If $ L_z $ is measured at time $ t $, the probability to measure $ \pm \hbar $ or $ 0 $ is given by
            \begin{equation}
                \Pr(L_z = + \hbar) = \abs{\bra{+1}\ket{\psi(t)}}^2 = \abs{\frac{1}{2 \sqrt{2}} (e^{- \imath \omega t} + e^{\imath \omega t})}^2 = \frac{1}{2} \cos[2](\omega t)
            \end{equation}
            Similarly,
            \begin{equation}
                \Pr(L_z = 0) = \abs{\bra{0}\ket{\psi(t)}}^2 = \sin[2](\omega t)
            \end{equation}
            and
            \begin{equation}
                \Pr(L_z = - \hbar) = \abs{\bra{-1}\ket{\psi(t)}}^2 = \frac{1}{2} \cos[2](\omega t)
            \end{equation}
        \end{problem}
    \item[c.] Calculate the mean values $ \ev{L_x}(t) $, $ \ev{L_y}(t) $, and $ \ev{L_z}(t) $ at $ t $. What is the motion performed by the vector $ \ev{ \va{L}} $?
        \begin{problem}
            Using
            \begin{equation}
                L_y = \frac{\hbar}{\sqrt{2} \imath} \mqty(0 & 1 & 0 \\ -1 & 0 & 1 \\ 0 & -1 & 0),
            \end{equation}
            I used Mathematica to quickly calculate
            \begin{equation}
                \ev{L_x}_t = \frac{1}{4} e^{-2 \imath \omega t} (e^{2 \imath \omega t} - 1)^2 \hbar
            \end{equation}
            \begin{equation}
                \ev{L_y}_t = \frac{1}{4} \imath e^{- 2 \imath \omega t} (e^{4 \imath \omega t} - 1) \hbar
            \end{equation}
            \begin{equation}
                \ev{L_z}_t = \frac{1}{4} e^{-2 \imath \omega t} (1 + e^{4 \imath \omega t}) \hbar
            \end{equation}
            I'm not sure what the second part of the question is asking.
        \end{problem}
    \item[d.] At $ t $, a measurement of $ L_z^2 $ is performed.
        \subitem(i) Do times exist when only one result is possible?
        \begin{problem}
            As seen by the probabilities above, there are certainly times when only $ \hbar^2 $ is possible, and that happens when $ \sin[2](\omega t) = 0 $ or $ t = n \pi $.
        \end{problem}
        \subitem(ii) Assume that this measurement has yielded the result $ \hbar^2 $. What is the state of the system immediately after the measurement? Indicate, without calculation, its subsequent evolution.
        \begin{problem}
            I'm not quite sure, but I'd imagine that once it's measured, the system immediately after measurement is now in a superposition of $\ket{+1} $ and $\ket{-1} $ but I'm not sure how it evolves after that.
        \end{problem}
\end{itemize}


\section*{Cohen-Tannoudji 7.2: 3D Harmonic Oscillator in Magnetic Field}
Consider a particle of mass $ \mu $, whose Hamiltonian is:
\begin{equation}
    H_0 = \frac{ \va{P}^2}{2 \mu} + \frac{1}{2} \mu \omega_0^2 \va{R}^2
\end{equation}
(an isotropic three-dimensional harmonic oscillator), where $ \omega_0 $ is a given positive constant.
\begin{itemize}
    \item[a.] Find the energy levels of the particle and their degrees of degeneracy. Is it possible to construct a basis of eigenstates common to $ H_0 $, $ \va{L}^2 $, $ L_z $?
        \begin{problem}
            For an isotropic 3D harmonic oscillator, the energy eigenstates can be described as a tensor product of three harmonic oscillators:
            \begin{equation}
                \mathcal{H} = \mathcal{H}_x \otimes \mathcal{H}_y \otimes \mathcal{H}_z
            \end{equation}
            where the energy levels of each particular Hilbert space are given by
            \begin{equation}
                E_i = \left( n_i + \frac{1}{2} \right) \hbar \omega_0
            \end{equation}
            Therefore, the total energy of a given eigenstate will be
            \begin{equation}
                E = \sum_i E_i = \left( n_x + n_y + n_z + \frac{3}{2} \right) \hbar \omega_0
            \end{equation}
            The ground state $\ket{000} $ is not degenerate. However, each excited state is increasingly degenerate. The first excited state has threefold degeneracy: $ E_{100} = E_{010} = E_{001} $, the second excited state has sixfold degeneracy: $ E_{200} = E_{020} = E_{002} = E_{110} = E_{011} = E_{101} $, and the third excited state has tenfold degeneracy.

            Because the potential is spherically symmetric, it is possible to construct a basis of eigenstates common to $ H_0 $, $ \va{L}^2 $, and $ L_z $ because the angular momentum operators commute with the Hamiltonian.
        \end{problem}
    \item[b.] Now assume that the particle, which has a charge $ q $, is placed in a uniform magnetic field $ \va{B} $ parallel to $ Oz $. We set $ \omega_L = - \frac{qB}{2 \mu} $. The Hamiltonian $ H $ of the particle is then, if we chose the gauge $ \va{A} = - \frac{1}{2} \va{r} \cross \va{B} $:
        \begin{equation}
            H = H_0 + H_1(\omega_L)
        \end{equation}
        where $ H_1 $ is the sum of an operator which is linearly dependent on $ \omega_L $ (the paramagnetic term) and an operator which is quadratically dependent on $ \omega_L $ (the diamagnetic term). Show that the new stationary states of the system and their degrees of degeneracy can be determined exactly.
        \begin{problem}
            From class, we showed that the contribution using the Coulomb gauge is
            \begin{equation}
                H_1 = \omega_L L_z + \frac{1}{2} \mu \omega_L^2 \left( X^2 + Y^2 \right)
            \end{equation}
            Therefore, the total Hamiltonian can be written as
            \begin{equation}
                H = \frac{P_x^2 + P_y^2}{2 \mu} + \omega_L L_z + \frac{1}{2} \mu (\omega_0^2 + \omega_L^2)\left( X^2 + Y^2 \right) + \left( \frac{P_z^2}{2 \mu} + \frac{1}{2} \omega_0^2 Z^2 \right)
            \end{equation}
            Note that the Hilbert space $ \mathcal{H}_z $ is essentially unaffected by the introduction of a magnetic field in the $ \vu{z} $-direction. This is a reflection of how a particle traveling parallel to the magnetic field will not experience a force from it. From here, we can use the left and right circularized number operators to rewrite the Hamiltonian as
            \begin{equation}
                H = (N_d + N_g + 1) \hbar \sqrt{\omega_0^2 + \omega_L^2} + (N_d - N_g) \hbar \omega_L + \left( N_z + \frac{1}{2} \right) \hbar \omega_0
            \end{equation}
            If we use the following basis of energy eigenstates:
            \begin{equation}
                \ket{\psi_{n_d n_g n_z}} = \frac{1}{\sqrt{n_d! n_g! n_z !}} (a_d^\dagger)^{n_d} (a_g^\dagger)^{n_g} (a_z^\dagger)^{n_z}\ket{000}
            \end{equation}
            we can find the energy levels of any state to be
            \begin{equation}
                (n_d + n_g + 1) \hbar \sqrt{\omega_0^2 + \omega_L^2} + (n_d - n_g) \hbar \omega_L + \left( n_z + \frac{1}{2} \right) \hbar \omega_0
            \end{equation}
        \end{problem}
    \item[c.] Show that if $ \omega_L $ is much smaller than $ \omega_0 $, the effect of the diamagnetic term is negligible compared to that of the paramagnetic term.
        \begin{problem}
            If $ \omega_L << \omega_0 $, then the term $ \omega_0^2 + \omega_L^2 \sim \omega_0^2 $, so the Hamiltonian will be approximately
            \begin{equation}
                H \approx H_0 + \omega_L L_z
            \end{equation}
        \end{problem}
    \item[d.] We now consider the first excited state of the oscillator, that is, the states whose energies approach $ \frac{5 \hbar \omega_0}{2} $ when $ \omega_L \to 0 $. To first order in $ \frac{\omega_L}{\omega_0} $, what are the energy levels in the presence of the field $ \va{B} $ and their degrees of degeneracy (the Zeeman effect for a three-dimensional harmonic oscillator?) Repeat for the second excited state.
        \begin{problem}
            Expanding the Hamiltonian to first-order, we find
            \begin{equation}
                H = \frac{\hbar \omega_0}{2} \left[ (N_d + N_g + 1) + \frac{\omega_L}{\omega_0} (N_d - N_g) + \left( N_z + \frac{1}{2} \right) \right]
            \end{equation}
            We can rewrite this as
            \begin{equation}
                H = \frac{\hbar \omega_0}{2} \left[ N_d \left( 1 + \frac{\omega_L}{\omega_0} \right) + N_g \left( 1 - \frac{\omega_L}{\omega_0} \right) + N_z + \frac{3}{2} \right]
            \end{equation}
            Assuming $ \omega_L < \omega_0 $, the first excited states will be when $ n_d = 1 $, $ n_g = 1 $, or $ n_z = 1 $. These states will have energy
            \begin{align}
                E_{100} &= \frac{5}{4} \hbar \omega_0 + \frac{1}{2} \hbar \omega_L \\
                E_{010} &= \frac{5}{4} \hbar \omega_0 - \frac{1}{2} \hbar \omega_L \\
                E_{001} &= \frac{5}{4} \hbar \omega_0
            \end{align}
            using the shorthand $ E_{n_d n_g n_z} $

            The second excited states will have the following degeneracy:
            \begin{align}
                E_{200} &= \frac{7}{4} \hbar \omega_0 + \hbar \omega_L \\
                E_{020} &= \frac{7}{4} \hbar \omega_0 - \hbar \omega_L \\
                E_{002} &= \frac{7}{4} \hbar \omega_0 \\
                E_{110} &= \frac{7}{4} \hbar \omega_0 \\
                E_{101} &= \frac{7}{4} \hbar \omega_0 + \frac{1}{2} \hbar \omega_L \\
                E_{011} &= \frac{7}{4} \hbar \omega_0 - \frac{1}{2} \hbar \omega_L
            \end{align}
        \end{problem}
\end{itemize}

\end{document}
