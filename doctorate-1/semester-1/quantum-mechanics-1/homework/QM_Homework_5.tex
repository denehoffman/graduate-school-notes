\documentclass[a4paper,twoside]{article}
% My LaTeX preamble file - by Nathaniel Dene Hoffman
% Credit for much of this goes to Olivier Pieters (https://olivierpieters.be/tags/latex)
% and Gilles Castel (https://castel.dev)
% There are still some things to be done:
% 1. Update math commands using mathtools package (remove ddfrac command and just override)
% 2. Maybe abbreviate \imath somehow?
% 3. Possibly format for margin notes and set new margin sizes
% First, some encoding packages and useful formatting
%--------------------------------------------------------------------------------------------
\usepackage{import}
\usepackage{pdfpages}
\usepackage{transparent}
\usepackage[l2tabu,orthodox]{nag}   % force newer (and safer) LaTeX commands
\usepackage[utf8]{inputenc}         % set character set to support some UTF-8
                                    %   (unicode). Do NOT use this with
                                    %   XeTeX/LuaTeX!
\usepackage[T1]{fontenc}
\usepackage[english]{babel}         % multi-language support
\usepackage{sectsty}                % allow redefinition of section command formatting
\usepackage{tabularx}               % more table options
\usepackage{booktabs}
\usepackage{titling}                % allow redefinition of title formatting
\usepackage{imakeidx}               % create and index of words
\usepackage{xcolor}                 % more colour options
\usepackage{enumitem}               % more list formatting options
\usepackage{tocloft}                % redefine table of contents, new list like objects
\usepackage{subfiles}               % allow for multifile documents

% Next, let's deal with the whitespaces and margins
%--------------------------------------------------------------------------------------------
\usepackage[centering,margin=1in]{geometry}
\setlength{\parindent}{0cm}
\setlength{\parskip}{2ex plus 0.5ex minus 0.2ex} % whitespace between paragraphs

% Redefine \maketitle command with nicer formatting
%--------------------------------------------------------------------------------------------
\pretitle{
  \begin{flushright}         % align text to right
    \fontsize{40}{60}        % set font size and whitespace
    \usefont{OT1}{phv}{b}{n} % change the font to bold (b), normally shaped (n)
                             %   Helvetica (phv)
    \selectfont              % force LaTeX to search for metric in its mapping
                             %   corresponding to the above font size definition
}
\posttitle{
  \par                       % end paragraph
  \end{flushright}           % end right align
  \vskip 0.5em               % add vertical spacing of 0.5em
}
\preauthor{
  \begin{flushright}
    \large                   % font size
    \lineskip 0.5em          % inter line spacing
    \usefont{OT1}{phv}{m}{n}
}
\postauthor{
  \par
  \end{flushright}
}
\predate{
  \begin{flushright}
  \large
  \lineskip 0.5em
  \usefont{OT1}{phv}{m}{n}
}
\postdate{
  \par
  \end{flushright}
}

% Mathematics Packages
\usepackage[Gray,squaren,thinqspace,cdot]{SIunits}      % elegant units
\usepackage{amsmath}                                    % extensive math options
\usepackage{amsfonts}                                   % special math fonts
\usepackage{mathtools}                                  % useful formatting commands
\usepackage{amsthm}                                     % useful commands for building theorem environments
\usepackage{amssymb}                                    % lots of special math symbols
\usepackage{mathrsfs}                                   % fancy scripts letters
\usepackage{cancel}                                     % cancel lines in math
\usepackage{esint}                                      % fancy integral symbols
\usepackage{relsize}                                    % make math things bigger or smaller
%\usepackage{bm}                                         % bold math!
\usepackage{slashed}

\newcommand\ddfrac[2]{\frac{\displaystyle #1}{\displaystyle #2}}    % elegant fraction formatting
\allowdisplaybreaks[1]                                              % allow align environments to break on pages

% Ensure numbering is section-specific
%--------------------------------------------------------------------------------------------
\numberwithin{equation}{section}
\numberwithin{figure}{section}
\numberwithin{table}{section}

% Citations, references, and annotations
%--------------------------------------------------------------------------------------------
\usepackage[small,bf,hang]{caption}        % captions
\usepackage{subcaption}                    % adds subfigure & subcaption
\usepackage{sidecap}                       % adds side captions
\usepackage{hyperref}                      % add hyperlinks to references
\usepackage[noabbrev,nameinlink]{cleveref} % better references than default \ref
\usepackage{autonum}                       % only number referenced equations
\usepackage{url}                           % urls
\usepackage{cite}                          % well formed numeric citations
% format hyperlinks
\colorlet{linkcolour}{black}
\colorlet{urlcolour}{blue}
\hypersetup{colorlinks=true,
            linkcolor=linkcolour,
            citecolor=linkcolour,
            urlcolor=urlcolour}

% Plotting and Figures
%--------------------------------------------------------------------------------------------
\usepackage{tikz}          % advanced vector graphics
\usepackage{pgfplots}      % data plotting
\usepackage{pgfplotstable} % table plotting
\usepackage{placeins}      % display floats in correct sections
\usepackage{graphicx}      % include external graphics
\usepackage{longtable}     % process long tables

% use most recent version of pgfplots
\pgfplotsset{compat=newest}

% Misc.
%--------------------------------------------------------------------------------------------
\usepackage{todonotes}  % add to do notes
\usepackage{epstopdf}   % process eps-images
\usepackage{float}      % floats
\usepackage{stmaryrd}   % some more nice symbols
\usepackage{emptypage}  % suppress page numbers on empty pages
\usepackage{multicol}   % use this for creating pages with multiple columns
\usepackage{etoolbox}   % adds tags for environment endings
\usepackage{tcolorbox}  % pretty colored boxes!


% Custom Commands
%--------------------------------------------------------------------------------------------
\newcommand\hr{\noindent\rule[0.5ex]{\linewidth}{0.5pt}}                % horizontal line
\newcommand\N{\ensuremath{\mathbb{N}}}                                  % blackboard set characters
\newcommand\R{\ensuremath{\mathbb{R}}}
\newcommand\Z{\ensuremath{\mathbb{Z}}}
\newcommand\Q{\ensuremath{\mathbb{Q}}}
%\newcommand\C{\ensuremath{\mathbb{C}}}
\renewcommand{\arraystretch}{1.2}                                       % More space between table rows (could be 1.3)
\newcommand{\Cov}{\mathrm{Cov}}
\newcommand\D{\mathrm{D}}
\newcommand*{\dbar}{\ensuremath{\text{\dj}}}

\newcommand{\incfig}[2][1]{%
    \def\svgwidth{#1\columnwidth}
    \import{./figures/}{#2.pdf_tex}
}

% Custom Environments
%--------------------------------------------------------------------------------------------
\newcommand{\lecture}[3]{\hr\\{\centering{\large\textsc{Lecture #1: #3}}\\#2\\}\hr\markboth{Lecture #1: #3}{\rightmark}}   % command to title lectures
\usepackage{mdframed}
\theoremstyle{plain}
\newmdtheoremenv[nobreak]{theorem}{Theorem}[section]
\newtheorem{corollary}{Corollary}[theorem]
\newtheorem{lemma}[theorem]{Lemma}
\theoremstyle{definition}
\newtheorem*{ex}{Example}
\newmdtheoremenv[nobreak]{definition}{Definition}[section]
\theoremstyle{remark}
\newtheorem*{remark}{Remark}
\newtheorem*{claim}{Claim}
\AtEndEnvironment{ex}{\null\hfill$\diamond$}%
% Note: A proof environment is already provided in the amsthm package
\tcbuselibrary{breakable}
\newenvironment{note}[1]{\begin{tcolorbox}[
    arc=0mm,
    colback=white,
    colframe=white!60!black,
    title=#1,
    fonttitle=\sffamily,
    breakable
]}{\end{tcolorbox}}
\newenvironment{problem}{\begin{tcolorbox}[
    arc=0mm,
    breakable,
    colback=white,
    colframe=black
]}{\end{tcolorbox}}

% Header and Footer
%--------------------------------------------------------------------------------------------
% set header and footer
\usepackage{fancyhdr}                       % header and footer
\pagestyle{fancy}                           % use package
\fancyhf{}
\fancyhead[LE,RO]{\textsl{\rightmark}}      % E for even (left pages), O for odd (right pages)
\fancyfoot[LE,RO]{\thepage}
\fancyfoot[LO,RE]{\textsl{\leftmark}}
\setlength{\headheight}{15pt}


% Physics
%--------------------------------------------------------------------------------------------
\usepackage[arrowdel]{physics}      % all the usual useful physics commands
\usepackage{feyn}                   % for drawing Feynman diagrams
%\usepackage{bohr}                   % for drawing Bohr diagrams
%\usepackage{tikz-feynman}
\usepackage{elements}               % for quickly referencing information of various elements
\usepackage{tensor}                 % for writing tensors and chemical symbols

% Finishing touches
%--------------------------------------------------------------------------------------------
\author{Nathaniel D. Hoffman}

\title{33-755 Homework 5}
\date{\today}
\begin{document}
\maketitle

\section*{1. Bell States and Teleportation}

Suppose Alice and Bob share the fully-entangled state $ \ket{B^1} $, and Alice is to teleport an unknown state $ \ket{\psi} $ to Bob by measuring her half (b) of the entangled state along with $ \ket{\psi} $.

\begin{itemize}
    \item[(a)] Assume that $ \ket{\psi} = \alpha \ket{0} + \beta \ket{1} \in \mathcal{H}_{a} $. Express the state $ \ket{\Psi} \equiv \ket{\psi} \otimes \ket{B^1} \in \mathcal{H}_{a} \otimes \mathcal{H}_{b} \otimes \mathcal{H}_{c} $, using the tensor product basis.
        \begin{tcolorbox}[breakable]
            \begin{align}
                \ket{\Psi} &= \ket{\psi} \otimes \ket{B^1}\\
                &= \alpha \ket{0} + \beta \ket{1} \otimes \frac{1}{\sqrt{2}} ( \ket{01} + \ket{10})\\
                &= \frac{1}{\sqrt{2}} [\alpha \ket{001} + \alpha \ket{010} + \beta \ket{101} + \beta \ket{110}]
            \end{align}
        \end{tcolorbox}
    \item[(b)] The same state can be expressed in the basis of Bell states $ \{B^k\} $ on $ \mathcal{H}_{a} \otimes \mathcal{H}_{b} $ as
        \begin{equation}
            \ket{\Psi} = \frac{1}{2} \sum_{k=0}^{3} B^k \otimes V_k \ket{\psi} 
        \end{equation}
        where $ \{V_k\} $ is a set of unitary maps from $ \mathcal{H}_{a} $ to $ \mathcal{H}_{c} $ (i.e. $ V_k \ket{\psi} \in \mathcal{H}_{c} $). Express $ V_k \ket{\psi} $ in the basis $ \{\ket{0}, \ket{1}\} $ of $ \mathcal{H}_{c} $, for $ k = 0,1,2,3 $.
        \begin{tcolorbox}[breakable]
            Note that there are two ways to express $ \ket{\Psi} $ on a basis of the Bell states:
            \begin{equation}
                \ket{\Psi} = (\ket{B^0} \otimes \beta \ket{0} + \alpha \ket{1}) + (\ket{B^1} \otimes \alpha \ket{0} + \beta \ket{1})
            \end{equation}
            and a second one using the antisymmetric states:
            \begin{equation}
                \ket{\Psi} = (\ket{B^2} \otimes - \beta \ket{0} + \alpha \ket{1}) + (\ket{B^3} \otimes \alpha \ket{0} - \beta \ket{1})
            \end{equation}
            Adding these together will give us twice the total state $ \ket{\Psi} $, which is where the $ \frac{1}{2} $ in the formula comes from:
            \begin{equation}
                \begin{split}
                \ket{\Psi} =& \frac{1}{2} \left[ (\ket{B^0} \otimes \beta \ket{0} + \alpha \ket{1}) + (\ket{B^1} \otimes \alpha \ket{0} + \beta \ket{1})\right.\\
                &+ \left.(\ket{B^2} \otimes - \beta \ket{0} + \alpha \ket{1}) + (\ket{B^3} \otimes \alpha \ket{0} - \beta \ket{1}) \right]
                \end{split}
            \end{equation}
            By equating the right halves of the product spaces to $ V_k \ket{\psi} $, we find:
            \begin{align}
                V_0 \ket{\psi} = V_0 (\alpha \ket{0} + \beta \ket{1}) &= \beta \ket{0} + \alpha \ket{1} \\
                & \implies V_0 = 
                \begin{bmatrix}
                    0&1\\
                    1&0
                \end{bmatrix}\\
                V_1 \ket{\psi} &= \alpha \ket{0} + \beta \ket{1} \\
                & \implies V_1 = 
                \begin{bmatrix}
                    1&0\\
                    0&1
                \end{bmatrix}\\
                V_2 \ket{\psi} &= - \beta \ket{0} + \alpha \ket{1} \\
                & \implies V_2 = 
                \begin{bmatrix}
                    0&-1\\
                    1&0
                \end{bmatrix}\\
                V_3 \ket{\psi} &= \alpha \ket{0} - \beta \ket{1} \\
                & \implies V_3 = 
                \begin{bmatrix}
                    1&0\\
                    0&-1
                \end{bmatrix}\\
            \end{align}
        \end{tcolorbox}
    \item[(c)] Alice measures the combination of $ \ket{\psi} $ and $ b $, in the basis of Bell states on $ \mathcal{H}_{a} \otimes \mathcal{H}_{b} $ mentioned above, yielding a specific outcome $ k (0 \leq k \leq 3 ) $. She then e-mails the result $ k $ to Bob, who must apply the unitary operator $ U_k $ to his half (c) of the original entangled state $ \ket{B^1} $ in order to complete the teleportation process. What are these operators $ U_k $?
        \begin{tcolorbox}[breakable]
            When Alice measures her state and sends the result, she, in effect, puts the system into the state described above. Bob's particle is now in the state $ V_k \ket{\psi} $, so he needs to perform the inverse of $ V_k $ to retrieve Alice's message state $ \ket{\psi} $. Because the matrices are unitary, the inverse is just the transpose:
            \begin{align}
                U_0 &= 
                \begin{bmatrix}
                    0&1\\
                    1&0
                \end{bmatrix}\\
                U_1 &= 
                \begin{bmatrix}
                    1&0\\
                    0&1
                \end{bmatrix}\\
                U_2 &= 
                \begin{bmatrix}
                    0&1\\
                    -1&0
                \end{bmatrix}\\
                U_3 &= 
                \begin{bmatrix}
                    1&0\\
                    0&-1
                \end{bmatrix}
            \end{align}
        \end{tcolorbox}
    \item[(d)] Check your result by transmitting the basis state $ \ket{\psi} = \ket{0} $. Show that whatever bell state $ k $ Alice measures on her pair of bits in $ \mathcal{H}_{a} \otimes \mathcal{H}_{b} $, Bob will obtain $ \ket{0} \in \mathcal{H}_{c} $ after applying $ U_k $ to $ \ket{c} $.
        \begin{tcolorbox}[breakable]
            Alice will be sent a particle in a superposition of $ \ket{0} $ and $ \ket{1} $. The total state will therefore be
            \begin{equation}
                \ket{\Psi} = \ket{0} \otimes \frac{1}{\sqrt{2}} (\ket{01} + \ket{10}) = \frac{1}{\sqrt{2}} (\ket{001} + \ket{010})
            \end{equation}
            When she measures the first two particles (the ones she actually has) she will get, with equal likelihood, $ \ket{00} $ or $ \ket{01} $. We can rewrite this state in terms of the Bell state expansion above:
            \begin{equation}
                \ket{\Psi} = \frac{1}{2} [\ket{B^0} \otimes \ket{1} + \ket{B^1} \otimes \ket{0} + \ket{B^2} \otimes \ket{1} + \ket{B^3} \otimes \ket{0}]
            \end{equation}
            Therefore, if she measures $ \ket{00} $, she can tell Bob either $ k = 0 $ or $ k = 2 $, since she could be in either of those Bell states. Regardless of her choice, Bob will be able to perform the operator $ U_0 $ or $ U_2 $ on his particle, which must now be in the $ \ket{1} $ state, and this operation will give him the original $ \ket{0} $ state. Similarly, if she measures $ \ket{01} $, she can tell him $ k = 1 $ or $ k = 3 $ and he will be able to use those operators on his particle, which must be in the $ \ket{0} $ state, to get $ \ket{0} $.
        \end{tcolorbox}

\end{itemize}
\section*{2. Infinite Dimensional Commutation Relation}
Let two operators A and B satisfy the commutation relation $ [B,A] = \imath I $. Explain why the Hilbert space must be infinite dimensional.
\begin{tcolorbox}[breakable]
    Suppose these operators acted on a finite Hilbert space of dimension $ n $. We would then know that
    \begin{equation}
        \Tr (AB) = \Tr (BA)
    \end{equation}
    so
    \begin{equation}
        \Tr ([B,A]) = \Tr (BA-AB) = \Tr (BA) - \Tr (AB) = 0
    \end{equation}
    However, $ \Tr (\imath I) = n\imath $ for an $ n $-dimensional space, so the space cannot have a finite dimension. In an infinite dimensional space, the operators need not have a finite trace, and therefore the trace of the commutator is meaningless ($ \Tr (AB) \to \infty \implies \Tr ([B,A]) = \infty - \infty $ is undefined), so it is perfectly reasonable to have $ [B,A] = \imath I $.
\end{tcolorbox}


\end{document}
