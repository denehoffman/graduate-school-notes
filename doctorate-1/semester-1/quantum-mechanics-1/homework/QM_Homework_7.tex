\documentclass[a4paper,twoside]{article}
% My LaTeX preamble file - by Nathaniel Dene Hoffman
% Credit for much of this goes to Olivier Pieters (https://olivierpieters.be/tags/latex)
% and Gilles Castel (https://castel.dev)
% There are still some things to be done:
% 1. Update math commands using mathtools package (remove ddfrac command and just override)
% 2. Maybe abbreviate \imath somehow?
% 3. Possibly format for margin notes and set new margin sizes
% First, some encoding packages and usefull formatting
%--------------------------------------------------------------------------------------------
\usepackage[l2tabu,orthodox]{nag}   % force newer (and safer) LaTeX commands
\usepackage[utf8]{inputenc}         % set character set to support some UTF-8
                                    %   (unicode). Do NOT use this with
                                    %   XeTeX/LuaTeX!
\usepackage[T1]{fontenc}
\usepackage[english]{babel}         % multi-language support
\usepackage{sectsty}                % allow redefinition of section command formatting
\usepackage{tabularx}               % more table options
\usepackage{booktabs}
\usepackage{titling}                % allow redefinition of title formatting
\usepackage{imakeidx}               % create and index of words
\usepackage{xcolor}                 % more colour options
\usepackage{enumitem}               % more list formatting options
\usepackage{tocloft}                % redefine table of contents, new list like objects
\usepackage{subfiles}               % allow for multifile documents

% Next, let's deal with the whitespaces and margins
%--------------------------------------------------------------------------------------------
\usepackage[centering,margin=1in]{geometry}
\setlength{\parindent}{0cm}
\setlength{\parskip}{2ex plus 0.5ex minus 0.2ex} % whitespace between paragraphs

% Redefine \maketitle command with nicer formatting
%--------------------------------------------------------------------------------------------
\pretitle{
  \begin{flushright}         % align text to right
    \fontsize{40}{60}        % set font size and whitespace
    \usefont{OT1}{phv}{b}{n} % change the font to bold (b), normally shaped (n)
                             %   Helvetica (phv)
    \selectfont              % force LaTeX to search for metric in its mapping
                             %   corresponding to the above font size definition
}
\posttitle{
  \par                       % end paragraph
  \end{flushright}           % end right align
  \vskip 0.5em               % add vertical spacing of 0.5em
}
\preauthor{
  \begin{flushright}
    \large                   % font size
    \lineskip 0.5em          % inter line spacing
    \usefont{OT1}{phv}{m}{n}
}
\postauthor{
  \par
  \end{flushright}
}
\predate{
  \begin{flushright}
  \large
  \lineskip 0.5em
  \usefont{OT1}{phv}{m}{n}
}
\postdate{
  \par
  \end{flushright}
}

% Mathematics Packages
\usepackage[Gray,squaren,thinqspace,cdot]{SIunits}      % elegant units
\usepackage{amsmath}                                    % extensive math options
\usepackage{amsfonts}                                   % special math fonts
\usepackage{mathtools}                                  % useful formatting commands
\usepackage{amsthm}                                     % useful commands for building theorem environments
\usepackage{amssymb}                                    % lots of special math symbols
\usepackage{mathrsfs}                                   % fancy scripts letters
\usepackage{cancel}                                     % cancel lines in math
\usepackage{esint}                                      % fancy integral symbols
\usepackage{relsize}                                    % make math things bigger or smaller
\usepackage{bm}                                         % bold math!

\newcommand\ddfrac[2]{\frac{\displaystyle #1}{\displaystyle #2}}    % elegant fraction formatting
\allowdisplaybreaks[1]                                              % allow align environments to break on pages

% Ensure numbering is section-specific
%--------------------------------------------------------------------------------------------
\numberwithin{equation}{section}
\numberwithin{figure}{section}
\numberwithin{table}{section}

% Citations, references, and annotations
%--------------------------------------------------------------------------------------------
\usepackage[small,bf,hang]{caption}        % captions
\usepackage{subcaption}                    % adds subfigure & subcaption
\usepackage{sidecap}                       % adds side captions
\usepackage{hyperref}                      % add hyperlinks to references
\usepackage[noabbrev,nameinlink]{cleveref} % better references than default \ref
\usepackage{autonum}                       % only number referenced equations
\usepackage{url}                           % urls
\usepackage{cite}                          % well formed numeric citations
% format hyperlinks
\colorlet{linkcolour}{black}
\colorlet{urlcolour}{blue}
\hypersetup{colorlinks=true,
            linkcolor=linkcolour,
            citecolor=linkcolour,
            urlcolor=urlcolour}

% Plotting and Figures
%--------------------------------------------------------------------------------------------
\usepackage{tikz}          % advanced vector graphics
\usepackage{pgfplots}      % data plotting
\usepackage{pgfplotstable} % table plotting
\usepackage{placeins}      % display floats in correct sections
\usepackage{graphicx}      % include external graphics
\usepackage{longtable}     % process long tables

% use most recent version of pgfplots
\pgfplotsset{compat=newest}

% Misc.
%--------------------------------------------------------------------------------------------
\usepackage{todonotes}  % add to do notes
\usepackage{epstopdf}   % process eps-images
\usepackage{float}      % floats
\usepackage{stmaryrd}   % some more nice symbols
\usepackage{emptypage}  % suppress page numbers on empty pages
\usepackage{multicol}   % use this for creating pages with multiple columns
\usepackage{etoolbox}   % adds tags for environment endings
\usepackage{tcolorbox}  % pretty colored boxes!


% Custom Commands
%--------------------------------------------------------------------------------------------
\newcommand\hr{\noindent\rule[0.5ex]{\linewidth}{0.5pt}}                % horizontal line
\newcommand\N{\ensuremath{\mathbb{N}}}                                  % blackboard set characters
\newcommand\R{\ensuremath{\mathbb{R}}}
\newcommand\Z{\ensuremath{\mathbb{Z}}}
\newcommand\Q{\ensuremath{\mathbb{Q}}}
\newcommand\C{\ensuremath{\mathbb{C}}}
\renewcommand{\arraystretch}{1.2}                                       % More space between table rows (could be 1.3)
\newcommand{\Cov}{\mathrm{Cov}}
\newcommand*{\dbar}{\ensuremath{\text{\dj}}}
% Custom Environments
%--------------------------------------------------------------------------------------------
\newcommand{\lecture}[3]{\hr\\{\centering{\large\textsc{Lecture #1: #3}}\\#2\\}\hr\markboth{Lecture #1: #3}{\rightmark}}   % command to title lectures
\usepackage{mdframed}
\theoremstyle{plain}
\newmdtheoremenv[nobreak]{theorem}{Theorem}[section]
\newtheorem{corollary}{Corollary}[theorem]
\newtheorem{lemma}[theorem]{Lemma}
\theoremstyle{definition}
\newtheorem*{ex}{Example}
\newmdtheoremenv[nobreak]{definition}{Definition}[section]
\theoremstyle{remark}
\newtheorem*{remark}{Remark}
\AtEndEnvironment{ex}{\null\hfill$\diamond$}%
% Note: A proof environment is already provided in the amsthm package
\tcbuselibrary{breakable}
\newenvironment{note}[1]{\begin{tcolorbox}[
    arc=0mm,
    colback=white,
    colframe=white!60!black,
    title=#1,
    fonttitle=\sffamily,
    breakable
]}{\end{tcolorbox}}
\newenvironment{problem}{\begin{tcolorbox}[
    arc=0mm,
    breakable,
    colback=white,
    colframe=black
]}{\end{tcolorbox}}

% Header and Footer
%--------------------------------------------------------------------------------------------
% set header and footer
\usepackage{fancyhdr}                       % header and footer
\pagestyle{fancy}                           % use package
\fancyhf{}
\fancyhead[LE,RO]{\textsl{\rightmark}}      % E for even (left pages), O for odd (right pages)
\fancyfoot[LE,RO]{\thepage}
\fancyfoot[LO,RE]{\textsl{\leftmark}}
\setlength{\headheight}{15pt}


% Physics
%--------------------------------------------------------------------------------------------
\usepackage[arrowdel]{physics}      % all the usual useful physics commands
%\usepackage{feyn}                   % for drawing Feynman diagrams
%\usepackage{bohr}                   % for drawing Bohr diagrams
\usepackage{elements}               % for quickly referencing information of various elements
\usepackage{tensor}                 % for writing tensors and chemical symbols

% Finishing touches
%--------------------------------------------------------------------------------------------
\author{Nathaniel D. Hoffman}

\title{33-755 Homework 7}
\date{October 25, 2019}
\begin{document}
\maketitle

\section*{5. Well Consisting of Two Delta Functions}
Consider a particle of mass $ m $ whose potential energy is
\begin{equation}
    V(x) = - \alpha \delta(x) - \alpha \delta(x - l) \quad \alpha > 0
\end{equation}
where $ l $ is a constant length.
\begin{itemize}
    \item[a)] Calculate the bound states of the particle, setting $ E = - \frac{\hbar^2 \rho^2}{2m} $. Show that the possible energies are given by the relation
    \begin{equation}
        e^{- \rho l} = \pm \left( 1 - \frac{2 \rho}{\mu} \right)
    \end{equation}
    where $\mu$ is defined by $ \mu = \frac{2m \alpha}{\hbar^2} $. Give a graphic solution of this equation.
    \begin{problem}
        To simplify this problem, we can translate the potential and rescale it so that the $\delta$-functions are centered around $ l $ and $ -l $. This transformation is $ y = 2(x - \frac{l}{2}) $. Now that we have symmetry about zero, we can write the even state as
        \begin{equation}
            \varphi_+(y) =
            \begin{cases}
                A e^{\rho y} & y < -l\\
                B(e^{\rho y} + e^{-rho y}) & -l < y < l\\
                A e^{- \rho y} & l < y
            \end{cases}
        \end{equation}
        and the odd states as
        \begin{equation}
            \varphi_-(y) =
            \begin{cases}
                A e^{\rho y} & y < -l\\
                B(e^{\rho y} - e^{-rho y}) & -l < y < l\\
                A e^{- \rho y} & l < y
            \end{cases}
        \end{equation}
        Now we apply the boundary conditions. At the $\delta$-functions, we must maintain a continuous wave function, so $ \varphi(y^-) = \varphi(y^+) $. However, the derivatives will have a jump, since
        \begin{equation}
            \varphi'(y^+) - \varphi'(y^-) = \lim_{\epsilon \to 0} \int_{y-\epsilon}^{y+ \epsilon} \frac{2m}{\hbar} V(y') \psi(y') \dd{y'}
        \end{equation}
        Let's first examine the even solutions at $ y = l $. By the continuity condition,
        \begin{equation}
            Ae^{- \rho l} = B(e^{\rho l} + e^{- \rho l}) \implies A = B(e^{2 \rho l} + 1)
        \end{equation}
        Next, the jump condition for the derivative gives us
        \begin{equation}
            \varphi'_+(l^+)- \varphi'_+(l^-) = - \mu \varphi(l)
        \end{equation}
        so
        \begin{align}
            - \rho A e^{- \rho l} - \rho B(e^{\rho l} - e^{- \rho l}) &= - \mu A e^{- \rho l}\\
            A + B(e^{2 \rho l} - 1) &= \frac{\mu}{\rho} A\\
            B(e^{2 \rho l} - 1) &= A(\frac{\mu}{\rho} - 1)\\
        \end{align}
        from the first continuity condition, we can make the whole equation in terms of $ B $:
        \begin{align}
            B(e^{2 \rho l} - 1) &= B(e^{2 \rho l} + 1)(\frac{\mu}{\rho} -1)\\
            e^{2 \rho l} - 1 &= e^{2 \rho l}(\frac{\mu}{\rho} - 1) + \frac{\mu}{\rho} - 1\\
            e^{2 \rho l} &= e^{2 \rho l} (\frac{\mu}{\rho} - 1) + \frac{\mu}{\rho} \\
            e^{2 \rho l} &= \frac{\mu}{\rho} \left( e^{2 \rho l} + 1 \right) - e^{2 \rho l}\\
            2 e^{2 \rho l} &= \frac{\mu}{\rho} \left( e^{2 \rho l} + 1 \right)\\
            \frac{2 \rho}{\mu} e^{2 \rho l} &= e^{2 \rho l} + 1\\
            \frac{2 \rho}{\mu} e^{\rho l} &= e^{\rho l} + e^{- \rho l}\\
            e^{- \rho l} &= e^{\rho l} \left( \frac{2 \rho}{\mu} - 1 \right)
            \implies e^{- 2 \rho l} &= -\left( 1 - \frac{2 \rho}{\mu} \right)
        \end{align}
        when we translate and scale our coordinate system, $ 2l \to l $, so we find that, for even solutions,
        \begin{equation}
            e^{- \rho l} = -\left( 1 - \frac{2 \rho}{\mu} \right)
        \end{equation}
        For odd solutions, we follow the same procedure:
        \begin{align}
            A &= B\left( e^{2 \rho l} - 1 \right)\\
            \left( \frac{\mu}{\rho} - 1 \right)A &= B(e^{2 \rho l} + 1)\\
            &= B(e^{2 \rho l} - 1)\left( \frac{\mu}{\rho} - 1 \right)\\
            e^{2 \rho l} &= e^{2 \rho l} \frac{\mu}{\rho} - \frac{\mu}{\rho} - e^{2 \rho l}\\
            \implies e^{-2 \rho l} = +\left( 1 - \frac{2 \rho}{\mu} \right)
        \end{align}
        When rescaled, this says that for odd solutions,
        \begin{equation}
            e^{- \rho l} = +\left( 1 - \frac{2 \rho}{\mu} \right)
        \end{equation}
        giving us both parts of the desired energy relation given in the problem.
        Graphically, if we look at the even solutions, we see that there should always be exactly one solution. See the attached sheet for graphs of both the even and odd solutions.

        However, for odd solutions, since we don't count the solution at $ \rho = 0 $ (this is the trivial solution to the Schr\"odinger equation), it is possible to get a single bound state solution. However, if the slope of $ 1 - \frac{2 \rho}{\mu} $ is too negative, there will be no solution. To see how negative it has to be, we look at the slope of $ e^{- \rho l} $ at $ \rho = 0 $, which is $ - l $. The slope of the other side of the equation is always $ -\frac{2}{\mu} $. If $ - \frac{2}{\mu} > - l $, or $ \frac{2}{\mu} < l $, there will be no odd solution.
    \end{problem}
    \subitem[(i)] \textit{Ground State}. Show that this state is even (invariant with respect to reflection about the point $ x = l/2 $), and that its energy $ E_S $ is less than the energy $ -E_L $ introduced in problem 3. Interpret this result physically. Represent graphically the corresponding wave function.
    \begin{problem}
        First, we substitute our definition for $ \mu $ back into the energy relation:
        \begin{align}
            e^{- \rho l} &= \frac{2 \rho}{\mu} - 1\\
            &= \frac{\rho \hbar^2}{m \alpha} - 1
        \end{align}
        Next, we solve for $ \alpha $:
        \begin{equation}
            \alpha = \frac{\rho\hbar^2}{m} \frac{1}{e^{- \rho l} + 1}
        \end{equation}
        Now we insert this $\alpha$ into the definition of $ - E_L $ from problem 3:
        \begin{align}
            - E_L &= - \frac{m \alpha^2}{2 \hbar^2}\\
            - E_L &= - \frac{m}{2 \hbar^2} \frac{\rho^2 \hbar^4}{m^2}\left( \frac{1}{e^{- \rho l} + 1} \right)^2\\
            &= - \frac{\rho^2 \hbar^2}{2 m} \left( \frac{1}{e^{- \rho l} +1} \right)^2\\
            - E_L &= E_S \left( \frac{1}{e^{- \rho l} +1} \right)^2
        \end{align}
        Since the fraction multiplying $ E_S $ is probably less than unity, it would appear that the energy relation is backwards, and $ E_S > -E_L $, so I'm not quite sure what I did wrong here. Assuming this is correct, this means that the ground state energy is smaller than the energy in the ground state of a single $\delta$-function barrier, I think. Graphically, the wave function looks like (i) on the attached sheet.

    \end{problem}
    \subitem[(ii)] \textit{Excited State}. Show that, when $ l $ is greater than a value which you are to specify, there exists an odd excited state, of energy $ E_A $ greater than $ -E_L $. Find the corresponding wave function.
    \begin{problem}
        Following the same method as above, I find that $ - E_L = E_S \left( \frac{1}{1 - e^{- \rho l}} \right)^2 $, and again, I can't see how this could give the conclusion given in the problem. From above, there only exists an odd excited state when $ l > \frac{2}{\mu}  $. For a graph of the corresponding wave function, see (ii) on the attached sheet.
    \end{problem}
    \subitem[(iii)] Explain how the preceding calculations enable us to construct a model which represents an ionized diatomic molecule ($ H^+_2 $, for example) whose nuclei are separated by a distance $ l $. How do the energies of the two levels vary with respect to $ l $? What happens at the limit where $ l \to 0 $ and at the limit where $ l \to \infty $? If the repulsion of the two nuclei is taken into account, what is the total energy of the system? Show that the curve which gives the variation with respect to $ l $ of the energies thus obtained enables us to predict in certain cases the existence of bound states of $ H_2^+ $, and to determine the value of $ l $ at equilibrium. In this way we obtain a very elementary model of the chemical bond.
    \begin{problem}
        In electrostatics, the positive ions act like $\delta$-potentials to the negative electron, since the potential grows as $ 1/r $ where $ r $ is the distance between the ions. Let's examine the energy relation as $ l \to 0 $:
        \begin{equation}
            1 = \pm\left(1- \frac{2 \rho}{\mu}\right)
        \end{equation}
        For the even solutions:
        \begin{align}
            1 &= \frac{2 \rho}{\mu} - 1\\
            \frac{2 \rho}{\mu} &= 2\\
            \rho &= \mu\\
            \implies \rho^2 &= \mu^2\\
            E_S \frac{-2m}{\hbar^2} &= \frac{4 m^2}{\hbar^2} \alpha^2\\
            - E_S &= 2 m \alpha^2\\
            &= 4 E_L \hbar^2
        \end{align}
        I probably did something wrong here since the units don't seem correct, but I can't find it. The odd solution doesn't exist as $ l \to 0 $ because it only exists when $ l > \frac{2}{\mu} $, and $ \mu $ is positive.

        For the other limit, as $ l \to \infty $, the even solution becomes
        \begin{align}
            0 &= \frac{2 \rho}{\mu} - 1\\
            2 \rho &= \mu\\
            \rho &= \frac{2 m \alpha}{\hbar^2 }\\
            \alpha = \frac{\rho \hbar^2}{2m}\\
            -E_L &= - \frac{m}{2 \hbar^2} \frac{\rho^2 \hbar^4}{4 m^2}\\
            &= \frac{\rho^2 \hbar^2}{2 m} = -E_S
        \end{align}
        The odd solution will be
        \begin{equation}
            0 = 1 - \frac{2 \rho}{\mu} \implies -E_L = -E_A
        \end{equation}
        so the two solutions will have the same energy and be equivalent to a bound state of a single $\delta$-function. This is basically the same as having a $\delta$-function with another at infinity, so it makes sense that the contribution of the second one vanishes.

        If we factor in the repulsion force of the two nuclei, we get that there is an additional energy $ E(x) = -\frac{1}{4 \pi \epsilon_0} \left( \frac{e_-^2}{\abs{x}} + \frac{e_-^2}{\abs{l-x}} \right) $. Let's imagine that the electron spends most of its time in the middle of the two protons. This means the repulsion energy is $E = - \frac{1}{4 \pi \epsilon_0} \left( \frac{4 e_-^2}{l} \right) = \frac{e_-^2}{\pi \epsilon_0 l} $. At equilibrium, this exactly cancels the binding energy, but I don't see how I can relate them in a more meaningful way which would allow for a simple calculation of the equilibrium length given the proper values of the variables, but I'm sure it's somehow possible or the question wouldn't be asked and we would have no idea how long chemical bonds are.
    \end{problem}
\item[b)] Calculate the reflection and transmission coefficients of the system of two delta function barriers. Study their variations with respect to $ l $. Do the resonances thus obtained occur when $ l $ is an integral multiple of the de Broglie wavelength of the particle? Why?
\begin{problem}
    Now we have an incoming wave. In general,
    \begin{equation}
        \psi(x) =
        \begin{cases}
            A e^{\imath kx} + B e^{-\imath kx} & x<0\\
            C e^{\imath kx} + D e^{-\imath kx} & 0 < x < L\\
            F e^{\imath kx} & L < x
        \end{cases}
    \end{equation}
    Next, at the $ x = 0 $ boundary, we have
    \begin{equation}
        A + B = C + D
    \end{equation}
    and
    \begin{equation}
        \imath k(C - D - A + B) = -\mu(A + B)
    \end{equation}
    At $ x = l $, we have
    \begin{equation}
        C e^{\imath k l} + D e^{-\imath k l} = F e^{\imath k l}
    \end{equation}
    and
    \begin{equation}
        \imath k(F e^{\imath kl} - C e^{\imath kl} + D e^{-\imath kl}) = -\mu F e^{\imath kl}
    \end{equation}
    We have four equations and five unknowns. We want the transmission and reflection coefficients $ T = \frac{\abs{F}^2}{\abs{A}^2} $ and $ R = 1-T $, so we can simply set $ A = 1 $ and solve the system. I only care about $ F $ in this scenario, since I can find $ R $ from $ T $:
    \begin{equation}
        F = \frac{4 k^2}{4 k^2 - 4\imath k \mu - \mu^2 + \mu^2e^{2\imath k l}}
    \end{equation}
    Since $ A = 1 $, $ T = \abs{F}^2 $:
    \begin{equation}
        T = \frac{8 k^4}{8 k^4 + 4 k^2 \mu^2 + \mu^4 - 4 k \mu^3 \sin(2kl) + \mu^2 \cos(2kl)(4 k^2 - \mu^2)}
    \end{equation}
    and
    \begin{equation}
        R = 1-T = \frac{4 k^2 \mu^2 + \mu^4 - 4 k \mu^3 \sin(2kl) + \mu^2 \cos(2kl)(4 k^2 - \mu^2)}{8 k^4 + 4 k^2 \mu^2 + \mu^4 - 4 k \mu^3 \sin(2kl) + \mu^2 \cos(2kl)(4 k^2 - \mu^2)}
    \end{equation}

    The only things that vary with $ l $ here are the sine and cosine in the denominator. I guess these can cause resonances at a certain proportion of $ l $ to $ k $, since they will be maximized and minimized when $ k = \frac{n \pi}{4 l} $ for integer $ n $. I can write out the energy now as
    \begin{equation}
        E(k) = \frac{\hbar^2}{2m} \frac{n^2\pi^2}{16 l^2} = \frac{h^2 n^2}{128 l^2}
    \end{equation}
    The energy of a particle is related to the de Broglie wavelength by
    \begin{equation}
        E = \frac{h}{c\lambda}
    \end{equation}
    so I guess there is some proportionality between $ l^2 $ and $ \lambda $. Physically, I'm not sure why, but I would guess it has something to do with the fact that when the de Broglie wavelength is a multiple of $ l $, there can exist a standing wave between the two $\delta$-functions, so reflections would be minimized and transmissions would be maximized (I think).
\end{problem}
\end{itemize}

\section*{10. Find $ \bra{x} XP \ket{\psi} $ and $ \bra{x} PX \ket{\psi} $ in terms of $ \psi(x) $.}
Use the relation $ \bra{x}\ket{p} = (2 \pi \hbar)^{-1/2} e^{\imath p x/ \hbar} $. Can these results be found directly by using the fact that in the $ \{\ket{x}\} $ representation, $ P $ acts like $ \frac{\hbar}{\imath} \dv{x} $?
\begin{problem}
    Using the relation given in the problem, we insert the identity in momentum space:

    \begin{align}
        \bra{x} XP \ket{\psi} &= \int \dd{p} \bra{x} XP \ket{p} \psi(p)\\
        &= x\int \dd{p} p \bra{x}\ket{p} \psi(p)\\
        &= x \int \dd{p} (2 \pi \hbar)^{-1/2} e^{\imath px/\hbar} p \psi(p)\\
        &= \imath\hbar x \psi'(x)
    \end{align}
    since this is just the Fourier transform of $ p \psi(p) $ scaled by $ \hbar $.

    \begin{align}
        \bra{x} PX \ket{\psi} &= \bra{x} XP - \imath\hbar\ket{\psi}\\
        &= \imath\hbar x\psi'(x) - \imath\hbar\psi(x)
    \end{align}
    
    \begin{align}
        \bra{x} XP \ket{\psi} &= \int \dd{x'} \bra{x} XP \ket{x'}\bra{x'}\ket{\psi}\\
        &= \int \dd{x'} x \bra{x} P \ket{x'} \psi(x')\\
        &= \int \dd{x'} \frac{\hbar}{\imath} \pdv{x'} \bra{x}\ket{x'} \psi(x')\\
        &= \int \dd{x'} \frac{\hbar}{\imath} \delta'(x-x') \psi(x')\\
        &= \imath\hbar x \psi'(x)
    \end{align}
    \begin{align}
        \bra{x} PX \ket{\psi} &= \int \dd{x'} \bra{x} PX \ket{x'} \bra{x'}\ket{\psi}\\
        &= \int \dd{x'} \bra{x} P \ket{x'} x' \psi(x')\\
        &= \int \dd{x'} \frac{\hbar}{\imath} \delta'(x-x') x' \psi(x')\\
        &= \imath\hbar \int \dd{x'} \delta(x-x') \pdv{x'}(x'\psi(x'))\\
        &= \imath\hbar \int \dd{x'} \delta(x-x') (x' \psi'(x') + \psi(x'))\\
        &= \imath\hbar (x\psi(x) + \psi(x))
    \end{align}
    I'm not sure which of the $ PX $ equations I messed up, but I'm missing a minus sign between them.
\end{problem}

\end{document}
