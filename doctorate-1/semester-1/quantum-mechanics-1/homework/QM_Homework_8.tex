\documentclass[a4paper,twoside]{article}
% My LaTeX preamble file - by Nathaniel Dene Hoffman
% Credit for much of this goes to Olivier Pieters (https://olivierpieters.be/tags/latex)
% and Gilles Castel (https://castel.dev)
% There are still some things to be done:
% 1. Update math commands using mathtools package (remove ddfrac command and just override)
% 2. Maybe abbreviate \imath somehow?
% 3. Possibly format for margin notes and set new margin sizes
% First, some encoding packages and usefull formatting
%--------------------------------------------------------------------------------------------
\usepackage[l2tabu,orthodox]{nag}   % force newer (and safer) LaTeX commands
\usepackage[utf8]{inputenc}         % set character set to support some UTF-8
                                    %   (unicode). Do NOT use this with
                                    %   XeTeX/LuaTeX!
\usepackage[T1]{fontenc}
\usepackage[english]{babel}         % multi-language support
\usepackage{sectsty}                % allow redefinition of section command formatting
\usepackage{tabularx}               % more table options
\usepackage{booktabs}
\usepackage{titling}                % allow redefinition of title formatting
\usepackage{imakeidx}               % create and index of words
\usepackage{xcolor}                 % more colour options
\usepackage{enumitem}               % more list formatting options
\usepackage{tocloft}                % redefine table of contents, new list like objects
\usepackage{subfiles}               % allow for multifile documents

% Next, let's deal with the whitespaces and margins
%--------------------------------------------------------------------------------------------
\usepackage[centering,margin=1in]{geometry}
\setlength{\parindent}{0cm}
\setlength{\parskip}{2ex plus 0.5ex minus 0.2ex} % whitespace between paragraphs

% Redefine \maketitle command with nicer formatting
%--------------------------------------------------------------------------------------------
\pretitle{
  \begin{flushright}         % align text to right
    \fontsize{40}{60}        % set font size and whitespace
    \usefont{OT1}{phv}{b}{n} % change the font to bold (b), normally shaped (n)
                             %   Helvetica (phv)
    \selectfont              % force LaTeX to search for metric in its mapping
                             %   corresponding to the above font size definition
}
\posttitle{
  \par                       % end paragraph
  \end{flushright}           % end right align
  \vskip 0.5em               % add vertical spacing of 0.5em
}
\preauthor{
  \begin{flushright}
    \large                   % font size
    \lineskip 0.5em          % inter line spacing
    \usefont{OT1}{phv}{m}{n}
}
\postauthor{
  \par
  \end{flushright}
}
\predate{
  \begin{flushright}
  \large
  \lineskip 0.5em
  \usefont{OT1}{phv}{m}{n}
}
\postdate{
  \par
  \end{flushright}
}

% Mathematics Packages
\usepackage[Gray,squaren,thinqspace,cdot]{SIunits}      % elegant units
\usepackage{amsmath}                                    % extensive math options
\usepackage{amsfonts}                                   % special math fonts
\usepackage{mathtools}                                  % useful formatting commands
\usepackage{amsthm}                                     % useful commands for building theorem environments
\usepackage{amssymb}                                    % lots of special math symbols
\usepackage{mathrsfs}                                   % fancy scripts letters
\usepackage{cancel}                                     % cancel lines in math
\usepackage{esint}                                      % fancy integral symbols
\usepackage{relsize}                                    % make math things bigger or smaller
\usepackage{bm}                                         % bold math!

\newcommand\ddfrac[2]{\frac{\displaystyle #1}{\displaystyle #2}}    % elegant fraction formatting
\allowdisplaybreaks[1]                                              % allow align environments to break on pages

% Ensure numbering is section-specific
%--------------------------------------------------------------------------------------------
\numberwithin{equation}{section}
\numberwithin{figure}{section}
\numberwithin{table}{section}

% Citations, references, and annotations
%--------------------------------------------------------------------------------------------
\usepackage[small,bf,hang]{caption}        % captions
\usepackage{subcaption}                    % adds subfigure & subcaption
\usepackage{sidecap}                       % adds side captions
\usepackage{hyperref}                      % add hyperlinks to references
\usepackage[noabbrev,nameinlink]{cleveref} % better references than default \ref
\usepackage{autonum}                       % only number referenced equations
\usepackage{url}                           % urls
\usepackage{cite}                          % well formed numeric citations
% format hyperlinks
\colorlet{linkcolour}{black}
\colorlet{urlcolour}{blue}
\hypersetup{colorlinks=true,
            linkcolor=linkcolour,
            citecolor=linkcolour,
            urlcolor=urlcolour}

% Plotting and Figures
%--------------------------------------------------------------------------------------------
\usepackage{tikz}          % advanced vector graphics
\usepackage{pgfplots}      % data plotting
\usepackage{pgfplotstable} % table plotting
\usepackage{placeins}      % display floats in correct sections
\usepackage{graphicx}      % include external graphics
\usepackage{longtable}     % process long tables

% use most recent version of pgfplots
\pgfplotsset{compat=newest}

% Misc.
%--------------------------------------------------------------------------------------------
\usepackage{todonotes}  % add to do notes
\usepackage{epstopdf}   % process eps-images
\usepackage{float}      % floats
\usepackage{stmaryrd}   % some more nice symbols
\usepackage{emptypage}  % suppress page numbers on empty pages
\usepackage{multicol}   % use this for creating pages with multiple columns
\usepackage{etoolbox}   % adds tags for environment endings
\usepackage{tcolorbox}  % pretty colored boxes!


% Custom Commands
%--------------------------------------------------------------------------------------------
\newcommand\hr{\noindent\rule[0.5ex]{\linewidth}{0.5pt}}                % horizontal line
\newcommand\N{\ensuremath{\mathbb{N}}}                                  % blackboard set characters
\newcommand\R{\ensuremath{\mathbb{R}}}
\newcommand\Z{\ensuremath{\mathbb{Z}}}
\newcommand\Q{\ensuremath{\mathbb{Q}}}
\newcommand\C{\ensuremath{\mathbb{C}}}
\renewcommand{\arraystretch}{1.2}                                       % More space between table rows (could be 1.3)
\newcommand{\Cov}{\mathrm{Cov}}
\newcommand*{\dbar}{\ensuremath{\text{\dj}}}
% Custom Environments
%--------------------------------------------------------------------------------------------
\newcommand{\lecture}[3]{\hr\\{\centering{\large\textsc{Lecture #1: #3}}\\#2\\}\hr\markboth{Lecture #1: #3}{\rightmark}}   % command to title lectures
\usepackage{mdframed}
\theoremstyle{plain}
\newmdtheoremenv[nobreak]{theorem}{Theorem}[section]
\newtheorem{corollary}{Corollary}[theorem]
\newtheorem{lemma}[theorem]{Lemma}
\theoremstyle{definition}
\newtheorem*{ex}{Example}
\newmdtheoremenv[nobreak]{definition}{Definition}[section]
\theoremstyle{remark}
\newtheorem*{remark}{Remark}
\AtEndEnvironment{ex}{\null\hfill$\diamond$}%
% Note: A proof environment is already provided in the amsthm package
\tcbuselibrary{breakable}
\newenvironment{note}[1]{\begin{tcolorbox}[
    arc=0mm,
    colback=white,
    colframe=white!60!black,
    title=#1,
    fonttitle=\sffamily,
    breakable
]}{\end{tcolorbox}}
\newenvironment{problem}{\begin{tcolorbox}[
    arc=0mm,
    breakable,
    colback=white,
    colframe=black
]}{\end{tcolorbox}}

% Header and Footer
%--------------------------------------------------------------------------------------------
% set header and footer
\usepackage{fancyhdr}                       % header and footer
\pagestyle{fancy}                           % use package
\fancyhf{}
\fancyhead[LE,RO]{\textsl{\rightmark}}      % E for even (left pages), O for odd (right pages)
\fancyfoot[LE,RO]{\thepage}
\fancyfoot[LO,RE]{\textsl{\leftmark}}
\setlength{\headheight}{15pt}


% Physics
%--------------------------------------------------------------------------------------------
\usepackage[arrowdel]{physics}      % all the usual useful physics commands
%\usepackage{feyn}                   % for drawing Feynman diagrams
%\usepackage{bohr}                   % for drawing Bohr diagrams
\usepackage{elements}               % for quickly referencing information of various elements
\usepackage{tensor}                 % for writing tensors and chemical symbols

% Finishing touches
%--------------------------------------------------------------------------------------------
\author{Nathaniel D. Hoffman}

\title{33-755 Homework 8}
\date{\today}
\begin{document}
\maketitle

\section*{5. Particle Subject to a Constant Force}
In a one-dimensional problem, consider a particle of potential energy $ V(X) = -fX $, where $ f $ is a positive constant.
\begin{itemize}
    \item[a.] Write Ehrenfest's theorem for the mean values of the position $ X $ and the momentum $ P $ of the particle. Integrate these equations; compare with the classical motion.
        \begin{problem}
            Ehrenfest's theorem states that
            \begin{equation}
                \partial_t \expval{X} = \frac{1}{m} \expval{P}
            \end{equation}
            and
            \begin{equation}
                \partial_t \expval{P} = \frac{1}{\imath\hbar} \expval{\comm{P}{H}}
            \end{equation}
            In the second equation, $ \comm{P}{H} = \comm{P}{V(X)} = -\imath\hbar V'(X) = \imath\hbar f $ so by Ehrenfest's theorem,
            \begin{equation}
                \expval{P} = \int f \dd{t} = f\Delta t
            \end{equation}
            Similarly,
            \begin{equation}
                \expval{X} = \frac{1}{m} \int \expval{P} \dd{t} = \frac{1}{2m}f\Delta t^2
            \end{equation}
            Classically, if $ f = ma $, $ f = \partial_t p $, and $ x = \frac{1}{2} at^2 $, assuming constant acceleration.
        \end{problem}
    \item[b.] Show that the root-mean-square deviation $ \Delta P $ does not vary over time.
        \begin{equation}
            \Delta P = \sqrt{\expval{P^2} - \expval{P}^2}
        \end{equation}
        Since $ \comm{P^2}{H} = \comm{P^2}{V(X)} = \hbar^2 V''(X) $,
        \begin{equation}
            \partial_t\expval{P^2} = \imath\hbar\expval{V''(X)} = 0
        \end{equation}
        since $ V(X) = fX $ so $ V''(X) = 0 $. We know that $ \partial_t\expval{P}^2 = f^2 $ from the previous section, and this is independent of $ t $, so $ \partial_t \Delta P = 0 $.
    \item[c.] Write the Schr\"odinger equation in the $ \{\ket{p}\} $ representation. Deduce from it a relation between $ \partial_t \abs{\bra{p}\ket{\psi(t)}}^2 $ and $ \partial_p \abs{\bra{p}\ket{\psi(t)}}^2 $. Integrate the equation thus obtained; give a physical interpretation.
        \begin{problem}
            If we write
            \begin{equation}
                \imath\hbar\partial_t \ket{\psi(t)} = \hat{H} \ket{\psi(t)} 
            \end{equation}
            in the $ \{\ket{p}\} $ representation, we get
            \begin{align}
                \imath\hbar\partial_t \psi(p, t) &= \int \dd{p} \frac{P^2}{2m} \ket{p} \bra{p}\ket{\psi} + f X \ket{p}\bra{p}\ket{\psi}\\
                &= \frac{p^2}{2m} \psi(p, t) - \imath\hbar f \partial_p \psi(p, t)
            \end{align}
        \end{problem}
\end{itemize}

\section*{9.7.6 Reflection Delay}
\begin{itemize}
    \item[(a)] Consider the reflection coefficient $ B $ for wave $ e^{\imath k x} $ incident at a potential step of height $ V_0 $, where $ E = \hbar^2 k^2 / 2m < V_0 $. Show that $ \abs{B} = 1 $ so we can write $ B = e^{-\imath\phi} $ with $ \phi $ a real quantity which you must determine.
        \begin{problem}
            If we normalize the incident wave to $ 1 $, the probability flux must be conserved, so $ \frac{\hbar k}{m} = \abs{B^2} \frac{\hbar k}{m} $, which means $ \abs{B^2} = 1 = \abs{B} $.
            Next, we know that $ B $ is the reflection coefficient, and for this scenario it is equal to $ B = - \frac{\kappa + \imath k}{\kappa - \imath k} $, with $ \kappa^2 = \frac{2m(E-V_0)}{\hbar^2} $. Suppose $ B = e^{-\imath\phi} $. We now have to find $ \phi = \imath\ln{\frac{\kappa - \imath k}{\kappa + \imath k}} $. Note that
            \begin{equation}
                \arctan(z) = \frac{\imath}{2} \ln\left( \frac{\imath + z}{\imath - z} \right)
            \end{equation}
            If $ z = \frac{k}{\kappa} $, we find that $ \phi = 2\arctan\left( \frac{k}{\kappa} \right) $ satisfies the equation for the proper reflection coefficient.
        \end{problem}
    \item[(b)] Given the incident wave packet
        \begin{equation}
            \phi(x,t) = \int \dd{k} \frac{A(k)}{\sqrt{2 \pi}} e^{\imath\left( kx-\omega(k) t \right)}
        \end{equation}
        determine the reflected wave packet and show that the reflection occurs with a delay $ \tau = - \hbar \dv{\phi}{E} > 0 $. Interpret your result in terms of motion in the classically forbidden region.
        \begin{problem}
            Noting the addition of the phase factor from part (a), we can write the reflected wave packet as
            \begin{equation}
                \phi_r(x,t) = \int \dd{k} \frac{A(k)}{\sqrt{2 \pi}} e^{\imath\left( -kx + \omega(k) t - \phi \right)}
            \end{equation}
            If we set the phase to be a constant, in the incident wave packet, we find that the phase velocity comes from  $ kx - \omega t = 0 \implies x = \frac{\omega}{k} t $ where $ v_p = \frac{\omega}{k} $ is the phase velocity. Equivalently, if the wave packet was Gaussian with a peak at $ \overline{k} $, the group velocity would be $ v_g = \dv{\omega}{k}\eval_{ \overline{k}} $. However, with the added phase in the reflected wave packet, the constant phase calculation gives $ kx = \omega t - \phi $, or $ x = \frac{\omega}{k} t - \frac{\phi}{k} $. If we look at the group velocity, we find that the peaks of the wave are delayed by a factor of $ -\dv{\phi}{k} = \hbar\dv{\phi}{E} $. This occurs because the wave function can enter the forbidden region, and while the distance it goes depends on the energy, if we average over the wave packet, the packet enters some set distance into the classically forbidden region. The time it spends there is the delay $ \tau $ which was found in this problem.
        \end{problem}
\end{itemize}


\end{document}
