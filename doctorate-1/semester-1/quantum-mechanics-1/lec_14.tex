\documentclass[a4paper,twoside,master.tex]{subfiles}
\begin{document}
\lecture{14}{Fri Sep 20 2019}{Compatible Properties in Histories}
We don't want to make the mistake of discussing incompatible properties. To do this with histories, we start with some family of histories $\{Y^\alpha\} $ and we require $Y^\alpha Y^\beta = Y^\beta Y^\alpha$, $\forall\alpha\beta$. This is a complete family of histories, so $\sum_{\vec{\alpha}} Y^{\vec{\alpha}} = \tilde{I}$. Therefore, \textbf{logical negation} is $\neg Y^{\vec{\alpha}} = \tilde{I}-Y^{\vec{\alpha}}$.
\begin{ex}
    Coin toss: $\{(H,H), (H,T), (T,H), (T,T)\} $. The negation of $(H,H)$ is $\neg(H,H) = \{(H,T),(T,H),(T,T)\} $
\end{ex}
\begin{ex}
    \begin{equation}
        \neg [z+] \odot [x+] = [z-]\odot [x+] + [z+]\odot [x-] + [z-]\odot [x-]
    \end{equation}
\end{ex}
We can also have \textbf{conjunction}, $Y \wedge Y' = YY'$.
\begin{ex}
    \begin{equation}
        Y=[z+]_0\odot I_1
    \end{equation}
    \begin{equation}
        Y' = I_0\odot [x+]_1
    \end{equation}
    \begin{equation}
        YY' = [z+]\odot [x+]
    \end{equation}
\end{ex}
Additionally, \textbf{disjunction} is defined by $Y\vee Y' = Y+Y' - YY'$.

\section{Chainket}%
\label{sec:chainket_}

If we start in a pure state, we can define a product history
\begin{equation}
    Y^{\vec{\alpha}} = [\psi_0]\odot P_1^{\alpha_1}\odot\ldots\odot P_f^{\alpha_f}\in\tilde{\mathcal{H}}
\end{equation}
\begin{definition}
    A \textbf{chainket} is defined by
    \begin{equation}
        \ket{\vec{\alpha}} =P_f^{\alpha_f} T_{f,f-1}\ldots T_{21}P_1^{\alpha_1}T_{10}\ket{\psi_0}
    \end{equation}
\end{definition}
\begin{theorem}
    Generalized Born Rule:
    \begin{equation}
        Pr(\vec{\alpha}) = \ip{\vec{\alpha}}
    \end{equation}
\end{theorem}
Is it correct? Let's check some cases.
\subsection{Two-Time History}%
\label{sub:two_time_history}

\begin{equation}
    Y^k = [\psi_0]\odot[\phi_1^k]
\end{equation}
All of our states start at $\ket{\psi_0}$, so we need to define a complete set of histories by adding all the things that don't start there:
\begin{equation}
    Z = (I-[\psi_0])\odot I_1
\end{equation}
The chainket for this state is
\begin{equation}
    \ket{k} = [\phi_1^k]T_{10}\ket{\psi_0}
\end{equation}
\begin{equation}
    \ip{k} = \color{green}\bra{\psi_0} T_{01}\color{blue}\op{\phi_1^k}\color{green}T_{10}\ket{\psi_0}\color{black} = |\ip{\phi_1^k}{\psi_1}|^2 = Pr([\phi_1^k]\mid\psi_0)
\end{equation}
\begin{note}{N.B.}
    \begin{equation}
        |\ip{\phi_1^k}{\psi_1}|^2 = \ip{\phi_1^k}{\psi_1}\ip{\psi_1}{\phi_1^k} = \bra{\psi_1}(\phi_1^k)^2\ket{\psi_1} = \bra{\psi_1}[\phi_1^k]\ket{\psi_1} 
    \end{equation}
\end{note}
\subsection{Unitary History}%
\label{sub:unitary_history}

\begin{equation}
    \ket{\psi_0}\to\ket{\psi_1} = T_{10}\ket{\psi_0}\to\ket{\psi_2} = T_{21}\ket{\psi_1} = T_{20}\ket{\psi_0}
\end{equation}
\begin{equation}
    U = \color{purple}[\psi_0]\odot\color{blue}[\psi_1]\odot\color{green}[\psi_2]\color{black}
\end{equation}
\begin{equation}
    \ket{U} = \color{green}\op{\psi_2}\color{black}T_{21}\color{blue}\op{\psi_1}\color{black}T_{10}\color{purple}{\ket{\psi_0}}\color{black}
\end{equation}
There's another way of thinking about this:
\begin{equation}
    \ket{U} = \ket{\psi_2}\color{red}\cancelto{\bra{\psi_1}}{\bra{\psi_2}T_{21}}\color{black}\ket{\psi_1}\color{yellow}\cancelto{\bra{\psi_0}}{\bra{\psi_1}T_{10}}\color{black}{\ket{\psi_0}}\color{black}
\end{equation}
so
\begin{equation}
    Pr(U) = \ip{U} = \ip{\psi_2} = 1
\end{equation}
\end{document}
