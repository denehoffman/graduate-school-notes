\documentclass[a4paper,twoside]{article}
% My LaTeX preamble file - by Nathaniel Dene Hoffman
% Credit for much of this goes to Olivier Pieters (https://olivierpieters.be/tags/latex)
% and Gilles Castel (https://castel.dev)
% There are still some things to be done:
% 1. Update math commands using mathtools package (remove ddfrac command and just override)
% 2. Maybe abbreviate \imath somehow?
% 3. Possibly format for margin notes and set new margin sizes
% First, some encoding packages and usefull formatting
%--------------------------------------------------------------------------------------------
\usepackage[l2tabu,orthodox]{nag}   % force newer (and safer) LaTeX commands
\usepackage[utf8]{inputenc}         % set character set to support some UTF-8
                                    %   (unicode). Do NOT use this with
                                    %   XeTeX/LuaTeX!
\usepackage[T1]{fontenc}
\usepackage[english]{babel}         % multi-language support
\usepackage{sectsty}                % allow redefinition of section command formatting
\usepackage{tabularx}               % more table options
\usepackage{booktabs}
\usepackage{titling}                % allow redefinition of title formatting
\usepackage{imakeidx}               % create and index of words
\usepackage{xcolor}                 % more colour options
\usepackage{enumitem}               % more list formatting options
\usepackage{tocloft}                % redefine table of contents, new list like objects
\usepackage{subfiles}               % allow for multifile documents

% Next, let's deal with the whitespaces and margins
%--------------------------------------------------------------------------------------------
\usepackage[centering,margin=1in]{geometry}
\setlength{\parindent}{0cm}
\setlength{\parskip}{2ex plus 0.5ex minus 0.2ex} % whitespace between paragraphs

% Redefine \maketitle command with nicer formatting
%--------------------------------------------------------------------------------------------
\pretitle{
  \begin{flushright}         % align text to right
    \fontsize{40}{60}        % set font size and whitespace
    \usefont{OT1}{phv}{b}{n} % change the font to bold (b), normally shaped (n)
                             %   Helvetica (phv)
    \selectfont              % force LaTeX to search for metric in its mapping
                             %   corresponding to the above font size definition
}
\posttitle{
  \par                       % end paragraph
  \end{flushright}           % end right align
  \vskip 0.5em               % add vertical spacing of 0.5em
}
\preauthor{
  \begin{flushright}
    \large                   % font size
    \lineskip 0.5em          % inter line spacing
    \usefont{OT1}{phv}{m}{n}
}
\postauthor{
  \par
  \end{flushright}
}
\predate{
  \begin{flushright}
  \large
  \lineskip 0.5em
  \usefont{OT1}{phv}{m}{n}
}
\postdate{
  \par
  \end{flushright}
}

% Mathematics Packages
\usepackage[Gray,squaren,thinqspace,cdot]{SIunits}      % elegant units
\usepackage{amsmath}                                    % extensive math options
\usepackage{amsfonts}                                   % special math fonts
\usepackage{mathtools}                                  % useful formatting commands
\usepackage{amsthm}                                     % useful commands for building theorem environments
\usepackage{amssymb}                                    % lots of special math symbols
\usepackage{mathrsfs}                                   % fancy scripts letters
\usepackage{cancel}                                     % cancel lines in math
\usepackage{esint}                                      % fancy integral symbols
\usepackage{relsize}                                    % make math things bigger or smaller
\usepackage{bm}                                         % bold math!

\newcommand\ddfrac[2]{\frac{\displaystyle #1}{\displaystyle #2}}    % elegant fraction formatting
\allowdisplaybreaks[1]                                              % allow align environments to break on pages

% Ensure numbering is section-specific
%--------------------------------------------------------------------------------------------
\numberwithin{equation}{section}
\numberwithin{figure}{section}
\numberwithin{table}{section}

% Citations, references, and annotations
%--------------------------------------------------------------------------------------------
\usepackage[small,bf,hang]{caption}        % captions
\usepackage{subcaption}                    % adds subfigure & subcaption
\usepackage{sidecap}                       % adds side captions
\usepackage{hyperref}                      % add hyperlinks to references
\usepackage[noabbrev,nameinlink]{cleveref} % better references than default \ref
\usepackage{autonum}                       % only number referenced equations
\usepackage{url}                           % urls
\usepackage{cite}                          % well formed numeric citations
% format hyperlinks
\colorlet{linkcolour}{black}
\colorlet{urlcolour}{blue}
\hypersetup{colorlinks=true,
            linkcolor=linkcolour,
            citecolor=linkcolour,
            urlcolor=urlcolour}

% Plotting and Figures
%--------------------------------------------------------------------------------------------
\usepackage{tikz}          % advanced vector graphics
\usepackage{pgfplots}      % data plotting
\usepackage{pgfplotstable} % table plotting
\usepackage{placeins}      % display floats in correct sections
\usepackage{graphicx}      % include external graphics
\usepackage{longtable}     % process long tables

% use most recent version of pgfplots
\pgfplotsset{compat=newest}

% Misc.
%--------------------------------------------------------------------------------------------
\usepackage{todonotes}  % add to do notes
\usepackage{epstopdf}   % process eps-images
\usepackage{float}      % floats
\usepackage{stmaryrd}   % some more nice symbols
\usepackage{emptypage}  % suppress page numbers on empty pages
\usepackage{multicol}   % use this for creating pages with multiple columns
\usepackage{etoolbox}   % adds tags for environment endings
\usepackage{tcolorbox}  % pretty colored boxes!


% Custom Commands
%--------------------------------------------------------------------------------------------
\newcommand\hr{\noindent\rule[0.5ex]{\linewidth}{0.5pt}}                % horizontal line
\newcommand\N{\ensuremath{\mathbb{N}}}                                  % blackboard set characters
\newcommand\R{\ensuremath{\mathbb{R}}}
\newcommand\Z{\ensuremath{\mathbb{Z}}}
\newcommand\Q{\ensuremath{\mathbb{Q}}}
\newcommand\C{\ensuremath{\mathbb{C}}}
\renewcommand{\arraystretch}{1.2}                                       % More space between table rows (could be 1.3)
\newcommand{\Cov}{\mathrm{Cov}}
\newcommand*{\dbar}{\ensuremath{\text{\dj}}}
% Custom Environments
%--------------------------------------------------------------------------------------------
\newcommand{\lecture}[3]{\hr\\{\centering{\large\textsc{Lecture #1: #3}}\\#2\\}\hr\markboth{Lecture #1: #3}{\rightmark}}   % command to title lectures
\usepackage{mdframed}
\theoremstyle{plain}
\newmdtheoremenv[nobreak]{theorem}{Theorem}[section]
\newtheorem{corollary}{Corollary}[theorem]
\newtheorem{lemma}[theorem]{Lemma}
\theoremstyle{definition}
\newtheorem*{ex}{Example}
\newmdtheoremenv[nobreak]{definition}{Definition}[section]
\theoremstyle{remark}
\newtheorem*{remark}{Remark}
\AtEndEnvironment{ex}{\null\hfill$\diamond$}%
% Note: A proof environment is already provided in the amsthm package
\tcbuselibrary{breakable}
\newenvironment{note}[1]{\begin{tcolorbox}[
    arc=0mm,
    colback=white,
    colframe=white!60!black,
    title=#1,
    fonttitle=\sffamily,
    breakable
]}{\end{tcolorbox}}
\newenvironment{problem}{\begin{tcolorbox}[
    arc=0mm,
    breakable,
    colback=white,
    colframe=black
]}{\end{tcolorbox}}

% Header and Footer
%--------------------------------------------------------------------------------------------
% set header and footer
\usepackage{fancyhdr}                       % header and footer
\pagestyle{fancy}                           % use package
\fancyhf{}
\fancyhead[LE,RO]{\textsl{\rightmark}}      % E for even (left pages), O for odd (right pages)
\fancyfoot[LE,RO]{\thepage}
\fancyfoot[LO,RE]{\textsl{\leftmark}}
\setlength{\headheight}{15pt}


% Physics
%--------------------------------------------------------------------------------------------
\usepackage[arrowdel]{physics}      % all the usual useful physics commands
%\usepackage{feyn}                   % for drawing Feynman diagrams
%\usepackage{bohr}                   % for drawing Bohr diagrams
\usepackage{elements}               % for quickly referencing information of various elements
\usepackage{tensor}                 % for writing tensors and chemical symbols

% Finishing touches
%--------------------------------------------------------------------------------------------
\author{Nathaniel D. Hoffman}

\title{33-756 Homework 2}
\date{\today}
\begin{document}
\maketitle

\section*{1. Total Time Derivative}
Show that if you add a total time derivative to the Lagrangian $ (\dv{t}F(x)) $, it has no effect on the equations of motion.
\begin{problem}
    Take a modified Lagrangian $ L \to L' = L + \dot{F}(x) $. We can derive the modified Euler-Lagrange equations of motion as
    \begin{align}
        \dv{t} \pdv{L'}{\dot{q}} &= \pdv{L'}{q} \\
        \dv{t} \left[\pdv{L}{\dot{q}} + \pdv{\dot{F}}{\dot{q}}\right] &= \pdv{L}{q} + \pdv{\dot{F}}{q} \\
    \end{align}
    Next, we can remove the unmodified equations of motion $ \dv{t} \pdv{L}{\dot{q}} - \pdv{L}{q} = 0 $:
    \begin{equation}
        \dv{t} \pdv{\dot{F}}{\dot{q}} = \pdv{\dot{F}}{q}
    \end{equation}
    Next, because $ \dv{t}{t} = 1 $,
    \begin{align}
        \pdv{\dot{F}}{\dot{q}} &= \pdv{F}{q} \\
        \dv{t} \pdv{\dot{F}}{\dot{q}} &= \dv{t} \pdv{F}{q} \\
    \end{align}
    so
    \begin{equation}
        \dv{t} \pdv{\dot{F}}{\dot{q}} \to \dv{t} \pdv{F}{q} = \pdv{\dot{F}}{q}
    \end{equation}
    This is true because the derivatives commute. Therefore, the total time derivative has no effect on the equations of motion because it satisfies the Euler-Lagrange equations.
\end{problem}

\section*{2. Transformations under Rotations}
Show that $ \va{P} $ transforms as a vector under an infinitesimal rotation using the representation of $ \va{P} $ and $ \va{L} $ as differential operators.
\begin{problem}
    Consider a rotation of the operator $ \va{P}' = U^\dagger \va{P} U $ where $ U = e^{\frac{\imath}{\hbar} \va{L} \vdot \vu{n} \theta} $. If we expand to first order, we find
    \begin{align}
        \va{P}' &= \left( I - \frac{\imath}{\hbar} \va{L} \vdot \vu{n} \theta \right) \va{P} \left( I + \frac{\imath}{\hbar} \va{L} \vdot \vu{n} \theta \right) \\
        &= \va{P} - \left( \frac{\imath}{\hbar} \va{L} \vdot \vu{n} \theta \right) \va{P} + \va{P} \left( \frac{\imath}{\hbar} \va{L} \vdot \vu{n} \theta \right)
    \end{align}
    so
    \begin{equation}
        \delta \va{P} = \va{P} \left( \frac{\imath}{\hbar} \va{L} \vdot \vu{n} \theta \right) - \left( \frac{\imath}{\hbar} \va{L} \vdot \vu{n} \theta \right) \va{P}
    \end{equation}
    or
    \begin{equation}
        \delta P_i = \comm{P_i}{L_j} \frac{\imath}{\hbar} n_j \theta
    \end{equation}
    We can write these operators in differential form as
    \begin{equation}
        P_i = \frac{\hbar}{\imath} \partial_i
    \end{equation}
    and
    \begin{equation}
        L_j = X_a P_b \epsilon_{abj} = \frac{\hbar}{\imath} X_a \partial_b \epsilon_{abj}
    \end{equation}
    The commutator is therefore
    \begin{align}
        \comm{P_i}{L_j} &= - \hbar^2 \epsilon_{abj} \comm{\partial_i}{X_a \partial_b} \\
        &= - \hbar^2 \epsilon_{abj} \left( \underbrace{\comm{\partial_i}{X_a}}_{\delta_{ia}} \partial_b + X_a \underbrace{\comm{\partial_i}{\partial_b}}_{0} \right) \\
        &= - \hbar^2 \epsilon_{abj} \partial_b \delta_{ai} = - \hbar^2 \epsilon_{ibj} \partial_b = - \frac{\hbar}{\imath} \epsilon_{ibj} P_b
    \end{align}
    Therefore
    \begin{equation}
        \delta P_i = \epsilon_{ibj} P_b n_j \theta
    \end{equation}
    or
    \begin{equation}
        \delta \va{P} = (\va{P} \cross \vu{n}) \theta = - (\vu{n} \cross \va{P}) \theta
    \end{equation}
    Therefore, the infinitesimal rotation leads to a differential which acts moves the end of a vector perpendicular to itself, which is how vectors transform under rotations.
\end{problem}

\section*{3. 3D Isotropic Harmonic Oscillator}
Consider the isotropic harmonic oscillator in three dimensions. Its Hamiltonian can be written as 
\begin{equation}
    H = \hbar \omega\left(a^\dagger_x a_x + a^\dagger_y a_y + a^\dagger_z a_z + \frac{3}{2} \right)
\end{equation}
Naively, one would think that the only symmetry is rotational. Let's assume that's the case\textemdash then the degeneracy would be $ g = (2N+1) $. But this underestimates the degeneracy considerably. Show that for $ n_x + n_y + n_z = N $, the degeneracy is $ g = \frac{1}{2} (N+1)(N+2) $. To do this as a counting problem, you have $ 3 $ buckets and $ N $ marbles. How many ways are there to distribute the marbles?
\begin{problem}
    Let's assume we know how many marbles are being put into the first bucket and call this number $ n_1 \in [0,N] $. The other two bucket's contents must sum to $ N - n_1 $. How many ways are there to divide up $ N - n_1 $ marbles into two bins? We could lay out all the marbles in a row and place a separator to choose which marbles would go into which bin. Since we can also have the case where zero marbles go into a particular bin, there are $ N - n_1 + 1 $ spaces in which the separator can be placed.
    
    Finally, to find the total number of ways we can separate $ N $ marbles, we have to sum over all the possible values of $ n_1 $:
    \begin{equation}
        \sum_{n_1=0}^{N} N-n_1+1 = \sum_{n_1=0}^{N} N+1 - \sum_{n_1=0}^{N} n_1
    \end{equation}
    The first sum is just the value $ N + 1 $ summed over $ n_1 + 1 $ times (since $ n_1 $ goes from $ 0 $ to $ N $). The second sum is the sum of all the numbers from $ 0 $ to $ N $, which is well-known thanks to Gauss: $ \frac{1}{2} N(N+1) $. Therefore, the degeneracy is
    \begin{equation}
        (N+1)(N+1) - \frac{1}{2} N(N+1) = \frac{1}{2} (n+1)(n+2)
    \end{equation}
\end{problem}
So we see that there must be another symmetry that explains why the degeneracy grows more quickly with $ N $. For the 3D SHO we have the symmetric tensor
\begin{equation}
    T_{ij} = \frac{1}{\lambda}(p_i p_j + \lambda^2 r_i r_j)
\end{equation}
Determine what $ \lambda $ must be in order for this tensor to be conserved. Show that the number of conserved quantities is equal to the number of generators in the group $ \text{U}(3) $. To determine the number of generators, determine how many independent parameters exist for the most general unitary $ 3 $-by-$ 3 $ matrix. One can in fact show that this system has a $ \text{U}(3) $ symmetry.
\begin{problem}
    If $ T_{ij} $ is conserved, $ \comm{T}{H} = 0 $. To calculate the commutator, I will first rewrite the Hamiltonian in terms of $ p $ and $ r $:
    \begin{equation}
        H = \frac{p_k^2}{2m} + \frac{m \omega^2 r_k^2}{2}
    \end{equation}
    so (factoring an $ \frac{2}{m} $ to make some of this more symmetric)
    \begin{align}
        0 = \comm{T}{H} &= \frac{2}{\lambda m}\left\{ (p_i p_j + \lambda^2 r_i r_j)(p_k p_k + m^2 \omega^2 r_k r_k) - (p_k p_k + m^2 \omega^2 r_k r_k)(p_i p_j + \lambda^2 r_i r_j)\right\} \\
        &= \color{blue}(p_i p_j p_k p_k) \color{black}+ m^2 \omega^2 (p_i p_j r_k r_k) + \lambda^2 (r_i r_j p_k p_k) + \color{red}\lambda^2 m^2 \omega^2 (r_i r_j r_k r_k)\color{black} \\
        &- \color{blue}(p_k p_k p_i p_j)\color{black} - m^2 \omega^2 (r_k r_k p_i p_j) - \lambda^2 (p_k p_k r_i r_j) - \color{red}\lambda^2 m^2 \omega^2 (r_k r_k r_i r_j)\color{black}
    \end{align}
    Since the position and momentum operators are all self-commuting, the red and blue terms cancel. The remaining terms can be written as:
    \begin{equation}
        0 = m^2 \omega^2 \comm{p_i p_j}{r_k r_k} - \lambda^2 \comm{p_k p_k}{r_i r_j}
    \end{equation}
    The first commutator is
    \begin{align}
        \comm{p_i p_j}{r_k r_k} &= p_i \color{blue}\comm{p_j}{r_k r_k}\color{black} + \color{red}\comm{p_i}{r_k r_k}\color{black} p_j \\
        &= p_i\color{blue}(\comm{p_j}{r_k} r_k + r_k \comm{p_j}{r_k})\color{black} + \color{red}(\comm{p_i}{r_k} r_k + r_k \comm{p_i}{r_k})\color{black} p_j \\
        &= p_i\color{blue}(-2 \imath \hbar r_k \delta_{jk})\color{black} + \color{red}(-2 \imath \hbar r_k \delta_{ik})\color{black} p_j \\
        &= - 2 \imath \hbar (\color{blue}p_i r_j\color{black} + \color{red}r_i p_j\color{black}) \\
        &= - 2 \imath \hbar \pb{p_i}{r_j}
    \end{align}
    The second commutator is
    \begin{align}
        \comm{p_k p_k}{r_i r_j} &= \color{blue}\comm{p_k p_k}{r_i}\color{black} r_j + r_i \color{red}\comm{p_k p_k}{r_j} \color{black} \\
        &= \color{blue} (p_k \comm{p_k}{r_i} + \comm{p_k}{r_i} p_k) \color{black} r_j + r_i \color{red} (p_k \comm{p_k}{r_j} + \comm{p_k}{r_j} p_k) \color{black} \\
        &= \color{blue} (- 2 \imath \hbar p_k \delta_{ik}) \color{black} r_j + r_i \color{red} (- 2 \imath \hbar p_k \delta_{jk}) \color{black} \\
        &= - 2 \imath \hbar (\color{blue} p_i r_j \color{black} + \color{red} r_i p_j \color{black}) \\
        &= - 2 \imath \hbar \pb{p_i}{r_j}
    \end{align}
    As we can see, these commutators are the same, so we can cancel them, leaving
    \begin{equation}
        \lambda^2 = m^2 \omega^2 \qor \lambda = \pm \sqrt{m \omega}
    \end{equation}

    

    All unitary matrices are diagonalizable. Diagonalizing an $ N $-by-$ N $ matrix will result in a diagonal matrix with $ N $ independent eigenvalue coefficients and a transformation matrix of $ N $ orthogonal basis vectors of dimension $ N $. However, you only need to know $ N - 1 $ orthogonal vectors to be able to exactly determine the last one, so in total a general unitary matrix has $ N + N(N - 1) = N^2 $ independent coefficients (generators). Therefore, a $ 3 $-by-$ 3 $ unitary matrix has $ 9 $ generators. Since $ T_{ij} $ is symmetric, it contains $ 6 $ independent parameters. The other $ 3 $ come from the 3D rotational symmetry group $\text{SO}(3)$, which has $ 3 $ generators.
\end{problem}

\end{document}
