\documentclass[a4paper,twoside]{article}
% My LaTeX preamble file - by Nathaniel Dene Hoffman
% Credit for much of this goes to Olivier Pieters (https://olivierpieters.be/tags/latex)
% and Gilles Castel (https://castel.dev)
% There are still some things to be done:
% 1. Update math commands using mathtools package (remove ddfrac command and just override)
% 2. Maybe abbreviate \imath somehow?
% 3. Possibly format for margin notes and set new margin sizes
% First, some encoding packages and usefull formatting
%--------------------------------------------------------------------------------------------
\usepackage[l2tabu,orthodox]{nag}   % force newer (and safer) LaTeX commands
\usepackage[utf8]{inputenc}         % set character set to support some UTF-8
                                    %   (unicode). Do NOT use this with
                                    %   XeTeX/LuaTeX!
\usepackage[T1]{fontenc}
\usepackage[english]{babel}         % multi-language support
\usepackage{sectsty}                % allow redefinition of section command formatting
\usepackage{tabularx}               % more table options
\usepackage{booktabs}
\usepackage{titling}                % allow redefinition of title formatting
\usepackage{imakeidx}               % create and index of words
\usepackage{xcolor}                 % more colour options
\usepackage{enumitem}               % more list formatting options
\usepackage{tocloft}                % redefine table of contents, new list like objects
\usepackage{subfiles}               % allow for multifile documents

% Next, let's deal with the whitespaces and margins
%--------------------------------------------------------------------------------------------
\usepackage[centering,margin=1in]{geometry}
\setlength{\parindent}{0cm}
\setlength{\parskip}{2ex plus 0.5ex minus 0.2ex} % whitespace between paragraphs

% Redefine \maketitle command with nicer formatting
%--------------------------------------------------------------------------------------------
\pretitle{
  \begin{flushright}         % align text to right
    \fontsize{40}{60}        % set font size and whitespace
    \usefont{OT1}{phv}{b}{n} % change the font to bold (b), normally shaped (n)
                             %   Helvetica (phv)
    \selectfont              % force LaTeX to search for metric in its mapping
                             %   corresponding to the above font size definition
}
\posttitle{
  \par                       % end paragraph
  \end{flushright}           % end right align
  \vskip 0.5em               % add vertical spacing of 0.5em
}
\preauthor{
  \begin{flushright}
    \large                   % font size
    \lineskip 0.5em          % inter line spacing
    \usefont{OT1}{phv}{m}{n}
}
\postauthor{
  \par
  \end{flushright}
}
\predate{
  \begin{flushright}
  \large
  \lineskip 0.5em
  \usefont{OT1}{phv}{m}{n}
}
\postdate{
  \par
  \end{flushright}
}

% Mathematics Packages
\usepackage[Gray,squaren,thinqspace,cdot]{SIunits}      % elegant units
\usepackage{amsmath}                                    % extensive math options
\usepackage{amsfonts}                                   % special math fonts
\usepackage{mathtools}                                  % useful formatting commands
\usepackage{amsthm}                                     % useful commands for building theorem environments
\usepackage{amssymb}                                    % lots of special math symbols
\usepackage{mathrsfs}                                   % fancy scripts letters
\usepackage{cancel}                                     % cancel lines in math
\usepackage{esint}                                      % fancy integral symbols
\usepackage{relsize}                                    % make math things bigger or smaller
\usepackage{bm}                                         % bold math!

\newcommand\ddfrac[2]{\frac{\displaystyle #1}{\displaystyle #2}}    % elegant fraction formatting
\allowdisplaybreaks[1]                                              % allow align environments to break on pages

% Ensure numbering is section-specific
%--------------------------------------------------------------------------------------------
\numberwithin{equation}{section}
\numberwithin{figure}{section}
\numberwithin{table}{section}

% Citations, references, and annotations
%--------------------------------------------------------------------------------------------
\usepackage[small,bf,hang]{caption}        % captions
\usepackage{subcaption}                    % adds subfigure & subcaption
\usepackage{sidecap}                       % adds side captions
\usepackage{hyperref}                      % add hyperlinks to references
\usepackage[noabbrev,nameinlink]{cleveref} % better references than default \ref
\usepackage{autonum}                       % only number referenced equations
\usepackage{url}                           % urls
\usepackage{cite}                          % well formed numeric citations
% format hyperlinks
\colorlet{linkcolour}{black}
\colorlet{urlcolour}{blue}
\hypersetup{colorlinks=true,
            linkcolor=linkcolour,
            citecolor=linkcolour,
            urlcolor=urlcolour}

% Plotting and Figures
%--------------------------------------------------------------------------------------------
\usepackage{tikz}          % advanced vector graphics
\usepackage{pgfplots}      % data plotting
\usepackage{pgfplotstable} % table plotting
\usepackage{placeins}      % display floats in correct sections
\usepackage{graphicx}      % include external graphics
\usepackage{longtable}     % process long tables

% use most recent version of pgfplots
\pgfplotsset{compat=newest}

% Misc.
%--------------------------------------------------------------------------------------------
\usepackage{todonotes}  % add to do notes
\usepackage{epstopdf}   % process eps-images
\usepackage{float}      % floats
\usepackage{stmaryrd}   % some more nice symbols
\usepackage{emptypage}  % suppress page numbers on empty pages
\usepackage{multicol}   % use this for creating pages with multiple columns
\usepackage{etoolbox}   % adds tags for environment endings
\usepackage{tcolorbox}  % pretty colored boxes!


% Custom Commands
%--------------------------------------------------------------------------------------------
\newcommand\hr{\noindent\rule[0.5ex]{\linewidth}{0.5pt}}                % horizontal line
\newcommand\N{\ensuremath{\mathbb{N}}}                                  % blackboard set characters
\newcommand\R{\ensuremath{\mathbb{R}}}
\newcommand\Z{\ensuremath{\mathbb{Z}}}
\newcommand\Q{\ensuremath{\mathbb{Q}}}
\newcommand\C{\ensuremath{\mathbb{C}}}
\renewcommand{\arraystretch}{1.2}                                       % More space between table rows (could be 1.3)
\newcommand{\Cov}{\mathrm{Cov}}
\newcommand*{\dbar}{\ensuremath{\text{\dj}}}
% Custom Environments
%--------------------------------------------------------------------------------------------
\newcommand{\lecture}[3]{\hr\\{\centering{\large\textsc{Lecture #1: #3}}\\#2\\}\hr\markboth{Lecture #1: #3}{\rightmark}}   % command to title lectures
\usepackage{mdframed}
\theoremstyle{plain}
\newmdtheoremenv[nobreak]{theorem}{Theorem}[section]
\newtheorem{corollary}{Corollary}[theorem]
\newtheorem{lemma}[theorem]{Lemma}
\theoremstyle{definition}
\newtheorem*{ex}{Example}
\newmdtheoremenv[nobreak]{definition}{Definition}[section]
\theoremstyle{remark}
\newtheorem*{remark}{Remark}
\AtEndEnvironment{ex}{\null\hfill$\diamond$}%
% Note: A proof environment is already provided in the amsthm package
\tcbuselibrary{breakable}
\newenvironment{note}[1]{\begin{tcolorbox}[
    arc=0mm,
    colback=white,
    colframe=white!60!black,
    title=#1,
    fonttitle=\sffamily,
    breakable
]}{\end{tcolorbox}}
\newenvironment{problem}{\begin{tcolorbox}[
    arc=0mm,
    breakable,
    colback=white,
    colframe=black
]}{\end{tcolorbox}}

% Header and Footer
%--------------------------------------------------------------------------------------------
% set header and footer
\usepackage{fancyhdr}                       % header and footer
\pagestyle{fancy}                           % use package
\fancyhf{}
\fancyhead[LE,RO]{\textsl{\rightmark}}      % E for even (left pages), O for odd (right pages)
\fancyfoot[LE,RO]{\thepage}
\fancyfoot[LO,RE]{\textsl{\leftmark}}
\setlength{\headheight}{15pt}


% Physics
%--------------------------------------------------------------------------------------------
\usepackage[arrowdel]{physics}      % all the usual useful physics commands
%\usepackage{feyn}                   % for drawing Feynman diagrams
%\usepackage{bohr}                   % for drawing Bohr diagrams
\usepackage{elements}               % for quickly referencing information of various elements
\usepackage{tensor}                 % for writing tensors and chemical symbols

% Finishing touches
%--------------------------------------------------------------------------------------------
\author{Nathaniel D. Hoffman}

\title{33-756 Homework 3}
\date{\today}
\begin{document}
\maketitle
\section*{1. Representations of $\text{SO}(3)$}
In class, we talked about the fact that there are matrix as well as functional representations of a Lie group\textemdash that is, the operators can be matrices or differential operators. A good way to understand the difference is to consider our friend $\text{SO}(3)$. Suppose we have a representation $\ket{l,m} $. At this point, this is an abstract vector in a Hilbert space. Now consider the action of a rotation $ R(\vu{n}, \delta) $ on this state:
\begin{equation}
    \ket{l'm'} = R(\vu{n}, \delta)\ket{lm}.
\end{equation}
\begin{itemize}
    \item[(a)] Prove that $ l = l' $.
        \begin{problem}
            \begin{align}
                \ket{l'm'} &= R\ket{lm} \\
                L^2\ket{l'm'} &= L^2 R\ket{lm} \\
                &= R L^2\ket{lm} \\
                \hbar^2 l'(l'+1)\ket{l'm'} &= R \left( \hbar^2 l(l+1) \right)\ket{lm} \\
                &\implies l'(l'+1) = l(l+1) \\
                &\implies l' = \pm l
            \end{align}
            but we define $ l \geq 0 $, so $ l' = l $. We can do the second step because $ L^2 $ commutes with rotations since $ R \propto e^{- \imath \vu{n} \vdot \vu{L} \delta / \hbar} $.
        \end{problem}
    \item[(b)] We may therefore write $ R $ as a matrix $ D\indices{^l_{m,m'}} $. These are called the Wigner D-matrices. We can derive these matrices in one of two ways depending on whether we are considering a representation on the space of functions or on matrices. Consider the state $ l = 1 $, $ m = 0 $. Perform an infinitesimal rotation around the $ x $ axis by an amount $ \epsilon $ by using the $ 3 $ by $ 3 $ representation of the generator $ L_x $. This example is a matrix representation.
        \begin{problem}
            \begin{equation}
                R(\vu{x}, \epsilon) \approx I - \frac{\imath}{\hbar} L_x \epsilon
            \end{equation}
            We can find the matrix elements of $ L_x $ starting with the matrix elements of $ L_{\pm} $:
            \begin{equation}
                \mel{lm'}{L_{\pm}}{lm} = \hbar \delta_{m',m+1} \sqrt{(l\mp 1)(l\pm m + 1)}
            \end{equation}
            and $ L_x = \frac{1}{2} (L_+ + L_-) $ so
            \begin{equation}
                L_x = \frac{\hbar}{\sqrt{2}} \mqty(0&1&0\\1&0&1\\0&1&0)
            \end{equation}
            so
            \begin{equation}
                R(\vu{x}, \epsilon) = \mqty(1 & -\frac{\imath \epsilon}{\sqrt{2}} & 0 \\ -\frac{\imath \epsilon}{\sqrt{2}} & 1 & -\frac{\imath \epsilon}{\sqrt{2}} \\ 0 & -\frac{\imath \epsilon}{\sqrt{2}} & 1)
            \end{equation}
            Therefore,
            \begin{equation}
                R(\vu{x}, \epsilon)\ket{1,0} = \mqty(1 & -\frac{\imath \epsilon}{\sqrt{2}} & 0 \\ -\frac{\imath \epsilon}{\sqrt{2}} & 1 & -\frac{\imath \epsilon}{\sqrt{2}} \\ 0 & -\frac{\imath \epsilon}{\sqrt{2}} & 1) \mqty(0\\1\\0) = \mqty(- \frac{\imath \epsilon}{\sqrt{2}} \\ 1 \\ - \frac{\imath \epsilon}{\sqrt{2}}) = - \frac{\imath \epsilon}{\sqrt{2}}\ket{1,-1} +\ket{1,0} - \frac{\imath \epsilon}{\sqrt{2}}\ket{1,+1}
            \end{equation}
        \end{problem}
    \item[(c)] Now consider the representation on the space of functions $\bra{\theta, \varphi}\ket{l,m} $. We know the eigenstates of $ L^2 $ and $ L_z $ are just $ Y_l^m(\theta, \varphi) =\bra{\theta, \varphi}\ket{l,m} $. Perform the same infinitesimal rotation as in the previous problem only now in function space and show that the resulting linear combination of $ m $'s is the same as in the previous problem.
        \begin{problem}
            In the $ Y_{lm} $ basis, we can write the $ L_x $ operator as
            \begin{equation}
                L_x = - \imath \hbar \left[ - \sin(\varphi) \pdv{\theta} - \cot(\theta) \cos(\varphi) \pdv{\varphi} \right]
            \end{equation}
            so that
            \begin{equation}
                R(\vu{x}, \epsilon) = 1 + \epsilon \cos(\varphi) \pdv{\theta} + \epsilon \cos(\theta) \cos(\varphi) \pdv{\varphi}
            \end{equation}
            The $\ket{1,0}$ state can be written as
            \begin{equation}
                Y_{10}(\theta, \varphi) = \frac{1}{2} \sqrt{\frac{3}{\pi}} \cos(\theta)
            \end{equation}
            so
            \begin{align}
                R(\vu{x}, \epsilon) Y_{10}(\theta, \varphi) &= \left( 1 + \epsilon \cos(\varphi) \pdv{\theta} \right) \frac{1}{2} \sqrt{\frac{3}{\pi}} \cos(\theta) \\
                &= \frac{1}{2} \sqrt{\frac{3}{\pi}} \left(\cos(\theta) - \epsilon \underbrace{\cos(\varphi)}_{\frac{e^{\imath \varphi} + e^{- \imath \varphi}}{2}} \sin(\theta) \right) = Y_{10} - \frac{\imath \epsilon}{\sqrt{2}} \left( Y_{1,-1} + Y_{1,+1} \right)
            \end{align}
        \end{problem}
\end{itemize}

\section*{2. Spherical and Cartesian Bases}
When we consider the $ 3 $ dimensional (defining) representation of $\text{SO}(3)$, we can think of the states either as Cartesian basis vectors or as spherical basis vectors $\ket{l=1,m=1} $, $\ket{l=1, m=-1} $, and $\ket{l=1,m=0} $. Calculate the relation between the Cartesian basis vectors $\ket{\vu{x}} $, $\ket{\vu{y}} $, and $\ket{\vu{z}} $. First determine which of the three spherical vectors corresponds to $\ket{\vu{z}} $ using the fact that this vector is invariant under rotations around the $ z $ axis. Then determine the relation between the $ x $ and $ y $ basis vectors in terms of the spherical basis vectors by using the fact that they are invariant under rotations around the $ x $ and $ y $ axes respectively. Next, consider a general vector $ A_x \vu{x} + A_y \vu{y} + A_z \vu{z} $ and write the components in the spherical basis in terms of $ A_x $, $ A_y $, and $ A_z $.
\begin{problem}
    We can first attribute $\ket{\vu{z}} =\ket{1,0} $ since $ L_z\ket{lm} = \hbar m\ket{lm} $. If we were to act an infinitesimal rotation on this vector, we would find that $ \left( I - \frac{\imath}{\hbar} \epsilon L_z \right)\ket{10} = \left(\ket{10} + 0\ket{10} \right) =\ket{10} $. However, the other vectors will give a factor of $ \pm 1 $ from the eigenvalue of $ L_z $, which means they are not invariant under this rotation. The other two basis vectors are invariant under rotations about the other two axes, but since we are in the $ L_z $ basis, it will be convenient to use linear combinations of $ L_{\pm} $. We can act $ L_x $ on a general vector and see what conditions are required for that product to go to $ 0 $, since $ R(\vu{x}) \propto I - \frac{\imath}{\hbar} L_z $.
    
    \begin{align}
        L_x \left( a\ket{1,-1} + b\ket{1,0} + c\ket{1,+1} \right) &= \frac{1}{2} \left( L_+ + L_- \right) \left( a\ket{1,-1} + b\ket{1,0} + c\ket{1,+1} \right) \\
        &= \frac{\hbar \sqrt{2}}{2} \left( a \ket{1,0} + b \ket{1,+1} + b\ket{1,-1} + c\ket{1,0} \right) \\
        &= 0
    \end{align}
    so $ b = 0 $ and $ a = -c $. Therefore,
    \begin{equation}
        \vu{x} = \frac{1}{\sqrt{2}} \left(\ket{1,-1} -\ket{1,+1} \right)
    \end{equation}

    Next, for $ L_y = \frac{1}{2 \imath} (L_+ - L_-) $,
    \begin{align}
        L_y \left( a\ket{1,-1} + b\ket{1,0} + c\ket{1,+1} \right) &= \frac{1}{2 \imath} \left( L_+ - L_- \right) \left( a\ket{1,-1} + b\ket{1,0} + c\ket{1,+1} \right) \\
        &= \frac{\hbar \sqrt{2}}{2 \imath} \left( a\ket{1,0} + b\ket{1,+1} - b\ket{1,-1} - c\ket{1,0} \right) \\
        &= 0
    \end{align}
    so $ b = 0 $ and $ a = c $. Therefore,
    \begin{equation}
        \vu{y} = \frac{1}{\sqrt{2}} \left(\ket{1,-1} +\ket{1,+1} \right)
    \end{equation}

    Now let's consider a general vector $ A_x \vu{x} + A_y \vu{y} + A_z \vu{z} $ and transform it into the spherical basis:
    \begin{align}
        A_x \vu{x} + A_y \vu{y} + A_z \vu{z} &= A_x \left(\ket{1,-1} -\ket{1,+1} \right) + A_y \left(\ket{1,-1} +\ket{1,+1} \right) + A_z\ket{1,0} \\
        &= \left( A_x + A_y \right)\ket{1,-1} + \left( A_y - A_x \right)\ket{1,+1} + A_z\ket{1,0}
    \end{align}
\end{problem}

\section*{3. Basis in Functional Space}
We can think of $ Y_{1m}(\theta, \varphi) $ as $\bra{\vu{n}(\theta, \varphi)}\ket{1,m} $ where $ \vu{n} $ is a Cartesian unit vector in the direction $ (\theta, \varphi) $. Choose values for $ \theta $ and $ \varphi $ that place $ \vu{n} $ along the three axes, and compare the resulting projection for each $ m $ to the result you got from the previous problem.
\begin{problem} 
    We can imagine a general scalar which is the functional projection of a vector in spherical coordinates:
    \begin{equation}
        A_{-1} Y_{1,-1}(\theta, \varphi) + A_{0} Y_{1,0}(\theta, \varphi) + A_{+1} Y_{1,+1}(\theta, \varphi)
    \end{equation}
    Along the $ \vu{z} $ axis, $\theta = 0$ and $ \varphi = 0 $, so the $ Y_{1,-1} $ and $ Y_{1,+1} $ spherical harmonics will cancel out:
    \begin{equation}
        Y_{1,-1}(0, 0) = Y_{1,+1}(0, 0) = 0
    \end{equation}
    so $ \bra{\vu{z}}\ket{l,m}  = A_0 Y_{1,0} $, where $ A_0 = 1 $ for normalization.

    Along $ \vu{x} $, $ \theta = \frac{\pi}{2} $ and $ \varphi = 0 $ so $ Y_{1,0} = 0 $ and $ A_{-1} = - A_{+1} $ since
    \begin{equation}
        Y_{1,-1} = \frac{1}{2} \sqrt{\frac{3}{2 \pi}} = - Y_{1,+1}
    \end{equation}
    Again, for normalization, this means that $ \bra{\vu{x}}\ket{l,m} = \frac{1}{\sqrt{2}} \left( Y_{1,-1} - Y_{1,+1} \right)$.

    Finally, along the $ \vu{y} $ axis, $ \theta = \frac{\pi}{2} $ and $ \varphi = \frac{\pi}{2} $. Now, $ A_{-1} = A_{+1} $ since
    \begin{equation}
        Y_{1,-1} = - \frac{1}{2} \imath \sqrt{\frac{3}{2 \pi}} = Y_{1, +1}
    \end{equation}
    so $ \bra{\vu{y}}\ket{l,m} = \frac{1}{\sqrt{2}} \left( Y_{1,-1} + Y_{1,+1} \right) $, which agrees with the result in the previous problem.
\end{problem}

\section*{4. The Runge-Lenz Vector}
Show that the Runge-Lenz vector is a Hermitian operator, and show that it commutes with the Hamiltonian.
\begin{problem}
    I'm assuming we are meant to check the commutation with the following Hamiltonian:
    \begin{equation}
        H = \frac{\va{p}^2}{2m} - \frac{e^2}{r}\va{r}
    \end{equation}
    In general, for two Hermitian operators $ A $ and $ B $, $ (AB)^\dagger = B^\dagger A^\dagger $. The Runge-Lenz vector is defined as:
    \begin{equation}
        \va{A} = \frac{1}{2m} \left[ \va{p} \cross \va{L} - \va{L} \cross \va{p} \right] - \frac{e^2}{r} \va{r}
    \end{equation}
    or
    \begin{equation}
        A_i = \frac{1}{2m} \left[ \epsilon_{ijk} p_j L_k - \epsilon_{ijk} L_j p_k\right] - \frac{e^2}{r} r_i = \frac{1}{m} 
    \end{equation}
    so
    \begin{equation}
        A_i^\dagger = \frac{1}{2m} \left[ \epsilon_{ijk} L_k^\dagger p_j^\dagger + \epsilon_{ijk} p_k^\dagger L_j^\dagger \right] - \frac{e^2}{r} r_i^\dagger 
    \end{equation}
    All of these operators are Hermitian, so
    \begin{equation}
        A_i^\dagger = \frac{1}{2m} \left[ \epsilon_{ijk} L_k p_j + \epsilon_{ijk} p_k L_j \right] - \frac{e^2}{r} r_i
    \end{equation}
    Finally, $ p_k L_j = L_j p_k + \imath \hbar \epsilon_{jkl} p_l $ and $ L_k p_j = p_j L_k - \imath \hbar \epsilon_{jkl} p_l $ so switching the order of the $ L $'s and $ p $'s just creates terms that cancel:
    \begin{equation}
        A_i^\dagger = \frac{1}{2m} \left[ \epsilon_{ijk} p_j L_k + \epsilon_{ijk} L_j p_k \right] - \frac{e^2}{r} r_i = A_i
    \end{equation}
    so $ \va{A} $ is Hermitian.

    Next, I will show that this vector commutes with the Hamiltonian. First, I will take a slight detour.
    \begin{equation}
        \comm{L_i}{p_l} = \comm{\epsilon_{ijk} r_j p_k}{p_l} = \epsilon_{ijk} \left( r_j \comm{p_k}{p_l} + \comm{r_j}{p_l} p_k \right) = \epsilon_{ijk} \delta_{jl} \imath \hbar p_k = \epsilon_{ilk} \imath \hbar p_k
    \end{equation}
    so
    \begin{equation}
        \comm{L^2}{\va{p}} \mapsto \comm{L_i L_i}{p_j} = \comm{L_i}{p_j} L_i + L_i \comm{L_i}{p_j} \mapsto \imath \hbar \left( \va{p} \cross \va{L} - \va{L} \cross \va{p} \right)
    \end{equation}
    We can then write the RL vector in the following form:
    \begin{equation}
        \va{A} = \frac{1}{2m \imath \hbar} \comm{L^2}{\va{p}} - \frac{e^2}{r} \va{r}
    \end{equation}
    Therefore,
    \begin{align}
        \comm{\va{A}}{H} &= \frac{1}{2m \imath \hbar} \comm{\comm{L^2}{\va{p}}}{H} - e^2 \comm{\frac{\va{r}}{r}}{H} \\
        &= \frac{1}{2m \imath \hbar} \left( \comm{\comm{\va{p}}{H}}{L^2} + \comm{\comm{H}{L^2}}{\va{p}} \right) - \frac{e^2}{2m} \comm{\frac{\va{r}}{r}}{\va{p}^2} \\
        &= \frac{- e^2}{2m \imath \hbar} \left(\comm{\comm{\va{p}}{\frac{\va{r}}{r}}}{L^2}\right) - \frac{e^2}{2m} \comm{\frac{\va{r}}{r}}{\va{p}^2}
    \end{align}
    since $ \comm{H}{L^2} = 0 $ because the Hamiltonian is spherically symmetric.
    \begin{equation}
        \comm{p_i}{\frac{r_j}{r}} = - \imath \hbar \left( \partial_i \frac{r_j}{r} - \frac{r_j}{r} \partial_i \right) = - \imath \hbar \left( \frac{\delta_{ij}}{r} - \frac{r_i r_j}{r^3} - \frac{r_j}{r} \partial_i \right)
    \end{equation}
    I ran out of time to complete the other commutators, sorry.
\end{problem}

\end{document}
