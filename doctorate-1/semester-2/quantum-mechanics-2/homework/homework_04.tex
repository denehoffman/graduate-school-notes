\documentclass[a4paper,twoside]{article}
% My LaTeX preamble file - by Nathaniel Dene Hoffman
% Credit for much of this goes to Olivier Pieters (https://olivierpieters.be/tags/latex)
% and Gilles Castel (https://castel.dev)
% There are still some things to be done:
% 1. Update math commands using mathtools package (remove ddfrac command and just override)
% 2. Maybe abbreviate \imath somehow?
% 3. Possibly format for margin notes and set new margin sizes
% First, some encoding packages and usefull formatting
%--------------------------------------------------------------------------------------------
\usepackage[l2tabu,orthodox]{nag}   % force newer (and safer) LaTeX commands
\usepackage[utf8]{inputenc}         % set character set to support some UTF-8
                                    %   (unicode). Do NOT use this with
                                    %   XeTeX/LuaTeX!
\usepackage[T1]{fontenc}
\usepackage[english]{babel}         % multi-language support
\usepackage{sectsty}                % allow redefinition of section command formatting
\usepackage{tabularx}               % more table options
\usepackage{booktabs}
\usepackage{titling}                % allow redefinition of title formatting
\usepackage{imakeidx}               % create and index of words
\usepackage{xcolor}                 % more colour options
\usepackage{enumitem}               % more list formatting options
\usepackage{tocloft}                % redefine table of contents, new list like objects
\usepackage{subfiles}               % allow for multifile documents

% Next, let's deal with the whitespaces and margins
%--------------------------------------------------------------------------------------------
\usepackage[centering,margin=1in]{geometry}
\setlength{\parindent}{0cm}
\setlength{\parskip}{2ex plus 0.5ex minus 0.2ex} % whitespace between paragraphs

% Redefine \maketitle command with nicer formatting
%--------------------------------------------------------------------------------------------
\pretitle{
  \begin{flushright}         % align text to right
    \fontsize{40}{60}        % set font size and whitespace
    \usefont{OT1}{phv}{b}{n} % change the font to bold (b), normally shaped (n)
                             %   Helvetica (phv)
    \selectfont              % force LaTeX to search for metric in its mapping
                             %   corresponding to the above font size definition
}
\posttitle{
  \par                       % end paragraph
  \end{flushright}           % end right align
  \vskip 0.5em               % add vertical spacing of 0.5em
}
\preauthor{
  \begin{flushright}
    \large                   % font size
    \lineskip 0.5em          % inter line spacing
    \usefont{OT1}{phv}{m}{n}
}
\postauthor{
  \par
  \end{flushright}
}
\predate{
  \begin{flushright}
  \large
  \lineskip 0.5em
  \usefont{OT1}{phv}{m}{n}
}
\postdate{
  \par
  \end{flushright}
}

% Mathematics Packages
\usepackage[Gray,squaren,thinqspace,cdot]{SIunits}      % elegant units
\usepackage{amsmath}                                    % extensive math options
\usepackage{amsfonts}                                   % special math fonts
\usepackage{mathtools}                                  % useful formatting commands
\usepackage{amsthm}                                     % useful commands for building theorem environments
\usepackage{amssymb}                                    % lots of special math symbols
\usepackage{mathrsfs}                                   % fancy scripts letters
\usepackage{cancel}                                     % cancel lines in math
\usepackage{esint}                                      % fancy integral symbols
\usepackage{relsize}                                    % make math things bigger or smaller
\usepackage{bm}                                         % bold math!

\newcommand\ddfrac[2]{\frac{\displaystyle #1}{\displaystyle #2}}    % elegant fraction formatting
\allowdisplaybreaks[1]                                              % allow align environments to break on pages

% Ensure numbering is section-specific
%--------------------------------------------------------------------------------------------
\numberwithin{equation}{section}
\numberwithin{figure}{section}
\numberwithin{table}{section}

% Citations, references, and annotations
%--------------------------------------------------------------------------------------------
\usepackage[small,bf,hang]{caption}        % captions
\usepackage{subcaption}                    % adds subfigure & subcaption
\usepackage{sidecap}                       % adds side captions
\usepackage{hyperref}                      % add hyperlinks to references
\usepackage[noabbrev,nameinlink]{cleveref} % better references than default \ref
\usepackage{autonum}                       % only number referenced equations
\usepackage{url}                           % urls
\usepackage{cite}                          % well formed numeric citations
% format hyperlinks
\colorlet{linkcolour}{black}
\colorlet{urlcolour}{blue}
\hypersetup{colorlinks=true,
            linkcolor=linkcolour,
            citecolor=linkcolour,
            urlcolor=urlcolour}

% Plotting and Figures
%--------------------------------------------------------------------------------------------
\usepackage{tikz}          % advanced vector graphics
\usepackage{pgfplots}      % data plotting
\usepackage{pgfplotstable} % table plotting
\usepackage{placeins}      % display floats in correct sections
\usepackage{graphicx}      % include external graphics
\usepackage{longtable}     % process long tables

% use most recent version of pgfplots
\pgfplotsset{compat=newest}

% Misc.
%--------------------------------------------------------------------------------------------
\usepackage{todonotes}  % add to do notes
\usepackage{epstopdf}   % process eps-images
\usepackage{float}      % floats
\usepackage{stmaryrd}   % some more nice symbols
\usepackage{emptypage}  % suppress page numbers on empty pages
\usepackage{multicol}   % use this for creating pages with multiple columns
\usepackage{etoolbox}   % adds tags for environment endings
\usepackage{tcolorbox}  % pretty colored boxes!


% Custom Commands
%--------------------------------------------------------------------------------------------
\newcommand\hr{\noindent\rule[0.5ex]{\linewidth}{0.5pt}}                % horizontal line
\newcommand\N{\ensuremath{\mathbb{N}}}                                  % blackboard set characters
\newcommand\R{\ensuremath{\mathbb{R}}}
\newcommand\Z{\ensuremath{\mathbb{Z}}}
\newcommand\Q{\ensuremath{\mathbb{Q}}}
\newcommand\C{\ensuremath{\mathbb{C}}}
\renewcommand{\arraystretch}{1.2}                                       % More space between table rows (could be 1.3)
\newcommand{\Cov}{\mathrm{Cov}}
\newcommand*{\dbar}{\ensuremath{\text{\dj}}}
% Custom Environments
%--------------------------------------------------------------------------------------------
\newcommand{\lecture}[3]{\hr\\{\centering{\large\textsc{Lecture #1: #3}}\\#2\\}\hr\markboth{Lecture #1: #3}{\rightmark}}   % command to title lectures
\usepackage{mdframed}
\theoremstyle{plain}
\newmdtheoremenv[nobreak]{theorem}{Theorem}[section]
\newtheorem{corollary}{Corollary}[theorem]
\newtheorem{lemma}[theorem]{Lemma}
\theoremstyle{definition}
\newtheorem*{ex}{Example}
\newmdtheoremenv[nobreak]{definition}{Definition}[section]
\theoremstyle{remark}
\newtheorem*{remark}{Remark}
\AtEndEnvironment{ex}{\null\hfill$\diamond$}%
% Note: A proof environment is already provided in the amsthm package
\tcbuselibrary{breakable}
\newenvironment{note}[1]{\begin{tcolorbox}[
    arc=0mm,
    colback=white,
    colframe=white!60!black,
    title=#1,
    fonttitle=\sffamily,
    breakable
]}{\end{tcolorbox}}
\newenvironment{problem}{\begin{tcolorbox}[
    arc=0mm,
    breakable,
    colback=white,
    colframe=black
]}{\end{tcolorbox}}

% Header and Footer
%--------------------------------------------------------------------------------------------
% set header and footer
\usepackage{fancyhdr}                       % header and footer
\pagestyle{fancy}                           % use package
\fancyhf{}
\fancyhead[LE,RO]{\textsl{\rightmark}}      % E for even (left pages), O for odd (right pages)
\fancyfoot[LE,RO]{\thepage}
\fancyfoot[LO,RE]{\textsl{\leftmark}}
\setlength{\headheight}{15pt}


% Physics
%--------------------------------------------------------------------------------------------
\usepackage[arrowdel]{physics}      % all the usual useful physics commands
%\usepackage{feyn}                   % for drawing Feynman diagrams
%\usepackage{bohr}                   % for drawing Bohr diagrams
\usepackage{elements}               % for quickly referencing information of various elements
\usepackage{tensor}                 % for writing tensors and chemical symbols

% Finishing touches
%--------------------------------------------------------------------------------------------
\author{Nathaniel D. Hoffman}

\title{33-756 Homework 4}
\date{\today}
\begin{document}
\maketitle

\section*{1. Delta Functions in $ d $ Dimensions}
Consider a $ d $-dimensional system with delta function potential. What are the symmetries of this system? Is there a particular dimension for which the symmetry is enhanced?
\begin{problem}
    In any dimension, there is a reflection symmetry about any coordinate, as long as the reflection is centered around the delta function. Since this potential is spherically symmetric, there is also an $ SO(d) $ symmetry for a $ d $-dimensional potential. In $ d = 2 $, there is an additional symmetry of scale invariance. In $ d $ dimensions, the action is
    \begin{equation}
        S = \int \dd{t} \frac{1}{2} m\dot{x}^2 - A \delta^{(d)}(x)
    \end{equation}
    If we scale both time and space,
    \begin{equation}
        (x', t') = (\gamma x, \lambda t)
    \end{equation}
    we want to find what value of $ d $ makes the action invariant. $ \dd{t'} = \lambda \dd{t} $ and $ \dot{x}' = \dv{t'}x' = \frac{\gamma}{\lambda} \dot{x} $, so
    \begin{align}
        S' &= \int \lambda \dd{t} \left( \frac{1}{2} m \frac{\gamma^2}{\lambda^2} \dot{x}^2 - A \delta^{(d)}(\gamma x) \right) \\
        &= \int \dd{t} \frac{1}{2} m \frac{\gamma^2}{\lambda} \dot{x}^2 - \frac{\lambda}{\gamma^d} \delta^{(d)}(x) 
    \end{align}
    For this to be invariant, we require $ \frac{\gamma^2}{\lambda} = 1 $ and $ \frac{\lambda}{\gamma^d} = 1 $, so $ d = 2 $.
\end{problem}

\section*{2. Inverse Square Potential}
Consider the problem of a particle moving in a spherical potential
\begin{equation}
    V(r) = \frac{A \hbar^2}{2M r^2}.
\end{equation}
Write out the Schr\"odinger equation for this system utilizing the spherical symmetry. Next, rewrite the equation in terms of the dimensionless variable $ \rho = kr $ where $ k = \sqrt{2ME/ \hbar^2} $. The solution to this differential equation is a special function. Show that there are no bound state solutions to the equation which are physically sensible. Finally, explain how you could have guessed this using physical reasoning using the results of the previous homework. As a side note, classically there is a theorem by Bertrand that states the only potentials with closed bound orbits are $ r^2 $ and $ 1/r $.
\begin{problem}
    Because of the spherical symmetry of the potential, we can expand the wave function as
    \begin{equation}
        \sum_{lm} Y_{lm}(\theta, \varphi) R(r)
    \end{equation}
    so we can write the Schr\"odinger equation in terms of the angular momentum operator and act it on the spherical harmonics:
    \begin{equation}
        \left[- \frac{\hbar^2}{2M} \frac{1}{r^2} \pdv{r} \left( r^2 \pdv{r} \right) + \frac{\hbar^2 (l(l+1) + A)}{2M} \frac{1}{r^2}\right] R(r) = ER(r)
    \end{equation}
    Next, by the chain rule, we can reduce the radial differential to
    \begin{equation}
        \frac{\hbar^2}{2M} \left[ - \frac{1}{r} \pdv[2]{r} r + (l(l+1) + A) \frac{1}{r^2}\right] R(r) = ER(r)
    \end{equation}
    We then make the following substitution:
    \begin{equation}
        \rho = kr \qquad \pdv[2]{\rho} = \left(\pdv{r}{\rho}\right)^2 \pdv[2]{r} = \frac{1}{k^2} \pdv[2]{r}
    \end{equation}
    \begin{equation}
        \left[ - E \frac{1}{\rho} \pdv[2]{\rho} \rho + E(l(l+1) + A) \frac{1}{\rho^2} \right] R = ER
    \end{equation}
    Substituting $ R(r) = \frac{U(\rho)}{\sqrt{\rho}} $, we get
    \begin{equation}
        \frac{U - 4 \rho (U' + \rho U'')}{4 \rho^{5/2}} + \frac{(l(l+1) + A)}{\rho^{5/2}} U - \frac{\rho^2}{\rho^{5/2}} U = 0
    \end{equation}
    or
    \begin{equation}
        0 = \rho^2 U''(\rho) + \rho U'(\rho) + \left[ \rho^2 - \underbrace{ \left\{ l(l+1) + A + \frac{1}{4} \right\}}_{\alpha^2} \right]U(\rho)
    \end{equation}
    This is the Bessel equation, and solutions for $ R $ are the spherical Bessel functions
    \begin{equation}
        R(r) = \sqrt{\frac{\pi}{2kr}} J_{\alpha}(kr) \equiv j_{\alpha - \frac{1}{2}}(kr)
    \end{equation}
    However, for bound states, $ E < 0 $, and $ E = \frac{\hbar^2 k^2}{2M} $ so then $ k^2 < 0 $ or $ k $ is imaginary. The Bessel equation can be solved for imaginary arguments, but then the solutions are the modified Bessel functions, which all diverge at $ r \to \infty $, so any bound states would not be normalizable (are not physically sensible). In the previous homework, we found that the Runge-Lenz vector is conserved for conservative central forces, but this vector will not commute with a $ \frac{1}{r^2} $ potential (we barely got it to work with a $ \frac{1}{r} $ potential).
\end{problem}

\section*{3. Perturbed Particle in a Box}
Consider a particle in a box of length $ L $. Suppose we place a delta function $ V(x) = AL \delta(x) $ potential at the center of the box. Under what conditions (on $ A $) do you expect to be able to treat this potential as a perturbation? Calculate the first-order energy shift to the $ n $th level of the system as well as the shift in the wave function.
\begin{problem}
    If $ A << 1 $ we can expand our eigenstates around the unperturbed Hamiltonian and find the low-order corrections for this system. In class, we learned that
    \begin{equation}
        E_n^{(1)} = \ev{A L \delta(x)}{\psi_n^{(0)}}
    \end{equation}
    In this case, the unperturbed wave functions for a particle in a box come in two different forms:
    \begin{equation}
        \psi_n^{(0)}(x) = \sqrt{\frac{2}{L}} \begin{cases} \sin(\frac{n \pi}{L} x) & n \text{ even} \\ \cos(\frac{n \pi}{L} x) & n \text{ odd} \end{cases} 
    \end{equation}
    Evaluating the expectation value of the potential, we find that for even $ n $, there is no first-order perturbation to the energy:
    \begin{equation}
        E_n^{(1)} = \frac{2AL}{L} \int \delta(x) \sin[2](\frac{n \pi}{L} x) \sim \sin[2](0) = 0
    \end{equation}
    However, for odd $ n $, we get
    \begin{equation}
        E_n^{(1)} = 2A \int \delta(x) \cos[2](\frac{n \pi}{L} x) = 2A
    \end{equation}
    We can then calculate the first-order shift in the wave function:
    \begin{equation}
        \ket{\psi_n^{(1)}} = \sum_{k \neq n} \frac{\mel{\psi_k^{(0)}}{AL \delta(x)}{\psi_n^{(0)}}}{E_n^{(0)} - E_k^{(0)}}\ket{\psi_k^{(0)}}
    \end{equation}
    In the numerator, if there is a single sine function in the integral, the whole thing will be zero, meaning that for $ n $ or $ k $ even, there is no first-order shift. However, if both are odd, then the numerator becomes:
    \begin{equation}
        \mel{\psi_k^{(0)}}{AL \delta(x)}{\psi_n^{(0)}} = 2A \int \delta(x) \cos(\frac{n \pi}{L} x) \cos(\frac{k \pi}{L} x) = 2A
    \end{equation}
    In the denominator, we just have the unperturbed energies for a particle in a box, so the first-order correction to the wave function is
    \begin{equation}
        \ket{\psi_n^{(1)}} = \sum_{k \neq n} \frac{2A}{\frac{n^2 \pi^2 \hbar^2}{2mL^2} - \frac{k^2 \pi^2 \hbar^2}{2mL^2}}\ket{\psi_k^{(0)}} = \sum_{k \neq n} \frac{4AmL^2}{(n^2 - k^2) \pi^2 \hbar^2}\ket{\psi_k^{(0)}} \quad n,k \text{ odd}
    \end{equation}
\end{problem}

\section*{4. Second-Order Energy Shift}
Using the algorithm discussed in class, derive an expansion for the second-order energy shift in perturbation theory.
\begin{problem}
    In perturbation theory, we had set up the expansion
    \begin{align}
        (H_0 + \lambda H_I)&\left(\ket{\psi_n} + \lambda C_n^{[1]} \ket{\psi_n} + \lambda^2 C_n^{[2]} \ket{\psi_n} + \cdots\right) \\
        &= \left(E_n^{[0]} + \lambda E_n^{[1]} + \lambda^2 E_n^{[2]} + \cdots\right)\left(\ket{\psi_n} + \lambda C_n^{[1]} \ket{\psi_n} + \lambda^2 C_n^{[2]} \ket{\psi_n} + \cdots\right)
    \end{align}
    For shorthand, I will write $ C_n^{[i]}\ket{\psi_n} \equiv\ket{\psi_n^{[i]}} $ with $ C_n^{[0]} = 1 $:
    \begin{align}
        (H_0 + \lambda H_I)&\left( \ket{\psi_n^{[0]}} + \lambda \ket{\psi_n^{[1]}} + \lambda^2 \ket{\psi_n^{[2]}} + \cdots\right) \\
        &= (E_n^{[0]} + \lambda E_n^{[1]} + \lambda^2 E_n^{[2]} +\cdots) \left( \ket{\psi_n^{[0]}} + \lambda \ket{\psi_n^{[1]}} + \lambda^2 \ket{\psi_n^{[2]}} + \cdots \right)
    \end{align}
    From the first-order derivation, we already know what $ \ket{\psi_n^{[1]}} $ and $ E_n^{[1]} $ are, but now we want to find $ E_n^{[2]} $. This value has a prefactor of $ \lambda^2 $ so we can multiply through and match powers of $ \lambda^2 $:
    \begin{equation}
        H_0 \ket{\psi_n^{[2]}} + H_I\ket{\psi_n^{[1]}} = E_n^{[0]}\ket{\psi_n^{[2]}} + E_n^{[1]}\ket{\psi_n^{[1]}} + E_n^{[2]}\ket{\psi_n^{[0]}} 
    \end{equation}
    Next, we will project onto the unperturbed wave function:
    \begin{align}
        \bra{\psi_n^{[0]}} H_0\ket{\psi_n^{[2]}} +\bra{\psi_n^{[0]}} H_I\ket{\psi_n^{[1]}} &= E_n^{[0]}\bra{\psi_n^{[0]}}\ket{\psi_n^{[2]}} + E_n^{[1]}\bra{\psi_n^{[0]}}\ket{\psi_n^{[1]}} + E_n^{[2]}\cancelto{1}{\bra{\psi_n^{[0]}}\ket{\psi_n^{[0]}}} \\
        \cancel{E_n^{[0]}\bra{\psi_n^{[0]}}\ket{\psi_n^{[2]}}} +\bra{\psi_n^{[0]}} H_I\ket{\psi_n^{[1]}} &= \cancel{E_n^{[0]}\bra{\psi_n^{[0]}}\ket{\psi_n^{[2]}}} + E_n^{[1]}\bra{\psi_n^{[0]}}\ket{\psi_n^{[1]}} + E_n^{[2]} \\
        \bra{\psi_n^{[0]}} H_I\ket{\psi_n^{[1]}} &= E_n^{[1]}\cancelto{0}{\bra{\psi_n^{[0]}}\ket{\psi_n^{[1]}}} + E_n^{[2]} \\
        \bra{\psi_n^{[0]}} H_I\ket{\psi_n^{[1]}} &= E_n^{[2]} 
    \end{align}
    From the first-order results, we know that
    \begin{equation}
        \ket{\psi_n^{[1]}} = \sum_{k \neq n} \frac{\bra{\psi_k^{[0]}} H_I\ket{\psi_n^{[0]}}}{E_n^{[0]} - E_k^{[0]}}\ket{\psi_k^{[0]}}
    \end{equation}
    so
    \begin{equation}
        E_n^{[2]} = \sum_{k \neq n} \frac{\bra{\psi_k^{[0]}} H_I\ket{\psi_n^{[0]}}}{E_n^{[0]} - E_k^{[0]}}\bra{\psi_n^{[0]}} H_I\ket{\psi_k^{[0]}} = \sum_{k \neq n} \frac{\abs{\bra{\psi_k^{[0]}} H_I\ket{\psi_n^{[0]}}}^2}{E_n^{[0]} - E_k^{[0]}}
    \end{equation}
\end{problem}

\end{document}
