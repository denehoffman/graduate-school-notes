\documentclass[a4paper,twoside]{article}
% My LaTeX preamble file - by Nathaniel Dene Hoffman
% Credit for much of this goes to Olivier Pieters (https://olivierpieters.be/tags/latex)
% and Gilles Castel (https://castel.dev)
% There are still some things to be done:
% 1. Update math commands using mathtools package (remove ddfrac command and just override)
% 2. Maybe abbreviate \imath somehow?
% 3. Possibly format for margin notes and set new margin sizes
% First, some encoding packages and usefull formatting
%--------------------------------------------------------------------------------------------
\usepackage[l2tabu,orthodox]{nag}   % force newer (and safer) LaTeX commands
\usepackage[utf8]{inputenc}         % set character set to support some UTF-8
                                    %   (unicode). Do NOT use this with
                                    %   XeTeX/LuaTeX!
\usepackage[T1]{fontenc}
\usepackage[english]{babel}         % multi-language support
\usepackage{sectsty}                % allow redefinition of section command formatting
\usepackage{tabularx}               % more table options
\usepackage{booktabs}
\usepackage{titling}                % allow redefinition of title formatting
\usepackage{imakeidx}               % create and index of words
\usepackage{xcolor}                 % more colour options
\usepackage{enumitem}               % more list formatting options
\usepackage{tocloft}                % redefine table of contents, new list like objects
\usepackage{subfiles}               % allow for multifile documents

% Next, let's deal with the whitespaces and margins
%--------------------------------------------------------------------------------------------
\usepackage[centering,margin=1in]{geometry}
\setlength{\parindent}{0cm}
\setlength{\parskip}{2ex plus 0.5ex minus 0.2ex} % whitespace between paragraphs

% Redefine \maketitle command with nicer formatting
%--------------------------------------------------------------------------------------------
\pretitle{
  \begin{flushright}         % align text to right
    \fontsize{40}{60}        % set font size and whitespace
    \usefont{OT1}{phv}{b}{n} % change the font to bold (b), normally shaped (n)
                             %   Helvetica (phv)
    \selectfont              % force LaTeX to search for metric in its mapping
                             %   corresponding to the above font size definition
}
\posttitle{
  \par                       % end paragraph
  \end{flushright}           % end right align
  \vskip 0.5em               % add vertical spacing of 0.5em
}
\preauthor{
  \begin{flushright}
    \large                   % font size
    \lineskip 0.5em          % inter line spacing
    \usefont{OT1}{phv}{m}{n}
}
\postauthor{
  \par
  \end{flushright}
}
\predate{
  \begin{flushright}
  \large
  \lineskip 0.5em
  \usefont{OT1}{phv}{m}{n}
}
\postdate{
  \par
  \end{flushright}
}

% Mathematics Packages
\usepackage[Gray,squaren,thinqspace,cdot]{SIunits}      % elegant units
\usepackage{amsmath}                                    % extensive math options
\usepackage{amsfonts}                                   % special math fonts
\usepackage{mathtools}                                  % useful formatting commands
\usepackage{amsthm}                                     % useful commands for building theorem environments
\usepackage{amssymb}                                    % lots of special math symbols
\usepackage{mathrsfs}                                   % fancy scripts letters
\usepackage{cancel}                                     % cancel lines in math
\usepackage{esint}                                      % fancy integral symbols
\usepackage{relsize}                                    % make math things bigger or smaller
\usepackage{bm}                                         % bold math!

\newcommand\ddfrac[2]{\frac{\displaystyle #1}{\displaystyle #2}}    % elegant fraction formatting
\allowdisplaybreaks[1]                                              % allow align environments to break on pages

% Ensure numbering is section-specific
%--------------------------------------------------------------------------------------------
\numberwithin{equation}{section}
\numberwithin{figure}{section}
\numberwithin{table}{section}

% Citations, references, and annotations
%--------------------------------------------------------------------------------------------
\usepackage[small,bf,hang]{caption}        % captions
\usepackage{subcaption}                    % adds subfigure & subcaption
\usepackage{sidecap}                       % adds side captions
\usepackage{hyperref}                      % add hyperlinks to references
\usepackage[noabbrev,nameinlink]{cleveref} % better references than default \ref
\usepackage{autonum}                       % only number referenced equations
\usepackage{url}                           % urls
\usepackage{cite}                          % well formed numeric citations
% format hyperlinks
\colorlet{linkcolour}{black}
\colorlet{urlcolour}{blue}
\hypersetup{colorlinks=true,
            linkcolor=linkcolour,
            citecolor=linkcolour,
            urlcolor=urlcolour}

% Plotting and Figures
%--------------------------------------------------------------------------------------------
\usepackage{tikz}          % advanced vector graphics
\usepackage{pgfplots}      % data plotting
\usepackage{pgfplotstable} % table plotting
\usepackage{placeins}      % display floats in correct sections
\usepackage{graphicx}      % include external graphics
\usepackage{longtable}     % process long tables

% use most recent version of pgfplots
\pgfplotsset{compat=newest}

% Misc.
%--------------------------------------------------------------------------------------------
\usepackage{todonotes}  % add to do notes
\usepackage{epstopdf}   % process eps-images
\usepackage{float}      % floats
\usepackage{stmaryrd}   % some more nice symbols
\usepackage{emptypage}  % suppress page numbers on empty pages
\usepackage{multicol}   % use this for creating pages with multiple columns
\usepackage{etoolbox}   % adds tags for environment endings
\usepackage{tcolorbox}  % pretty colored boxes!


% Custom Commands
%--------------------------------------------------------------------------------------------
\newcommand\hr{\noindent\rule[0.5ex]{\linewidth}{0.5pt}}                % horizontal line
\newcommand\N{\ensuremath{\mathbb{N}}}                                  % blackboard set characters
\newcommand\R{\ensuremath{\mathbb{R}}}
\newcommand\Z{\ensuremath{\mathbb{Z}}}
\newcommand\Q{\ensuremath{\mathbb{Q}}}
\newcommand\C{\ensuremath{\mathbb{C}}}
\renewcommand{\arraystretch}{1.2}                                       % More space between table rows (could be 1.3)
\newcommand{\Cov}{\mathrm{Cov}}
\newcommand*{\dbar}{\ensuremath{\text{\dj}}}
% Custom Environments
%--------------------------------------------------------------------------------------------
\newcommand{\lecture}[3]{\hr\\{\centering{\large\textsc{Lecture #1: #3}}\\#2\\}\hr\markboth{Lecture #1: #3}{\rightmark}}   % command to title lectures
\usepackage{mdframed}
\theoremstyle{plain}
\newmdtheoremenv[nobreak]{theorem}{Theorem}[section]
\newtheorem{corollary}{Corollary}[theorem]
\newtheorem{lemma}[theorem]{Lemma}
\theoremstyle{definition}
\newtheorem*{ex}{Example}
\newmdtheoremenv[nobreak]{definition}{Definition}[section]
\theoremstyle{remark}
\newtheorem*{remark}{Remark}
\AtEndEnvironment{ex}{\null\hfill$\diamond$}%
% Note: A proof environment is already provided in the amsthm package
\tcbuselibrary{breakable}
\newenvironment{note}[1]{\begin{tcolorbox}[
    arc=0mm,
    colback=white,
    colframe=white!60!black,
    title=#1,
    fonttitle=\sffamily,
    breakable
]}{\end{tcolorbox}}
\newenvironment{problem}{\begin{tcolorbox}[
    arc=0mm,
    breakable,
    colback=white,
    colframe=black
]}{\end{tcolorbox}}

% Header and Footer
%--------------------------------------------------------------------------------------------
% set header and footer
\usepackage{fancyhdr}                       % header and footer
\pagestyle{fancy}                           % use package
\fancyhf{}
\fancyhead[LE,RO]{\textsl{\rightmark}}      % E for even (left pages), O for odd (right pages)
\fancyfoot[LE,RO]{\thepage}
\fancyfoot[LO,RE]{\textsl{\leftmark}}
\setlength{\headheight}{15pt}


% Physics
%--------------------------------------------------------------------------------------------
\usepackage[arrowdel]{physics}      % all the usual useful physics commands
%\usepackage{feyn}                   % for drawing Feynman diagrams
%\usepackage{bohr}                   % for drawing Bohr diagrams
\usepackage{elements}               % for quickly referencing information of various elements
\usepackage{tensor}                 % for writing tensors and chemical symbols

% Finishing touches
%--------------------------------------------------------------------------------------------
\author{Nathaniel D. Hoffman}

\title{33-756 Homework 5}
\date{\today}
\begin{document}
\maketitle

\section*{1. Corrections to the Hydrogen Atom}
In class, we said that there are two other corrections (aside from spin-orbit) which contribute at order $ v^2 / c^2 $. Let us consider the correction to the kinetic energy
\begin{equation}
    H_{\text{KE}} = - \frac{\va{p}^4}{8m^3 c^2}.
\end{equation}

Calculate the shift in the $ n = 1 $, $ l = 0 $ level.

\begin{problem}
    \begin{equation}
        \Delta E_{\text{KE}}^{100} = - \frac{1}{8m^3 c^2} \ev{\va{p}^4}{\psi_{100}}
    \end{equation}
    However, rather than evaluate the momentum operator to the fourth power in spherical coordinates, we can rewrite this perturbation as
    \begin{align}
        - \frac{\va{p}^4}{8m^3 c^2} &= - \frac{1}{2mc^2} \left( \frac{\va{p}^2}{2m} \right)^2 = - \frac{1}{2 m c^2} \left( H_0 + \frac{e^2}{r} \right)^2 \\
        &= - \frac{1}{2m c^2} \left( H_0^2 + \frac{2 e^2 H_0}{r} + \frac{e^4}{r^2} \right)
    \end{align}
    Using this, we can write the energy shift as
    \begin{equation}
        \Delta E_{\text{KE}}^{100} = - \frac{1}{2mc^2} \left( E_n^2 + 2 e^2 E_n \ev{\frac{1}{r}} + e^4 \ev{\frac{1}{r^2}} \right)
    \end{equation}
    From class, online, or through a thorough proof involving the Hellmann-Feynman theorem, we could show that the expectation value for $ \frac{1}{r} $ is
    \begin{equation}
        \ev{\frac{1}{r}} = \frac{1}{a_0}
    \end{equation}
    and
    \begin{equation}
        \ev{\frac{1}{r^2}} = \frac{2}{a_0^2}
    \end{equation}
    Plugging all of this in, we find that
    \begin{equation}
        \Delta E_{\text{KE}}^{100} = - \frac{5}{8} mc^2 \alpha^4
    \end{equation}
\end{problem}

Now consider the $ n = 2 $ level. Ignoring spin, this level has four degenerate states. We showed in class that in such cases we may need to use degenerate perturbation theory where we diagonalize the perturbing Hamiltonian. However, this is not necessary if the Hamiltonian has no non-vanishing, off-diagonal matrix elements. Why is this true?

\begin{problem}
    From perturbation theory, the first-order correction to the perturbed wave function is
    \begin{equation}
        \ket{\psi_n^{(1)}} = \sum_{m \neq n} \frac{\mel{\psi_m^{(0)}}{V}{\psi_n^{(0)}}}{E_n^{(0)} - E_m^{(0)}}\ket{\psi_m^{(0)}}
    \end{equation}
    If all of the off-diagonal matrix elements vanish, the numerator will be zero, since these are the off-diagonal elements of the Hamiltonian (since it's a sum over $ m \neq n $). Therefore, for such Hamiltonians, there is no first-order correction to the wave function.
\end{problem}

Determine whether or not $ H_{\text{KE}} $ has non-vanishing, off-diagonal matrix elements within the $ n = 2 $ system. That is, determine the selection rules for this operator. Can it change $ l $ or $ m $?

\begin{problem}
    Since $ \va{p}^4 $ is a scalar under rotations, $ \comm{\va{p}^4}{\va{L}^2} = \comm{\va{p}^4}{L_z} = 0 $. Therefore
    \begin{align}
        \mel{n'l'm'}{\comm{H_{\text{KE}}}{\va{L}^2}}{nlm} &= \mel{n'l'm'}{H_{\text{KE}} \va{L}^2}{nlm} - \mel{n'l'm'}{\va{L}^2 H_{\text{KE}}}{nlm} \\
        0 &= \hbar^2 \left( (l(l+1)) - (l'(l'+1)) \mel{n'l'm'}{H_{\text{KE}}}{nlm} \right)
    \end{align}
    The nonzero matrix elements are possible when $ l'(l'+1) = l(l+1) $. Solving for this, we find $ l = l' $. This selection rule gives ``on-diagonal'' matrix elements only, and so far, there are no nonzero off-diagonal elements.

    Next
    \begin{align}
        \mel{n'l'm'}{\comm{H_{\text{KE}}}{L_z}}{nlm} &= \mel{n'l'm'}{H_{\text{KE} L_z}}{nlm} - \mel{n'l'm'}{L_z H_{\text{KE}}}{nlm} \\
        0 &= \hbar (m - m') \mel{n'l'm'}{H_{\text{KE}}}{nlm}
    \end{align}
    Again, the selection rule is $ m = m' $, so there are no non-vanishing, off-diagonal matrix elements.
\end{problem}

Next, consider the Darwin term which arises from the fact that the electron has a finite Compton wavelength.
\begin{equation}
    H_{\text{D}} = \frac{\pi e^2 \hbar^2}{2m^2 c^2} \delta^3(\va{R})
\end{equation}

Calculate the shift in the $ n = 1 $ energy level. If we consider the $ n = 2 $ state, are there any non-vanishing, off-diagonal matrix elements? Calculate the shift in the $ p $-wave orbitals.

\begin{problem}
    \begin{equation}
        \Delta E_{\text{D}}^{100} = \frac{\pi e^2 \hbar^2}{2m^2 c^2} \ev{\delta^3(\va{R})}_{100}
    \end{equation}
    and
    \begin{equation}
        \ev{\delta^3(\va{R})}_{100} = \frac{1}{\pi a_0^3} \int e^{-2 \frac{r}{a_0}} \delta(r) \dd{r} = \frac{1}{\pi a_0^3} e^{-2 \frac{0}{a_0}} = \frac{1}{\pi a_0^3}
    \end{equation}
    where $ a_0 = \frac{\hbar^2}{m e^2} $, so
    \begin{equation}
        \Delta E_{\text{D}}^{100} = \frac{e^8 m}{2 c^2 \hbar^4} = \frac{1}{2} mc^2 \alpha^4
    \end{equation}

    For the $ n = 2 $ state, there are no non-vanishing, off-diagonal matrix elements because all of the $ p $-wave orbitals have wave functions $ \psi \propto r e^{- \frac{r}{2a_0}} $, so they vanish at the origin. While this is not true for the $ 2s $ state, the only non-zero matrix element will be on the diagonal for that state, since the $ r $-dependence in the $ p $-wave orbitals will cause the whole term to be zero. Therefore,
    \begin{equation}
        \Delta E_{\text{D}}^{21\{-1,0,1\}} = 0
    \end{equation}
\end{problem}

Now we would like to determine the net shift in the $ n = 1 $ state. Collect your results from this problem to determine the net change in the $ n = 1 $ state. Present your answer in $ \electronvolt $. Why is it OK to ignore the spin coupling for this state?

\begin{problem}
    \begin{equation}
        \Delta E_{\text{KE}}^{100} + \Delta E_{\text{D}}^{100} = \left(- \frac{5}{8} + \frac{1}{2} \right) m c^2 \alpha^4 = - \frac{1}{8} mc^2 \alpha^4 \approx 510,998 \frac{\electronvolt}{c^2} c^2 \left( \frac{1}{137} \right)^4 = 0.00145056\electronvolt
    \end{equation}
    We can ignore the spin coupling because the Hamiltonian is proportional to $ \va{J}^2 - \va{L}^2 - \va{S}^2 $ and for the $ 1s $ state, $ l = 0 $, $ \va{J}^2\ket{1s} \propto \frac{1}{2} \left( \frac{1}{2} + 1 \right)\ket{1s} = \va{S}^2\ket{1s} $ so the $ \va{J}^2 - \va{S}^2 $ term will vanish. For higher energy states, we need to take into account the different ways of adding angular momentum and spin to form total angular momentum, but for the $ 1s $ state, there is no angular momentum except for spin.
\end{problem}

\section*{2. Clebsch-Gordan Coefficients}
Consider adding two spins, one of $ 3/2 $ and the other $ 1/2 $. Calculate all of the Clebsch-Gordan coefficients.

\begin{problem}
    From group theory, we know that
    \begin{equation}
        \frac{3}{2} \otimes \frac{1}{2} = \left( \frac{3}{2} + \frac{1}{2} \right) \oplus \left( \frac{3}{2} - \frac{1}{2} \right) = 2 \oplus 1
    \end{equation}
    since in general, $ J_1 \otimes J_2 = (J_1 + J_2) \oplus (J_1 + J_2 - 1) \oplus \cdots \oplus (J_1 - J_2) $. We first identify the highest states in both bases:
    \begin{equation}
        \ket{2,2} = \ket{\frac{3}{2}, \frac{3}{2}} \otimes\ket{\frac{1}{2}, \frac{1}{2}} \equiv\ket{\frac{3}{2}, \frac{3}{2}}\ket{\frac{1}{2}, \frac{1}{2}}
    \end{equation}
    Next, we act the lowering operator on both sides:
    \begin{equation}
        J_-\ket{2,2} = \hbar \sqrt{(J+M)(J-M+1)}\ket{2, 2-1} = \hbar \sqrt{(4)(1)}\ket{2,1} = 2 \hbar\ket{2,1}
    \end{equation}
    \begin{equation}
        j_-\ket{\frac{3}{2}, \frac{3}{2}}\ket{\frac{1}{2},\ket{\frac{1}{2}}} +\ket{\frac{3}{2}, \frac{3}{2}} j_-\ket{\frac{1}{2}, \frac{1}{2}} = \hbar \left( \sqrt{3}\ket{\frac{3}{2}, \frac{1}{2}}\ket{\frac{1}{2}, \frac{1}{2}} +\ket{\frac{3}{2}, \frac{3}{2}}\ket{\frac{1}{2}, - \frac{1}{2}} \right)
    \end{equation}
    so
    \begin{equation}
        \ket{2,1} = \frac{\sqrt{3}}{2}\ket{\frac{3}{2}, \frac{1}{2}}\ket{\frac{1}{2}, \frac{1}{2}} + \frac{1}{2} \ket{\frac{3}{2}, \frac{3}{2}}\ket{\frac{1}{2}, - \frac{1}{2}}
    \end{equation}
    We can continue using ladder operators to get the rest of the $ J = 2 $ states:
    \begin{equation}
        J_-\ket{2,1} = \sqrt{6} \hbar \ket{2,0}
    \end{equation}
    \begin{align}
        &\frac{\sqrt{3}}{2} j_-\ket{\frac{3}{2}, \frac{1}{2}}\ket{\frac{1}{2}, \frac{1}{2}} + \frac{1}{2} j_-\ket{\frac{3}{2}, \frac{3}{2}}\ket{\frac{1}{2}, - \frac{1}{2}} + \frac{\sqrt{3}}{2}\ket{\frac{3}{2}, \frac{1}{2}}j_-\ket{\frac{1}{2}, \frac{1}{2}} + \frac{1}{2} \ket{\frac{3}{2}, \frac{3}{2}}j_-\ket{\frac{1}{2}, - \frac{1}{2}} \\
        &= \sqrt{3} \hbar \ket{\frac{3}{2}, - \frac{1}{2}} \ket{\frac{1}{2}, \frac{1}{2}} + \frac{\sqrt{3}}{2} \hbar \ket{\frac{3}{2}, \frac{1}{2}}\ket{\frac{1}{2}, - \frac{1}{2}} + \frac{\sqrt{3}}{2} \hbar \ket{\frac{3}{2}, \frac{1}{2}}\ket{\frac{1}{2}, - \frac{1}{2}} + 0 \\
        &= \sqrt{3} \hbar \left[\ket{\frac{3}{2}, - \frac{1}{2}}\ket{\frac{1}{2}, \frac{1}{2}} +\ket{\frac{3}{2}, \frac{1}{2}}\ket{\frac{1}{2}, - \frac{1}{2}} \right]
    \end{align}

    Therefore,
    \begin{equation}
        \ket{2,0} = \frac{1}{\sqrt{2}} \left[\ket{\frac{3}{2}, - \frac{1}{2}}\ket{\frac{1}{2}, \frac{1}{2}} +\ket{\frac{3}{2}, \frac{1}{2}}\ket{\frac{1}{2}, - \frac{1}{2}} \right]
    \end{equation}

    Next, $ J_-\ket{2,1} = \sqrt{6} \hbar\ket{2,-1} $ and

    \begin{align}
        & \frac{1}{\sqrt{2}} \left[j_-\ket{\frac{3}{2}, - \frac{1}{2}}\ket{\frac{1}{2}, \frac{1}{2}} + j_-\ket{\frac{3}{2}, \frac{1}{2}}\ket{\frac{1}{2}, - \frac{1}{2}} + \ket{\frac{3}{2}, - \frac{1}{2}}j_-\ket{\frac{1}{2}, \frac{1}{2}} +\ket{\frac{3}{2}, \frac{1}{2}}j_-\ket{\frac{1}{2}, - \frac{1}{2}}  \right] \\
        &= \frac{\hbar}{\sqrt{2}} \left[ \sqrt{3} \ket{\frac{3}{2}, - \frac{3}{2}}\ket{\frac{1}{2}, \frac{1}{2}} + 2 \ket{\frac{3}{2}, -\frac{1}{2}}\ket{\frac{1}{2}, - \frac{1}{2}} + \ket{\frac{3}{2}, - \frac{1}{2}}\ket{\frac{1}{2}, -\frac{1}{2}} + 0\right] \\
        &= \frac{\hbar}{\sqrt{2}} \left[ \sqrt{3}\ket{\frac{3}{2}, - \frac{3}{2}}\ket{\frac{1}{2}, \frac{1}{2}} + 3 \ket{\frac{3}{2}, - \frac{1}{2}}\ket{\frac{1}{2},-\frac{1}{2}} \right]
    \end{align}

    Therefore
    \begin{equation}
        \ket{2,-1} = \frac{1}{2} \ket{\frac{3}{2}, - \frac{3}{2}}\ket{\frac{1}{2}, \frac{1}{2}} + \frac{\sqrt{3}}{2} \ket{\frac{3}{2}, - \frac{1}{2}}\ket{\frac{1}{2}, -\frac{1}{2}}
    \end{equation}

    The lowest state with $ J = 2 $ can be deduced by the same reasoning as with the highest state:
    \begin{equation}
        \ket{2,-2} =\ket{\frac{3}{2}, - \frac{3}{2}}\ket{\frac{1}{2},-\frac{1}{2}}
    \end{equation}

    For $ J = 1 $, we first start an arbitrary state with $ M = m_1 + m_2 = 1 $:
    \begin{equation}
        \ket{1,1} = \alpha\ket{\frac{3}{2}, \frac{1}{2}}\ket{\frac{1}{2}, \frac{1}{2}} + \beta\ket{\frac{3}{2}, \frac{3}{2}}\ket{\frac{1}{2}, - \frac{1}{2}}
    \end{equation}
    We want this to be orthogonal to $\ket{2,1} $, so
    \begin{equation}
        \ket{1, 1} = - \frac{1}{2} \ket{\frac{3}{2}, \frac{1}{2}}\ket{\frac{1}{2}, \frac{1}{2}} + \frac{\sqrt{3}}{2} \ket{\frac{3}{2}, \frac{3}{2}}\ket{\frac{1}{2}, - \frac{1}{2}}
    \end{equation}

    Applying the lowering operator gives us $ J_-\ket{1, 1} = \sqrt{2} \hbar\ket{1,0} $ and
    \begin{align}
        &- \frac{1}{2} j_-\ket{\frac{3}{2}, \frac{1}{2}}\ket{\frac{1}{2}, \frac{1}{2}} + \frac{\sqrt{3}}{2} j_-\ket{\frac{3}{2}, \frac{3}{2}}\ket{\frac{1}{2}, - \frac{1}{2}} - \frac{1}{2} \ket{\frac{3}{2}, \frac{1}{2}}j_-\ket{\frac{1}{2}, \frac{1}{2}} + \frac{\sqrt{3}}{2} \ket{\frac{3}{2}, \frac{3}{2}}j_-\ket{\frac{1}{2}, - \frac{1}{2}} \\
        &= - \hbar \ket{\frac{3}{2}, -\frac{1}{2}}\ket{\frac{1}{2}, \frac{1}{2}} + \frac{3}{2} \hbar\ket{\frac{3}{2}, \frac{1}{2}}\ket{\frac{1}{2}, - \frac{1}{2}} - \frac{1}{2} \hbar \ket{\frac{3}{2}, \frac{1}{2}}\ket{\frac{1}{2}, -\frac{1}{2}} + 0 \\
        &= - \hbar\ket{\frac{3}{2}, - \frac{1}{2}}\ket{\frac{1}{2}, \frac{1}{2}} + \hbar\ket{\frac{3}{2}, \frac{1}{2}}\ket{\frac{1}{2}, - \frac{1}{2}}
    \end{align}
    so
    \begin{equation}
        \ket{1,0} = \frac{1}{\sqrt{2}} \left[-\ket{\frac{3}{2}, -\frac{1}{2}}\ket{\frac{1}{2}, \frac{1}{2}} +\ket{\frac{3}{2}, \frac{1}{2}}\ket{\frac{1}{2}, - \frac{1}{2}}\right]
    \end{equation}
    Finally, for the $\ket{1,-1} $ state, I don't want to work out the whole thing again. Technically all three of these can be solved by orthogonality:
    \begin{equation}
        \ket{1,-1} = - \frac{\sqrt{3}}{2}\ket{\frac{3}{2}, - \frac{3}{2}}\ket{\frac{1}{2}, \frac{1}{2}} + \frac{1}{2}\ket{\frac{3}{2}, - \frac{1}{2}}\ket{\frac{1}{2}, - \frac{1}{2}}
    \end{equation}
    Now that we know all the states, we can read out the (nonzero) Clebsch-Gordan coefficients:
    \begin{alignat}{4}
        C^{2,2}_{\frac{3}{2}, \frac{3}{2}, \frac{1}{2}, \frac{1}{2}} &= C^{2,-2}_{\frac{3}{2}, -\frac{3}{2}, \frac{1}{2}, -\frac{1}{2}} && && &&= 1 \\
        C^{2,1}_{\frac{3}{2}, \frac{1}{2}, \frac{1}{2}, \frac{1}{2}} &= C^{2,-1}_{\frac{3}{2}, -\frac{1}{2}, \frac{1}{2}, -\frac{1}{2}} &&= C^{1,1}_{\frac{3}{2}, \frac{3}{2}, \frac{1}{2}, -\frac{1}{2}} &&= - C^{1,-1}_{\frac{3}{2}, -\frac{3}{2}, \frac{1}{2}, \frac{1}{2}} &&= \sqrt{\frac{3}{4}} \\
        C^{2,1}_{\frac{3}{2}, \frac{3}{2}, \frac{1}{2}, -\frac{1}{2}} &= C^{2,-1}_{\frac{3}{2}, -\frac{3}{2}, \frac{1}{2}, \frac{1}{2}} &&= - C^{1,1}_{\frac{3}{2}, \frac{1}{2}, \frac{1}{2}, \frac{1}{2}} &&= C^{1,-1}_{\frac{3}{2}, -\frac{1}{2}, \frac{1}{2}, -\frac{1}{2}} &&= \frac{1}{2} \\
        C^{2,0}_{\frac{3}{2}, -\frac{1}{2}, \frac{1}{2}, \frac{1}{2}} &= C^{2,0}_{\frac{3}{2}, \frac{1}{2}, \frac{1}{2}, -\frac{1}{2}} &&= -C^{1,0}_{\frac{3}{2}, -\frac{1}{2}, \frac{1}{2}, \frac{1}{2}} &&= C^{2,0}_{\frac{3}{2}, \frac{1}{2}, \frac{1}{2}, -\frac{1}{2}} &&= \sqrt{\frac{1}{2}} \\
    \end{alignat}
\end{problem}

\end{document}
