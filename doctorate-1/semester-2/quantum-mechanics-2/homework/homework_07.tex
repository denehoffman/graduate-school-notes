\documentclass[a4paper,twoside]{article}
% My LaTeX preamble file - by Nathaniel Dene Hoffman
% Credit for much of this goes to Olivier Pieters (https://olivierpieters.be/tags/latex)
% and Gilles Castel (https://castel.dev)
% There are still some things to be done:
% 1. Update math commands using mathtools package (remove ddfrac command and just override)
% 2. Maybe abbreviate \imath somehow?
% 3. Possibly format for margin notes and set new margin sizes
% First, some encoding packages and usefull formatting
%--------------------------------------------------------------------------------------------
\usepackage[l2tabu,orthodox]{nag}   % force newer (and safer) LaTeX commands
\usepackage[utf8]{inputenc}         % set character set to support some UTF-8
                                    %   (unicode). Do NOT use this with
                                    %   XeTeX/LuaTeX!
\usepackage[T1]{fontenc}
\usepackage[english]{babel}         % multi-language support
\usepackage{sectsty}                % allow redefinition of section command formatting
\usepackage{tabularx}               % more table options
\usepackage{booktabs}
\usepackage{titling}                % allow redefinition of title formatting
\usepackage{imakeidx}               % create and index of words
\usepackage{xcolor}                 % more colour options
\usepackage{enumitem}               % more list formatting options
\usepackage{tocloft}                % redefine table of contents, new list like objects
\usepackage{subfiles}               % allow for multifile documents

% Next, let's deal with the whitespaces and margins
%--------------------------------------------------------------------------------------------
\usepackage[centering,margin=1in]{geometry}
\setlength{\parindent}{0cm}
\setlength{\parskip}{2ex plus 0.5ex minus 0.2ex} % whitespace between paragraphs

% Redefine \maketitle command with nicer formatting
%--------------------------------------------------------------------------------------------
\pretitle{
  \begin{flushright}         % align text to right
    \fontsize{40}{60}        % set font size and whitespace
    \usefont{OT1}{phv}{b}{n} % change the font to bold (b), normally shaped (n)
                             %   Helvetica (phv)
    \selectfont              % force LaTeX to search for metric in its mapping
                             %   corresponding to the above font size definition
}
\posttitle{
  \par                       % end paragraph
  \end{flushright}           % end right align
  \vskip 0.5em               % add vertical spacing of 0.5em
}
\preauthor{
  \begin{flushright}
    \large                   % font size
    \lineskip 0.5em          % inter line spacing
    \usefont{OT1}{phv}{m}{n}
}
\postauthor{
  \par
  \end{flushright}
}
\predate{
  \begin{flushright}
  \large
  \lineskip 0.5em
  \usefont{OT1}{phv}{m}{n}
}
\postdate{
  \par
  \end{flushright}
}

% Mathematics Packages
\usepackage[Gray,squaren,thinqspace,cdot]{SIunits}      % elegant units
\usepackage{amsmath}                                    % extensive math options
\usepackage{amsfonts}                                   % special math fonts
\usepackage{mathtools}                                  % useful formatting commands
\usepackage{amsthm}                                     % useful commands for building theorem environments
\usepackage{amssymb}                                    % lots of special math symbols
\usepackage{mathrsfs}                                   % fancy scripts letters
\usepackage{cancel}                                     % cancel lines in math
\usepackage{esint}                                      % fancy integral symbols
\usepackage{relsize}                                    % make math things bigger or smaller
\usepackage{bm}                                         % bold math!

\newcommand\ddfrac[2]{\frac{\displaystyle #1}{\displaystyle #2}}    % elegant fraction formatting
\allowdisplaybreaks[1]                                              % allow align environments to break on pages

% Ensure numbering is section-specific
%--------------------------------------------------------------------------------------------
\numberwithin{equation}{section}
\numberwithin{figure}{section}
\numberwithin{table}{section}

% Citations, references, and annotations
%--------------------------------------------------------------------------------------------
\usepackage[small,bf,hang]{caption}        % captions
\usepackage{subcaption}                    % adds subfigure & subcaption
\usepackage{sidecap}                       % adds side captions
\usepackage{hyperref}                      % add hyperlinks to references
\usepackage[noabbrev,nameinlink]{cleveref} % better references than default \ref
\usepackage{autonum}                       % only number referenced equations
\usepackage{url}                           % urls
\usepackage{cite}                          % well formed numeric citations
% format hyperlinks
\colorlet{linkcolour}{black}
\colorlet{urlcolour}{blue}
\hypersetup{colorlinks=true,
            linkcolor=linkcolour,
            citecolor=linkcolour,
            urlcolor=urlcolour}

% Plotting and Figures
%--------------------------------------------------------------------------------------------
\usepackage{tikz}          % advanced vector graphics
\usepackage{pgfplots}      % data plotting
\usepackage{pgfplotstable} % table plotting
\usepackage{placeins}      % display floats in correct sections
\usepackage{graphicx}      % include external graphics
\usepackage{longtable}     % process long tables

% use most recent version of pgfplots
\pgfplotsset{compat=newest}

% Misc.
%--------------------------------------------------------------------------------------------
\usepackage{todonotes}  % add to do notes
\usepackage{epstopdf}   % process eps-images
\usepackage{float}      % floats
\usepackage{stmaryrd}   % some more nice symbols
\usepackage{emptypage}  % suppress page numbers on empty pages
\usepackage{multicol}   % use this for creating pages with multiple columns
\usepackage{etoolbox}   % adds tags for environment endings
\usepackage{tcolorbox}  % pretty colored boxes!


% Custom Commands
%--------------------------------------------------------------------------------------------
\newcommand\hr{\noindent\rule[0.5ex]{\linewidth}{0.5pt}}                % horizontal line
\newcommand\N{\ensuremath{\mathbb{N}}}                                  % blackboard set characters
\newcommand\R{\ensuremath{\mathbb{R}}}
\newcommand\Z{\ensuremath{\mathbb{Z}}}
\newcommand\Q{\ensuremath{\mathbb{Q}}}
\newcommand\C{\ensuremath{\mathbb{C}}}
\renewcommand{\arraystretch}{1.2}                                       % More space between table rows (could be 1.3)
\newcommand{\Cov}{\mathrm{Cov}}
\newcommand*{\dbar}{\ensuremath{\text{\dj}}}
% Custom Environments
%--------------------------------------------------------------------------------------------
\newcommand{\lecture}[3]{\hr\\{\centering{\large\textsc{Lecture #1: #3}}\\#2\\}\hr\markboth{Lecture #1: #3}{\rightmark}}   % command to title lectures
\usepackage{mdframed}
\theoremstyle{plain}
\newmdtheoremenv[nobreak]{theorem}{Theorem}[section]
\newtheorem{corollary}{Corollary}[theorem]
\newtheorem{lemma}[theorem]{Lemma}
\theoremstyle{definition}
\newtheorem*{ex}{Example}
\newmdtheoremenv[nobreak]{definition}{Definition}[section]
\theoremstyle{remark}
\newtheorem*{remark}{Remark}
\AtEndEnvironment{ex}{\null\hfill$\diamond$}%
% Note: A proof environment is already provided in the amsthm package
\tcbuselibrary{breakable}
\newenvironment{note}[1]{\begin{tcolorbox}[
    arc=0mm,
    colback=white,
    colframe=white!60!black,
    title=#1,
    fonttitle=\sffamily,
    breakable
]}{\end{tcolorbox}}
\newenvironment{problem}{\begin{tcolorbox}[
    arc=0mm,
    breakable,
    colback=white,
    colframe=black
]}{\end{tcolorbox}}

% Header and Footer
%--------------------------------------------------------------------------------------------
% set header and footer
\usepackage{fancyhdr}                       % header and footer
\pagestyle{fancy}                           % use package
\fancyhf{}
\fancyhead[LE,RO]{\textsl{\rightmark}}      % E for even (left pages), O for odd (right pages)
\fancyfoot[LE,RO]{\thepage}
\fancyfoot[LO,RE]{\textsl{\leftmark}}
\setlength{\headheight}{15pt}


% Physics
%--------------------------------------------------------------------------------------------
\usepackage[arrowdel]{physics}      % all the usual useful physics commands
%\usepackage{feyn}                   % for drawing Feynman diagrams
%\usepackage{bohr}                   % for drawing Bohr diagrams
\usepackage{elements}               % for quickly referencing information of various elements
\usepackage{tensor}                 % for writing tensors and chemical symbols

% Finishing touches
%--------------------------------------------------------------------------------------------
\author{Nathaniel D. Hoffman}

\title{33-756 Homework 7}
\date{\today}
\begin{document}
\maketitle

\section*{1. Particle in Harmonic Oscillator Subject to Time-Dependent Force}
A particle in a harmonic oscillator potential is subject to a time-dependent force $ H = A x^2 e^{- t / \tau} $ that turns on at $ t = 0 $, where the system starts in the ground state. Find the probability of finding the system in the $ n $th excited state after a time $ t $. Assume $ t >> \tau $.

\begin{problem}
    To first-order, the probability of going from the ground state to the $ n $th excited state is
    \begin{equation}
        \Pr(0 \to n) = \abs{c_n^{(1)}}^2
    \end{equation}
    where
    \begin{align}
        c_n^{(1)} &= - \frac{\imath}{\hbar} \int_0^{t} A\bra{n}x^2\ket{0} e^{-t' / \tau} e^{\imath (E_n - E_0)} \dd{t'} \\
        &= - \frac{\imath}{\hbar} A \int_0^t\bra{n} x^2\ket{0} e^{-t' / \tau} e^{\imath \hbar \omega n} \dd{t'}
    \end{align}
    Next, we can write $\bra{n} = \frac{1}{\sqrt{n!}}\bra{0} a^n $ and $ x^2 = \frac{\hbar}{2m \omega} (a^\dagger + a)^2 $:
    \begin{equation}
        c_n^{(1)} = - \frac{\imath}{\hbar} \frac{\hbar A}{2m \omega \sqrt{n!}} e^{\imath \hbar \omega n} \int_0^t\bra{0} a^n (a^\dagger + a)(a^\dagger + a)\ket{0} e^{-t' / \tau} \dd{t'}
    \end{equation}
    \begin{align}
        \bra{0} a^n (a^\dagger + a)^2\ket{0} &=\bra{0} a^n ( (a^\dagger)^2 + aa^\dagger + a^\dagger a + a^2)\ket{0} \\
        &=\bra{0} a^n (a^\dagger)^2 + a^{n+1} a^\dagger + a^n a^\dagger a + a^{n+2}\ket{0} \\
        &=\bra{0} a^{n-2} (a a^\dagger)^2 +a^{n} (a a^\dagger) + a^n a^\dagger a _ a^{n+2}\ket{0} \\
        &=\bra{0} a^{n-2} (N+1)^2 + a^{n} (N+1) + a^{n} N + a^{n+2}\ket{0} \\
        &=\bra{0} a^{n-2} (N^2 + 2 N + 1) + a^n N + a^n + a^n N + a^{n+2}\ket{0} \\
        &=\bra{n - 2} N^2 + 2N + 1\ket{0}\sqrt{(n-2)!} +\bra{n} 2N + 1\ket{0} \sqrt{n!} +\bra{n+2}\ket{0} \sqrt{(n+2)!} \\
        &= \delta_{n-2,0} (0^2 + 2\times 0 + 1) \sqrt{(n-2)!} + \delta_{n,0} (2N +1) \sqrt{n!} + \delta_{n+2,0} \sqrt{(n+2)!} \\
        &= \begin{cases} 1 & n = 0,2, (-2)\\ 0 & \text{otherwise} \end{cases}
   \end{align}

   We only care about transitions to excited states, so we can ignore the $ 0 $ case:
   \begin{align}
       c_n^{(1)} &= - \frac{\imath}{\hbar} \frac{\hbar A}{2m \omega \sqrt{n!}} e^{\imath \hbar \omega n} \int_0^t e^{-t' / \tau} \dd{t'} \\
       &= - \frac{\imath A e^{\imath \hbar \omega 2}}{2m \omega \sqrt{2}} \tau \left( 1 - e^{-t / \tau} \right)
   \end{align}
   so
   \begin{equation}
       \Pr(0 \to n \neq 0) = \begin{cases} \frac{\abs{A}^2}{8m^2 \abs{\omega}^2} e^{-4 \hbar \Im{\omega}} \tau^2 \abs{1- e^{-t / \tau}}^2 & n = 2 \\ 0 & \text{otherwise} \end{cases}
   \end{equation}
\end{problem}

\section*{2. A Two-Level System with Time-Dependent Potential}
Consider a two-level system with $ E_1 < E_2 $. There is a time-dependent potential that connects the two levels as follows:
\begin{equation}
    V_{11} = V_{22} = 0,\qquad V_{12} = \gamma e^{\imath \omega t},\qquad \gamma e^{- \imath \omega t} \qquad (\gamma \in \R).
\end{equation}
At $ t = 0 $, it is known that only the lower level is populated\textemdash that is, $ c_1(0) = 1 $, $ c_2(0) = 0 $.
\begin{itemize}
    \item[(a)] Find $ \abs{c_1(t)}^2 $ and $ \abs{c_2(t)}^2 $ for $ t > 0 $ by \textit{exactly} solving the coupled differential equation
        \begin{equation}
            \imath \hbar \dot{c}_k = \sum_{n=1}^{2} V_{kn}(t) e^{\imath \omega_{kn} t} c_n,\qquad (k = 1,2).
        \end{equation}
        You should find Rabi's formula:
        \begin{align}
            \abs{c_2(t)}^2 &= \frac{\gamma^2 / \hbar^2}{\gamma^2 / \hbar^2 + (\omega - \omega_{21})^2 / 4} \sin[2](\left[ \frac{\gamma^2}{\hbar^2} + \frac{(\omega - \omega_{21})^2}{4} \right]^{1/2} t),\\
            \abs{c_1(t)}^2 &= 1 - \abs{c_2(t)}^2.
        \end{align}
        \begin{problem}
            We can write out our system of equations as
           \begin{align}
               \imath \hbar \dot{c}_1 &= \gamma e^{\imath \omega t} e^{\imath \omega_{12} t} c_2 = \gamma c_2 e^{\imath (\omega - \omega_{21}) t} \\
               \imath \hbar \dot{c}_2 &= \gamma e^{- \imath \omega t} e^{\imath \omega_{21} t} c_1 = \gamma c_1 e^{-\imath (\omega - \omega_{21}) t}
           \end{align}
           since $ \omega_{12} = - \omega_{21} $. We can write $ \omega_0 = \omega - \omega_{21} $ and rewrite the second equation as
           \begin{equation}
               \frac{\imath \hbar}{\gamma} \dot{c}_2 e^{\imath \omega_0 t} = c_1
           \end{equation}
           Taking the derivative of both sides, we get
           \begin{equation}
               \dot{c_1} = \frac{\imath \hbar}{\gamma} \left( \ddot{c}_2 e^{\imath \omega_0 t} + \dot{c}_2 \imath \omega_0 e^{\imath \omega_0 t} \right)
           \end{equation}
           We then plug this into the first equation:
           \begin{align}
               - \frac{\hbar^2}{\gamma} \left( \ddot{c}_2 e^{\imath \omega_0 t} + \dot{c}_2 \imath \omega_0 e^{\imath \omega_0 t}\right) &= \gamma c_2 e^{\imath \omega_0 t} \\
                \ddot{c_2} \hbar^2 + \dot{c}_2 \imath \hbar^2 \omega_0 + \gamma^2 c_2 &= 0
           \end{align}
           We want to solve this using the boundary condition that $ c_2(0) = 0 $, and doing so, we find that
           \begin{equation}
               c_2(t) = A \left( e^{- \imath t \left( \frac{\omega}{2} + \Omega \right)} - e^{- \imath t \left( \frac{\omega}{2} - \Omega \right)} \right)
           \end{equation}
           where $ \Omega = \sqrt{\frac{\gamma^2}{\hbar^2} + \frac{\omega_0^2}{4}} $. To find $ A $, we need to look at the second initial equation at $ t=0 $:
           \begin{equation}
               \imath \hbar \dot{c}_2(0) = \gamma c_1(0) \implies \dot{c}_2(0) = \frac{\gamma}{\imath \hbar}
           \end{equation}
           Taking the derivative of our solution for $ c_2(t) $ and evaluating at $ 0 $, we see that
           \begin{equation}
               A = \frac{\gamma}{2 \Omega \hbar}
           \end{equation}
           Our final answer for $ c_2(t) $ is
           \begin{equation}
               c_2(t) = \frac{\gamma}{2 \Omega \hbar} \left( e^{- \imath t \left( \frac{\omega}{2} + \Omega \right)} - e^{- \imath t \left( \frac{\omega}{2} - \Omega \right)} \right)
           \end{equation}
           Therefore,
           \begin{equation}
               \abs{c_2(t)}^2 = 4 \abs{A}^2 \sin[2](t \Omega) =  \frac{\gamma^2 / \hbar^2}{\gamma^2 / \hbar^2 + (\omega - \omega_{21})^2 / 4} \sin[2](\left[ \frac{\gamma^2}{\hbar^2} + \frac{(\omega - \omega_{21})^2}{4} \right]^{1/2} t)
           \end{equation}
           The solution for $ c_1(t) $ follows a similar yet equally boring process, and the general result is obvious from a probabilistic standpoint:
           \begin{equation}
               \abs{c_1(t)}^2 = 1 - \abs{c_2(t)}^2
           \end{equation}
       \end{problem}
   \item[(b)] Do the same problem using time-dependent perturbation theory to lowest non-vanishing order. Compare the two approaches for small values of $ \gamma $. Treat the following two cases separately: (i) $ \omega $ very different form $ \omega_{21} $ and (ii) $ \omega $ close to $ \omega_{21} $.
        \begin{problem}
            The derivation with perturbation theory is actually much simpler, since, to first-order,
            \begin{equation}
                c_2^{(1)}(t) = - \frac{\imath}{\hbar} \int_0^t e^{\imath \omega_{21} t'} \gamma e^{- \imath \omega t'} \dd{t'} = \gamma \frac{e^{- \imath t (\omega - \omega_{21})} - 1}{\hbar (\omega - \omega_{21})} = \gamma \frac{e^{- \imath t \omega_0} - 1}{\hbar \omega_0}
            \end{equation}
            and
            \begin{equation}
                \abs{c_2^{(1)}}^2 = \frac{\gamma^2}{\hbar^2 \frac{\omega_0^2}{4}} \sin[2](\frac{\omega_0}{2} t)
            \end{equation}
            This solution is nearly equivalent to the exact solution for case (i), since when $ \omega $ is very different from $ \omega_{21} $, $ \omega_0 $ is large compared to $ \gamma $ so $ \Omega \to \frac{\omega_0}{2} $. When $ \omega $ is close to $ \omega_{21} $ (ii) we find that these solutions become very different, since $ \Omega \to \frac{\gamma}{\hbar} $, which is not reflected at all in the perturbation theory solution.
        \end{problem}
\end{itemize}

\section*{3. Hydrogen Atom in Time-Dependent Potential}
A Hydrogen atom is subject to an external potential $ V = V_0 \cos(\omega t - kz) $. The initial state is the ground state. Calculate the rate the electron is ejected into the final plane wave state at a given angle. For the final state, put the system in a finite box so that $ \psi_f(x) = L^{-3/2} e^{\imath \va{p} \vdot \va{x} / \hbar} $. Your final answer should not depend upon $ L $.
\begin{problem}
    From Fermi's Golden rule, we know that
    \begin{equation}
        w_{\text{abs}} = \frac{2 \pi}{\hbar} \left( \frac{V_0}{2} \right)^2 \abs{\mel{f}{e^{\imath k z}}{0}}^2 \delta(E_f - E_0 - \hbar \omega)
    \end{equation}
    where we divide $ V_0 $ by $ 2 $ because we only care about the absorption part of the cycle, and by absorption we technically mean absorption to the final (emitted) state. All we have to do is evaluate the matrix element for an arbitrary scattering angle. Using the given final state and the known ground state of Hydrogen,
    \begin{align}
        \mel{f}{e^{\imath k z}}{0} &= \frac{1}{\sqrt{\pi}} \left( \frac{1}{a_0 L} \right)^{3/2} \int \dd[3]{x'} e^{\imath (kz' - \va{p} \vdot \va{x}')} e^{-r'/ a_0} \\
        &= \frac{1}{\sqrt{\pi}} \left( \frac{1}{a_0 L} \right)^{3/2} \int \dd[3]{x'} e^{\imath r' (k \cos(\theta') - (p/ \hbar) \cos(\theta'))} e^{- r'/a_0} \\
        &= \frac{1}{\sqrt{\pi}} \left( \frac{1}{a_0 L} \right)^{3/2} 2 \pi \int_0^{\infty} \dd{r'} r'^2 e^{-r' / a_0} \int_{-1}^1 \dd{\cos(\theta')} e^{\imath r' (k - p / \hbar) \cos(\theta')} \\
        &= \frac{1}{\sqrt{\pi}} \left( \frac{1}{a_0 L} \right)^{3/2} \frac{8 \pi a_0^3}{\left( 1 + a_0^2 \left( k - \frac{p}{\hbar} \right)^2 \right)^2} 
    \end{align}
    so
    \begin{equation}
        w_{\text{abs}} = \frac{32 a_0^3 \pi^2 V^2 \hbar^7}{L^3 \left( \hbar^2 + a_0^2 (p - k \hbar)^2 \right)^4} \delta(E_f - E_0 - \hbar \omega)
    \end{equation}
    Inserting the appropriate energies for the ground state of an electron in a Hydrogen atom and a particle in a box, we find
    \begin{equation}
        w_{\text{abs}} = \frac{32 a_0^3 \pi^2 V^2 \hbar^7}{L^3 \left( \hbar^2 + a_0^2 (p - k \hbar)^2 \right)^4} \delta\left(\frac{4 \pi^2 \hbar^2 n^2}{2m_e L} + \frac{m_e e^4}{2 \hbar^2} - \hbar \omega\right)
    \end{equation}
    This $\delta$-function sets
    \begin{equation}
        L = \frac{2n \pi \hbar^2}{\sqrt{2m \omega \hbar^3 - e^4 m^2}}
    \end{equation}
    so
    \begin{equation}
        w_{\text{abs}} = \frac{4 a_0^3 V^2 \hbar m^{3/2}}{\pi n^3} \frac{\left( 2 \omega \hbar^3 - m e^4 \right)^{3/2}}{(\hbar^2 + a_0^2 (p-k \hbar)^2)^4}
    \end{equation}
    I know this is not correct but I honestly don't understand the proper steps to get the correct answer.
\end{problem}


\end{document}
