\documentclass[a4paper,twoside]{article}
% My LaTeX preamble file - by Nathaniel Dene Hoffman
% Credit for much of this goes to Olivier Pieters (https://olivierpieters.be/tags/latex)
% and Gilles Castel (https://castel.dev)
% There are still some things to be done:
% 1. Update math commands using mathtools package (remove ddfrac command and just override)
% 2. Maybe abbreviate \imath somehow?
% 3. Possibly format for margin notes and set new margin sizes
% First, some encoding packages and useful formatting
%--------------------------------------------------------------------------------------------
\usepackage{import}
\usepackage{pdfpages}
\usepackage{transparent}
\usepackage[l2tabu,orthodox]{nag}   % force newer (and safer) LaTeX commands
\usepackage[utf8]{inputenc}         % set character set to support some UTF-8
                                    %   (unicode). Do NOT use this with
                                    %   XeTeX/LuaTeX!
\usepackage[T1]{fontenc}
\usepackage[english]{babel}         % multi-language support
\usepackage{sectsty}                % allow redefinition of section command formatting
\usepackage{tabularx}               % more table options
\usepackage{booktabs}
\usepackage{titling}                % allow redefinition of title formatting
\usepackage{imakeidx}               % create and index of words
\usepackage{xcolor}                 % more colour options
\usepackage{enumitem}               % more list formatting options
\usepackage{tocloft}                % redefine table of contents, new list like objects
\usepackage{subfiles}               % allow for multifile documents

% Next, let's deal with the whitespaces and margins
%--------------------------------------------------------------------------------------------
\usepackage[centering,margin=1in]{geometry}
\setlength{\parindent}{0cm}
\setlength{\parskip}{2ex plus 0.5ex minus 0.2ex} % whitespace between paragraphs

% Redefine \maketitle command with nicer formatting
%--------------------------------------------------------------------------------------------
\pretitle{
  \begin{flushright}         % align text to right
    \fontsize{40}{60}        % set font size and whitespace
    \usefont{OT1}{phv}{b}{n} % change the font to bold (b), normally shaped (n)
                             %   Helvetica (phv)
    \selectfont              % force LaTeX to search for metric in its mapping
                             %   corresponding to the above font size definition
}
\posttitle{
  \par                       % end paragraph
  \end{flushright}           % end right align
  \vskip 0.5em               % add vertical spacing of 0.5em
}
\preauthor{
  \begin{flushright}
    \large                   % font size
    \lineskip 0.5em          % inter line spacing
    \usefont{OT1}{phv}{m}{n}
}
\postauthor{
  \par
  \end{flushright}
}
\predate{
  \begin{flushright}
  \large
  \lineskip 0.5em
  \usefont{OT1}{phv}{m}{n}
}
\postdate{
  \par
  \end{flushright}
}

% Mathematics Packages
\usepackage[Gray,squaren,thinqspace,cdot]{SIunits}      % elegant units
\usepackage{amsmath}                                    % extensive math options
\usepackage{amsfonts}                                   % special math fonts
\usepackage{mathtools}                                  % useful formatting commands
\usepackage{amsthm}                                     % useful commands for building theorem environments
\usepackage{amssymb}                                    % lots of special math symbols
\usepackage{mathrsfs}                                   % fancy scripts letters
\usepackage{cancel}                                     % cancel lines in math
\usepackage{esint}                                      % fancy integral symbols
\usepackage{relsize}                                    % make math things bigger or smaller
%\usepackage{bm}                                         % bold math!
\usepackage{slashed}

\newcommand\ddfrac[2]{\frac{\displaystyle #1}{\displaystyle #2}}    % elegant fraction formatting
\allowdisplaybreaks[1]                                              % allow align environments to break on pages

% Ensure numbering is section-specific
%--------------------------------------------------------------------------------------------
\numberwithin{equation}{section}
\numberwithin{figure}{section}
\numberwithin{table}{section}

% Citations, references, and annotations
%--------------------------------------------------------------------------------------------
\usepackage[small,bf,hang]{caption}        % captions
\usepackage{subcaption}                    % adds subfigure & subcaption
\usepackage{sidecap}                       % adds side captions
\usepackage{hyperref}                      % add hyperlinks to references
\usepackage[noabbrev,nameinlink]{cleveref} % better references than default \ref
\usepackage{autonum}                       % only number referenced equations
\usepackage{url}                           % urls
\usepackage{cite}                          % well formed numeric citations
% format hyperlinks
\colorlet{linkcolour}{black}
\colorlet{urlcolour}{blue}
\hypersetup{colorlinks=true,
            linkcolor=linkcolour,
            citecolor=linkcolour,
            urlcolor=urlcolour}

% Plotting and Figures
%--------------------------------------------------------------------------------------------
\usepackage{tikz}          % advanced vector graphics
\usepackage{pgfplots}      % data plotting
\usepackage{pgfplotstable} % table plotting
\usepackage{placeins}      % display floats in correct sections
\usepackage{graphicx}      % include external graphics
\usepackage{longtable}     % process long tables

% use most recent version of pgfplots
\pgfplotsset{compat=newest}

% Misc.
%--------------------------------------------------------------------------------------------
\usepackage{todonotes}  % add to do notes
\usepackage{epstopdf}   % process eps-images
\usepackage{float}      % floats
\usepackage{stmaryrd}   % some more nice symbols
\usepackage{emptypage}  % suppress page numbers on empty pages
\usepackage{multicol}   % use this for creating pages with multiple columns
\usepackage{etoolbox}   % adds tags for environment endings
\usepackage{tcolorbox}  % pretty colored boxes!


% Custom Commands
%--------------------------------------------------------------------------------------------
\newcommand\hr{\noindent\rule[0.5ex]{\linewidth}{0.5pt}}                % horizontal line
\newcommand\N{\ensuremath{\mathbb{N}}}                                  % blackboard set characters
\newcommand\R{\ensuremath{\mathbb{R}}}
\newcommand\Z{\ensuremath{\mathbb{Z}}}
\newcommand\Q{\ensuremath{\mathbb{Q}}}
%\newcommand\C{\ensuremath{\mathbb{C}}}
\renewcommand{\arraystretch}{1.2}                                       % More space between table rows (could be 1.3)
\newcommand{\Cov}{\mathrm{Cov}}
\newcommand\D{\mathrm{D}}
\newcommand*{\dbar}{\ensuremath{\text{\dj}}}

\newcommand{\incfig}[2][1]{%
    \def\svgwidth{#1\columnwidth}
    \import{./figures/}{#2.pdf_tex}
}

% Custom Environments
%--------------------------------------------------------------------------------------------
\newcommand{\lecture}[3]{\hr\\{\centering{\large\textsc{Lecture #1: #3}}\\#2\\}\hr\markboth{Lecture #1: #3}{\rightmark}}   % command to title lectures
\usepackage{mdframed}
\theoremstyle{plain}
\newmdtheoremenv[nobreak]{theorem}{Theorem}[section]
\newtheorem{corollary}{Corollary}[theorem]
\newtheorem{lemma}[theorem]{Lemma}
\theoremstyle{definition}
\newtheorem*{ex}{Example}
\newmdtheoremenv[nobreak]{definition}{Definition}[section]
\theoremstyle{remark}
\newtheorem*{remark}{Remark}
\newtheorem*{claim}{Claim}
\AtEndEnvironment{ex}{\null\hfill$\diamond$}%
% Note: A proof environment is already provided in the amsthm package
\tcbuselibrary{breakable}
\newenvironment{note}[1]{\begin{tcolorbox}[
    arc=0mm,
    colback=white,
    colframe=white!60!black,
    title=#1,
    fonttitle=\sffamily,
    breakable
]}{\end{tcolorbox}}
\newenvironment{problem}{\begin{tcolorbox}[
    arc=0mm,
    breakable,
    colback=white,
    colframe=black
]}{\end{tcolorbox}}

% Header and Footer
%--------------------------------------------------------------------------------------------
% set header and footer
\usepackage{fancyhdr}                       % header and footer
\pagestyle{fancy}                           % use package
\fancyhf{}
\fancyhead[LE,RO]{\textsl{\rightmark}}      % E for even (left pages), O for odd (right pages)
\fancyfoot[LE,RO]{\thepage}
\fancyfoot[LO,RE]{\textsl{\leftmark}}
\setlength{\headheight}{15pt}


% Physics
%--------------------------------------------------------------------------------------------
\usepackage[arrowdel]{physics}      % all the usual useful physics commands
\usepackage{feyn}                   % for drawing Feynman diagrams
%\usepackage{bohr}                   % for drawing Bohr diagrams
%\usepackage{tikz-feynman}
\usepackage{elements}               % for quickly referencing information of various elements
\usepackage{tensor}                 % for writing tensors and chemical symbols

% Finishing touches
%--------------------------------------------------------------------------------------------
\author{Nathaniel D. Hoffman}

\title{33-756 Homework 9}
\date{\today}
\begin{document}
\maketitle
\section*{1. Generating Functions}
In class, we discussed the generating functions of canonical transformations. These are functions $ F(q, Q, p, P) $ where the old variables are $ \{q,p\} $ and the new variables are $ \{Q, P\} $. Why are we calling these the ``generating functions''? This problem explores this question.
\begin{itemize}
    \item[(a)] Begin by showing that $ F = qP $ leads to the identity transformation $ q = Q $, $ p = P $.
        \begin{problem}
            We begin with $ F_2 = qP $ and $ F = F_2 - QP $:
            \begin{align}
                p\dot{q} - H' &= P\dot{Q} - K + \pdv{F}{t} \\
                &= P\dot{Q} - K + \pdv{F}{t} + \pdv{F}{q}\dot{q} + \pdv{F}{P}\dot{P} \\
                &= -Q\dot{P} - K + P\dot{q} + q\dot{P}
            \end{align}
            Matching up the derivatives from each side, we find that
            \begin{equation}
                p\dot{q} = P\dot{q} \qand Q\dot{P} = q\dot{P}
            \end{equation}
            so $ P = p $ and $ Q = q $.
        \end{problem}
    \item[(b)] Next, consider an infinitesimal transformation $ F = qP + \epsilon G(q,P) $. Show that
        \begin{equation}
            \delta q = Q - q = \epsilon \pb{q}{G(q,P)} \approx \epsilon \pb{q}{G(q,p)} 
        \end{equation}
        \begin{equation}
            \delta p = P - p = \epsilon \pb{p}{G(q,P)} \approx \epsilon \pb{p}{G(q,p)}
        \end{equation}
        where $ G(q,p) $ is the generator of the transformation. Then compare this to how operators shift under symmetry transformations when we talk about the notion of generators. Compare the classical and quantum results to explain the terminology. Then show that classically, $ H $ generates time translations, $ p $ generates spatial translations, and $ \va{L} $ generates rotations. Note the $ F $'s are often called the ``generating functions'' but $ G $'s are called the generators. Finally, we see that symmetries and conservation laws are identical in the sense that any generator which is conserved in time also generates a symmetry of the system.
        \begin{problem}
            \begin{equation}
                F = qP + \epsilon G(q,P) - QP
            \end{equation}
            so
            \begin{equation}
                \pdv{F}{t} = \left( P + \epsilon \pdv{G}{q} \right)\dot{q} + \left( q + \epsilon \pdv{G}{P} \right)\dot{P} - Q\dot{P} - P\dot{Q}
            \end{equation}
            so
            \begin{equation}
                p\dot{q} - H = - Q\dot{P} - K + \left( P + \epsilon \pdv{G}{q} \right)\dot{q} + \left( q + \epsilon \pdv{G}{P} \right)\dot{P}
            \end{equation}
            Again, matching derivatives, we find that
            \begin{equation}
                p = P + \epsilon \pdv{G}{q} \qand Q = q + \epsilon \pdv{G}{P}
            \end{equation}
            so
            \begin{equation}
                \delta p = - \epsilon \pdv{G}{q} \qand \delta q = \epsilon \pdv{G}{P}
            \end{equation}
            In quantum mechanics, symmetry operations act in such a way that there is a direct equivalence between $ \pb{q}{G} $ and $ \frac{\imath}{\hbar} \comm{q}{G} $ and so on. We can write the Poisson bracket as
            \begin{equation}
                \pb{f}{g} = \pdv{f}{q} \pdv{g}{p} - \pdv{f}{p} \pdv{g}{q}
            \end{equation}
            The action of the Hamiltonian will therefore be
            \begin{align}
                \epsilon \pb{q}{H} = \epsilon \pdv{H}{q} = \epsilon \dot{q} = \delta q \qand \epsilon \pb{p}{H} = - \epsilon \pdv{H}{p} = \epsilon \dot{p} = \delta p
            \end{align}
            according to Hamilton's equations of motion. Therefore the Hamiltonian is the generator of time translations. The action of the momentum is
            \begin{equation}
                \epsilon \pb{q}{p} = \epsilon \pdv{q}{q} \pdv{p}{p} - \epsilon \pdv{q}{p} \pdv{p}{q} = \epsilon = \delta q
            \end{equation}
            and $ \pb{p}{p} = 0 $, so $ p $ generates infinitesimal translations.
            Finally, the action of the angular momentum vector $ L_{\alpha} $ will be
            \begin{align}
                \delta q_i = \epsilon \pb{q_i}{n_{\alpha}\epsilon_{\alpha jk} q_j p_k} &= - \epsilon n_{\alpha}\epsilon_{\alpha jk} \pb{q_j p_k}{q_i} \\
                &= - \epsilon n_{\alpha} \epsilon_{\alpha jk}\left( \pb{q_j}{q_i} p_k + q_j\pb{p_k}{q_i} \right) \\
                &= \epsilon n_{\alpha} \epsilon_{\alpha jk} q_j \delta_{ik} \\
                &= \epsilon n_{\alpha} \epsilon_{\alpha ji} q_j \\
                &= \epsilon \epsilon_{i \alpha j} n_{\alpha} q_j
            \end{align}
            so $ \delta \va{q} = \epsilon (\vu{n} \cross \va{q}) $, motion perpendicular to both the direction of the angular momentum vector (I define $ \va{L} \sim \vu{n} $) and the direction of the position vector. Equivalently,
            \begin{align}
                \delta p_i &= - \epsilon n_{\alpha} \epsilon_{\alpha jk} \left( \pb{q_j}{p_i} p_k + q_j \pb{p_k}{p_i} \right) \\
                &= - \epsilon n_{\alpha}\epsilon_{\alpha jk} \delta_{ij} p_k \\
                &= - \epsilon n_{\alpha} \epsilon_{\alpha ik} p_k \\
                &= \epsilon \epsilon_{i \alpha k} n_{\alpha} p_k
            \end{align}
            so $ \delta \va{p} = \epsilon (\vu{n} \cross \va{p}) $, a rotation around $ \vu{n} $.
        \end{problem}
    \item[(c)] Finally we would like to show that the canonical transformation preserves the Poisson bracket. Show that this is true for an infinitesimal transformation. Hint: Consider the action of the generator on $ \pb{q}{p} $ and show that it is zero, that is, it leaves the Poisson bracket invariant. Utilize the Jacobi identity obeyed by any anti-symmetric bracket operation:
        \begin{equation}
            \comm{A}{\comm{B}{C}} + \comm{C}{\comm{A}{B}} + \comm{B}{\comm{C}{A}} = 0.
        \end{equation}
        Note that proving invariance under an infinitesimal transformation is sufficient to prove invariance under finite transformations as long as the finite transformation can be reached by a sequence of infinitesimal transformations. This means that in the space of transformations, the system will be invariant under all transformations that can be continuously connected to the identity transformation.
        \begin{problem}
            The action of a generator on $ \pb{q}{p} $ is $ \pb{G}{\pb{q}{p}} $, so the Jacobi relation tells us that
            \begin{align}
                \pb{G}{\pb{q}{p}} &= -\pb{p}{\pb{G}{q}} - \pb{q}{\pb{p}{G}} \\
                &= - \pb{p}{- \pdv{G}{p}} - \pb{q}{- \pdv{G}{q}} \\
                &= - \left( -\pdv{p}{q}\pdv[2]{G}{p} + \pdv{G}{p}{q} \pdv{p}{p} \right) - \left( -\pdv{q}{q} \pdv{G}{q}{p} + \pdv[2]{G}{q}\right) \\
                &= - \pdv{G}{p}{q} + \pdv{G}{q}{p} \\
                &= 0
            \end{align}
        \end{problem}
\end{itemize}


\section*{2. Ambiguity in Path Integral Formulation}
We said in class that given a description of a classical system with Hamiltonian $ H $, the quantum system is ambiguous because we don't know how to order operators which involve both $ p $ and $ q $. However, in the path integral, we don't deal with operators, so it seems like there is no ambiguity. Going through the derivation of the path integral, determine which step leads to an ambiguity.
\begin{problem}
    When we perform time slicing, we are in essence choosing an order of integration for $ x $ and $ p $, such that the integration
    \begin{equation}
        K = \int_{x_i}^{x_f} Dx Dp e^{\frac{\imath}{\hbar} \int_{t_i}^{t_f} S[x,p]}
    \end{equation}
    has ambiguous order in the way the integration is performed. By Fubini's theorem, the order of integration can only be switched if the integral remains finite when the integrand is replaced by its magnitude. This is not necessarily true, since we are integrating over all possible paths in $ x $ and in $ p $, and we need to choose an ordering prescription to give us the correct result from the Schr\"odinger equation.
\end{problem}

\section*{3. SHO in the WKB Approximation}
Calculate the energy levels of a SHO in the WKB approximation. How does this result compare to the exact one?
\begin{problem}
    We first need to find the classical turning points, which occur at $ E = V(x) $. In the case of the harmonic oscillator, $ V(x) = \frac{1}{2} m \omega^2 x^2 $, so
    \begin{equation}
        x_{\{1,2\}} = \mp \sqrt{\frac{2E}{m \omega^2}}
    \end{equation}
    B\"ohm-Sommerfeld quantization tells us that
    \begin{align}
        \int_{x_1}^{x_2} \sqrt{2m\left( E - \frac{1}{2} m \omega^2 x^2 \right)} \dd{x} &= \left( n + \frac{1}{2}\right) \hbar \pi \\
        \eval{\frac{1}{2} \sqrt{m(2E - m x^2 \omega^2)} \left( x + \frac{E\arctan(\sqrt{\frac{\frac{1}{2} m \omega^2 x^2}{E - \frac{1}{2}m \omega^2 x^2}})}{\sqrt{\frac{1}{2} m \omega^2 \left( E - \frac{1}{2} m \omega^2 x^2 \right)}} \right)}_{x_1}^{x_2} &= \left( n + \frac{1}{2} \right) \hbar \pi\\
        \frac{E \pi}{\omega} &= \left( n + \frac{1}{2} \right) \hbar \pi \\
        E &= \left( n + \frac{1}{2} \right) \hbar \omega
    \end{align}
    The conclusion from the WKB approximation is the same as the exact solution.
\end{problem}

\section*{4. Integral of a Quadratic Exponential}
In performing the path integral, we run into the following integral:
\begin{equation}
    I = \int \dd{z_i} e^{-z_a A_{ab} z_b}
\end{equation}
where $ A $ is a symmetric ($ N $ by $ N $ dimensional) matrix. Show that the result of the integral is given by
\begin{equation}
    I = \frac{\pi^{N/2}}{(\det{A})^{1/2}}.
\end{equation}
\begin{problem}
    We can begin by inserting the identity $ I = TT^{-1} $ such that $ T_{ij} $ is the $ j $th right eigenvector of $ A $. The similarity transformation $ T^{-1} A T $ results in a matrix in which the diagonal elements are the right eigenvalues of $ A $ and the off-diagonal elements are zero. We can represent this diagonalized matrix as $ \lambda_{\epsilon} \delta^{\epsilon}_{jk} $, where $ \lambda_{\epsilon} $ is the $ \epsilon $th right eigenvalue of $ A $ and $ \delta^{\epsilon}_{jk} = 1 $ if $ \epsilon = j = k $ and $ 0 $ otherwise:
    \begin{equation}
        I = \int \dd{z_i} e^{-z_A A_{ab} z_b} = \int \dd{z_i} e^{- z_a T_{aj} T_{j \alpha}^{-1} A_{\alpha \beta} T_{\beta k} T_{kb}^{-1} z_b}
    \end{equation}
    We can absorb the outermost $ T $ matrices into the variables of integration, effectively changing bases:
    \begin{equation}
        y_j \equiv z_a T_{aj}
    \end{equation}
    Because $ A $ is symmetric, $ T $ must be an orthogonal matrix, so it has unit determinant. We can also always find a unique matrix such that the determinant will also be positive. Therefore, $ \dd{z_i} = \dd{y_i} \det(T) = \dd{y_i} $ and $ T^{-1} = T^\top $, its transpose.
    \begin{align}
        I &= \int \dd{y_i} e^{- y_j T_{j \alpha}^{-1} A_{\alpha \beta} T_{\beta k} y_k} \\
        &= \int \dd{y_i} e^{- y_j \lambda_{\epsilon} \delta^{\epsilon}_{jk} y_k} \\
        &= \int \dd{y_i} e^{- \lambda_{\epsilon} y_{\epsilon}^2} \\
        &= \int \dd{y_i} \prod_{\epsilon=1}^{N} e^{- \lambda_{\epsilon} y_{\epsilon}^2} \\
        &= \prod_{\epsilon=1}^{N} \int \dd{y_{\epsilon}} e^{- \lambda_{\epsilon} y_{\epsilon}^2} \\
        &= \prod_{\epsilon=1}^{N} \sqrt{\frac{\pi}{\lambda_{\epsilon}}} \\
        &= \sqrt{\frac{\prod_{\epsilon=1}^{N} \pi}{\prod_{\epsilon=1}^{N} \lambda_{\epsilon}}} \\
        &= \sqrt{\frac{\pi^N}{\det(A)}} \\
        &= \frac{\pi^{N/2}}{\det(A)^{1/2}}
    \end{align}
    since, by definition, $ \det(A) $ is equal to the product of the eigenvalues of $ A $.
\end{problem}

\section*{5. Energy Fluctuations for the Harmonic Oscillator}
Derive the energy eigenvalues of the harmonic oscillator by taking the propagator for this system, analytically continuing via $ t = - \imath \tau $, setting $ x' = x $, and integrating over $ x $. Notice the remarkable relation between the path integral and the partition function in statistical mechanics. Both systems involve fluctuations. What is the difference between thermodynamic and quantum fluctuations?
\begin{problem}
    The propagator can be written
    \begin{equation}
        K(x'', t; x', t_0 = 0) = \sqrt{\frac{m \omega}{2 \pi \imath \hbar \sin(\omega t)}} e^{\frac{\imath m \omega}{2 \hbar \sin(\omega t)}\left( [{x''}^2 + {x'}^2] \cos(\omega t) - 2x''x' \right)}
    \end{equation}
    I assume I am using a different edition, and that I should change $ x'' = x' = x $ and integrate over $ x $:
    \begin{align}
        \int \dd{x} K(x, \tau; x, 0) &= \int \dd{x} \sqrt{\frac{m \omega}{2 \pi \imath \hbar \sin(- \imath \omega \tau)}} e^{\frac{\imath m \omega}{2 \hbar \sin(- \imath \omega \tau)} \left( 2x^2 (\cos(- \imath \omega \tau) - 1) \right)} \\
        &= \sqrt{\frac{m \omega}{2 \pi \hbar \sinh(\tau \omega)}} \int \dd{x} e^{- \frac{m \omega x^2}{\hbar \sinh(\tau \omega)} (\cosh(\tau \omega) - 1)} \\
        &=  \sqrt{\frac{m \omega}{2 \pi \hbar \sinh(\tau \omega)}} \left( \sqrt{\pi} \sqrt{\frac{\hbar \sinh(\tau \omega)}{m \omega (\cosh(\tau \omega) - 1)}} \right) \\
        &= \frac{1}{\sqrt{2(\cosh(\tau \omega) - 1)}} \\
        &= e^{- \frac{\tau \omega}{2}} \frac{1}{1 - e^{- \tau \omega}} \\
        &= \sum_{n=0} e^{- \tau \omega \left( n + \frac{1}{2} \right)}
    \end{align}
    We know the energy of the harmonic oscillator:
    \begin{equation}
        \sum_n e^{- \frac{\tau}{\hbar} E_n} = Z
    \end{equation}
    The integration is exactly a sum over states, which is equivalent to the partition function for the harmonic oscillator when $ \tau \to \beta \hbar $ where $ \beta = k_B T $.
\end{problem}





\end{document}
