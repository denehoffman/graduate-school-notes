\documentclass[a4paper,twoside]{article}
% My LaTeX preamble file - by Nathaniel Dene Hoffman
% Credit for much of this goes to Olivier Pieters (https://olivierpieters.be/tags/latex)
% and Gilles Castel (https://castel.dev)
% There are still some things to be done:
% 1. Update math commands using mathtools package (remove ddfrac command and just override)
% 2. Maybe abbreviate \imath somehow?
% 3. Possibly format for margin notes and set new margin sizes
% First, some encoding packages and usefull formatting
%--------------------------------------------------------------------------------------------
\usepackage[l2tabu,orthodox]{nag}   % force newer (and safer) LaTeX commands
\usepackage[utf8]{inputenc}         % set character set to support some UTF-8
                                    %   (unicode). Do NOT use this with
                                    %   XeTeX/LuaTeX!
\usepackage[T1]{fontenc}
\usepackage[english]{babel}         % multi-language support
\usepackage{sectsty}                % allow redefinition of section command formatting
\usepackage{tabularx}               % more table options
\usepackage{booktabs}
\usepackage{titling}                % allow redefinition of title formatting
\usepackage{imakeidx}               % create and index of words
\usepackage{xcolor}                 % more colour options
\usepackage{enumitem}               % more list formatting options
\usepackage{tocloft}                % redefine table of contents, new list like objects
\usepackage{subfiles}               % allow for multifile documents

% Next, let's deal with the whitespaces and margins
%--------------------------------------------------------------------------------------------
\usepackage[centering,margin=1in]{geometry}
\setlength{\parindent}{0cm}
\setlength{\parskip}{2ex plus 0.5ex minus 0.2ex} % whitespace between paragraphs

% Redefine \maketitle command with nicer formatting
%--------------------------------------------------------------------------------------------
\pretitle{
  \begin{flushright}         % align text to right
    \fontsize{40}{60}        % set font size and whitespace
    \usefont{OT1}{phv}{b}{n} % change the font to bold (b), normally shaped (n)
                             %   Helvetica (phv)
    \selectfont              % force LaTeX to search for metric in its mapping
                             %   corresponding to the above font size definition
}
\posttitle{
  \par                       % end paragraph
  \end{flushright}           % end right align
  \vskip 0.5em               % add vertical spacing of 0.5em
}
\preauthor{
  \begin{flushright}
    \large                   % font size
    \lineskip 0.5em          % inter line spacing
    \usefont{OT1}{phv}{m}{n}
}
\postauthor{
  \par
  \end{flushright}
}
\predate{
  \begin{flushright}
  \large
  \lineskip 0.5em
  \usefont{OT1}{phv}{m}{n}
}
\postdate{
  \par
  \end{flushright}
}

% Mathematics Packages
\usepackage[Gray,squaren,thinqspace,cdot]{SIunits}      % elegant units
\usepackage{amsmath}                                    % extensive math options
\usepackage{amsfonts}                                   % special math fonts
\usepackage{mathtools}                                  % useful formatting commands
\usepackage{amsthm}                                     % useful commands for building theorem environments
\usepackage{amssymb}                                    % lots of special math symbols
\usepackage{mathrsfs}                                   % fancy scripts letters
\usepackage{cancel}                                     % cancel lines in math
\usepackage{esint}                                      % fancy integral symbols
\usepackage{relsize}                                    % make math things bigger or smaller
\usepackage{bm}                                         % bold math!

\newcommand\ddfrac[2]{\frac{\displaystyle #1}{\displaystyle #2}}    % elegant fraction formatting
\allowdisplaybreaks[1]                                              % allow align environments to break on pages

% Ensure numbering is section-specific
%--------------------------------------------------------------------------------------------
\numberwithin{equation}{section}
\numberwithin{figure}{section}
\numberwithin{table}{section}

% Citations, references, and annotations
%--------------------------------------------------------------------------------------------
\usepackage[small,bf,hang]{caption}        % captions
\usepackage{subcaption}                    % adds subfigure & subcaption
\usepackage{sidecap}                       % adds side captions
\usepackage{hyperref}                      % add hyperlinks to references
\usepackage[noabbrev,nameinlink]{cleveref} % better references than default \ref
\usepackage{autonum}                       % only number referenced equations
\usepackage{url}                           % urls
\usepackage{cite}                          % well formed numeric citations
% format hyperlinks
\colorlet{linkcolour}{black}
\colorlet{urlcolour}{blue}
\hypersetup{colorlinks=true,
            linkcolor=linkcolour,
            citecolor=linkcolour,
            urlcolor=urlcolour}

% Plotting and Figures
%--------------------------------------------------------------------------------------------
\usepackage{tikz}          % advanced vector graphics
\usepackage{pgfplots}      % data plotting
\usepackage{pgfplotstable} % table plotting
\usepackage{placeins}      % display floats in correct sections
\usepackage{graphicx}      % include external graphics
\usepackage{longtable}     % process long tables

% use most recent version of pgfplots
\pgfplotsset{compat=newest}

% Misc.
%--------------------------------------------------------------------------------------------
\usepackage{todonotes}  % add to do notes
\usepackage{epstopdf}   % process eps-images
\usepackage{float}      % floats
\usepackage{stmaryrd}   % some more nice symbols
\usepackage{emptypage}  % suppress page numbers on empty pages
\usepackage{multicol}   % use this for creating pages with multiple columns
\usepackage{etoolbox}   % adds tags for environment endings
\usepackage{tcolorbox}  % pretty colored boxes!


% Custom Commands
%--------------------------------------------------------------------------------------------
\newcommand\hr{\noindent\rule[0.5ex]{\linewidth}{0.5pt}}                % horizontal line
\newcommand\N{\ensuremath{\mathbb{N}}}                                  % blackboard set characters
\newcommand\R{\ensuremath{\mathbb{R}}}
\newcommand\Z{\ensuremath{\mathbb{Z}}}
\newcommand\Q{\ensuremath{\mathbb{Q}}}
\newcommand\C{\ensuremath{\mathbb{C}}}
\renewcommand{\arraystretch}{1.2}                                       % More space between table rows (could be 1.3)
\newcommand{\Cov}{\mathrm{Cov}}
\newcommand*{\dbar}{\ensuremath{\text{\dj}}}
% Custom Environments
%--------------------------------------------------------------------------------------------
\newcommand{\lecture}[3]{\hr\\{\centering{\large\textsc{Lecture #1: #3}}\\#2\\}\hr\markboth{Lecture #1: #3}{\rightmark}}   % command to title lectures
\usepackage{mdframed}
\theoremstyle{plain}
\newmdtheoremenv[nobreak]{theorem}{Theorem}[section]
\newtheorem{corollary}{Corollary}[theorem]
\newtheorem{lemma}[theorem]{Lemma}
\theoremstyle{definition}
\newtheorem*{ex}{Example}
\newmdtheoremenv[nobreak]{definition}{Definition}[section]
\theoremstyle{remark}
\newtheorem*{remark}{Remark}
\AtEndEnvironment{ex}{\null\hfill$\diamond$}%
% Note: A proof environment is already provided in the amsthm package
\tcbuselibrary{breakable}
\newenvironment{note}[1]{\begin{tcolorbox}[
    arc=0mm,
    colback=white,
    colframe=white!60!black,
    title=#1,
    fonttitle=\sffamily,
    breakable
]}{\end{tcolorbox}}
\newenvironment{problem}{\begin{tcolorbox}[
    arc=0mm,
    breakable,
    colback=white,
    colframe=black
]}{\end{tcolorbox}}

% Header and Footer
%--------------------------------------------------------------------------------------------
% set header and footer
\usepackage{fancyhdr}                       % header and footer
\pagestyle{fancy}                           % use package
\fancyhf{}
\fancyhead[LE,RO]{\textsl{\rightmark}}      % E for even (left pages), O for odd (right pages)
\fancyfoot[LE,RO]{\thepage}
\fancyfoot[LO,RE]{\textsl{\leftmark}}
\setlength{\headheight}{15pt}


% Physics
%--------------------------------------------------------------------------------------------
\usepackage[arrowdel]{physics}      % all the usual useful physics commands
%\usepackage{feyn}                   % for drawing Feynman diagrams
%\usepackage{bohr}                   % for drawing Bohr diagrams
\usepackage{elements}               % for quickly referencing information of various elements
\usepackage{tensor}                 % for writing tensors and chemical symbols

% Finishing touches
%--------------------------------------------------------------------------------------------
\author{Nathaniel D. Hoffman}

\title{33-756 Homework 10}
\date{\today}
\begin{document}
\maketitle

\section*{1. Fresnel-Like Integrals}
\begin{itemize}
    \item[(a)] $ I = \int_{- \infty}^{\infty} e^{\imath \lambda x^2} \dd{x} $.
        \begin{problem}
            We can define the following parameterization of a closed loop in complex space: $ \Gamma = \gamma_1 + \gamma_2 + \gamma_3 $ where
            \begin{equation}
                \gamma_1\colon\quad x = t \qfor t \in [0,R)
            \end{equation}
            \begin{equation}
                \gamma_2\colon\quad x = R e^{\imath t} \qfor t \in \left[ 0, \frac{\pi}{4} \right].
            \end{equation}
            \begin{equation}
                \gamma_3\colon\quad x = t e^{\imath \frac{\pi}{4}}
            \end{equation}
            The integral we want to calculate is
            \begin{equation}
                I = 2\int_{\gamma_1} e^{\imath \lambda x^2} \dd{x}
            \end{equation}
            as $ R \to \infty $. In order to find this, we can show that the integral along $ \gamma_2 $ vanishes as $ R \to \infty $, and because $ e^{\imath \lambda x^2} $ is holomorphic, it will have no residues in this region so the integral around $ \Gamma $ will be equal to $ 0 $. Therefore, $ \int_{\gamma_1} - \int_{\gamma_2} = \int_{\gamma_3} $ (here I have defined $ \gamma_3 $ in the anticlockwise direction so as to cancel the minus sign).
            

            First, I will show that $ \int_{\gamma_2} \to 0 $ as $ R \to 0 $. The integral along this path is
            \begin{equation}
                I_{\gamma_2} = \int_{\gamma_2} e^{\imath \lambda x^2} \dd{x} = \int_0^{\pi / 4} e^{\imath \lambda R^2 (\cos(2t) + \imath \sin(2t))} i R e^{\imath t} \dd{t}
            \end{equation}
            since $ \dd{z} = \imath R e^{\imath t} \dd{t} $. We want to show that this integral vanishes, so we can equivalently show that its magnitude vanishes. by the triangle inequality,
            \begin{align}
                \abs{\int_0^{\pi / 4} e^{\imath \lambda R^2 (\cos(2t) + \imath \sin(2t))} \imath R e^{\imath t} \dd{t}} &\leq \int_0^{\pi / 4} \abs{e^{\imath \lambda R^2 (\cos(2t) + \imath \sin(2t))}} \abs{\imath R e^{\imath t}} \dd{t}\\
                &= R \int_0^{\pi / 4} e^{- \lambda R^2 \sin(2t)} \dd{t}
            \end{align}
            We can then use Jordan's inequality: $ \frac{4t}{\pi} \leq \sin(2t) \leq 2t $ for $ 0 \leq t \leq \frac{\pi}{4} $:
            \begin{equation}
                \abs{I_{\gamma_2}} \leq R \int_0^{\pi / 4} e^{-4 \lambda R^2 t / \pi} \dd{t} = \frac{\pi}{4 \lambda R} \left( 1 - e^{- \lambda R^2} \right)
            \end{equation}
            It is obvious from here that this vanishes as $ R \to \infty $, as long as $ \lambda > 0 $. Therefore, $ I \equiv I_{\gamma_1} = I_{\gamma_3} $:
            \begin{equation}
                I = 2 e^{\imath \frac{\pi}{4}} \int_0^{R \to \infty} e^{- \lambda x^2} \dd{x} = e^{\imath \frac{\pi}{4}} \int_{- \infty}^{\infty} e^{- \lambda x^2} \dd{x} = \sqrt{\frac{\pi}{\lambda}} \left(e^{\imath \frac{\pi}{2}}\right)^{1/2} = \sqrt{\frac{i \pi}{\lambda}}
            \end{equation}
        \end{problem}
    \item[(b)] $ I = \int_{- \infty}^{\infty} e^{\imath (ax^2 + b x + c)} \dd{x} $.
        \begin{problem}
            In one dimension, the stationary phase approximation says that
            \begin{equation}
                \int_{- \infty}^{\infty} e^{\imath f(x)} \dd{x} \approx \sum_{x_0 \in \Sigma} e^{\imath f(x_0) + \text{sign}(f''(x_0)) \frac{\imath \pi}{4}} \sqrt{\frac{2 \pi}{\abs{f''(x_0)}}}
            \end{equation}
            where $ \Sigma $ is the set of critical points $ \partial_x f(x_0) = 0 $. For this problem, there is only one critical point at $ x_0 = - \frac{b}{2a} $, so we can plug in the proper values into the formula above to find that
            \begin{equation}
                I \approx e^{\imath \left( c - \frac{b^2}{4a} + \imath \frac{\pi}{4} \right)} \sqrt{\frac{2 \pi}{2a}} = \sqrt{\frac{\imath \pi}{a}} e^{\imath \left( c - \frac{b^2}{4a} \right)}
            \end{equation}
            To find the exact solution, we first must complete the square in the exponential:
            \begin{equation}
                a x^2 + b x + c = a\left( x + \frac{b}{2a} \right)^2 + \left( c - \frac{b^2}{4a} \right)
            \end{equation}
            Therefore, the integral will be
            \begin{equation}
                I = e^{\imath \left(c - \frac{b^2}{4a}\right)} \int_{- \infty}^{\infty} e^{\imath a \left( x + \frac{b}{2a} \right)^2} \dd{x}
            \end{equation}
            We can do a change of variables $ x' = x + \frac{b}{2a} $, $ \dd{x'} = \dd{x} $. The bounds of integration won't change under this transformation, and we have already done the integral of $ e^{\imath a x'^2} $ above. We will therefore find
            \begin{equation}
                I = e^{\imath \left( c - \frac{b^2}{4a} \right)} \sqrt{\frac{\imath \pi}{a}}
            \end{equation}
            which happens to be equal to the stationary phase approximation.
        \end{problem}
    \item[(c)] $ I = \int_{- \infty}^{\infty} e^{\imath (ax^2 + A x^4)} $.
        \begin{problem}
            Expanding in $ A $ we find
            \begin{equation}
                I \approx \int e^{\imath a x^2} \dd{x} + \int \imath x^4 A e^{\imath a x^2} \dd{x} + \order{A^2}
            \end{equation}
            We already know the first integral, and the second one can be done by realizing that
            \begin{equation}
                \int \imath x^4 A e^{\imath a x^2} \dd{x} = \int - \imath A \partial_a^2 \left( e^{\imath a x^2} \right) \dd{x} = - \imath A \partial^2_a \int e^{\imath a x^2} \dd{x}
            \end{equation}
            so
            \begin{equation}
                I = \sqrt{\frac{\imath \pi}{a}} - \imath A \partial^2_a \sqrt{\frac{\imath \pi}{a}} = \sqrt{\frac{\imath \pi}{a}} - \frac{3A}{4a^{5/2}} \imath\sqrt{\pi \imath}
            \end{equation}
            I'd guess the condition for a convergent expansion is that $ A << 1 $ or maybe $ A << a $.

            
            After some consideration, I don't think this was the correct way to do the problem. If we consider the stationary points to be $ 0, \pm \sqrt{- \frac{a}{2A}} $, we can see that expanding around $ 0 $ gives us
            \begin{align}
                I &= e^{\imath 0 + \imath \frac{\pi}{4}} \sqrt{\frac{2 \pi}{\abs{2a}}} + 2 e^{- \imath \frac{a^2}{4A} - \imath \frac{\pi}{4}} \sqrt{\frac{2 \pi}{6a}} \\
                &= \sqrt{\frac{\imath \pi}{a}} + 2 e^{- \imath \frac{a^2}{4A}} \sqrt{\frac{\imath \pi}{3a}}
            \end{align}
        \end{problem}
\end{itemize}

\section*{2. Frequency Space Propagator}
The frequency space propagator for a particle moving in a potential $ V $ is given by
\begin{equation}
    K(x_f, x_i, \omega) = \int_{0}^{\infty} K(x_f, x_i, t) e^{\imath \omega t} \dd{t} = A \sum_n \frac{\sin(nrx_f) \sin(nrx_i)}{(E - \frac{\hbar^2 r^2}{2m} n^2)}
\end{equation}
\begin{itemize}
    \item[(a)] Determine the potential $ V $.
        \begin{problem}
            It's just a bit obvious that this is the propagator for a particle in a box. The sine functions in the numerator are eigenfunctions of that Hamiltonian and the energy in the denominator contains the eigenvalues with $ r = \frac{\pi}{L} $. I will therefore derive the propagator for a particle in a box and show that it's Fourier transform is equal to the given equation. The particle in a box (in one dimension) can be described as
            \begin{equation}
                H = \frac{p^2}{2m} + V(x) \qquad V(x) = \begin{cases} 0 & 0 \leq x \leq L \\ \infty &\qotherwise \end{cases}
            \end{equation}
            The eigensystem is
            \begin{equation}
                \varphi_n(x) = \sqrt{\frac{2}{L}} \sin(\frac{n \pi x}{L})
            \end{equation}
            \begin{equation}
                E_n = \frac{\hbar^2 \pi^2}{2mL^2} n^2
            \end{equation}
            The propagator is easily calculated:
            \begin{align}
                K(x_f, x_i ; t) &= \bra{x} e^{-\imath Ht / \hbar}\ket{x_0} \\
                &= \sum_n\bra{x} e^{-\imath H t / \hbar}\ket{n}\bra{n}\ket{x_0} \\
                &= \sum_n e^{- E_n t / \hbar} \psi_n(x) \psi_0^*(x) \\
                &= \frac{2}{L} \sum_n e^{\imath E_n t / \hbar} \sin(\frac{n \pi x}{L}) \sin(\frac{n \pi x_0}{L})
            \end{align}
            We can then Fourier transform this statement:
            \begin{align}
                K(x_f, x_i ; \omega) &= \frac{2}{L} \int_0^{\infty} \sum_n \sin(\frac{2 \pi x}{L}) \sin(\frac{2 \pi x_0}{L}) e^{- \imath E t / \hbar} e^{\imath \omega t} \dd{t} \\
                &= \frac{2}{L} \sum_n \sin(\frac{2 \pi x}{L}) \sin(\frac{2 \pi x_0}{L}) \int_0^{\infty} e^{\imath (\hbar \omega - E_n)t} \dd{t} \\
                &= \frac{2}{L} \sum_n \sin(\frac{2 \pi x}{L}) \sin(\frac{2 \pi x_0}{L}) \frac{\imath \hbar}{\hbar \omega - E_n} \\
                &= A \sum_n \frac{\sin(nrx_f) \sin(nrx_i)}{(E - \frac{\hbar^2 r^2}{2m} n^2)}
            \end{align}
            with $ r = \frac{\pi}{L} $, $ E = \hbar \omega $, and $ A = \frac{2 \imath \hbar}{L} = \frac{2 \imath \hbar r}{\pi} $.
        \end{problem}
    \item[(b)] Determine the constant $ A $ in terms of the other parameters in the problem.
        \begin{problem}
            See the end result of 2(a).
        \end{problem}
\end{itemize}

\section*{3. Sakurai 2.34}
\begin{itemize}
    \item[(a)] Write down an expression for the classical action for a simple harmonic oscillator for a finite time interval.
        \begin{problem}
            The classical action for the interval $ [t_a, t_b] $ for a simple harmonic oscillator is
            \begin{equation}
                S(t_a, t_b) \equiv \int_{t_a}^{t_b} \dd{t} L(x(t), \dot{x}(t)) = \int_{t_a}^{t_b} \dd{t} \left( \frac{1}{2} m\dot{x}(t)^2 - \frac{1}{2} m \omega^2 x(t)^2 \right)
            \end{equation}
        \end{problem}
    \item[(b)] Construct $\bra{x_n, t_n}\ket{x_{n-1}, t_{n-1}} $ for a simple harmonic oscillator using Feynman's prescription for $ t_n - t_{n-1} = \Delta t $ small. Keeping only terms up to order $ (\Delta t)^2 $, show that it is in complete agreement with the $ t - t_0 \to 0 $ limit of the propagator given by (2.6.46).
        \begin{problem}
            We can write the action as
            \begin{align}
                S(t_{n-1}, t_n) &= \int_{t_{n-1}}^{t_n} \dd{t} \left[ \frac{1}{2} m\dot{x}^2 - V(x) \right] \\
                &= \Delta t \left[ \frac{1}{2} m \left( \frac{x_n - x_{n-1}}{\Delta t} \right)^2 - V\left( \frac{x_n + x_{n-1}}{2} \right) \right] \\
                &= \frac{1}{2 \Delta t} m \left[ x^2_n - 2 x_n x_{n-1} + x^2_{n-1} - \frac{\omega^2}{2} (x_n + x_{n-1})^2 \Delta t \right]
            \end{align}
            Therefore, the propagator is
            \begin{align}
                \bra{x_n, t_n}\ket{x_{n-1}, t_{n-1}} &= \sqrt{\frac{m}{2 \pi \imath \hbar \Delta t}} e^{\frac{\imath}{\hbar} \int \dd{t} L} \\
                &= \sqrt{\frac{m}{2 \pi \imath \hbar \Delta t}} \exp\left[ \frac{\imath \Delta t}{\hbar} \frac{m}{2} \left( \left[ \frac{x_n - x_{n-1}}{\Delta t} \right]^2 - \omega^2 \left[ \frac{x_n + x_{n-1}}{2} \right]^2\right) \right]\\
                &= \sqrt{\frac{m}{2 \pi \imath \hbar \Delta t}} \exp \left[ \frac{\imath}{2 \Delta t} m \left[ x^2_n - 2 x_n x_{n-1} + x^2_{n-1} - \frac{\omega^2}{2} (x_n + x_{n-1})^2 \Delta t \right] \right]
            \end{align}
            As $ \Delta t \to 0 $, we can see that this will resemble the original formulation of the propagator, which was
            \begin{align}
                K(x_n, t_n ; x_{n-1}, t_{n-1}) = &\sqrt{\frac{m \omega}{2 \pi \imath \hbar \sin(\omega \Delta t)}}\\
                & \times \exp\left[ \left( \frac{\imath m \omega}{2 \hbar \sin(\omega \Delta t)} \right) \left( (x_n^2 - x_{n-1}^2) \cos(\omega \Delta t) - 2 x_{n} x_{n-1} \right) \right]
            \end{align}
            For small $ \Delta t $, $ \sin(\omega \Delta t) \approx \omega $ and $ \cos(\omega \Delta t) \approx 1 - \frac{\omega^2}{2} $, which will make this equation resemble the derived version from the classical action. 
        \end{problem}
\end{itemize}
\end{document}
