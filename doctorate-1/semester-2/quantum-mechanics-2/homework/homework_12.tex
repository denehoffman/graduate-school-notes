\documentclass[a4paper,twoside]{article}
% My LaTeX preamble file - by Nathaniel Dene Hoffman
% Credit for much of this goes to Olivier Pieters (https://olivierpieters.be/tags/latex)
% and Gilles Castel (https://castel.dev)
% There are still some things to be done:
% 1. Update math commands using mathtools package (remove ddfrac command and just override)
% 2. Maybe abbreviate \imath somehow?
% 3. Possibly format for margin notes and set new margin sizes
% First, some encoding packages and useful formatting
%--------------------------------------------------------------------------------------------
\usepackage{import}
\usepackage{pdfpages}
\usepackage{transparent}
\usepackage[l2tabu,orthodox]{nag}   % force newer (and safer) LaTeX commands
\usepackage[utf8]{inputenc}         % set character set to support some UTF-8
                                    %   (unicode). Do NOT use this with
                                    %   XeTeX/LuaTeX!
\usepackage[T1]{fontenc}
\usepackage[english]{babel}         % multi-language support
\usepackage{sectsty}                % allow redefinition of section command formatting
\usepackage{tabularx}               % more table options
\usepackage{booktabs}
\usepackage{titling}                % allow redefinition of title formatting
\usepackage{imakeidx}               % create and index of words
\usepackage{xcolor}                 % more colour options
\usepackage{enumitem}               % more list formatting options
\usepackage{tocloft}                % redefine table of contents, new list like objects
\usepackage{subfiles}               % allow for multifile documents

% Next, let's deal with the whitespaces and margins
%--------------------------------------------------------------------------------------------
\usepackage[centering,margin=1in]{geometry}
\setlength{\parindent}{0cm}
\setlength{\parskip}{2ex plus 0.5ex minus 0.2ex} % whitespace between paragraphs

% Redefine \maketitle command with nicer formatting
%--------------------------------------------------------------------------------------------
\pretitle{
  \begin{flushright}         % align text to right
    \fontsize{40}{60}        % set font size and whitespace
    \usefont{OT1}{phv}{b}{n} % change the font to bold (b), normally shaped (n)
                             %   Helvetica (phv)
    \selectfont              % force LaTeX to search for metric in its mapping
                             %   corresponding to the above font size definition
}
\posttitle{
  \par                       % end paragraph
  \end{flushright}           % end right align
  \vskip 0.5em               % add vertical spacing of 0.5em
}
\preauthor{
  \begin{flushright}
    \large                   % font size
    \lineskip 0.5em          % inter line spacing
    \usefont{OT1}{phv}{m}{n}
}
\postauthor{
  \par
  \end{flushright}
}
\predate{
  \begin{flushright}
  \large
  \lineskip 0.5em
  \usefont{OT1}{phv}{m}{n}
}
\postdate{
  \par
  \end{flushright}
}

% Mathematics Packages
\usepackage[Gray,squaren,thinqspace,cdot]{SIunits}      % elegant units
\usepackage{amsmath}                                    % extensive math options
\usepackage{amsfonts}                                   % special math fonts
\usepackage{mathtools}                                  % useful formatting commands
\usepackage{amsthm}                                     % useful commands for building theorem environments
\usepackage{amssymb}                                    % lots of special math symbols
\usepackage{mathrsfs}                                   % fancy scripts letters
\usepackage{cancel}                                     % cancel lines in math
\usepackage{esint}                                      % fancy integral symbols
\usepackage{relsize}                                    % make math things bigger or smaller
%\usepackage{bm}                                         % bold math!
\usepackage{slashed}

\newcommand\ddfrac[2]{\frac{\displaystyle #1}{\displaystyle #2}}    % elegant fraction formatting
\allowdisplaybreaks[1]                                              % allow align environments to break on pages

% Ensure numbering is section-specific
%--------------------------------------------------------------------------------------------
\numberwithin{equation}{section}
\numberwithin{figure}{section}
\numberwithin{table}{section}

% Citations, references, and annotations
%--------------------------------------------------------------------------------------------
\usepackage[small,bf,hang]{caption}        % captions
\usepackage{subcaption}                    % adds subfigure & subcaption
\usepackage{sidecap}                       % adds side captions
\usepackage{hyperref}                      % add hyperlinks to references
\usepackage[noabbrev,nameinlink]{cleveref} % better references than default \ref
\usepackage{autonum}                       % only number referenced equations
\usepackage{url}                           % urls
\usepackage{cite}                          % well formed numeric citations
% format hyperlinks
\colorlet{linkcolour}{black}
\colorlet{urlcolour}{blue}
\hypersetup{colorlinks=true,
            linkcolor=linkcolour,
            citecolor=linkcolour,
            urlcolor=urlcolour}

% Plotting and Figures
%--------------------------------------------------------------------------------------------
\usepackage{tikz}          % advanced vector graphics
\usepackage{pgfplots}      % data plotting
\usepackage{pgfplotstable} % table plotting
\usepackage{placeins}      % display floats in correct sections
\usepackage{graphicx}      % include external graphics
\usepackage{longtable}     % process long tables

% use most recent version of pgfplots
\pgfplotsset{compat=newest}

% Misc.
%--------------------------------------------------------------------------------------------
\usepackage{todonotes}  % add to do notes
\usepackage{epstopdf}   % process eps-images
\usepackage{float}      % floats
\usepackage{stmaryrd}   % some more nice symbols
\usepackage{emptypage}  % suppress page numbers on empty pages
\usepackage{multicol}   % use this for creating pages with multiple columns
\usepackage{etoolbox}   % adds tags for environment endings
\usepackage{tcolorbox}  % pretty colored boxes!


% Custom Commands
%--------------------------------------------------------------------------------------------
\newcommand\hr{\noindent\rule[0.5ex]{\linewidth}{0.5pt}}                % horizontal line
\newcommand\N{\ensuremath{\mathbb{N}}}                                  % blackboard set characters
\newcommand\R{\ensuremath{\mathbb{R}}}
\newcommand\Z{\ensuremath{\mathbb{Z}}}
\newcommand\Q{\ensuremath{\mathbb{Q}}}
%\newcommand\C{\ensuremath{\mathbb{C}}}
\renewcommand{\arraystretch}{1.2}                                       % More space between table rows (could be 1.3)
\newcommand{\Cov}{\mathrm{Cov}}
\newcommand\D{\mathrm{D}}
\newcommand*{\dbar}{\ensuremath{\text{\dj}}}

\newcommand{\incfig}[2][1]{%
    \def\svgwidth{#1\columnwidth}
    \import{./figures/}{#2.pdf_tex}
}

% Custom Environments
%--------------------------------------------------------------------------------------------
\newcommand{\lecture}[3]{\hr\\{\centering{\large\textsc{Lecture #1: #3}}\\#2\\}\hr\markboth{Lecture #1: #3}{\rightmark}}   % command to title lectures
\usepackage{mdframed}
\theoremstyle{plain}
\newmdtheoremenv[nobreak]{theorem}{Theorem}[section]
\newtheorem{corollary}{Corollary}[theorem]
\newtheorem{lemma}[theorem]{Lemma}
\theoremstyle{definition}
\newtheorem*{ex}{Example}
\newmdtheoremenv[nobreak]{definition}{Definition}[section]
\theoremstyle{remark}
\newtheorem*{remark}{Remark}
\newtheorem*{claim}{Claim}
\AtEndEnvironment{ex}{\null\hfill$\diamond$}%
% Note: A proof environment is already provided in the amsthm package
\tcbuselibrary{breakable}
\newenvironment{note}[1]{\begin{tcolorbox}[
    arc=0mm,
    colback=white,
    colframe=white!60!black,
    title=#1,
    fonttitle=\sffamily,
    breakable
]}{\end{tcolorbox}}
\newenvironment{problem}{\begin{tcolorbox}[
    arc=0mm,
    breakable,
    colback=white,
    colframe=black
]}{\end{tcolorbox}}

% Header and Footer
%--------------------------------------------------------------------------------------------
% set header and footer
\usepackage{fancyhdr}                       % header and footer
\pagestyle{fancy}                           % use package
\fancyhf{}
\fancyhead[LE,RO]{\textsl{\rightmark}}      % E for even (left pages), O for odd (right pages)
\fancyfoot[LE,RO]{\thepage}
\fancyfoot[LO,RE]{\textsl{\leftmark}}
\setlength{\headheight}{15pt}


% Physics
%--------------------------------------------------------------------------------------------
\usepackage[arrowdel]{physics}      % all the usual useful physics commands
\usepackage{feyn}                   % for drawing Feynman diagrams
%\usepackage{bohr}                   % for drawing Bohr diagrams
%\usepackage{tikz-feynman}
\usepackage{elements}               % for quickly referencing information of various elements
\usepackage{tensor}                 % for writing tensors and chemical symbols

% Finishing touches
%--------------------------------------------------------------------------------------------
\author{Nathaniel D. Hoffman}

\title{33-756 Homework 12}
\date{\today}
\begin{document}
\maketitle
\section*{1. Particle in a 1D Box}
Consider a particle in a one-dimensional box. The length of the wall increases as a function of time such that $ L = v_0 t + L_0 $. The state of the systems at $ t = 0 $ is $\ket{n(L_0)} $.
\begin{itemize}
    \item[(a)] Under what conditions do you expect the system to stay in the states $\ket{n(L)} $ (i.e. adiabatically)?
        \begin{problem}
            
        \end{problem}
    \item[(b)] Calculate the state of the system at a later time $ t $ and show that the Berry phase vanishes.
        \begin{problem}
            
        \end{problem}
    \item[(c)] The question arises whether or not a priori one can tell that the Berry phase will vanish. The answer is yes. The condition is that somewhere in parameter space there is a degeneracy. Show that the fictitious magnetic field is ill defined at some point(s) in parameter space when a degeneracy point is reached.
        \begin{problem}
            
        \end{problem}
    \item[(d)] Now we would like to understand why this singularity implies there is a non-vanishing phase. To understand this, we will utilize the analogy with magnetic fields. First, start off by showing that there can be (na\"ively) no magnetic charges in Maxwell's equations, because it would be mathematically inconsistent.
        \begin{problem}
            
        \end{problem}
    \item[(e)] There can be magnetic charges if we allow for the vector potential to be ambiguous in the sense that you can define it as follows:
        \begin{equation}
            \va{A} = \begin{cases} \va{A}_1 & 0 \leq \theta < \frac{\pi}{2} \\ \va{A}_2 & \frac{\pi}{2} < \theta \leq \pi \end{cases}
        \end{equation}
        Notice that this is fine since $ \va{A} $ is not physical. $ \va{A}_1 $ and $ \va{A}_2 $ only have to be equal up to a gauge transformation $ \va{A}_1 = \va{A}_2 + \va{\partial}_{\chi} $ on the equator so that $ \va{B} $ is a globally well-defined function of $ x $. Show that the magnetic charge will be given by
        \begin{equation}
            4 \pi Q_M = \oint (\va{A}_1 - \va{A}_2) \vdot \dd{l},
        \end{equation}
        where the integral is along the $ \vu{\varphi} $ direction. Then show that $ chii = 2Q_m \varphi $.
        \begin{problem}
            
        \end{problem}
    \item[(f)] Sticking with our diversion into magnetic monopoles, show that if a magnetic monopole existed, then every charged particle would have its charge quantized in units of $ (\hbar c)/(2Q_M) $. This condition comes from making sure that the wave function for an electron in the monopole background is single valued.
        \begin{problem}
            
        \end{problem}
    \item[(g)] Finally, returning to Berry's phase, we see that if there is going to be a non-zero Berry's phase, the vector potential should be ambiguous (singular at some point). Let us consider the case of a particle in a slowly varying time dependent magnetic field $ \va{B}(t) $. We wish to construct the vector potential
        \begin{equation}
            \va{A} = \mel{n}{\grad_R}{n}
        \end{equation}
        Take the Hamiltonian to be $ H = \va{R}(t) \vdot \va{\sigma} $. Construct $ n $ by performing a rotation in spin space using polar coordinates and show that the spinor wave function is singular at $ \theta = \pi $. That is, as the south poll is approached, the value of the spinor changes if we approach it from opposite sides. Thus we have found our ambiguity in the vector potential that allows for a non-zero Berry's phase.
\end{itemize}
\section*{2. Prove 7.5.49 in Sakurai}
Sakurai tells us that the potential energy term for interacting Fermions can be written as
\begin{equation}
    \mathcal{V} = \frac{e^2}{2V} \sum_{\va{k}_1 \lambda_1} \sum_{\va{k}_2 \lambda_2} \sum_{\va{k_3}} \sum_{\va{k_4}} \delta_{\va{k}_1 + \va{k}_2, \va{k}_3 + \va{k}_4} \frac{4 \pi}{\va{q}^2 + \mu^2} a^\dagger_{\va{k}_1, \lambda_1} a^\dagger_{\va{k}_2 \lambda_2} a_{\va{k}_4 \lambda_2} a_{\va{k}_3 \lambda_1}
\end{equation}
\begin{problem}
    Sakurai deduces this from 7.5.46:
    \begin{equation}
        \mathcal{V} = \frac{1}{2} \sum_{\va{k}_1 \lambda_1} \sum_{\va{k}_2 \lambda_2} \sum_{\va{k}_3 \lambda_3} \sum_{\va{k}_4 \lambda_4} \mel{\va{k}_1 \lambda_1 \va{k}_2 \lambda_2}{V}{\va{k}_3 \lambda_3 \va{k}_4 \lambda_4} a^\dagger_{\va{k}_1 \lambda_1} a^\dagger_{\va{k}_2 \lambda_2} a_{\va{k}_4 \lambda_4} a_{\va{k}_3 \lambda_3} 
    \end{equation}
    We next need to evaluate the matrix element. To do this, we can write it as a double integral with one integrated variable governing the momentum transfer between states $ 1 $ and $ 3 $ ($ \va{x}' $) and the other governing states $ 2 $ and $ 4 $ ($ \va{x}'' $):
    \begin{align}
        \bra{12} V\ket{43} &= \int \dd[3]{x'} \int \dd[3]{x''} V(\va{x}', \va{x}'')\bra{\va{k}_1 \lambda_1}\ket{\va{x}'}\bra{\va{x}'}\ket{\va{k}_3 \lambda_3}\bra{\va{k}_2 \lambda_2}\ket{\va{x}''}\bra{\va{x}''}\ket{\va{k}_4 \lambda_4} \\
        &= \frac{e^2}{V^2} \iint \dd[3]{(x',x'')} \frac{e^{- \mu \abs{\va{x}' - \va{x}''}}}{\abs{\va{x}' - \va{x}''}} e^{- \imath \va{k}_1 \vdot \va{x}'} \chi^\dagger_{\lambda_1} e^{\imath \va{k}_3 \vdot \va{x}'} \chi_{\lambda_3} e^{- \imath \va{k}_2 \vdot \va{x}''} \chi^\dagger_{\lambda_2} e^{\imath \va{k}_4 \vdot \va{x}''} \chi_{\lambda_4}
    \end{align}
\end{problem}
\section*{3. Fock Space}
Use the Fock space representation to calculate the partition function for free Bosons and free Fermions
\begin{equation}
    Z = \Tr(e^{- \beta H})
\end{equation}
where $ H = \sum_k a_k a^\dagger_k \epsilon(k) $ and $ \epsilon(k) $ is the energy of a state with momentum $ \hbar k $.
\end{document}
