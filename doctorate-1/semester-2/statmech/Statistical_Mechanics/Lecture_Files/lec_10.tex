\documentclass[a4paper,twoside,master.tex]{subfiles}
\begin{document}
\lecture{10}{Wednesday, February 05, 2020}{}

\begin{note}{Excursion \textendash\ Exact and Inexact Differentials}
    Consider $ f(x) \dd{x} $. This can always be written as the differential of some function $ F(x) $:
    \begin{align}
        F(x) &= \int^x \dd{t} f(t) \\
        \dd{F(x)} = \pdv{F(x)}{x} \dd{x} = f(x) \dd{x}
    \end{align}

    However, this is no longer generally true in more than one dimension. Consider $ f_x(x,y) \dd{x} + f_y(x,y) \dd{y} $. Does a function exist such that
    \begin{equation}
        \dd{F} = f_x \dd{x} + f_y \dd{y}
    \end{equation}
    If so, this differential is exact. If this is true,
    \begin{equation}
        \dd{F} = \pdv{F}{x} \dd{x} + \pdv{F}{y} \dd{y}
    \end{equation}
    As a consequence, 
    \begin{align}
        \pdv{f_x}{y} &= \pdv{y}\left( \pdv{F}{x} \right) \\
        &= \pdv{x}\left( \pdv{F}{y} \right) \\
        \pdv{f_x}{y} &= \pdv{f_y}{x}
    \end{align}
    If this doesn't hold, our differential will never be exact. This condition is called an ``integrability condition''. If it holds, the differential is called ``closed''. Closed differentials are exact if the domain is simply connected.

    The ``bar'' notation denotes differentials that are not exact. There is no $ F $ such that the equations are satisfied. In the equation $ \dd{U} = \dbar{Q} + \dbar{W} $ 
\end{note}

How would you integrate an inexact differential?
\begin{equation}
    \dbar{F} = f_x(x,y) \dd{x} + f_y(x,y) \dd{y}
\end{equation}
Define
\begin{equation}
    \va{f}(x,y) = \mqty(f_x(x,y) \\ f_y(x,y)) \qquad \dd{\va{r}} = \mqty(\dd{x} \\ \dd{y})
\end{equation}
so that
\begin{equation}
    \int_C \dbar{F} = \int_C \va{f} \vdot \dd{\va{r}} = \int_a^b \va{f}(\va{r}(s)) \vdot \pdv{\va{r}}{s} \dd{s}
\end{equation}
If the differential was exact, any path integral would only depend on the endpoints and not on the path taken. Therefore, any closed loop integral would be zero.

Differentials that are not closed still have integrating factors which close them!
\begin{equation}
    \dbar{F}(x,y) \to a(x,y) \dbar F(x,y) = \dd{G}(x,y)
\end{equation}
For example:
\begin{equation}
    \dbar{F} = y \dd{x} + 1 \dd{y}
\end{equation}
Obviously, $ \pdv{y}{y} = 1 \neq \pdv{1}{x} = 0 $, so this differential is not exact, but look at
\begin{equation}
    e^{x} \dbar{F} = e^x y \dd{x} + e^x \dd{y}
\end{equation}
which is exact.

Can we find the $ G $ that gives this $ e^x \dbar{F} $? Yes, we can pick a convenient starting point $ (0,0) $ and integrate to some fixed point $ (x,y) $. If the differential is exact, then the answer does not depend on the path.
\begin{align}
    G(x,y) &= \int_S \dbar{F}(x',y') e^{x'} \\
    &= \int_S^{(x,y)} \left[ y' e^{x'} \dd{x'} + e^{x'} \dd{y'} \right] \\
    &= \int_0^x \dd{x'} 0 e^{x'} + \int_0^y \dd{y} e^x = y e^x
\end{align}
Here, we take a straight line path along the x-axis till $ x' = x $, then move straight up/down to $ y $. Let's test if this is an exact differential.
\begin{equation}
    \dd{G} = \pdv{G}{x} \dd{x} + \pdv{G}{y} \dd{y} = y e^y \dd{x} + e^x \dd{y}
\end{equation}


\end{document}
