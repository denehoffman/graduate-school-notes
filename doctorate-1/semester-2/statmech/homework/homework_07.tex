\documentclass[a4paper,twoside]{article}
% My LaTeX preamble file - by Nathaniel Dene Hoffman
% Credit for much of this goes to Olivier Pieters (https://olivierpieters.be/tags/latex)
% and Gilles Castel (https://castel.dev)
% There are still some things to be done:
% 1. Update math commands using mathtools package (remove ddfrac command and just override)
% 2. Maybe abbreviate \imath somehow?
% 3. Possibly format for margin notes and set new margin sizes
% First, some encoding packages and usefull formatting
%--------------------------------------------------------------------------------------------
\usepackage[l2tabu,orthodox]{nag}   % force newer (and safer) LaTeX commands
\usepackage[utf8]{inputenc}         % set character set to support some UTF-8
                                    %   (unicode). Do NOT use this with
                                    %   XeTeX/LuaTeX!
\usepackage[T1]{fontenc}
\usepackage[english]{babel}         % multi-language support
\usepackage{sectsty}                % allow redefinition of section command formatting
\usepackage{tabularx}               % more table options
\usepackage{booktabs}
\usepackage{titling}                % allow redefinition of title formatting
\usepackage{imakeidx}               % create and index of words
\usepackage{xcolor}                 % more colour options
\usepackage{enumitem}               % more list formatting options
\usepackage{tocloft}                % redefine table of contents, new list like objects
\usepackage{subfiles}               % allow for multifile documents

% Next, let's deal with the whitespaces and margins
%--------------------------------------------------------------------------------------------
\usepackage[centering,margin=1in]{geometry}
\setlength{\parindent}{0cm}
\setlength{\parskip}{2ex plus 0.5ex minus 0.2ex} % whitespace between paragraphs

% Redefine \maketitle command with nicer formatting
%--------------------------------------------------------------------------------------------
\pretitle{
  \begin{flushright}         % align text to right
    \fontsize{40}{60}        % set font size and whitespace
    \usefont{OT1}{phv}{b}{n} % change the font to bold (b), normally shaped (n)
                             %   Helvetica (phv)
    \selectfont              % force LaTeX to search for metric in its mapping
                             %   corresponding to the above font size definition
}
\posttitle{
  \par                       % end paragraph
  \end{flushright}           % end right align
  \vskip 0.5em               % add vertical spacing of 0.5em
}
\preauthor{
  \begin{flushright}
    \large                   % font size
    \lineskip 0.5em          % inter line spacing
    \usefont{OT1}{phv}{m}{n}
}
\postauthor{
  \par
  \end{flushright}
}
\predate{
  \begin{flushright}
  \large
  \lineskip 0.5em
  \usefont{OT1}{phv}{m}{n}
}
\postdate{
  \par
  \end{flushright}
}

% Mathematics Packages
\usepackage[Gray,squaren,thinqspace,cdot]{SIunits}      % elegant units
\usepackage{amsmath}                                    % extensive math options
\usepackage{amsfonts}                                   % special math fonts
\usepackage{mathtools}                                  % useful formatting commands
\usepackage{amsthm}                                     % useful commands for building theorem environments
\usepackage{amssymb}                                    % lots of special math symbols
\usepackage{mathrsfs}                                   % fancy scripts letters
\usepackage{cancel}                                     % cancel lines in math
\usepackage{esint}                                      % fancy integral symbols
\usepackage{relsize}                                    % make math things bigger or smaller
\usepackage{bm}                                         % bold math!

\newcommand\ddfrac[2]{\frac{\displaystyle #1}{\displaystyle #2}}    % elegant fraction formatting
\allowdisplaybreaks[1]                                              % allow align environments to break on pages

% Ensure numbering is section-specific
%--------------------------------------------------------------------------------------------
\numberwithin{equation}{section}
\numberwithin{figure}{section}
\numberwithin{table}{section}

% Citations, references, and annotations
%--------------------------------------------------------------------------------------------
\usepackage[small,bf,hang]{caption}        % captions
\usepackage{subcaption}                    % adds subfigure & subcaption
\usepackage{sidecap}                       % adds side captions
\usepackage{hyperref}                      % add hyperlinks to references
\usepackage[noabbrev,nameinlink]{cleveref} % better references than default \ref
\usepackage{autonum}                       % only number referenced equations
\usepackage{url}                           % urls
\usepackage{cite}                          % well formed numeric citations
% format hyperlinks
\colorlet{linkcolour}{black}
\colorlet{urlcolour}{blue}
\hypersetup{colorlinks=true,
            linkcolor=linkcolour,
            citecolor=linkcolour,
            urlcolor=urlcolour}

% Plotting and Figures
%--------------------------------------------------------------------------------------------
\usepackage{tikz}          % advanced vector graphics
\usepackage{pgfplots}      % data plotting
\usepackage{pgfplotstable} % table plotting
\usepackage{placeins}      % display floats in correct sections
\usepackage{graphicx}      % include external graphics
\usepackage{longtable}     % process long tables

% use most recent version of pgfplots
\pgfplotsset{compat=newest}

% Misc.
%--------------------------------------------------------------------------------------------
\usepackage{todonotes}  % add to do notes
\usepackage{epstopdf}   % process eps-images
\usepackage{float}      % floats
\usepackage{stmaryrd}   % some more nice symbols
\usepackage{emptypage}  % suppress page numbers on empty pages
\usepackage{multicol}   % use this for creating pages with multiple columns
\usepackage{etoolbox}   % adds tags for environment endings
\usepackage{tcolorbox}  % pretty colored boxes!


% Custom Commands
%--------------------------------------------------------------------------------------------
\newcommand\hr{\noindent\rule[0.5ex]{\linewidth}{0.5pt}}                % horizontal line
\newcommand\N{\ensuremath{\mathbb{N}}}                                  % blackboard set characters
\newcommand\R{\ensuremath{\mathbb{R}}}
\newcommand\Z{\ensuremath{\mathbb{Z}}}
\newcommand\Q{\ensuremath{\mathbb{Q}}}
\newcommand\C{\ensuremath{\mathbb{C}}}
\renewcommand{\arraystretch}{1.2}                                       % More space between table rows (could be 1.3)
\newcommand{\Cov}{\mathrm{Cov}}
\newcommand*{\dbar}{\ensuremath{\text{\dj}}}
% Custom Environments
%--------------------------------------------------------------------------------------------
\newcommand{\lecture}[3]{\hr\\{\centering{\large\textsc{Lecture #1: #3}}\\#2\\}\hr\markboth{Lecture #1: #3}{\rightmark}}   % command to title lectures
\usepackage{mdframed}
\theoremstyle{plain}
\newmdtheoremenv[nobreak]{theorem}{Theorem}[section]
\newtheorem{corollary}{Corollary}[theorem]
\newtheorem{lemma}[theorem]{Lemma}
\theoremstyle{definition}
\newtheorem*{ex}{Example}
\newmdtheoremenv[nobreak]{definition}{Definition}[section]
\theoremstyle{remark}
\newtheorem*{remark}{Remark}
\AtEndEnvironment{ex}{\null\hfill$\diamond$}%
% Note: A proof environment is already provided in the amsthm package
\tcbuselibrary{breakable}
\newenvironment{note}[1]{\begin{tcolorbox}[
    arc=0mm,
    colback=white,
    colframe=white!60!black,
    title=#1,
    fonttitle=\sffamily,
    breakable
]}{\end{tcolorbox}}
\newenvironment{problem}{\begin{tcolorbox}[
    arc=0mm,
    breakable,
    colback=white,
    colframe=black
]}{\end{tcolorbox}}

% Header and Footer
%--------------------------------------------------------------------------------------------
% set header and footer
\usepackage{fancyhdr}                       % header and footer
\pagestyle{fancy}                           % use package
\fancyhf{}
\fancyhead[LE,RO]{\textsl{\rightmark}}      % E for even (left pages), O for odd (right pages)
\fancyfoot[LE,RO]{\thepage}
\fancyfoot[LO,RE]{\textsl{\leftmark}}
\setlength{\headheight}{15pt}


% Physics
%--------------------------------------------------------------------------------------------
\usepackage[arrowdel]{physics}      % all the usual useful physics commands
%\usepackage{feyn}                   % for drawing Feynman diagrams
%\usepackage{bohr}                   % for drawing Bohr diagrams
\usepackage{elements}               % for quickly referencing information of various elements
\usepackage{tensor}                 % for writing tensors and chemical symbols

% Finishing touches
%--------------------------------------------------------------------------------------------
\author{Nathaniel D. Hoffman}

\title{33-765 Homework 6}
\date{\today}
\begin{document}
\maketitle

\section*{24. Examples of Simple Thermodynamic Identities}
Rewrite the following thermodynamic derivatives only using the ``standard'' derivatives $ \alpha $, $ \kappa_T $, $ c_P $, and $ c_V $ (and possibly factors, such as $ T $ or $ P $, or numbers, such as $ 2 $ or $ \pi $) occur:
\begin{itemize}
    \item[1.] $ \left( \pdv{T}{P} \right)_{S,N} = ? $
        \begin{problem}
            \begin{align}
                \left( \pdv{T}{P} \right)_{S,N} &\sim \pdv{(T,S)}{(P,S)} \\
                &= \pdv{(T,S)}{(T,P)} \pdv{(T,P)}{(P,S)} \\
                &= -\pdv{(T,S)}{(T,P)} \pdv{(P,T)}{(P,S)} \\
                &= - \left( \pdv{S}{P} \right)_{T} \left( \pdv{T}{S} \right)_{P} \\
                &= - \left( \pdv{S}{P} \right)_{T} \frac{T}{N c_p} \\
                &= - \pdv{(S,T)}{(P,V)} \pdv{(P,V)}{(P,T)} \frac{T}{N c_p} \\
                &= \pdv{(P,V)}{(P,T)} \frac{T}{N c_p} \\
                &= \left( \pdv{V}{T} \right)_{P} \frac{T}{N c_p} \\
                &= \frac{V \alpha T}{N c_p}
            \end{align}
        \end{problem}
    \item[2.] $ \left( \pdv{F}{S} \right)_{T,N} = ? $
        \begin{problem}
            \begin{align}
                \left( \pdv{F}{S} \right)_{T,N} &\sim \pdv{(F,T)}{(S,T)} \\
                &= \pdv{(F,T)}{(V,T)} \pdv{(V,T)}{(S,T)} \\
                &= \left( \pdv{F}{V} \right)_{T} \pdv{(V,T)}{(S,T)} \\
                &= - P \pdv{(V,T)}{(S,T)} \\
                &= - P \pdv{(V,T)}{(V,P)} \pdv{(V,P)}{(S,T)} \\
                &= - P \left( \pdv{T}{P} \right)_{V} \\
                &= - P \frac{\kappa_T}{\alpha} 
            \end{align}
        \end{problem}
\end{itemize}

\section*{25. Maxwell Relations and Jacobians in Tedious Disguise}
\begin{itemize}
    \item[1.] Prove the first $ T \dd{S} $ equation: $ T \dd{S} = N c_V \dd{T} + \frac{\alpha T}{\kappa_T} \dd{V} $.
        \begin{problem}
            \begin{align}
                \dd{S} &= \eval{\pdv{S}{T}}_{V} \dd{T} + \eval{\pdv{S}{V}}_{T} \dd{V} \\
                &=  \frac{N c_V}{T} \dd{T} + \pdv{(S,T)}{(V,T)} \dd{V} \\
                &= \frac{Nc_V}{T} \dd{T} + \pdv{(S,T)}{\bm{(P,T)}} \pdv{\bm{(P,T)}}{(V,T)} \dd{V} \\
                &= \frac{Nc_V}{T} \dd{T} + \pdv{(S,T)}{(P,T)} \frac{1}{\left( \pdv{V}{P} \right)_T} \dd{V} \\
                &= \frac{N c_V}{T} \dd{T} - \pdv{(S,T)}{(P,T)} \frac{1}{V \kappa_T} \dd{V} \\
                &= \frac{N c_V}{T} \dd{T} - \frac{1}{V \kappa_T} \underbrace{\pdv{(S,T)}{\bm{(P,V)}}}_{-1} \underbrace{\pdv{\bm{(P,V)}}{(P,T)}}_{V \alpha} \dd{V} \\
                &= \frac{N c_V}{T} \dd{T} + \frac{\alpha}{\kappa_T} \dd{V} \\
                T \dd{S} &= N c_V \dd{T} + \frac{\alpha T}{\kappa_T} \dd{V} 
            \end{align}
        \end{problem}
    \item[2.] Prove the second $ T \dd{S} $ equation: $ T \dd{S} = N c_p \dd{T} - \alpha TV \dd{P} $.
        \begin{problem}
            \begin{align}
                \dd{S} &= \eval{\pdv{S}{T}}_{P} \dd{T} + \eval{\pdv{S}{P}}_{T} \dd{P} \\
                &= \frac{N c_P}{T} \dd{T} + \pdv{(S,T)}{(P,T)} \dd{P} \\
                &= \frac{N c_P}{T} \dd{T} + \alpha V \underbrace{\pdv{(P,T)}{(P,V)}}_{\frac{1}{\alpha V}} \pdv{(S,T)}{(P,T)} \dd{P} \\
                &= \frac{N c_P}{T} \dd{T} + \underbrace{\alpha V \pdv{(S,T)}{(P,V)}}_{-1} \dd{P} \\
                &= \frac{N c_P}{T} \dd{T} - \alpha V \dd{P} \\
                T \dd{S} &= N c_P \dd{T} - \alpha TV \dd{P} 
            \end{align} 
        \end{problem}
\end{itemize}

\section*{26. ``A pearl of theoretical physics'' \ldots}
\ldots that's what H.A. Lorentz called Boltzmann's following brilliant insight. Consider some mystery system, of which we only know that it is extensive, the chemical potential vanishes, and it satisfies the equation of state $ PV = \frac{1}{3} U $.
\begin{itemize}
    \item[1.] Explain why in such a situation we must have $ U(T,V) = V u(T) $.
        \begin{problem}
            If the system is extensive, the differential of $ U $ must be homogeneous and first-order. There is only one extensive variable in the equation of state, so the energy must scale linearly with that variable, $ V $.
        \end{problem}
    \item[2.] Express the entropy as a function of temperature and volume. (This will involve $ u(T) $, which you need not eliminate.)
        \begin{problem}
            Because the system is extensive, we can use the equation
            \begin{equation}
                U = TS - PV
            \end{equation}
            so
            \begin{align}
                TS &= PV + U \\
                S &= \frac{PV}{T} + \frac{U}{T} \\
                &= \frac{V \frac{U}{3V}}{T} + \frac{U}{T} \\
                &= \frac{V \frac{Vu(T)}{3V}}{T} + \frac{Vu(T)}{T} \\
                &= \frac{1}{T} \left( \frac{V u(T)}{3} + V u(T) \right) \\
                &= \frac{4}{3} \frac{V u(T)}{T} 
            \end{align}
        \end{problem}
    \item[3.] Find a differential equation for $ u(T) $ by pondering over the temperature dependence of the pressure.
        \begin{problem}
            \begin{equation}
                u'(T) = 3 \pdv{P}{T} = 3 \pdv{S}{V} = 3 \left[ \frac{4}{3} \frac{u(T)}{T} \right] = 4 \frac{u(T)}{T}
            \end{equation}
        \end{problem}
    \item[4.] Solve the differential equation and thus predict how the energy density and the entropy depend on the temperature.
        \begin{problem}
            \begin{align}
                u'(T) &= 4 \frac{u(T)}{T} \\
                \dv{u}{T} &= 4 \frac{u}{T} \\
                \frac{\dd{u}}{u} &= \frac{4 \dd{T}}{T} \\
                \ln(u) &= 4\ln(T) + \ln(c) \\
                u &= e^{\ln\left( T^4 \right) + \ln(c)} \\
                u &= cT^4 \\
                &\implies u \propto T^4 \qand S \propto \frac{u}{T} \propto T^3
            \end{align}
        \end{problem}
\end{itemize}


\section*{27. Relation Between the Isothermal and the Adiabatic Compressibilities}
In analogy to the well-known relation between the isobaric and the isochoric heat capacities, $ c_P $ and $ c_V $, derive the following very similar formula for the isothermal and adiabatic compressibilities, $ \kappa_T $ and $ \kappa_S $:
\begin{equation}
    \kappa_T - \kappa_S = \frac{TV \alpha^2}{N c_p}
\end{equation}

\begin{problem}
    Let's start with the definition of $ \kappa_T $:
    \begin{equation}
        \kappa_T = - \frac{1}{V} \left( \pdv{V}{P} \right)_T = - \frac{1}{V} \pdv{(V,T)}{(P,T)}
    \end{equation}
    Next, we'll expand this in a vain attempt to get a $ c_P $ on one side (this was the first time I did it, and while it worked out, there's probably a faster way to get this result):
    \begin{align}
        \kappa_T &= - \frac{1}{V} \pdv{(V,T)}{(P,T)} \\
        &= - \frac{1}{V} \pdv{(V,T)}{\bm{(P,S)}} \pdv{\bm{(P,S)}}{(P,T)} \\
        &= - \frac{1}{V} \pdv{(V,T)}{(P,S)}\left( \pdv{S}{T} \right)_P \\
        &= - \frac{1}{V} \pdv{(V,T)}{(P,S)} \frac{N c_P}{T}
    \end{align}
    While this seems to give us terms which are the reciprocal of what we want right now, it does eventually work out. As we did with the heat capacities, we will expand the remaining Jacobian into its determinant form:
    \begin{align}
        \kappa_T &= - \frac{N c_P}{TV} \left[ \left( \pdv{V}{P} \right)_S \left( \pdv{T}{S} \right)_P - \left( \pdv{T}{P} \right)_S \left( \pdv{V}{S} \right)_P \right] \\
        &= - \frac{N c_P}{TV} \left[ \left( -V \kappa_S \right) \left( \frac{1}{\left( \pdv{S}{T} \right)_P}\right) - \left( \pdv{T}{P} \right)_S \left( \pdv{V}{S} \right)_P \right] \\
        &= - \frac{Nc_P}{TV} \left[ \left( -V \kappa_S \right) \left( \frac{T}{Nc_P} \right) - \left( \pdv{T}{P} \right)_S \left( \pdv{V}{S} \right)_P \right] \\
        &= \kappa_S + \frac{N c_P}{VT} \left( \pdv{T}{P} \right)_S \left( \pdv{V}{S} \right)_P
    \end{align}
    It now seems reasonable to move $ \kappa_S $ to the other side. I'm going to need some $ \alpha $'s here, and I also need some factors $ c_P $ to move it to the denominator:
    \begin{align}
        \kappa_T - \kappa_S &= \frac{Nc_p}{TV} \pdv{(T,S)}{(P,S)} \pdv{(V,P)}{(S,P)} \\
        &= \frac{Nc_p}{TV} \left[ \pdv{(T,S)}{\bm{(P,T)}} \pdv{\bm{(P,T)}}{(P,S)} \right] \left[ \pdv{(V,P)}{\bm{(T,P)}} \pdv{\bm{(T,P)}}{(S,P)} \right] \\
        &= \frac{Nc_P}{TV} \left[ \pdv{(T,S)}{(P,T)} \frac{T}{Nc_p} \right] \left[ (V \alpha) \left( \frac{T}{Nc_p} \right) \right] \\
        &= \frac{T\alpha}{N c_p} \pdv{(T,S)}{(P,V)} \\
        &= \frac{T \alpha}{N c_p} \underbrace{\pdv{(T,S)}{\bm{(P,V)}}}_{1} \underbrace{\pdv{\bm{(P,V)}}{(P,T)}}_{V \alpha} \\
        \kappa_T - \kappa_S &= \frac{TV \alpha^2}{N c_P} 
    \end{align}
\end{problem}

\section*{28. Adiabatic Compression}
\begin{itemize}
    \item[1.] Show that, quite generally, $ \frac{\kappa_T}{\kappa_S} = \frac{c_P}{c_V} =\colon \gamma $, where $ \gamma $ is called the adiabatic index.
        \begin{problem}
            \begin{equation}
                \frac{\kappa_T}{\kappa_S} = \frac{\left( \pdv{V}{P} \right)_T}{\left( \pdv{V}{P} \right)_S} = \pdv{\color{red}(V,T)\color{black}}{(P,T)} \pdv{(P,S)}{\color{red}(V,S)\color{black}} = \pdv{\color{red}(V,T)\color{black}}{\color{red}(V,S)\color{black}} \pdv{(P,S)}{(P,T)} = \frac{\left( \pdv{S}{T} \right)_P}{\left( \pdv{S}{T} \right)_V} = \frac{c_P}{c_V}
            \end{equation}
        \end{problem}
    \item[2.] Calculate $ \kappa_T $, $ c_V $, $ c_P $, and $ \gamma $ for the monoatomic ideal gas.
        \begin{problem}
            We begin with the equation of state for the ideal gas:
            \begin{equation}
                PV = N k_B T
            \end{equation}
            \begin{align}
                \kappa_T &= - \frac{1}{V} \left( \pdv{V}{P} \right)_{T,N} \\
                &= - \frac{1}{V} \left( - \frac{Nk_B T}{P^2} \right) \\
                \kappa_T &= \frac{N k_B T}{VP^2} = \frac{1}{P}
            \end{align}
            At constant volume, $ \dd{U} = T \dd{S} $ so
            \begin{align}
                c_V &= \frac{T}{N} \left( \pdv{S}{T} \right)_{V,N} \\
                &= \frac{T}{N} \frac{1}{T} \left( \pdv{U}{T} \right)_{V,N}  \\
                &= \frac{1}{N} \left( \frac{3}{2} N k_B \right) \\
                &= \frac{3}{2} k_B
            \end{align}
            Next, $ c_P = \frac{TV \alpha^2}{N \kappa_T} + c_V $, so
            \begin{equation}
                c_P = \frac{TV \alpha^2}{N} P + \frac{3}{2} k_B = \frac{T V P \alpha^2}{N} + \frac{3}{2} k_B = k_B + \frac{3}{2} k_B = \frac{5}{2} k_B
            \end{equation}
            since, for an ideal gas, $ \alpha = \frac{1}{T} $.

            Finally, $ \gamma = \frac{c_P}{c_V} $, so
            \begin{equation}
                \gamma = \frac{5}{3}
            \end{equation}
        \end{problem}
    \item[3.] Show that for adiabatic (constant entropy) compression of an ideal gas we get $ P \propto V^{- \gamma} $.
        \begin{problem}
            \begin{equation}
                - \frac{1}{V} \left( \pdv{V}{P} \right)_{S,N} = \kappa_S = \frac{\kappa_T}{\gamma} = \frac{1}{\gamma P}
            \end{equation}
            so
            \begin{equation}
                - \gamma \frac{\dd{V}}{V} = \frac{\dd{P}}{P}
            \end{equation}
            Solving this, we get
            \begin{equation}
                - \gamma \ln(V) = \ln(P) + \ln(c)
            \end{equation}
            so
            \begin{equation}
                P = c V^{- \gamma} \qor P \propto V^{- \gamma}
            \end{equation}
        \end{problem}
\end{itemize}



\end{document}
