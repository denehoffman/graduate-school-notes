\documentclass[a4paper,twoside]{article}
% My LaTeX preamble file - by Nathaniel Dene Hoffman
% Credit for much of this goes to Olivier Pieters (https://olivierpieters.be/tags/latex)
% and Gilles Castel (https://castel.dev)
% There are still some things to be done:
% 1. Update math commands using mathtools package (remove ddfrac command and just override)
% 2. Maybe abbreviate \imath somehow?
% 3. Possibly format for margin notes and set new margin sizes
% First, some encoding packages and usefull formatting
%--------------------------------------------------------------------------------------------
\usepackage[l2tabu,orthodox]{nag}   % force newer (and safer) LaTeX commands
\usepackage[utf8]{inputenc}         % set character set to support some UTF-8
                                    %   (unicode). Do NOT use this with
                                    %   XeTeX/LuaTeX!
\usepackage[T1]{fontenc}
\usepackage[english]{babel}         % multi-language support
\usepackage{sectsty}                % allow redefinition of section command formatting
\usepackage{tabularx}               % more table options
\usepackage{booktabs}
\usepackage{titling}                % allow redefinition of title formatting
\usepackage{imakeidx}               % create and index of words
\usepackage{xcolor}                 % more colour options
\usepackage{enumitem}               % more list formatting options
\usepackage{tocloft}                % redefine table of contents, new list like objects
\usepackage{subfiles}               % allow for multifile documents

% Next, let's deal with the whitespaces and margins
%--------------------------------------------------------------------------------------------
\usepackage[centering,margin=1in]{geometry}
\setlength{\parindent}{0cm}
\setlength{\parskip}{2ex plus 0.5ex minus 0.2ex} % whitespace between paragraphs

% Redefine \maketitle command with nicer formatting
%--------------------------------------------------------------------------------------------
\pretitle{
  \begin{flushright}         % align text to right
    \fontsize{40}{60}        % set font size and whitespace
    \usefont{OT1}{phv}{b}{n} % change the font to bold (b), normally shaped (n)
                             %   Helvetica (phv)
    \selectfont              % force LaTeX to search for metric in its mapping
                             %   corresponding to the above font size definition
}
\posttitle{
  \par                       % end paragraph
  \end{flushright}           % end right align
  \vskip 0.5em               % add vertical spacing of 0.5em
}
\preauthor{
  \begin{flushright}
    \large                   % font size
    \lineskip 0.5em          % inter line spacing
    \usefont{OT1}{phv}{m}{n}
}
\postauthor{
  \par
  \end{flushright}
}
\predate{
  \begin{flushright}
  \large
  \lineskip 0.5em
  \usefont{OT1}{phv}{m}{n}
}
\postdate{
  \par
  \end{flushright}
}

% Mathematics Packages
\usepackage[Gray,squaren,thinqspace,cdot]{SIunits}      % elegant units
\usepackage{amsmath}                                    % extensive math options
\usepackage{amsfonts}                                   % special math fonts
\usepackage{mathtools}                                  % useful formatting commands
\usepackage{amsthm}                                     % useful commands for building theorem environments
\usepackage{amssymb}                                    % lots of special math symbols
\usepackage{mathrsfs}                                   % fancy scripts letters
\usepackage{cancel}                                     % cancel lines in math
\usepackage{esint}                                      % fancy integral symbols
\usepackage{relsize}                                    % make math things bigger or smaller
\usepackage{bm}                                         % bold math!

\newcommand\ddfrac[2]{\frac{\displaystyle #1}{\displaystyle #2}}    % elegant fraction formatting
\allowdisplaybreaks[1]                                              % allow align environments to break on pages

% Ensure numbering is section-specific
%--------------------------------------------------------------------------------------------
\numberwithin{equation}{section}
\numberwithin{figure}{section}
\numberwithin{table}{section}

% Citations, references, and annotations
%--------------------------------------------------------------------------------------------
\usepackage[small,bf,hang]{caption}        % captions
\usepackage{subcaption}                    % adds subfigure & subcaption
\usepackage{sidecap}                       % adds side captions
\usepackage{hyperref}                      % add hyperlinks to references
\usepackage[noabbrev,nameinlink]{cleveref} % better references than default \ref
\usepackage{autonum}                       % only number referenced equations
\usepackage{url}                           % urls
\usepackage{cite}                          % well formed numeric citations
% format hyperlinks
\colorlet{linkcolour}{black}
\colorlet{urlcolour}{blue}
\hypersetup{colorlinks=true,
            linkcolor=linkcolour,
            citecolor=linkcolour,
            urlcolor=urlcolour}

% Plotting and Figures
%--------------------------------------------------------------------------------------------
\usepackage{tikz}          % advanced vector graphics
\usepackage{pgfplots}      % data plotting
\usepackage{pgfplotstable} % table plotting
\usepackage{placeins}      % display floats in correct sections
\usepackage{graphicx}      % include external graphics
\usepackage{longtable}     % process long tables

% use most recent version of pgfplots
\pgfplotsset{compat=newest}

% Misc.
%--------------------------------------------------------------------------------------------
\usepackage{todonotes}  % add to do notes
\usepackage{epstopdf}   % process eps-images
\usepackage{float}      % floats
\usepackage{stmaryrd}   % some more nice symbols
\usepackage{emptypage}  % suppress page numbers on empty pages
\usepackage{multicol}   % use this for creating pages with multiple columns
\usepackage{etoolbox}   % adds tags for environment endings
\usepackage{tcolorbox}  % pretty colored boxes!


% Custom Commands
%--------------------------------------------------------------------------------------------
\newcommand\hr{\noindent\rule[0.5ex]{\linewidth}{0.5pt}}                % horizontal line
\newcommand\N{\ensuremath{\mathbb{N}}}                                  % blackboard set characters
\newcommand\R{\ensuremath{\mathbb{R}}}
\newcommand\Z{\ensuremath{\mathbb{Z}}}
\newcommand\Q{\ensuremath{\mathbb{Q}}}
\newcommand\C{\ensuremath{\mathbb{C}}}
\renewcommand{\arraystretch}{1.2}                                       % More space between table rows (could be 1.3)
\newcommand{\Cov}{\mathrm{Cov}}
\newcommand*{\dbar}{\ensuremath{\text{\dj}}}
% Custom Environments
%--------------------------------------------------------------------------------------------
\newcommand{\lecture}[3]{\hr\\{\centering{\large\textsc{Lecture #1: #3}}\\#2\\}\hr\markboth{Lecture #1: #3}{\rightmark}}   % command to title lectures
\usepackage{mdframed}
\theoremstyle{plain}
\newmdtheoremenv[nobreak]{theorem}{Theorem}[section]
\newtheorem{corollary}{Corollary}[theorem]
\newtheorem{lemma}[theorem]{Lemma}
\theoremstyle{definition}
\newtheorem*{ex}{Example}
\newmdtheoremenv[nobreak]{definition}{Definition}[section]
\theoremstyle{remark}
\newtheorem*{remark}{Remark}
\AtEndEnvironment{ex}{\null\hfill$\diamond$}%
% Note: A proof environment is already provided in the amsthm package
\tcbuselibrary{breakable}
\newenvironment{note}[1]{\begin{tcolorbox}[
    arc=0mm,
    colback=white,
    colframe=white!60!black,
    title=#1,
    fonttitle=\sffamily,
    breakable
]}{\end{tcolorbox}}
\newenvironment{problem}{\begin{tcolorbox}[
    arc=0mm,
    breakable,
    colback=white,
    colframe=black
]}{\end{tcolorbox}}

% Header and Footer
%--------------------------------------------------------------------------------------------
% set header and footer
\usepackage{fancyhdr}                       % header and footer
\pagestyle{fancy}                           % use package
\fancyhf{}
\fancyhead[LE,RO]{\textsl{\rightmark}}      % E for even (left pages), O for odd (right pages)
\fancyfoot[LE,RO]{\thepage}
\fancyfoot[LO,RE]{\textsl{\leftmark}}
\setlength{\headheight}{15pt}


% Physics
%--------------------------------------------------------------------------------------------
\usepackage[arrowdel]{physics}      % all the usual useful physics commands
%\usepackage{feyn}                   % for drawing Feynman diagrams
%\usepackage{bohr}                   % for drawing Bohr diagrams
\usepackage{elements}               % for quickly referencing information of various elements
\usepackage{tensor}                 % for writing tensors and chemical symbols

% Finishing touches
%--------------------------------------------------------------------------------------------
\author{Nathaniel D. Hoffman}

\title{33-765 Homework 9}
\date{\today}
\begin{document}
\maketitle

\section*{33. Statistical Physics of the $ N $-dimensional Quadratic Hamiltonian}
Consider a phase space with $ N $ degrees of freedom $ \{x_i\}_{i = 1,\ldots,N} $ and on it a quadratic Hamiltonian
\begin{equation}
    H = \frac{1}{2} \vb{x}^\top \vb{K} \vb{x}
\end{equation}
or in components,
\begin{equation}
    H = \frac{1}{2} x_i K_{ij} x_j
\end{equation}
where the ``kernel'' $ \vb{K} $ is symmetric and positive definite. To de-clutter the problem, we will ignore the beauty factor $ \frac{1}{N!h^N} $.
\begin{itemize}
    \item[1.] Show that the canonical partition function is given by $ Z:= \Tr(e^{- \beta H}) = \int \dd[N]{x} e^{- \frac{1}{2}\beta x_i K_{ij} x_j} = \left( \det \frac{\beta \vb{K}}{2 \pi} \right)^{-1/2} $.
        \begin{problem}
            If we change bases to one in which $ K_{ij} $ is diagonal ($ y_i = T_{ij} x_j $, $ \det(T) = 1 $ since $ \vb{T} $ must be orthogonal),
            \begin{align}
                Z &= \int \dd[N]{x} e^{- \frac{1}{2} \beta x_i K_{ij} x_j} \\
                &= \int \dd[N]{y} \det(T) e^{- \frac{1}{2} \beta y_i T_{ij}^{-1} K_{ij} T_{ij} y_j} \\
                &= \int \dd[N]{y} e^{- \frac{1}{2} \beta y_i K_{ii} y_i} \\
                &= \int \dd[N]{y} e^{\sum_{i=1}^{N} - \frac{1}{2} \beta K_{ii} y_i^2} \\
                &= \int \dd[N]{y} \prod_{i=1}^{N} e^{- \frac{1}{2} \beta K_{ii} y_i^2} \\
                &= \prod_{i=1}^{N} \int \dd{y_i} e^{- \frac{1}{2} \beta K_{ii} y_i^2} \\
                &= \prod_{i=1}^{N} \left( \sqrt{\frac{2\pi}{\beta K_{ii}}} \right) \\
                &= \left( \sqrt{\frac{2^N \pi^N}{\beta^N \prod_{i=1}^{N} K_{ii}}} \right)
            \end{align}
            where
            \begin{equation}
                \prod_{i=1}^{N} K_{ii} = \det(\vb{K})
            \end{equation}
            since the diagonalized $ \vb{K} $ has its eigenvalues on the diagonal and the determinant can be defined as the product of the eigenvalues.
            \begin{align}
                Z &= \left( \sqrt{\frac{2^N \pi^N}{\beta^N \det(\vb{K})}} \right) \\
                &= \left( \sqrt{\det\left( \frac{2 \pi}{\beta \vb{K}} \right)} \right) \\
                &= \left(\det \frac{\beta \vb{K}}{2 \pi} \right)^{-1/2}
            \end{align}
            We can bring the other constants in because they are multiplied $ N $-times, and
            \begin{equation}
                \det(\alpha \vb{K}) = \alpha^N \det(\vb{K})
            \end{equation}
            where $ N $ is the dimension of $ \vb{K} $.
        \end{problem}
    \item[2.] Starting with the result from problem 31.1, show that the equipartition theorem in this case can be written as
        \begin{equation}
            \ev{\vb{x} \otimes \vb{x}} \equiv \ev{\vb{x} \vb{x}^\top} = k_B T \vb{K}^{-1}
        \end{equation}
        or, in components,
        \begin{equation}
            \ev{x_i x_j} = k_B T K_{ij}^{-1}.
        \end{equation}
        \begin{problem}
            From problem 31.1, we know that
            \begin{equation}
                \ev{x_i \pdv{H}{x_j}} = k_B T \delta_{ij}
            \end{equation}
            \begin{align}
                \pdv{H}{x_k} &= \frac{1}{2} \partial_k \left( x_i K_{ij} x_j \right) = \frac{1}{2} \left( \delta_{ik} K_{ij} x_j + x_i K_{ij} \delta_{jk} \right) \\
                &= \frac{1}{2} \left( K_{kj} x_j + K_{ik} x_i \right) \\
                &= \frac{1}{2} \left( K_{kj} x_j + K_{ki} x_i \right) \\
                &= \frac{1}{2} \left( K_{kj} x_j + K_{kj} x_j \right) \\
                &= K_{kj} x_j \\
                \pdv{H}{x_j} &= K_{jk} x_k
            \end{align}
            so
            \begin{align}
                \ev{x_i K_{jk} x_k} &= k_B T \delta_{ij} \\
                \ev{x_i x_k} &= k_B T \delta_{ij} K_{jk} \\
                &= k_B T K_{ik} \\
                \ev{x_i x_j} &= k_B T K_{ij}
            \end{align}
        \end{problem}
    \item[3.] We now amend the Hamiltonian by a ``source term'', $ H = \frac{1}{2} \vb{x}^\top \vb{K} \vb{x} - \vb{J} \vdot \vb{x} $. This Hamiltonian is still quadratic, but it takes its minimum not at $ \vb{x} = \vb{0} $ but at some displaced value $ \vb{x}^* $. Find it!
        \begin{problem}
            To minimize the Hamiltonian, we set its derivative to zero and solve:
            \begin{align}
                \pdv{H}{\vb{x}} &= \frac{1}{2} \left( \vb{x}^\top \vdot \vb{K} + \vb{K}^\top \vdot \vb{x} \right) - \vb{J} \\
                0 &= \vb{K}^\top \vb{x}^* - \vb{J} \\
                \vb{x}^* &= \vb{K}^{-1} \vdot \vb{J}
            \end{align}
        \end{problem}
    \item[4.] Use your result from the previous part to complete the square of this shifted quadratic matrix expression. In other words, write it in the form $ H = \frac{1}{2} (\vb{x} - \vb{x}^*)^\top \vb{K} (\vb{x} - \vb{x}^*) + \text{stuff} $.
        \begin{problem}
            \begin{align}
                \frac{1}{2} \left( \vb{x} - \vb{x}^* \right)^\top \vb{K} (\vb{x} - \vb{x}^*) &= \frac{1}{2}\left( \vb{x}^\top \vb{K} \vb{x} + (\vb{x}^*)^\top \vb{K} \vb{x}^* - (\vb{x}^*)^\top \vb{K} \vb{x} - \vb{x}^\top \vb{K} \vb{x}^*\right) \\
                &= \frac{1}{2} \left( \vb{x}^\top \vb{K} \vb{x} + (\vb{K}^{-1} \vdot \vb{J})^\top \vb{K} \vb{K}^{-1} \vdot \vb{J} - (\vb{K}^{-1} \vdot \vb{J})^\top \vb{K} \vb{x} - \vb{x}^\top \vb{K} \vb{K}^{-1} \vdot \vb{J} \right) \\
                &= \frac{1}{2} \left( \vb{x}^\top \vb{K} \vb{x} + \vb{J}^\top \vdot (\vb{K}^{-1})^\top \vb{K} \vb{K}^{-1} \vdot \vb{J} - \vb{J}^\top \vdot (\vb{K}^{-1})^\top \vb{K} \vb{x} - \vb{x}^\top \vb{K} \vb{K}^{-1} \vdot \vb{J} \right) \\
                &= \frac{1}{2} \left( \vb{x}^\top \vb{K} \vb{x} + \vb{J}^\top \vb{K}^{-1} \vb{J} - \vb{J}^\top \vb{x} - \vb{x}^\top \vb{J} \right) \\
                &= \frac{1}{2} \vb{x}^\top \vb{K} \vb{x} + \frac{1}{2} \vb{J}^\top \vb{K}^{-1} \vb{J} - \vb{J} \vdot \vb{x} \\
                &= H + \frac{1}{2} \vb{J}^\top \vb{K}^{-1} \vb{J}
            \end{align}
            so
            \begin{equation}
                H = \frac{1}{2} (\vb{x} - \vb{x}^*)^\top \vb{K} (\vb{x} - \vb{x}^*) - \frac{1}{2} \vb{J}^\top \vb{K}^{-1} \vb{J}
            \end{equation}
        \end{problem}
    \item[5.] Show that for a general $ \vb{J} \neq \vb{0} $, the partition function is given by $ Z = \Tr e^{- \beta H} = \left( \det \frac{\beta \vb{K}}{2 \pi} \right)^{- 1/2}e^{\frac{1}{2} \beta \vb{J}^\top \vb{K}^{-1} \vb{J}} $.
        \begin{problem}
            This follows easily from the first part of this problem along with the fact that the part we just added by completing the square doesn't depend on $ \vb{x} $, so we can factor it out:
            \begin{align}
                Z = \int \dd[N]{x} e^{- \frac{1}{2} \beta x_i K_{ij} x_j} e^{ \frac{1}{2} \beta \vb{J}^\top \vb{K}^{-1} \vb{J}} &= e^{\frac{1}{2} \beta \vb{J}^\top \vb{K}^{-1} \vb{J}} \int \dd[N]{x} e^{- \frac{1}{2} \beta x_i K_{ij} x_j}\\
                &= \left( \det \frac{\beta \vb{K}}{2 \pi} \right)^{- 1/2}e^{\frac{1}{2} \beta \vb{J}^\top \vb{K}^{-1} \vb{J}}
            \end{align}
        \end{problem}
    \item[6.] Prove that $ \ev{x_i x_j} = \ev{x_i x_j}_{\vb{J} = \vb{0}} + \ev{x_i} \ev{x_j} $. Hence, unsurprisingly, the covariance $ \text{Cov}(x_i, x_j) $ does not depend on $ \vb{J} $.
        \begin{problem}
            \begin{align}
                \ev{x_i \pdv{H}{x_j}} &= \ev{x_i (x_k K_{kj} - J_j)} \\
                &= \ev{x_i x_k K_{kj} - x_i J_j} \\
                &= K_{kj} \ev{x_i x_k - x_i J_i K_{kj}^{-1}} \\
                k_{B} T \delta_{ij} &= K_{kj} \ev{x_i x_k - x_i x_k^*} \\
                k_B T K_{kj}^{-1} \delta_{ij} &= \ev{x_i x_k} - \ev{x_i} x^*_j \\
                k_B T K^{-1}_{kj} &= \ev{x_i x_j} - \ev{x_i} \ev{x_j} \\
                \ev{x_i x_j}_{\vb{J} = \vb{0}} &= \ev{x_i x_j} - \ev{x_i} \ev{x_j}
            \end{align}
        \end{problem}
    \item[7.] Verifying that $ k_B T \pdv{J_k} e^{- \beta H} = x_k e^{- \beta H} $, re-derive the equipartition theorem by continuing the following calculation:
        \begin{equation}
            \text{Cov}(x_i, x_j) = \ev{x_i x_j}_{\vb{J} = \vb{0}} = \eval{\frac{\Tr(x_i x_j e^{- \beta H})}{\Tr(e^{- \beta H})}}_{\vb{J} = \vb{0}} = \eval{\frac{\Tr \left( \left( k_B T \pdv{J_i} \right) \left( k_B T \pdv{J_j} \right) e^{- \beta H} \right)}{\Tr(e^{- \beta H})}}_{\vb{J} = \vb{0}} = \cdots
        \end{equation}
        \begin{problem}
            Looking at the numerator, we can pull the derivatives out of the trace because the trace is an integral over $ x $ and $ J $ does not depend on $ x $. Ignoring the factors of $ K_B T $ (for now), the numerator reads:
            \begin{equation}
                \partial_{J_i} \partial_{J_j} \Tr e^{- \beta H} = \partial_{J_i} \partial_{J_j} \left( \det \frac{\beta \vb{K}}{2 \pi} \right)^{-1/2} e^{\frac{1}{2} \beta \vb{J}^\top \vb{K}^{-1} \vb{J}}
            \end{equation}
            For brevity, I will define $ A \equiv \left( \det \frac{\beta \vb{K}}{2 \pi} \right)^{-1/2} $.
            \begin{align}
                \partial_{J_i} \partial_{J_j} A e^{\frac{1}{2} \beta J_k K_{kl}^{-1} J_l} &= \partial_{J_i} A \left( e^{\frac{1}{2} \beta J_k K_{kl}^{-1} J_l} \right) \frac{\beta}{2} \left( \delta_{jk} K_{kl}^{-1} J_l + J_k K_{kl}^{-1} \delta_{jl} \right)\\
                &= \partial_{J_i} A \left( e^{\frac{1}{2} \beta J_k K_{kl}^{-1} J_l} \right) \frac{\beta}{2} \left( K_{jl}^{-1} J_l + J_k K_{kj}^{-1} \right) \\
                &= \partial_{J_i} A \left( e^{\frac{1}{2} \beta J_k K_{kl}^{-1} J_l} \right) \beta (K_{jl}^{-1} J_l) \\
                &= A \left( e^{\frac{1}{2} \beta J_k K_{kl}^{-1} J_l} \right) \partial_{J_i}\beta (K_{jl}^{-1} J_l) + A \partial_{J_i} \left( e^{\frac{1}{2} \beta J_k K_{kl}^{-1} J_l} \right) \beta (K_{jl}^{-1} J_l) \\
                &= A \left( e^{\frac{1}{2} \beta J_k K_{kl}^{-1} J_l} \right) \left[ \beta K_{jl}^{-1} \delta_{li} + K_{jl}^{-1} J_l \frac{\beta^2}{2} (\delta_{ik} K_{kl}^{-1} J_l + J_k K_{kl}^{-1} \delta_{il}) \right] \\
                &= A \left( e^{\frac{1}{2} \beta J_k K_{kl}^{-1} J_l} \right) \left[ \beta K_{jl}^{-1} \delta_{li} + K_{jl}^{-1} J_l \frac{\beta^2}{2} (K_{il}^{-1} J_l + J_k K_{ki}^{-1})\right] \\
                &= A \left( e^{\frac{1}{2} \beta J_k K_{kl}^{-1} J_l} \right) \left[ \beta K_{jl}^{-1} \delta_{li} + K_{jl}^{-1} J_l \beta^2 K_{il}^{-1} J_l \right] \\
                &= \Tr e^{- \beta H} \left[ \beta K_{jl}^{-1} \delta_{li} + K_{jl}^{-1} J_l \beta^2 K_{il}^{-1} J_l \right]
            \end{align}
            However, this whole thing is being evaluated at $ \vb{J} = \vb{0} $. If we reinsert this expression into the equation, the traces in the numerator and denominator cancel. I will write the factors of $ k_B T k_B T \equiv \frac{1}{\beta^2} $:
            \begin{align}
                \ev{x_i x_j} &= \frac{1}{\beta^2}  \left[ \beta K_{jl}^{-1} \delta_{li} + \cancelto{0}{K_{jl}^{-1} J_l \beta^2 K_{il}^{-1} J_l} \right] \\
                &= \frac{1}{\beta} K_{ij}^{-1} \\
                &= k_B T K_{ij}^{-1}
            \end{align}
        \end{problem}
\end{itemize}

\section*{34. Statistical Physics of the Double Pendulum}
Consider a planar double pendulum: two masses $ m_1 $ and $ m_2 $, two pendulum lengths $ l_1 $ and $ l_2 $, and two degrees of freedom $ \varphi_1 $ and $ \varphi_2 $.
\begin{itemize}
    \item[1.] Write down the Lagrangian $ L(\varphi_1, \varphi_2, \dot{\varphi}_1, \dot{\varphi}_2) $ of the system.
        \begin{problem}
            \begin{equation}
                L = \text{KE} - \text{PE}
            \end{equation}
            where
            \begin{equation}
                KE = \sum_i \frac{1}{2} m_i v_i^2 = \sum_i \frac{1}{2} m_i (\dot{x}_i^2 + \dot{y}_i^2)
            \end{equation}
            and
            \begin{equation}
                PE = \sum_i mgy_i
            \end{equation}
            Of course, this is in Cartesian coordinates, not the coordinates given by the problem. In the given coordinates,
            \begin{alignat}{3}
                x_1 &= l_1 \sin(\varphi_1) \qquad && y_1 &&= -l \cos(\varphi_1) \\
                x_2 &= l_1 \sin(\varphi_1) + l_2 \sin(\varphi_2) \qquad && y_2 &&= -l_1 \cos(\varphi_1) - l_2 \cos(\varphi_2)
            \end{alignat}
            In these coordinates, we find that
            \begin{equation}
                \text{KE} = \frac{m_1 + m_2}{2} l_1^2 \dot{\varphi}_1^2 + \frac{m_2}{2} l_2^2 \dot{\varphi}_2^2 + m_2 l_1 l_2 \dot{\varphi}_1 \dot{\varphi}_2 \cos(\varphi_1 - \varphi_2)
            \end{equation}
            and
            \begin{equation}
                \text{PE} = - (m_1 + m_2) l_1 g \cos(\varphi_1) - m_2 l_2 g \cos(\varphi_2)
            \end{equation}
            so
            \begin{equation}
                L = \frac{m_1 + m_2}{2} l_1^2 \dot{\varphi}_1^2 + \frac{m_2}{2} l_2^2 \dot{\varphi}_2^2 + m_2 l_1 l_2 \dot{\varphi}_1 \dot{\varphi}_2 \cos(\varphi_1 - \varphi_2a) + (m_1 + m_2) l_1 g \cos(\varphi_1) + m_2 l_2 g \cos(\varphi_2)
            \end{equation}
        \end{problem}
    \item[2.] Expand the Lagrangian to quadratic order.
        \begin{problem}
            To quadratic order, $ \cos(\varphi) \to 1 - \frac{\varphi^2}{2} $ and $ \cos(\varphi_1 - \varphi_2) \to 1 + \varphi_1 \varphi_2 - \frac{1}{2} \left( \varphi_1^2 - \varphi_2^2 \right) $. When we put these substitutions into the Lagrangian and cancel anything else that is now of higher than quadratic order, we find that
            \begin{align}
                \text{KE} &= \frac{m_1 + m_2}{2} l_1^2 \dot{\varphi}_1^2 + \frac{m_2}{2} l_2^2 \dot{\varphi}_2^2 + m_2 l_1 l_2 \dot{\varphi}_1 \dot{\varphi}_2 \\
                &= \dot{\vb{\varphi}}^\top \vb{A} \dot{\vb{\varphi}}
            \end{align}
            where
            \begin{equation}
                \vb{A} = \frac{1}{2} \mqty( (m_1 + m_2)l_1^2 & m_2 l_1 l_2 \\ m_2 l_1 l_2 & m_2 l_2^2)
            \end{equation}
            The potential energy becomes
            \begin{align}
                \text{PE} &= - (m_1 + m_2)l_1 g \left( 1 - \frac{1}{2} \varphi_1^2 \right) - m_2 l_2 g \left( 1 - \frac{1}{2} \varphi_2^2 \right) \\
                &= - g l_1 (m_1 + m_2) - g l_2 m_2 + \frac{1}{2} g \left( l_1 (m_1 + m_2) \varphi_1^2 + l_2 m_2 \varphi_2^2 \right) \\
                &= \vb{\varphi}^\top \vb{B} \vb{\varphi}
            \end{align}
            where
            \begin{equation}
                \vb{B} = \frac{g}{2} \mqty( (m_1 + m_2)l_1 & 0 \\ 0 & m_2 l_2)
            \end{equation}
            and we can eliminate the two constant terms out in front because they will not effect the equations of motion. All together, the Lagrangian now reads as
            \begin{equation}
                L = \dot{\vb{\varphi}}^\top \vb{A} \dot{\vb{\varphi}} - \vb{\varphi}^\top \vb{B} \vb{\varphi}
            \end{equation}
        \end{problem}
    \item[3.] Calculate the canonically conjugate momenta $ p_1 $ and $ p_2 $ belonging to $ \varphi_1 $ and $ \varphi_2 $.
        \begin{problem}
            The canonical conjugate momenta are defined as
            \begin{equation}
                \vb{p} = \pdv{L}{\dot{\vb{\varphi}}} = \vb{A}^\top \dot{\vb{\varphi}} + \dot{\vb{\varphi}}^\top \vb{A} = 2 \vb{A} \dot{\vb{\varphi}}
            \end{equation}
            so
            \begin{equation}
                \dot{\vb{\varphi}} = \frac{1}{2} \vb{A}^{-1} \vb{p}
            \end{equation}
            Note that one of the operations I did above was only possible because $ \vb{A} $ is symmetric, so it is equal to its transpose.
        \end{problem}
    \item[4.] Find the Hamiltonian $ H(\varphi_1, \varphi_2, p_1, p_2) $ of the system.
        \begin{problem}
            The Hamiltonian is defined by
            \begin{equation}
                H = \text{KE} + \text{PE}
            \end{equation}
            and substituting in the conjugate momenta from the previous problem, we find
            \begin{equation}
                H = \frac{1}{4} \vb{p}^\top \vb{A}^{-1} \vb{p} + \vb{\varphi}^\top \vb{B} \vb{\varphi}
            \end{equation}
        \end{problem}
    \item[5.] Show that the kinetic energy has the form $ \frac{1}{2} \vb{x}^\top \vb{K} \vb{x} $ that we discussed in the previous problem. What is $ \vb{x} $ and what is $ \vb{K} $?
        \begin{problem}
            The kinetic energy sure seems to have the same form with $ \vb{x} = \vb{p} $ and $ \vb{K} = \frac{1}{2} \vb{A}^{-1} $. We must determine if $ \vb{A}^{-1} $ is positive definite and symmetric. We can write it as
            \begin{equation}
                \frac{1}{2} \vb{A}^{-1} = \frac{1}{4\det(\vb{A})} \mqty( m_2 l_2^2 & - m_2 l_1 l_2 \\ - m_2 l_1 l_2 & (m_1 + m_2) l_1^2) \equiv \vb{K}
            \end{equation}
            By observation, it is symmetric. Additionally, it must be positive definite since all of it's eigenvalues
            \begin{equation}
                \lambda = \frac{1}{2} \Tr(\vb{K}) \pm \sqrt{\frac{1}{4} \Tr(\vb{K})^2 - \det(\vb{K})}
            \end{equation}
            are positive, as a result of $ \det(\vb{K}) = \frac{1}{m_1 m_2 l_1^2 l_2^2} > 0 $. 
        \end{problem}
    \item[6.] Calculate $ \vb{K}^{-1} $.
        \begin{problem}
            \begin{equation}
                \vb{K}^{-1} = 2 \vb{A} = \mqty( (m_1 + m_2)l_1^2 & m_2 l_1 l_2 \\ m_2 l_1 l_2 & m_2 l_2^2)
            \end{equation}
        \end{problem}
    \item[7.] Finally, let's turn up the heat: If this system is in contact with a heat bath at temperature $ T $, calculate the correlation coefficient between $ p_1 $ and $ p_2 $!
        \begin{problem}
            The correlation can be written as
            \begin{align}
                \rho_{p_i p_j} = \frac{\text{Cov}(p_i, p_j)}{\sigma_{p_i} \sigma_{p_j}} &= \frac{\text{Cov}(p_i, p_j)}{\sqrt{\text{Cov}(p_i, p_i)\text{Cov}(p_j, p_j)}}\\
                &= \frac{\ev{p_i p_j}}{\sqrt{\ev{p_i^2} \ev{p_j^2}}}\\
                &= \frac{\vb{K}_{ij}^{-1}}{\sqrt{\vb{K}_{ii}^{-1} \vb{K}_{jj}^{-1}}}\\
                \rho_{p_1 p_2} &= \frac{m_2 l_1 l_2}{\sqrt{( (m_1 + m_2)l_1^2)(m_2 l_2^2)}}\\
                &= \frac{m_2}{\sqrt{m_1 + m_2^2}}
            \end{align}
            In the limit of $ m_2 >> m_1 $, we see that $ \rho_{p_1 p_2} \to 1 $, indicating that as the middle mass becomes less consequential, the momenta become directly correlated. When $ m_2 << m_1 $, $ \rho_{p_i p_j} \to 0 $, indicating that in this limit, the momenta will have no correlation at all.
        \end{problem}
\end{itemize}

\end{document}
