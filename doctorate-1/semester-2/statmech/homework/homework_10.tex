\documentclass[a4paper,twoside]{article}
% My LaTeX preamble file - by Nathaniel Dene Hoffman
% Credit for much of this goes to Olivier Pieters (https://olivierpieters.be/tags/latex)
% and Gilles Castel (https://castel.dev)
% There are still some things to be done:
% 1. Update math commands using mathtools package (remove ddfrac command and just override)
% 2. Maybe abbreviate \imath somehow?
% 3. Possibly format for margin notes and set new margin sizes
% First, some encoding packages and usefull formatting
%--------------------------------------------------------------------------------------------
\usepackage[l2tabu,orthodox]{nag}   % force newer (and safer) LaTeX commands
\usepackage[utf8]{inputenc}         % set character set to support some UTF-8
                                    %   (unicode). Do NOT use this with
                                    %   XeTeX/LuaTeX!
\usepackage[T1]{fontenc}
\usepackage[english]{babel}         % multi-language support
\usepackage{sectsty}                % allow redefinition of section command formatting
\usepackage{tabularx}               % more table options
\usepackage{booktabs}
\usepackage{titling}                % allow redefinition of title formatting
\usepackage{imakeidx}               % create and index of words
\usepackage{xcolor}                 % more colour options
\usepackage{enumitem}               % more list formatting options
\usepackage{tocloft}                % redefine table of contents, new list like objects
\usepackage{subfiles}               % allow for multifile documents

% Next, let's deal with the whitespaces and margins
%--------------------------------------------------------------------------------------------
\usepackage[centering,margin=1in]{geometry}
\setlength{\parindent}{0cm}
\setlength{\parskip}{2ex plus 0.5ex minus 0.2ex} % whitespace between paragraphs

% Redefine \maketitle command with nicer formatting
%--------------------------------------------------------------------------------------------
\pretitle{
  \begin{flushright}         % align text to right
    \fontsize{40}{60}        % set font size and whitespace
    \usefont{OT1}{phv}{b}{n} % change the font to bold (b), normally shaped (n)
                             %   Helvetica (phv)
    \selectfont              % force LaTeX to search for metric in its mapping
                             %   corresponding to the above font size definition
}
\posttitle{
  \par                       % end paragraph
  \end{flushright}           % end right align
  \vskip 0.5em               % add vertical spacing of 0.5em
}
\preauthor{
  \begin{flushright}
    \large                   % font size
    \lineskip 0.5em          % inter line spacing
    \usefont{OT1}{phv}{m}{n}
}
\postauthor{
  \par
  \end{flushright}
}
\predate{
  \begin{flushright}
  \large
  \lineskip 0.5em
  \usefont{OT1}{phv}{m}{n}
}
\postdate{
  \par
  \end{flushright}
}

% Mathematics Packages
\usepackage[Gray,squaren,thinqspace,cdot]{SIunits}      % elegant units
\usepackage{amsmath}                                    % extensive math options
\usepackage{amsfonts}                                   % special math fonts
\usepackage{mathtools}                                  % useful formatting commands
\usepackage{amsthm}                                     % useful commands for building theorem environments
\usepackage{amssymb}                                    % lots of special math symbols
\usepackage{mathrsfs}                                   % fancy scripts letters
\usepackage{cancel}                                     % cancel lines in math
\usepackage{esint}                                      % fancy integral symbols
\usepackage{relsize}                                    % make math things bigger or smaller
\usepackage{bm}                                         % bold math!

\newcommand\ddfrac[2]{\frac{\displaystyle #1}{\displaystyle #2}}    % elegant fraction formatting
\allowdisplaybreaks[1]                                              % allow align environments to break on pages

% Ensure numbering is section-specific
%--------------------------------------------------------------------------------------------
\numberwithin{equation}{section}
\numberwithin{figure}{section}
\numberwithin{table}{section}

% Citations, references, and annotations
%--------------------------------------------------------------------------------------------
\usepackage[small,bf,hang]{caption}        % captions
\usepackage{subcaption}                    % adds subfigure & subcaption
\usepackage{sidecap}                       % adds side captions
\usepackage{hyperref}                      % add hyperlinks to references
\usepackage[noabbrev,nameinlink]{cleveref} % better references than default \ref
\usepackage{autonum}                       % only number referenced equations
\usepackage{url}                           % urls
\usepackage{cite}                          % well formed numeric citations
% format hyperlinks
\colorlet{linkcolour}{black}
\colorlet{urlcolour}{blue}
\hypersetup{colorlinks=true,
            linkcolor=linkcolour,
            citecolor=linkcolour,
            urlcolor=urlcolour}

% Plotting and Figures
%--------------------------------------------------------------------------------------------
\usepackage{tikz}          % advanced vector graphics
\usepackage{pgfplots}      % data plotting
\usepackage{pgfplotstable} % table plotting
\usepackage{placeins}      % display floats in correct sections
\usepackage{graphicx}      % include external graphics
\usepackage{longtable}     % process long tables

% use most recent version of pgfplots
\pgfplotsset{compat=newest}

% Misc.
%--------------------------------------------------------------------------------------------
\usepackage{todonotes}  % add to do notes
\usepackage{epstopdf}   % process eps-images
\usepackage{float}      % floats
\usepackage{stmaryrd}   % some more nice symbols
\usepackage{emptypage}  % suppress page numbers on empty pages
\usepackage{multicol}   % use this for creating pages with multiple columns
\usepackage{etoolbox}   % adds tags for environment endings
\usepackage{tcolorbox}  % pretty colored boxes!


% Custom Commands
%--------------------------------------------------------------------------------------------
\newcommand\hr{\noindent\rule[0.5ex]{\linewidth}{0.5pt}}                % horizontal line
\newcommand\N{\ensuremath{\mathbb{N}}}                                  % blackboard set characters
\newcommand\R{\ensuremath{\mathbb{R}}}
\newcommand\Z{\ensuremath{\mathbb{Z}}}
\newcommand\Q{\ensuremath{\mathbb{Q}}}
\newcommand\C{\ensuremath{\mathbb{C}}}
\renewcommand{\arraystretch}{1.2}                                       % More space between table rows (could be 1.3)
\newcommand{\Cov}{\mathrm{Cov}}
\newcommand*{\dbar}{\ensuremath{\text{\dj}}}
% Custom Environments
%--------------------------------------------------------------------------------------------
\newcommand{\lecture}[3]{\hr\\{\centering{\large\textsc{Lecture #1: #3}}\\#2\\}\hr\markboth{Lecture #1: #3}{\rightmark}}   % command to title lectures
\usepackage{mdframed}
\theoremstyle{plain}
\newmdtheoremenv[nobreak]{theorem}{Theorem}[section]
\newtheorem{corollary}{Corollary}[theorem]
\newtheorem{lemma}[theorem]{Lemma}
\theoremstyle{definition}
\newtheorem*{ex}{Example}
\newmdtheoremenv[nobreak]{definition}{Definition}[section]
\theoremstyle{remark}
\newtheorem*{remark}{Remark}
\AtEndEnvironment{ex}{\null\hfill$\diamond$}%
% Note: A proof environment is already provided in the amsthm package
\tcbuselibrary{breakable}
\newenvironment{note}[1]{\begin{tcolorbox}[
    arc=0mm,
    colback=white,
    colframe=white!60!black,
    title=#1,
    fonttitle=\sffamily,
    breakable
]}{\end{tcolorbox}}
\newenvironment{problem}{\begin{tcolorbox}[
    arc=0mm,
    breakable,
    colback=white,
    colframe=black
]}{\end{tcolorbox}}

% Header and Footer
%--------------------------------------------------------------------------------------------
% set header and footer
\usepackage{fancyhdr}                       % header and footer
\pagestyle{fancy}                           % use package
\fancyhf{}
\fancyhead[LE,RO]{\textsl{\rightmark}}      % E for even (left pages), O for odd (right pages)
\fancyfoot[LE,RO]{\thepage}
\fancyfoot[LO,RE]{\textsl{\leftmark}}
\setlength{\headheight}{15pt}


% Physics
%--------------------------------------------------------------------------------------------
\usepackage[arrowdel]{physics}      % all the usual useful physics commands
%\usepackage{feyn}                   % for drawing Feynman diagrams
%\usepackage{bohr}                   % for drawing Bohr diagrams
\usepackage{elements}               % for quickly referencing information of various elements
\usepackage{tensor}                 % for writing tensors and chemical symbols

% Finishing touches
%--------------------------------------------------------------------------------------------
\author{Nathaniel D. Hoffman}

\title{33-765 Homework 10}
\date{\today}
\begin{document}
\maketitle

\section*{35. Equivalent Characterizations of Pure Quantum States}
If we have a normalized state vector $\ket{\psi} \in \mathcal{H} $, then the quantum state $ W =\ket{\psi}\bra{\psi} $ is pure, and it is obvious that $ W = W^2 $. Prove that the converse is also true: If a quantum state satisfies $ W = W^2 $, then there exists a vector $\ket{\psi} $ such that $ W =\ket{\psi}\bra{\psi} $.
\begin{problem}
    If $ W $ is a quantum state, we can surely expand it in eigenstates:
    \begin{equation}
        W = \sum_i p_i\ket{\psi_i}\bra{\psi_i}
    \end{equation}
    Since $ W = W^2 $,
    \begin{equation}
        W^2 = \sum_{ij} p_i p_j\ket{\psi_i}\bra{\psi_i}\ket{\psi_j}\bra{\psi_j} = \sum_{ij} p_i p_j\ket{\psi_i}\bra{\psi_j} \delta_{ij} = \sum_{i} p_i^2\ket{\psi_i}\bra{\psi_i}
    \end{equation}
    so
    \begin{equation}
        p_i = p_i^2
    \end{equation}
    or $ p_i = 1 \qor 0 $. We require $ \sum_i p_i = 1 $ for normalization, so at most one of the $ p_i $ can be $ 1 $ while the others must be zero, so the state can be represented by a single eigenvector.
\end{problem}

\section*{36. Equivalent Characterizations of Eigenstates}
If a quantum state $ W $ is an eigenstate of an observable $ A $, then measurements of $ A $ in that state are sharp: $ \sigma_A^2 = \ev{A^2} - \ev{A}^2 = 0 $. But the converse also holds: $ \sigma^2_A = 0 \implies AW = \alpha W $ for some $ \alpha \in \R $. Prove this in the special case where $ W $ is a pure state!
\begin{problem}
    The Cauchy-Schwarz inequality states that
    \begin{equation}
        \abs{\bra{u}\ket{v}}^2 \leq\bra{u}\ket{u}\bra{v}\ket{v}
    \end{equation}
    The equality condition means that the vectors are linearly dependent, or $ u = \lambda v $. For our vectors, we have
    \begin{align}
        \abs{\bra{\psi} A\ket{\psi}}^2 &\leq\bra{\psi}\ket{\psi}\bra{\psi} A^\dagger A\ket{\psi} \\
        \ev{A}^2 &\leq \ev{A^2}
    \end{align}
    From our assumption that $ \sigma^2_A = 0 $, we know that this must be an equality, but that also implies that the vectors are linearly dependent:
    \begin{equation}
        A\ket{\psi} = \lambda\ket{\psi}
    \end{equation}
    or equivalently
    \begin{equation}
        A\ket{\psi}\bra{\psi} = \lambda\ket{\psi}\bra{\psi}
    \end{equation}
\end{problem}

\section*{37. Quantum Fluctuations Can Only Increase the Free Energy}
Consider the quantum mechanical one-particle Hamiltonian $ H(P,Q) = \frac{P^2}{2m} + V(Q) $ and its classical analogue. In this problem, we want to prove the folowing inequality between the quantum and the classical free energy for this system:
\begin{equation}
    F_{\text{classical}} \leq F_{\text{quantum}}.
\end{equation}
You will need to use the Golden-Thompson inequality, which states that for two self-adjoint operators $ A $ and $ B $ which might not commute, $ \Tr e^{A+B} \leq \Tr(e^{A e^{B}}) $. 
\begin{problem}
    \begin{align}
        Z_{\text{QM}} = \Tr \left[ e^{- \beta \left( \frac{P^2}{2m} + V(Q) \right)} \right] &\leq \Tr \left[ e^{- \beta \frac{P^2}{2m}} e^{- \beta V(Q)} \right] \\
        &\leq \int \dd{q}\bra{q} e^{- \beta \frac{P^2}{2m}} e^{- \beta V(Q)}\ket{q} \\
        &\leq \int \dd{q} \int \dd{p}\bra{q} e^{- \beta \frac{P^2}{2m}}\ket{p}\bra{p} e^{- \beta V(Q)}\bra{q} \\
        &\leq \int \dd{q} \int \dd{p} e^{- \beta \frac{p^2}{2m}} e^{- \beta V(q)} \frac{1}{\sqrt{h}} e^{\imath pq / \hbar} e^{- \imath pq / \hbar} \\
        &\leq \frac{1}{h} \int \dd{q} \int \dd{p} e^{- \beta \frac{p^2}{2m}} e^{- \beta V(q)} \\
        Z_{\text{QM}} &\leq Z_{\text{CM}}
    \end{align}
    Therefore, $ \ln(Z_{\text{QM}}) \leq \ln(Z_{\text{CM}}) $ so $ - F_{\text{QM}} \leq - F_{\text{CM}} $ or $ F_{\text{QM}} \geq F_{\text{CM}} $.
\end{problem}

\section*{38. A System of Spin-1 Particles on a Lattice}
Consider a macroscopic crystal with a spin-1 quantum mechanical moment located on each of $ N $ atoms. Assume further that we can represent the energy eigenvalues of the system with a Hamiltonian of the following form:
\begin{equation}
    H = B \sum_{n=1}^{N} \sigma_n + D \sum_{n=1}^{N} \sigma^2_n,
\end{equation}
where the $ \sigma_n $ can independently take values in $ \{-1, 0, +1\} $ and $ B $ and $ D $ are constants representing an external magnetic field and an internal ``crystal field'' respectively. The entire system is in contact with a heat bath at temperature $ T $.
\begin{itemize}
    \item[1.] Calculate the canonical partition function and the free energy of this system.
        \begin{problem}
            We can factorize over each particle, so $ Z = N e^{- \beta H} $. I'll then divide this up into exponentials where $ \sigma_n $ is equal to each of its respective values:
            \begin{align}
                Z &= N e^{- \beta \left( B \sum_n \sigma_n + D \sum_n \sigma^2_n \right)} \\
                &= N \prod_{n=1}^{N} e^{- \beta (B \sigma_n + D \sigma_n^2)} \\
                &= \left( e^{- \beta \underbrace{(-B + D)}_{\sigma_n = -1}} + e^{- \beta \underbrace{(0)}_{\sigma_n = 0}} + e^{- \beta \underbrace{(B + D)}_{\sigma_n = 1}} \right)^{N} \\
                &= \left( \left( e^{\beta B} + e^{\beta B} \right) e^{- \beta D} + 1 \right)^N \\
                Z &= \left( 2 \cosh(\beta B) e^{- \beta D} + 1 \right)^N
            \end{align}
            so
            \begin{equation}
                F = - k_B T \ln(\left[ 2 \cosh(\beta B) e^{- \beta D} + 1 \right]^N)
            \end{equation}
        \end{problem}
    \item[2.] Calculate the magnetization per spin, $ m = \frac{1}{N} \ev{\sum_{n=1}^{N} \sigma_n} $, and plot $ m(B) $ for selected values of $ \beta D $.
        \begin{problem}
            I don't understand what I'm supposed to do here.
        \end{problem}
\end{itemize}

\section*{39. Polylogarithms}
The quantum grand potential of ideal Bose and Fermi gases can be expressed analytically via special functions called polylogarithms. The polylogarithm is defined as follows:
\begin{equation}
    L_{\nu}(z) = \frac{1}{\Gamma(\nu)} \int_0^{\infty} \dd{t} \frac{t^{\nu - 1}}{z^{-1} e^{t} - 1} \qquad z < 1, \quad \nu > 0.
\end{equation}
Prove that the polylogarithm has the following properties:
\begin{itemize}
    \item[1.] For $ \nu > 1 $, we can rewrite it as $ L_{\nu}(z) = - \frac{1}{\Gamma(\nu - 1)} \int_0^{\infty} \dd{t} t^{\nu -2} \ln(1-ze^{-t}) $.
        \begin{problem}
            \begin{equation}
                L_{\nu}(z) = \frac{1}{\Gamma(\nu)} \int_0^{\infty}  \underbrace{t^{\nu - 1}}_{u} \underbrace{(z^{-1} e^{t} - 1) \dd{t}}_{\dd{v}}
            \end{equation}
            \begin{equation}
                v = \ln(1 - z e^{-t})
            \end{equation}
            \begin{equation}
                \dd{u} = (\nu - 1) t^{\nu - 2} \dd{t}
            \end{equation}
            so
            \begin{align}
                L_{\nu}(z) &= \frac{1}{\Gamma(\nu)} \left[ \cancelto{0}{\eval{t^{\nu - 1} \ln(1 - z e^{- t})}_{0}^{\infty}} - \int (\nu - 1)t^{\nu - 2} \ln(1 - z e^{- t}) \dd{t} \right] \\
                &= - \underbrace{\frac{(\nu - 1)}{\Gamma(\nu)}}_{\frac{1}{\Gamma(\nu - 1)}} \int_0^{\infty} \dd{t} t^{\nu - 2} \ln(1-z e^{-t})
            \end{align}
        \end{problem}
    \item[2.] $ z \dv{z} L_{\nu + 1}(z) = L_{\nu}(z) $.
        \begin{problem}
            \begin{align}
                z \dd{z} L_{\nu + 1}(z) &= - \frac{z}{\Gamma(\nu)} \int_{0}^{\infty} \dd{t} \dv{z} t^{\nu - 1} \ln(1 - z e^{-t}) \\
                &= - \frac{z}{\Gamma(\nu)} \int_0^{\infty} \dd{t} t^{\nu - 1} \frac{1}{1 - z e^{-t}} e^{-t} \\
                &= \frac{z}{\Gamma(\nu)} \int_0^{\infty} \dd{t} t^{\nu - 1} \frac{1}{e^t - z} \\
                &= \frac{1}{\Gamma(\nu)} \int_0^{\infty} \dd{t} t^{\nu - 1} \frac{1}{z^{-1} e^{t} - 1}
            \end{align}
        \end{problem}
    \item[3.] For $ \abs{z} \leq 1 $ we also have $ L_{\nu}(z) = \sum_{n=1}^{\infty} \frac{z^n}{n^{\nu}} $.
        \begin{problem}
            \begin{align}
                L_{\nu}(z) &= \frac{1}{\Gamma(\nu)} \int_0^{\infty} \dd{t} t^{\nu - 1} \left( \frac{1}{\frac{e^t}{z} \left( 1 - \frac{z}{e^t} \right)} \right) \\
                &= \frac{1}{\Gamma(\nu)} \int_0^{\infty} \dd{t} t^{\nu - 1} \frac{z}{e^t} \left( \frac{1}{1 - \frac{z}{e^t}} \right) \\
                &= \frac{1}{\Gamma(\nu)} \int_0^{\infty} \dd{t} t^{\nu - 1} \sum_{n=1}^{\infty} \left( \frac{z}{e^t} \right)^n \\
                &= \frac{1}{\Gamma(\nu)} \sum_{n=1}^{\infty} z^n \int_0^{\infty} e^{-nt} t^{\nu - 1} \dd{t} \\
                &= \frac{1}{\Gamma(\nu)} \sum_{n=1}^{\infty} z^n \frac{\Gamma(\nu)}{n^{\nu}} \\
                &= \sum_{n=1}^{\infty} \frac{z^n}{n^{\nu}}
            \end{align}
        \end{problem}
    \item[4.] $ \dv{z}L_{\nu}(z) > 0 $ and $ \dv{\nu}L_{\nu}(z) < 0 $.
        \begin{problem}
            \begin{equation}
                \dv{z}L_{\nu}(z) = \dv{z} \sum_n \frac{z^n}{n^{\nu}} = \sum_{n} \frac{z^{n - 1}}{n^{\nu - 1}}
            \end{equation}
            Both the numerator and denominator are positive as long as $ z > 0 $.
            \begin{equation}
                \dv{\nu}L_{\nu}(z) = \sum_n \left( - \frac{z^n}{n^{\nu}} \ln(n) \right)
            \end{equation}
            Again, every term is positive (aside from the explicit negative sign), which makes the entire thing negative.
        \end{problem}
    \item[5.] $ L_{\nu}(0) = 0 $, $ L_{\nu}(1) = \zeta(\nu) $, $ L_0(x) = \frac{x}{1-x} $, and $ L_1(x) = - \ln(1-x) $.
        \begin{problem}
            \begin{equation}
                L_{\nu}(0) = \sum_n \frac{0^n}{n^{\nu}} = \sum_n 0 = 0
            \end{equation}
            \begin{equation}
                L_{\nu}(1) = \sum_n \frac{1}{n^{\nu}} = \zeta(\nu)
            \end{equation}
            This is just how the Riemann zeta function is defined.
            \begin{equation}
                L_0(x) = \sum_n \frac{x^n}{n^0} = \sum_n x^n = \frac{x}{1 - x}
            \end{equation}
            This is by the properties of a geometric series.
            \begin{equation}
                L_1(x) = \sum_n \frac{x^n}{n}
            \end{equation}
            The Taylor series for $ \ln(z) $ is
            \begin{equation}
                \ln(z) = \sum_n \frac{(-1)^{n-1}}{n} z^n
            \end{equation}
            so
            \begin{equation}
                -\ln(1-x) = - \sum_n \frac{(-1)^{n-1}}{n} (1-x)^n = - \sum_n \frac{(-1)^{n-1}}{n} \left( \sum_{k=0}^{n} \binom{n}{k} 1^{n-k} (-x)^k \right) = \sum_n \frac{x^n}{n}
            \end{equation}
            since the binomial coefficient with $ (-1)^k $ will cancel out $ (-1)^{n-1} $ in the outside summation.
        \end{problem}
\end{itemize}


\end{document}
