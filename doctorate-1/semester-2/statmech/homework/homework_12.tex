\documentclass[a4paper,twoside]{article}
% My LaTeX preamble file - by Nathaniel Dene Hoffman
% Credit for much of this goes to Olivier Pieters (https://olivierpieters.be/tags/latex)
% and Gilles Castel (https://castel.dev)
% There are still some things to be done:
% 1. Update math commands using mathtools package (remove ddfrac command and just override)
% 2. Maybe abbreviate \imath somehow?
% 3. Possibly format for margin notes and set new margin sizes
% First, some encoding packages and useful formatting
%--------------------------------------------------------------------------------------------
\usepackage{import}
\usepackage{pdfpages}
\usepackage{transparent}
\usepackage[l2tabu,orthodox]{nag}   % force newer (and safer) LaTeX commands
\usepackage[utf8]{inputenc}         % set character set to support some UTF-8
                                    %   (unicode). Do NOT use this with
                                    %   XeTeX/LuaTeX!
\usepackage[T1]{fontenc}
\usepackage[english]{babel}         % multi-language support
\usepackage{sectsty}                % allow redefinition of section command formatting
\usepackage{tabularx}               % more table options
\usepackage{booktabs}
\usepackage{titling}                % allow redefinition of title formatting
\usepackage{imakeidx}               % create and index of words
\usepackage{xcolor}                 % more colour options
\usepackage{enumitem}               % more list formatting options
\usepackage{tocloft}                % redefine table of contents, new list like objects
\usepackage{subfiles}               % allow for multifile documents

% Next, let's deal with the whitespaces and margins
%--------------------------------------------------------------------------------------------
\usepackage[centering,margin=1in]{geometry}
\setlength{\parindent}{0cm}
\setlength{\parskip}{2ex plus 0.5ex minus 0.2ex} % whitespace between paragraphs

% Redefine \maketitle command with nicer formatting
%--------------------------------------------------------------------------------------------
\pretitle{
  \begin{flushright}         % align text to right
    \fontsize{40}{60}        % set font size and whitespace
    \usefont{OT1}{phv}{b}{n} % change the font to bold (b), normally shaped (n)
                             %   Helvetica (phv)
    \selectfont              % force LaTeX to search for metric in its mapping
                             %   corresponding to the above font size definition
}
\posttitle{
  \par                       % end paragraph
  \end{flushright}           % end right align
  \vskip 0.5em               % add vertical spacing of 0.5em
}
\preauthor{
  \begin{flushright}
    \large                   % font size
    \lineskip 0.5em          % inter line spacing
    \usefont{OT1}{phv}{m}{n}
}
\postauthor{
  \par
  \end{flushright}
}
\predate{
  \begin{flushright}
  \large
  \lineskip 0.5em
  \usefont{OT1}{phv}{m}{n}
}
\postdate{
  \par
  \end{flushright}
}

% Mathematics Packages
\usepackage[Gray,squaren,thinqspace,cdot]{SIunits}      % elegant units
\usepackage{amsmath}                                    % extensive math options
\usepackage{amsfonts}                                   % special math fonts
\usepackage{mathtools}                                  % useful formatting commands
\usepackage{amsthm}                                     % useful commands for building theorem environments
\usepackage{amssymb}                                    % lots of special math symbols
\usepackage{mathrsfs}                                   % fancy scripts letters
\usepackage{cancel}                                     % cancel lines in math
\usepackage{esint}                                      % fancy integral symbols
\usepackage{relsize}                                    % make math things bigger or smaller
%\usepackage{bm}                                         % bold math!
\usepackage{slashed}

\newcommand\ddfrac[2]{\frac{\displaystyle #1}{\displaystyle #2}}    % elegant fraction formatting
\allowdisplaybreaks[1]                                              % allow align environments to break on pages

% Ensure numbering is section-specific
%--------------------------------------------------------------------------------------------
\numberwithin{equation}{section}
\numberwithin{figure}{section}
\numberwithin{table}{section}

% Citations, references, and annotations
%--------------------------------------------------------------------------------------------
\usepackage[small,bf,hang]{caption}        % captions
\usepackage{subcaption}                    % adds subfigure & subcaption
\usepackage{sidecap}                       % adds side captions
\usepackage{hyperref}                      % add hyperlinks to references
\usepackage[noabbrev,nameinlink]{cleveref} % better references than default \ref
\usepackage{autonum}                       % only number referenced equations
\usepackage{url}                           % urls
\usepackage{cite}                          % well formed numeric citations
% format hyperlinks
\colorlet{linkcolour}{black}
\colorlet{urlcolour}{blue}
\hypersetup{colorlinks=true,
            linkcolor=linkcolour,
            citecolor=linkcolour,
            urlcolor=urlcolour}

% Plotting and Figures
%--------------------------------------------------------------------------------------------
\usepackage{tikz}          % advanced vector graphics
\usepackage{pgfplots}      % data plotting
\usepackage{pgfplotstable} % table plotting
\usepackage{placeins}      % display floats in correct sections
\usepackage{graphicx}      % include external graphics
\usepackage{longtable}     % process long tables

% use most recent version of pgfplots
\pgfplotsset{compat=newest}

% Misc.
%--------------------------------------------------------------------------------------------
\usepackage{todonotes}  % add to do notes
\usepackage{epstopdf}   % process eps-images
\usepackage{float}      % floats
\usepackage{stmaryrd}   % some more nice symbols
\usepackage{emptypage}  % suppress page numbers on empty pages
\usepackage{multicol}   % use this for creating pages with multiple columns
\usepackage{etoolbox}   % adds tags for environment endings
\usepackage{tcolorbox}  % pretty colored boxes!


% Custom Commands
%--------------------------------------------------------------------------------------------
\newcommand\hr{\noindent\rule[0.5ex]{\linewidth}{0.5pt}}                % horizontal line
\newcommand\N{\ensuremath{\mathbb{N}}}                                  % blackboard set characters
\newcommand\R{\ensuremath{\mathbb{R}}}
\newcommand\Z{\ensuremath{\mathbb{Z}}}
\newcommand\Q{\ensuremath{\mathbb{Q}}}
%\newcommand\C{\ensuremath{\mathbb{C}}}
\renewcommand{\arraystretch}{1.2}                                       % More space between table rows (could be 1.3)
\newcommand{\Cov}{\mathrm{Cov}}
\newcommand\D{\mathrm{D}}
\newcommand*{\dbar}{\ensuremath{\text{\dj}}}

\newcommand{\incfig}[2][1]{%
    \def\svgwidth{#1\columnwidth}
    \import{./figures/}{#2.pdf_tex}
}

% Custom Environments
%--------------------------------------------------------------------------------------------
\newcommand{\lecture}[3]{\hr\\{\centering{\large\textsc{Lecture #1: #3}}\\#2\\}\hr\markboth{Lecture #1: #3}{\rightmark}}   % command to title lectures
\usepackage{mdframed}
\theoremstyle{plain}
\newmdtheoremenv[nobreak]{theorem}{Theorem}[section]
\newtheorem{corollary}{Corollary}[theorem]
\newtheorem{lemma}[theorem]{Lemma}
\theoremstyle{definition}
\newtheorem*{ex}{Example}
\newmdtheoremenv[nobreak]{definition}{Definition}[section]
\theoremstyle{remark}
\newtheorem*{remark}{Remark}
\newtheorem*{claim}{Claim}
\AtEndEnvironment{ex}{\null\hfill$\diamond$}%
% Note: A proof environment is already provided in the amsthm package
\tcbuselibrary{breakable}
\newenvironment{note}[1]{\begin{tcolorbox}[
    arc=0mm,
    colback=white,
    colframe=white!60!black,
    title=#1,
    fonttitle=\sffamily,
    breakable
]}{\end{tcolorbox}}
\newenvironment{problem}{\begin{tcolorbox}[
    arc=0mm,
    breakable,
    colback=white,
    colframe=black
]}{\end{tcolorbox}}

% Header and Footer
%--------------------------------------------------------------------------------------------
% set header and footer
\usepackage{fancyhdr}                       % header and footer
\pagestyle{fancy}                           % use package
\fancyhf{}
\fancyhead[LE,RO]{\textsl{\rightmark}}      % E for even (left pages), O for odd (right pages)
\fancyfoot[LE,RO]{\thepage}
\fancyfoot[LO,RE]{\textsl{\leftmark}}
\setlength{\headheight}{15pt}


% Physics
%--------------------------------------------------------------------------------------------
\usepackage[arrowdel]{physics}      % all the usual useful physics commands
\usepackage{feyn}                   % for drawing Feynman diagrams
%\usepackage{bohr}                   % for drawing Bohr diagrams
%\usepackage{tikz-feynman}
\usepackage{elements}               % for quickly referencing information of various elements
\usepackage{tensor}                 % for writing tensors and chemical symbols

% Finishing touches
%--------------------------------------------------------------------------------------------
\author{Nathaniel D. Hoffman}

\title{33-765 Homework 12}
\date{\today}
\begin{document}
\maketitle

\section*{42. Occupation Number Fluctuations in the Free Ideal Fermi/Bose-Gas}
The occupation number $ n_{\alpha} $ of a single-particle energy eigenstate $ \alpha $ is a random number. Its expectation value is the Fermi/Bose-distribution, $ \ev{n_{\alpha}} = (e^{\beta (\epsilon_{\alpha} - \mu)} \pm 1)^{-1} $. Prove that its variance is given by $ \sigma^2_{n_{\alpha}} = \ev{n_{\alpha}} (1 \mp \ev{n_{\alpha}}) $.
\begin{problem}
    \begin{align}
        \ev{n_{\alpha}^2} &= \frac{1}{\mathcal{Z}} \left( \prod_{\alpha'} \sum_{n_{\alpha'}} n^2_{\alpha} e^{- \beta(\epsilon_{\alpha} - \mu) n_{\alpha'}} \right) \\
        &= \frac{1}{\mathcal{Z}} \left( \prod \sum \left( \frac{1}{\beta^2} \partial^2_{\epsilon_{\alpha}} \right) e^{- \beta(\epsilon_{\alpha} - \mu) n_{\alpha'}} \right) \\
        &= \frac{1}{\beta^2} \frac{1}{\mathcal{Z}} \partial^2_{\epsilon_{\alpha}} \mathcal{Z} \\
        &= \frac{1}{\beta^2} \left( \partial_{\epsilon_{\alpha}} \ln(\mathcal{Z}) + \frac{1}{\mathcal{Z}^2} (\partial_{\epsilon_{\alpha}} \mathcal{Z})^2 \right) \\
        &= - \frac{1}{\beta} \partial_{\epsilon_{\alpha}}^2 \Omega - (\partial_{\epsilon_{\alpha}} \Omega)^2 \\
        &= - \left( \frac{1}{\beta} \partial_{\epsilon_{\alpha}}^2 \Omega + \ev{n_{\alpha}}^2 \right) \\
        &= - \left( \frac{1}{\beta} \partial^2_{\epsilon_{\alpha}} \left[ \mp \frac{1}{\beta} \sum_{\alpha'} \ln(1\pm e^{- \beta(\epsilon_{\alpha'} - \mu)}) \right] + \ev{n_{\alpha}}^2\right) \\
        &= \left( \mp \ev{n_{\alpha}}^2 + \ev{n_{\alpha}} - \ev{n_{\alpha}}^2 \right)
    \end{align}
    so
    \begin{equation}
        \sigma^2_{n_{\alpha}} = \ev{n_{\alpha}^2} - \ev{n_{\alpha}}^2 = \mp \ev{n_{\alpha}}^2 + \ev{n_{\alpha}} - \ev{n_{\alpha}}^2 - \ev{n_{\alpha}}^2 = \ev{n_{\alpha}} (1 \mp \ev{n_{\alpha}})
    \end{equation}
\end{problem}

\section*{43. Bose-Einstein Condensation in a Harmonic Trap}
Consider bosonic atoms in an optically generated harmonic trap: $ V(x, y, z) = \frac{1}{2} m(\omega_x^2 x^2 + \omega_y^2 y^2 + \omega_z^2 z^2) $.
\begin{itemize}
    \item[1.] What are the single-particle energy eigenstates of this system? Write down the energy spectrum of the single-particle Hamiltonian and subtract away the irrelevant ground-state energy.
        \begin{problem}
            The eigenstates of a 3D anisotropic harmonic oscillator can be written as a tensor product of three eigenstates of 1D oscillators:
            \begin{equation}
                \ket{n} =\ket{n_x}\ket{n_y}\ket{n_z}
            \end{equation}
            with
            \begin{equation}
                \ket{n_{x_{\mu}}} = \frac{1}{\sqrt{2^n n!}} \left( \frac{m \omega_{\mu}}{\pi \hbar} \right)^{1/4} e^{-m \omega_{\mu} x_{\mu} / 2 \hbar} H_n \left( \sqrt{\frac{m \omega_{\mu}}{\hbar}} x_{\mu} \right)
            \end{equation}
            where $ H_n(x) $ are the Hermite polynomials. The energy spectrum is therefore a sum of the energy spectra for each 1D oscillator:
            \begin{equation}
                E = E_x + E_y + E_z = \sum_{\mu} \hbar \omega_{\mu} \left( n_{\mu} + \frac{1}{2} \right) = \frac{\hbar}{2} (\sum_{\mu} \omega_{\mu} + 2 n_{\mu \omega_{\mu}})
            \end{equation}
            If we consider the ground-state energy to be where $ n_{\mu} = 0 $,
            \begin{equation}
                E_0 = \frac{\hbar}{2} (\sum_{\mu} \omega_{\mu})
            \end{equation}
            and
            \begin{equation}
                E = E_0 + \hbar (\omega_x n_x + \omega_y n_y + \omega_z n_z)
            \end{equation}
        \end{problem}
    \item[2.] As a useful result, prove that $ \frac{1}{e^x - 1} = \sum_{k=1}^{\infty} e^{- k x} $ as long as $ x > 0 $.
        \begin{problem}
            We can rewrite $ \frac{1}{e^x - 1} = \frac{e^{- x}}{1 - e^{- x}} $. This is a geometric series if $ x > 0 $ (since we need $ e^{- x} $ to be less than $ 1 $):
            \begin{equation}
                \frac{a}{1 - r} = \sum_{k=0}^{\infty} a r^k
            \end{equation}
            so with $ a = e^{- x} $ and $ r = e^{- x} $, we have
            \begin{equation}
                \frac{1}{e^x - 1} = \sum_{k=0}^{\infty} e^{-x} (e^{-x})^k = \sum_{k=1}^{\infty} e^{- kx}
            \end{equation}
        \end{problem}
    \item[3.] Bose-Einstein condensation happens if the total number $ N' $ of particles contained in all excited states is bounded from above. Without using a continuum approximation, find an expression for $ N' $ and look for an upper bound by focusing on the ``best-case scenario'' $ \mu = 0 $. Why is this ``best case''? You will end up with an unwieldy looking sum.
        \begin{problem}
            \begin{equation}
                N' = \sum_{\alpha} \ev{n_{\alpha}} = \sum_{\alpha} \frac{1}{e^{\beta (\epsilon_{\alpha} - \mu)} - 1} \to \sum_{\alpha} \frac{1}{e^{\beta \epsilon_{\alpha}} - 1}
            \end{equation}
            We choose $ \mu = 0 $ because $ \mu > 0 $ would make $ e^{\beta (\epsilon - \mu)} $ smaller making $ N' $ bigger, so $ \mu = 0 $ gives the smallest upper bound.
        \end{problem}
    \item[4.] Rewrite the sum using the identity you proved in part (2), swap the two sums, sum up the inner one, and expand it for large temperatures. Now show that the leading order is
        \begin{equation}
            N' = \zeta(3) \left( \frac{k_B T}{\hbar \bar{\omega}} \right)^3 + \order{T^2} \qquad \text{with } \bar{\omega} = (\omega_x \omega_y \omega_z)^{1/3}.
        \end{equation}
        \begin{problem}
            \begin{align}
                N' &= \sum_{\alpha=1}^{\infty} \left( \sum_{k=1}^{\infty} e^{- \beta k \epsilon_{\alpha}} \right) \\
                &= \sum_{k=1}^{\infty} \sum_{\alpha=1}^{\infty} e^{- \beta k \epsilon_{\alpha}} \\
                &= \sum_{k=1}^{\infty} \left( \underbrace{\sum_{n_x}^{\infty} \sum_{n_y}^{\infty} \sum_{n_z}^{\infty}}_{n_x + n_y + n_z \geq 1} e^{-k \beta \hbar \omega_x n_x} e^{-k \beta \hbar \omega_y n_y} e^{-k \beta \hbar \omega_z n_z} \right) \\
                &= \sum_{k=1}^{\infty} \sum_{n_x=0}^{\infty} \sum_{n_y=0}^{\infty} \sum_{n_z=0}^{\infty} \left( e^{-k \beta \hbar \omega_x n_x} e^{-k \beta \hbar \omega_y n_y} e^{-k \beta \hbar \omega_z n_z} \right) - 1 \\
                &= \sum_{k=1}^{\infty} \prod_{\mu} (- e^{-k \beta \hbar \omega_{\mu} x_{\mu}} + 1) \\
                &= \sum_{k=1}^{\infty} \frac{1}{\omega_x \omega_y \omega_z} \left( \frac{k_B T}{k \hbar} \right)^3 + \frac{\omega_x + \omega_y + \omega_z}{2 \omega_x \omega_y \omega_z} \left( \frac{k_B T}{k \hbar} \right)^2 + \order{T} \\
                &= \sum_{k=1}^{\infty} \frac{1}{k^3} \left( \frac{k_B T}{\hbar \bar{\omega}} \right) + \order{T^2} \\
                &= \zeta(3) \left( \frac{k_B T}{\hbar \bar{\omega}} \right)^3 + \order{T^2}
            \end{align}
        \end{problem}
    \item[5.] What is the Einstein temperature $ T_E $ for this case? How is it different from what we found for the gas-in-a-box?
        \begin{problem}
            The Einstein temperature occurs when $ N = N' $ or 
            \begin{equation}
                N = \zeta(3) \left( \frac{k_B T_E}{\hbar \bar{\omega}} \right)^3
            \end{equation}
            so
            \begin{equation}
                T_E = \frac{\hbar \bar{\omega}}{k_B} \left( \frac{N}{\zeta(3)} \right)^{1/3}
            \end{equation}
            For the gas-in-a-box, we found
            \begin{equation}
                T_E = \frac{h^2}{2 \pi m k_B} \left( (2s+1) \frac{V}{N} \gamma \left( \frac{3}{2} \right) \right)^{-2/3}
            \end{equation}
            We have already included the degeneracy of the energy states in our derivation, and interestingly the mass doesn't seem to explicitly factor int, but the dependence on $ N $ is different:
            \begin{equation}
                N_E^{\text{SHO}} \propto N^{1/3} \qquad T_E^{\text{Box}} \propto N^{2/3}
            \end{equation}
        \end{problem}
    \item[6.] For $ \omega_x = 4681\hertz $, $ \omega_y = 1477\hertz $, and $ \omega_z = 2576\hertz $, how many atoms should be trapped in a BEC at $ 2\milli\kelvin $?
        \begin{problem}
            With these numbers, we get $ \bar{\omega} = 2611.49\hertz $. Plugging in values for everything, we find that
            \begin{equation}
                N \approx 1.21163\times 10^6
            \end{equation}
        \end{problem}
\end{itemize}

\section*{44. Bose-Einstein Condensation in a Harmonic Trap\textemdash Finite Size Corrections}
Let us look a bit more closely at a simplified version of the previous problem, namely the isotropic case $ \omega_x = \omega_y = \omega_z \equiv \omega $.
\begin{itemize}
    \item[1.] Show that continuing the series expansion that led to the answer for problem 43.4 leads to
        \begin{equation}
            N' = \zeta(3) \left( \frac{k_B T}{\hbar \omega} \right)^3 + \frac{3}{2} \zeta(2) \left( \frac{k_B T}{\hbar \omega} \right)^2 + \zeta(1) \left( \frac{k_B T}{\hbar \omega} \right) + \cdots
        \end{equation}
        But alas, this high temperature series expansion should strike you as a terrible disappointment. Why?
        \begin{problem}
            In the isotropic case, we can write
            \begin{align}
                N' &= \sum_{k=1}^{\infty} \prod_{\mu} \left( - e^{-k \beta \hbar \omega x_{\mu}} + 1 \right) \\
                &= \sum_{k=1}^{\infty} \frac{1}{k^3} \left( \frac{k_B T}{\hbar \omega} \right)^3 + \frac{3}{2} \frac{1}{k^2} \left( \frac{k_B T}{\hbar \omega} \right)^2 + \frac{1}{k \left( \frac{k_B T}{\hbar \omega} \right)} + \frac{3}{8} + \order{T^{-1}} \\
                &= \zeta(3) \left( \frac{k_B T}{\hbar \omega} \right)^3 + \frac{3}{2} \zeta(2) \left( \frac{k_B T}{\hbar \omega} \right)^2 + \zeta(1) \left( \frac{k_B T}{\hbar \omega} \right) + \cdots
            \end{align}
            Unfortunately, $ \gamma(1) = \tilde{\infty} $, so this series is really an asymptotic expansion which fails miserably at the third term.
        \end{problem}
    \item[2.] Boldly ignoring your perfectly reasonable anxieties, and taking only the first two terms from the previous equation, show that they predict a slight downward shift of the Bose-Einstein transition temperature that vanishes in the limit $ N \to \infty $ and is given by
        \begin{equation}
            \frac{T_{\text{transition}} - T_E}{T_E} = - \frac{\alpha}{N^{1/3}} \qquad \text{with } \alpha = \frac{\zeta(2)}{2 \zeta(3)^{2/3}} \approx 0.7275.
        \end{equation}
        \begin{problem}
            First, we can write the Einstein temperature for the leading-order as
            \begin{equation}
                T_E = \frac{\hbar \omega}{k_B} \left( \frac{N}{\zeta(3)} \right)^{1/3}
            \end{equation}
            so
            \begin{equation}
                \frac{k_B}{\hbar \omega} = \frac{1}{T_E} \left( \frac{N}{\zeta(3)} \right)^{1/3}
            \end{equation}
            We can now write the expression with the next-order in terms of $ \frac{T_T}{T_E} $ with $ T_T \equiv T_{\text{transition}} $ for brevity and later $ x \equiv\frac{T_T}{T_E} $:
            \begin{equation}
                N\left( \frac{T_T}{T_E} \right)  = \zeta(3) \left( \frac{N}{\zeta(3)} \right) \left( \frac{T_T}{T_E} \right)^3 + \frac{3}{2} \zeta(2) \left( \frac{N}{\zeta(3)} \right)^{2/3} \left( \frac{T_T}{T_E} \right)^2
            \end{equation}

            For $ N \to \infty $, the downward shift vanishes, so $ \frac{T_T}{T_E} \to 1 $. We can think of the downward shift as a linear perturbation in $ \frac{T_T - T_E}{T_E} = \epsilon $ for $ \frac{T_T}{T_E} \to 1 + \epsilon $, where we will ignore all $ \epsilon^2 $ and higher orders. Note that this makes $ \epsilon \propto N^{-1/3} $ so we will also ignore terms of order $ N^{-1/3} \epsilon $ (this is important in the final step of the derivation):
            \begin{align}
                N &= N x^3 + 3 \alpha N^{2/3} x^2 \\
                &= N(1 + 3 \epsilon + 3 \epsilon^2 + \epsilon^3) + 3 \alpha N^{2/3} (1 - 2 \epsilon + \epsilon^2) \\
                1 &= 1 + 3 \epsilon + 3 \frac{\alpha}{N^{1/3}} (1 + 2 \epsilon) \\
                1 &= 1 + 3 \epsilon + 3\frac{\alpha}{N^{1/3}} \\
                \implies \epsilon = \frac{T_T - T_E}{T_E} = - \frac{\alpha}{N^{1/3}}
            \end{align}
        \end{problem} 
\end{itemize}

\end{document}
