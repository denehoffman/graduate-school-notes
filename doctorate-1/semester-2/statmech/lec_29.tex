\documentclass[a4paper,twoside,master.tex]{subfiles}
\begin{document}
\lecture{29}{Friday, April 03, 2020}{}

\begin{theorem}[Spectral Theorem for a General Quantum State]
    Suppose we have a quantum state $ W $ with $ W kt \psi_n = p_n\ket{\psi_n} $ and $\bra{\psi_n}\ket{\psi_m} = \delta_{nm} $ and $ \sum_n\ket{\psi_n}\bra{\psi_n} = 1 $. This means that $ \{\ket{\psi_n}\} $ is an eigenbasis of the Hilbert space to the state $ W $. This means we can write
    \begin{equation}
        W = \sum_n p_n\ket{\psi_n}\bra{\psi_n}
    \end{equation}
    where $ 0 \leq p_n \leq 1 $ and $ \sum_n p_n = 1 $, which follows from $ \Tr(W) = 1 $.
\end{theorem}

In this framework, what do we mean by expectation value?
\begin{align}
    \ev{A} = \Tr(WA) &= \Tr(\sum_n p_n\ket{\psi_n}\bra{\psi_n} A) \\
    &= \sum_n p_n \Tr(\ket{\psi_n}\bra{\psi_n} A) \\
    &= \sum_n p_n \sum_m\bra{\psi_m}\ket{\psi_n}\bra{\psi_n} A\ket{\psi_m} \\
    &= \sum_n p_n \underbrace{\mel{\psi_n}{A}{\psi_n}}_{\text{objective indeterminacy}}
\end{align}
The term objective indeterminacy is something that is new in quantum mechanics. The value of this matrix element causes the result to not necessarily be deterministic. The $ p_n $ give subjective ignorance, a sense that we have not measured the state carefully enough. This is because the original state that we worked with is not pure.


The entropy of a state is a measure of our subjective ignorance plus, in quantum mechanics, the objective indeterminacy. Classicaly, the entropy is bounded by $ - \infty \leq S \leq k_B \ln(\int_{\Gamma} \frac{\dd{p} \dd{q}}{h}) = + \infty $ if $ \Gamma = \R^2 $. In quantum mechanics, however, $ 0 \leq S \leq k_B \ln(\text{dim} \mathcal{H}) = + \infty $ if $ \mathcal{H} = L^2(\R) $. The interesting thing is that the entropy cannot get arbitrarily small, it's bounded below by $ 0 $. This looks a lot like it has something to do with the third law of thermodynamics. For quantum mechanics, we can prove this:
\begin{equation}
    W \ln(W) = \sum_n p_n \ln(p_n)\ket{\psi_n}\bra{\psi_n}
\end{equation}
This implies
\begin{equation}
    S = - k_B \Tr(W \ln(W)) = - k_B \sum_n p_n \ln(p_n)
\end{equation}
If we have maximal knowledge about the system, then we have minimal entropy. Maximal knowledge means that $ p_n = \delta_{n,n_0} $ (we know the system is exactly in one pure state). This makes $ S = 0 $ since $ 1 \ln(1) = 0 \ln(0) = 0 $. Conversely, if we have minimal knowledge, we have maximal entropy. In that case, $ p_n = \frac{1}{N} $ where $ N = \text{dim} \mathcal{H} $. If so, then
\begin{equation}
    S = - k_B \sum_{n=1}^{N} \frac{1}{N} \ln(\frac{1}{N}) = - k_B \ln(\frac{1}{N}) = k_B \ln(N) = k_B \ln(\text{dim} \mathcal{H})
\end{equation}
which is infinite if the Hilbert space square-integrable over the reals.


\section{The Quantum Canonical Ensemble}
\label{sec:the_quantum_canonical_ensemble}

By analogy with the classical case, we can reasonably guess that the quantum canonical state is given by
\begin{equation}
    W = \frac{1}{Z} e^{= \beta H}
\end{equation}
Note that this is an operator, since $ H $ is an operator. The partition function must be
\begin{equation}
    Z = \Tr(e^{- \beta H})
\end{equation}

Since $ H $ is self-adjoint (and hopefully compact) its eigenvalues are real and its eigenvectors form an orthonormal basis of Hilbert space:
\begin{equation}
    H\ket{n} = E_n\ket{n}
\end{equation}
such that $\ket{n} $ is an energy eigenvector. This corresponds to a pure energy eigenstate $\ket{n}\bra{n} $ with the corresponding eigenvalue $ E_n $. The fact that this basis is orthonormal means $\bra{n}\ket{m} = \delta_{nm} $ and $ H = \sum_n E_n\ket{n}\bra{n} $. This is the spectral theorem for (compact) operators.


Since the canonical state is a function of $ H $, it is diagonal in the same basis as $ H $. Therefore,
\begin{align}
    Z = \Tr(e^{- \beta H}) &= \sum_n\bra{n} e^{- \beta H}\ket{n} \\
    &= \sum_n\bra{n} e^{- \beta E_n}\ket{n} \\
    &= \sum_n e^{- \beta E_n}
\end{align}

It could happen that the Hamiltonian is degenerate. If we have degenerate eigenstates of $ H $, we can (if we want) sum over the energy levels, but then we need to explicitly multiply back the degeneracy:
\begin{equation}
    Z = \sum_n e^{- \beta E_n} = \sum_l \Omega(l) e^{- \beta E_l}
\end{equation}
where $ l $ is an energy level and $ \Omega(l) $ is the degeneracy of the level.

\subsection{Thermal Averages}
\label{sub:thermal_averages}

\begin{equation}
    \ev{A} = \ev{A,W} = \Tr(AW) = \sum_n\bra{n} AW\ket{n}
\end{equation}
where $\ket{n} $ is the energy eigenbasis and $ W $ is the canoical state. Then we get that
\begin{align}
    \ev{A} &= \sum_n\bra{n} A e^{- \beta E_n} \frac{1}{Z}\ket{n} \\
    &= \sum_n p_n \mel{n}{A}{n}
\end{align}
The matrix element is agains a measure of objective indeterminacy while the $ p_n $ factors are a measure of subjective ignorance.


\end{document}
