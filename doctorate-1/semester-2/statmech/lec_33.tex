\documentclass[a4paper,twoside,master.tex]{subfiles}
\begin{document}
\lecture{33}{Monday, April 13, 2020}{The Harmonic Solid}

Last time we finished up talking about blackbody radiation. Today we will do something that at first looks rather different but is in fact very similar. Imagine a sequence of masses $ m $ connected by springs in one-dimension. The particles lie at a resting displacement of $ a $, and we will describe the displacement from that resting position as $ x $, so the position of the $ j $th particle is $ \va{r}_j = R_j + x_j = ja + x_j $. We can write down the energy of this system:
\begin{equation}
    E = \sum_{j=1}^{N} \left\{ \frac{1}{2} m\dot{x}_j^2 + \frac{1}{2} \left( x_j - x_{j+1} \right)^2 \right\}
\end{equation}
However, this final term needs some additional definition at the boundaries ($ j = 1,N $). In this case, we will use periodic boundary conditions such that $ x_{N+1} = x_1 $, as if the particles were actually connected in a circle. Unfortunately, the coupling between particles in the last term doesn't allow this to easily factor in the partition function, but we can use normal modes to decouple the system. It turns out we can find the normal modes via a Fourier transform:
\begin{equation}
    \tilde{x}_k = \frac{1}{\sqrt{N}} \sum_{j=1}^{N} x_j e^{- \imath k R_j} = \frac{1}{\sqrt{N}} \sum_{j=1}^{N} x_j e^{- \imath k a j}
\end{equation}
We can define the inverse transform as
\begin{equation}
    x_j = \frac{1}{\sqrt{N}} \sum_k \tilde{x}_k e^{+ \imath k a j}
\end{equation}
Let's check that this is actually the inverse transformation:
\begin{align}
    x_j &= \frac{1}{\sqrt{N}} \sum_k \frac{1}{\sqrt{N}} \sum_{j'} x_{j'} e^{- \imath k R_{j'}} e^{+ \imath k R_{j}} \\
    &= \sum_{j'} x_{j'} \frac{1}{N} \sum_{k} e^{\underbrace{\imath k (R_j - R_{j'})}_{\frac{2 \pi}{L} n a (j-j')} = \frac{2 \pi}{N} n (j-j')} \\
    &= \sum_{j'} x_{j'} \frac{1}{N} \begin{cases} N & j = j'\\ 0 & j \neq j' \end{cases}
\end{align}
because the phases average out if $ j \neq j' $
\begin{equation}
    x_j = \sum_{j'} x_{j'} \delta_{jj'} = x_j
\end{equation}
Now let's transform the Hamiltonian. We'll first look at the kinetic energy:
\begin{align}
    \frac{1}{2} m \sum_j \dot{x}_j^2 &= \frac{1}{2} m \sum_j \left( \frac{1}{\sqrt{N}} \sum_k \dot{\tilde{x}}_k e^{\imath k R_j} \right)\left( \frac{1}{\sqrt{N}} \sum_{k'} \dot{\tilde{x}}_{k'} e^{\imath k' R_j} \right) \\
    &= \frac{1}{2} m \sum_{k,k'} \dot{\tilde{x}}_{k} \dot{\tilde{x}}_{k'} \underbrace{\frac{1}{N} \sum_j e^{\imath (k+k')R_j}}_{\delta_{k, -k'}} \\
    &= \frac{1}{2} m \sum_{k} \dot{\tilde{x}}_{k} \dot{\tilde{x}}_{-k} \\
    &= \frac{1}{2} m \sum_{k} \abs{\dot{\tilde{x}}_{k}}^2
\end{align}
This last step can be justified by looking at how the Fourier transform is defined. $ -k $ will give the complex conjugate of the point.


Next, the potential energy:
\begin{align}
    \frac{1}{2} K \sum_j (x_j - x_{j + 1})^2 &= \frac{1}{2} K \sum_j \left( \frac{1}{\sqrt{N}} \sum_k \tilde{x}_k e^{\imath k R_j} - e^{\imath k R_{j+1}} \right) \left( \frac{1}{\sqrt{N}} \sum_{k'} \tilde{x}_{k'} e^{\imath k' R_j} - e^{\imath k' R_{j+1}} \right) \\
    \text{Recall that} R_{j+1} &= R_j + a \\
    &= \frac{1}{2} K \sum_{k,k'} \tilde{x}_k \left( 1 - e^{\imath ka} \right) \tilde{x}_{k'} \left( 1 - e^{\imath k' a} \right) \frac{1}{N} \sum_j e^{\imath (k+k') R_j} \\
    &= \frac{1}{2} K \sum_k \abs{\tilde{x}_k}^2 \underbrace{\left( 1 - e^{\imath k a} \right) \left( 1 - e^{- \imath k a} \right)}_{1 - e^{- \imath k a} - e^{\imath k a} + 1 = 2 - 2 \cos(ka) = 4 \sin[2](\frac{ka}{2})} \\
    &= \sum_k \frac{1}{2} K(k) \abs{\tilde{x}_k}^2
\end{align}
where $ K(k) = 4 K \sin[2](\frac{ka}{2}) $. The Hamiltonian is now diagonal, so we can factorize it! Every mode oscillates with frequency
\begin{equation}
    \omega(k) = \sqrt{\frac{K(k)}{m}} = \sqrt{\frac{K}{m}} 2 \abs{\sin(\frac{ka}{2})} = 2 \tilde{\omega} \abs{\sin(\frac{ka}{2})} 
\end{equation}
This is called the ``dispersion relation''. The region from $ - \frac{\pi}{a} $ to $ \frac{\pi}{a} $ is known as the ``first Brillouin zone''. For small $ ka $, $ \omega(k) \approx \tilde{\omega} \abs{ka} $ and the phase velocity is $ v = \frac{\omega(k)}{k} \approx_{k \to 0} \tilde{\omega}a $.


The classical energy can be easily calculated by equipartition theorem. There are $ N $ quadratic degrees of freedom for both position and momentum, so
\begin{equation}
    U = \ev{H} = N \cdot 2 \cdot \frac{1}{2} k_{B} T = N k_B T
\end{equation}
which tells us that
\begin{equation}
    c_V = k_B
\end{equation}
which follows from the ``Dulong-Petit'' law.


In quantum, the partition function factorizes, so
\begin{equation}
    Z_{\text{QM}} = \prod_K Z_{\text{QM}, k} = \prod_k \frac{1}{2 \sinh(\frac{\beta \hbar \omega(k)}{2})}
\end{equation}
so
\begin{equation}
    F_{\text{QM}} = - k_{B} T \ln(Z_{\text{QM}}) = k_B T \sum_k \ln(2 \sinh(\frac{\beta \hbar \omega(k)}{2}))
\end{equation}
and
\begin{equation}
    U_{\text{QM}} = \pdv{\beta F_{\text{QM}}}{\beta} = \sum_k \left\{ \frac{\hbar \omega(k)}{2} + \frac{\hbar \omega(k)}{e^{\beta \hbar oemga(k)} - 1} \right\}
\end{equation}
This is better than the Einstein assumption that all the modes have the same frequency. What now does this predict about the heat capacity of a 3D solid? This was first done by Debye using the following approximation:
\begin{itemize}
    \item[1] Restrict to low temperatures, so we only need low $ \omega $ or $ \hbar \omega(k) \approx \hbar \tilde{\omega} a \abs{k} = \hbar v \abs{k} = \hbar v \frac{\pi}{L} n = \hbar \omega(n) $.
    \item[2+3] Make the Brillouin zone a sphere which contains $ 3N $ modes:
        \begin{equation}
            3N = \frac{3}{8} \int_0^{n_D} \dd{n} 4 \pi n^2 = \frac{1}{2} \pi n_D^3 \implies n_D = \left(\frac{6N}{\pi}\right)^{1/3}
        \end{equation}
\end{itemize}



\end{document}
