\documentclass[a4paper,twoside,master.tex]{subfiles}
\begin{document}
\lecture{2}{Wednesday, September 02, 2020}{}

A final note on orbits: they are equivalence classes. We can prove this through the following propositions:


Proposition: If $ y \in G \cdot x $ then $ x \in G \cdot y $

Proof: $ y = g \cdot x $ then $ g^{-1} \cdot y = g^{-1} \cdot (g \cdot x) = (g^{-1} g) \cdot x = id \cdot x = x $

Proposition: If $ X $ is finite and $ G \curvearrowright X $, then $ \abs{G \cdot x} = \abs{G} / \abs{G_x} $ where $ G_x := \{g \in G \mid gx = x\} $ (stabilizer of $ x $)

Proof: $ \varphi: G \to G \cdot x $, $ g \mapsto g \cdot x $

\begin{note}{Note}
    $ h \in G_x =: H $ implies $ \varphi(gh) = \varphi(g) $ since $ ghx = gx $. $ \varphi $ is constant on the left cosets of $ H $, i.e. sets of the form $ gH $.
\end{note}

$ G/H $ is the set of left cosets. We can make maps $ \pi(g) = gH $ and $ \psi(gH) = \varphi(g) $. $ \varphi = \psi \pi $ and $ \abs{G/H} = \abs{G} / \abs{H} $. We claim $ \psi $ is a bijection, and if we prove it, we prove the proposition. First, we can say its a surjection because $ \psi $ is a surjection. Suppose $ \psi(g_1 H) = \psi(g_2 H) $. This is equivalent to $ \varphi(g_1) = \varphi(g_2) $.
\begin{align}
    g_1 \cdot x = g_2 \cdot x &\Leftrightarrow g_1^{-1} g_2 \cdot x = x\\
    &\Leftrightarrow g_1^{-1} g_2 \in G_x = H\\
    &\Leftrightarrow g_2 \in g_1 H\\
    &\Leftrightarrow g_2 H = g_1 H
\end{align}
which proves injectivity.

\subsection{Fixed Points}
\label{sub:fixed_points}

\begin{definition}
    An orbit of size $ 1 $ is called a \textit{fixed point}.
\end{definition}

\begin{definition}
    The orbit of $ G\curvearrowright G $ by conjugation\textemdash the orbit of this action is a conjugacy class. The \textit{centralizer} is the stabilizer for this action: $ C_G(x) = \{g \in G \mid gxg^{-1} = x\} $.
\end{definition}

\begin{definition}
    We can also look at mappings like $ G\curvearrowright 2^G $. If we look at subgroups $ H \leq G $, the stabilizer of $ H $ is called the \textit{normalizer}: $ N_G(H) = \{g \mid gHg^{-1} = H\} $.
\end{definition}

\begin{definition}
    The \textit{center} of $ G $ is $ Z(G) = \{h \in G \mid g^{-1} h g = h \quad\forall g \in G\} $.
\end{definition}
If $ G $ is finite,
\begin{equation}
    G = Z(G) \cup G \cdot x_1 \cup \cdots \cup G \cdot x_r
\end{equation}
where $ x_1 \cdots x_r $ are representations of non-central conjugacy classes. Additionally,
\begin{equation}
    \abs{G} = \abs{Z(G)} + \sum_{i = 1}^{r} [G \colon C_G(x_i)]
\end{equation}

\section{P-Groups}
\label{sec:p-groups}

\begin{definition}
    A group $ G $ is a \textit{p-group} if $ \abs{G} = p^m $ where $ p $ is prime.
\end{definition}

The reason these are important in group theory is that most finite groups are p-groups.

\begin{definition}
    For finite groups $ G $, if $ p $ divides $ \abs{G} $ and $ p^m $ divides $ \abs{G} $ but $ p^{m+1} $ does not divide $ \abs{G} $ and $ H \leq G $ of order $ \abs{H} = p^m $, we call $ H $ the \textit{p-Sylow subgroup} of $ G $.
\end{definition}

We will use the notation $ a \mid b $ sometimes for $ a $ ``divides'' $ b $.

\begin{theorem}[Sylow I]
    If $ p \mid \abs{G} $, then $ G $ contains a p-Sylow subgroup. 
\end{theorem}

\begin{lemma}
    If $ G $ is abelian, $ p \mid \abs{G} $ then $ G $ contains a subgroup of order $ p $.
\end{lemma}

\begin{proof}
    A number $ n $ is an \textit{exponent} of $ G $ if $ g^n = 1 \quad\forall g \in G$. We claim if $ n $ is an exponent of an abelian group $ G $, then $ \abs{G} \mid n^m \quad\exists m $. We can prove this by induction on $ \abs{G} $: $ b \in G $ and $ b \neq 1 $, we can define a subgroup $ H = <b> $, the powers of $ b $. Therefore, $ b^{\abs{H}} = 1 \implies \abs{H} \mid n $. Because $ n $ is an exponent of $ G $, $ n $ is also an exponent of $ G/H $: $ (gH)^n = g^n H $. By induction, $ \exists m $ such that $ \abs{G/H} \mid n^m $: $ \abs{G} = \abs{G/H} \cdot \abs{H} $ and $ n^m \cdot n = n^{m+1} $.

    Consider $ n = \prod_{g \in G} \abs{g} $. $ n $ is an exponent of $ G $. Therefore, we know that $ p\mid \abs{G} $ and $ \abs{G} \mid n^m $, so $ p \mid \abs{g} $ for some $ g \in G $. $ h = g^{\abs{g} / p} $, so $ \abs{h} = p $.
\end{proof}

Let's now use this lemma to prove the first Sylow theorem:
\begin{proof}
    Proof by induction on $ \abs{G} $. Let's assume $ \abs{G} = p^m n $ where $ p \slashed{\mid} n $. If $ H < G $ and $ p^m \mid \abs{H} $, then a p-Sylow subgroup of $ H $ is a p-Sylow subgroup of $ G $.

    For all $ H < G $, assume the largest power of $ p $ dividing $ \abs{H} $ is $ p^{l(H)} $ with $ l < m $. This is equivalent to $ p\mid [G \colon H] $. Recall that we can say
    \begin{equation}
        \abs{G} = \abs{Z(G)} + \sum_{i = 1}^{r} [G \colon C_G(x_i)]
    \end{equation}
    We know that $ p \mid \abs{G} $ and $ p \mid [G \colon C_G(x_i)] $, so $ p \mid \abs{Z(G)} $. By the lemma we just proved, $ \exists H \leq Z(G) $ such that $ \abs{H} = p $.

    $ Z(G) \triangleleft G $ (normal subgroup) and even $ H \triangleleft G $. Consider $ G' = G/H $. We want to use induction to find $ P' $, the p-Sylow subgroup in $ G' $.

    $ G \xrightarrow{\pi} G/H = G' \geq P' $. We are going to pull back $ P = \pi^{-1}(P') $. $ \abs{P} = p \cdot \abs{P'} $, so $ P $ is a subgroup.
\end{proof}



\end{document}
