\documentclass[a4paper,twoside,master.tex]{subfiles}
\begin{document}
\lecture{4}{Wednesday, September 09, 2020}{}

If $ X \subseteq G $, the group generated by $ X $, denoted by, $ \ev{X} $ is the smallest group containing $ X $:
\begin{equation}
    \ev{X} = \bigcap_{H\subseteq G,\ X\subseteq H} H
\end{equation}

Equivalently,

\begin{equation}
    \ev{X} = \{x_1^{\epsilon_1} x_2^{\epsilon_2} \cdots x_n^{\epsilon_n} \mid \epsilon \in \{\pm 1\},\ n \in \Z \geq 0\}
\end{equation}

\begin{definition}
    If $ S $ is a set, the \textit{free group} on $ S $, $ F(S) $, ``consists of'' all formal expressions of the form $ x_1^{\epsilon_1} x_2^{\epsilon_2} \cdots x_n^{\epsilon_n} $ with multiplication being the concatenation. This is not yet a group, but rather a monoid (no inverses). We also need to define equivalence classes of ``words'' in the obvious way, namely you can cancel adjacent pairs of elements and their inverses and insert these pairs arbitrarily.
\end{definition}

\begin{note}{Conjugation Notation}
    We don't want to write things like $ b^{-1} a b $ all the time so we will just call it $ a^b $. Additionally, we can write things like
    \begin{equation}
        (a^b)^c = a^{b c} = c^{-1} (b^{-1} a b) c
    \end{equation}

    Additionally, $ 1^a = 1 $ and $ a^1 = a $ by definition.
\end{note}

\begin{definition}
    The \textit{commutator} of $ a $ and $ b $ is denoted by $ \comm{a}{b} = a^{-1} b^{-1} a b $. By this definition, $ \comm{a}{b} = 1 \iff ab = b a $ or equivalently $ ab = b a \comm{a}{b} $.
\end{definition}

\begin{equation}
    \comm{a}{b c} = \comm{a}{c} \comm{a}{b} [a,b,c]
\end{equation}
and
\begin{equation}
    \comm{ab}{c} = \comm{a}{c} [a,c,b] \comm{b}{c}
\end{equation}
where $ [a,b,c] \equiv \comm{\comm{a}{b}}{c} $. More generally,
\begin{equation}
    [a,b,c,\cdots,z] = [ [\cdots \comm{a}{b} ,\cdots],z]
\end{equation}

Additionally,
\begin{equation}
    [a,b,c^a][c,a,b^c][b,c,a^b] = 1\tag{Hall's Identity}
\end{equation}

If $ A,B \triangleleft G $, we can define
\begin{equation}
    \comm{A}{B} \equiv \ev{\comm{a}{b} \mid a \in A,\ b \in B}
\end{equation}
and this group, $ A \cap B $, and $ AB $ are all normal in $ G $.

\begin{equation}
    \comm{A}{B} \subseteq A \cap B
\end{equation}
Why? $ \comm{a}{b} = a^{-1} b^{-1} a b = a^{-1} a^b \in A = (b^{-1})^a b \in B $.

If we define $ [A,B,C] = [\comm{A}{B}, C] $, we claim that $ [A,B,C] = \ev{[a,b,c]\mid a \in A,\ b \in B,\ c \in C} = H $.
\begin{proof}
    \begin{equation}
        [a,b,c] \in [A,B,C]
    \end{equation}
    \begin{equation}
        [ \prod_{i=1}^{n} \comm{a_i}{b_i}^{\epsilon_i}, c] = \underbrace{[\comm{a_1}{b_1}^{\epsilon_1}, c]}_{\in H}\underbrace{[\comm{a_1}{b_1}^{\epsilon_1}, c_1 \prod_{i=2}^{n} \comm{a_i}{b_i}^{\epsilon_i}]}_{[ [a,b,c], \prod \comm{a_i}{b_i}^{\epsilon_i} \in H \text{ since } [a,b,c] \in H}\underbrace{[\prod_{i=2}^{n} \comm{a_i}{b_i}, c]}_{\in H}
    \end{equation}
\end{proof}


\begin{definition}
    \textbf{Nilpotent Groups}: Take a series of subgroups $ \gamma_0(G) = G $, $ \gamma_i(G) = \comm{\gamma_{i - 1}(G)}{G} $, etc. It is clear from this definition that $ \gamma_0 \supseteq \gamma_1 \supseteq \gamma_2 \cdots $. All of these subgroups are normal in $ G $ ($ \gamma_i \triangleleft G $).

    We say $ G $ is nilpotent of step $ s $ if $ \gamma^{s+1}(G) = 1 $ yet $ \gamma^s(G) \neq 1 $.
\end{definition}


\end{document}
