\documentclass[a4paper,twoside,master.tex]{subfiles}
\begin{document}
\lecture{5}{Friday, September 11, 2020}{}

\begin{note}{Aside}
    
    In the last class, there was a question about the inverse of commutators in this derivation. We want to show that
    \begin{equation}
        \comm{\comm{a}{b}^{-1}}{c} = \comm{\comm{b}{a}}{c} \in [A,B,C]
    \end{equation}

    Let's examine
    \begin{align}
        \comm{\comm{b}{a}}{c}^{b^{-1}} &= \comm{\comm{b}{a}^{b^{-1}}}{c^{b^{-1}}} \\
        &= \comm{b(b^{-1} a^{-1} b a)b^{-1}}{c^{b^{-1}}} \\
        &= \comm{a^{-1} b a b^{-1}}{c^{b^{-1}}} \\
        &= \comm{\comm{a}{\underbrace{b^{-1}}_{\in B}}}{\underbrace{c^{b^{-1}}}_{\in C}} \in [A,B,C]
    \end{align}
\end{note}

Back to nilpotent groups. When we say that $ a \equiv b\ (\mod{H}) $, we mean $ H \triangleleft G $, $ aH = bH $ for $ a = b \in G/H $.

\begin{equation}
    ab = b a \mod{\gamma_1}
\end{equation}

\begin{claim}
    \begin{equation}
        [A,B,C] \subseteq [C,A,B] [B,C,A]
    \end{equation}
\end{claim}

\begin{claim}
    If $ a \in \gamma_i $, $ b \in \gamma_j $, then $ \comm{a}{b} \in \gamma_{i+j} $. In particular, $ a b \equiv b a \mod{\gamma_{i+j}} $.
\end{claim}
\begin{proof}
    We will prove this by induction on $ \min(i,j) $. If $ j=1 $, we are done by definition. Let's then take $ a \in \gamma_{i-1} $, $ b \in \gamma_{j+1} $.
    \begin{equation}
        \comm{\gamma_{i-1}}{\gamma_{j+1}} = \comm{\gamma_{i-1}}{\comm{\gamma_{j}}{G}}
    \end{equation}
    (proof unfinished)
\end{proof}

Examples of nilpotent groups
\begin{itemize}
    \item[(0)] Abelian groups are nilpotent of rank $ 0 $.
    \item[(1)] If $ G $ is nilpotent, $ G/Z(G) $ is nilpotent because $ \gamma_i(G/Z(G)) = \frac{\gamma_i(G) Z}{Z} $. Elements of $ \gamma_i $ are generated by nested commutators with $ i-1 $ bracket pairs. $ \comm{g_1 Z}{g_2 Z} = \comm{g_1}{g_2} Z $.

        In particular, if $ G $ is nilpotent of step $ s $, then $ G/Z(G) $ is nilpotent of step $ s-1 $ because $ \gamma_s \neq 1 $ and $ 1 = \comm{\gamma_s}{G} $ implies $ \gamma_s \subseteq Z(G) $.
    \item[(2)] p-groups have non-trivial center. By induction on $ \abs{G} $, you can prove that p-groups are nilpotent.

        If $ G/Z $ is nilpotent, say $ \gamma_s(G/Z) = 1 $, then $ \gamma_{s+1}(G) \subseteq Z $ which implies $ \gamma_{s+2}(G) = 1 $.
\end{itemize}

\begin{definition}
    A \textit{subring} $ R $ of a ring with a unit is nilpotent of step $ s $ if the product of any $ s+1 $ elements is $ 0 $.
\end{definition}

\begin{claim}
    The set $ 1_R + R $ (the unit plus the elements of the subring) is a nilpotent group.
\end{claim}

\begin{ex}
    \begin{equation}
        R = \mqty(0&a&b&\cdots\\\vdots&\ddots&c&\cdots)_{n \times n}
    \end{equation}
    is nilpotent of step $ n-1 $.
\end{ex}

(back to claim) For $ r \in R $,
\begin{equation}
    (1-r)^{-1} = 1 + r + r^2 + \cdots r^s
\end{equation}
\begin{equation}
    (1-r)(1+r+\cdots+r^s) = 1 - r^{s+1} = 1
\end{equation}

\begin{claim}
    If $ r \in R^k $, $ r' \in R $,
    \begin{equation}
        \comm{1-r}{1-r'} \in 1+R^{k+1}
    \end{equation}
    where $ R^k= \text{span} \{r_1r_2\ldots r_k\} $ or $ R^k = \{\sum_{i=1}^{m} \prod_{j=1}^{k} r_{ij} \mid r_{ij} \in R,\ m \in \Z \geq 0\} $, so for example, $ R^2 = \{r_1 r_2 + r_3 r_4 +\cdots + r_{99} r_{100}, \cdots\} $.
\end{claim}
\begin{proof}
    \begin{align}
        (1+\overbrace{r+r^2 +\ldots + r^s}^{\#1})&(1+\overbrace{r'+r'^2 + \ldots + r'^s}^{\#2})(1-\overbrace{r}^{\#3})(1-\overbrace{r'}^{\#4}) \\
        &= \overbrace{(1+r'+\ldots+r'^s)(1-r')}^{=1} +\overbrace{\#1}^{\in R^k} (\#2 + \#3 \ldots) \\
        &+ \#2\#3\#4 + \#1 \cdot 1 + \#3 \cdot 1\\
        &\in 1 + rR + RrR + rR + (r + r^2 + \ldots + r^s - r)\\
        &\in 1 + R^{k+1}
    \end{align}
\end{proof}


\end{document}
