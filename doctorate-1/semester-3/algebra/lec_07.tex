\documentclass[a4paper,twoside,master.tex]{subfiles}
\begin{document}
\lecture{7}{Wednesday, September 16, 2020}{}

We will continue with the proof from the last lecture. We ended with the claim that if $ H < G $, then $ N_G(H) > H $. Next, we claim that every Sylow subgroup is normal in $ G $.
\begin{proof}
    $ P $ is p-Sylow in $ G $. Consider the normalizer $ N = N_G(P) $. We can trivially say that $ P \triangleleft N $.
    
    Now consider the normalizer of the normalizer, $ H = N_G(N) $. For $ h \in H $, $ P^h \subseteq N^h = N $. $ P^h $ is a subgroup of size $ \abs{P} $. Since all p-Sylow subgroups are conjugate. This means that $ P $ is the only p-Sylow subgroup of $ N $.

    $ P^h = P $ and hence $ P \triangleleft N_G(N) $. That means that $ N_G(N) \subseteq N $. Therefore, $ N_G(N) = N $. Therefore, by our first claim, $ N = G $. 
\end{proof}

The final step in the original proof is by induction.

\begin{claim}
    \begin{equation}
        P_1 P_2 \cdots P_s \cong P_1 \times P_2 \times \cdots \times P_s
    \end{equation}
\end{claim}
\begin{proof}
    For $ s = 1 $, this is trivial. Now suppose $ H = P_1 P_2 \cdots P_s \triangleleft G $. Let's look at $ H \cap P_{s+1} $. $ \abs{H \cap P_{s+1}} $ divides both $ \abs{H} $ and $ \abs{P_{s+1}} $. The order of $ H $ is the power of primes up to $ s $ while the order of $ P_{s+1} $ is a power of that prime, so $ H \cap P_{s+1} = 1 $.

    Recall $ \comm{H}{P_{s+1}} \subseteq H \cap P_{s+1} = 1 $, so these commute, and we have an isomorphism.
\end{proof}


\section{Solvable Groups}
\label{sec:solvable_groups}

Like nilpotent groups, we can describe another kind of series, called a derived series:
\begin{equation}
    G^{(0)} = G\qquad G^{(1)} = \comm{G}{G}\qquad G^{(i+1)} = \comm{G^{(i)}}{G^{(i)}}
\end{equation}

\begin{definition}
    A group $ G $ is \textit{solvable} if $ G^{(s)} = 1 $ for some $ s $. The smallest such $ s $ is called the \textit{solvable length}.
\end{definition}

\begin{lemma}
    If $ N\triangleleft G $, then $ \comm{G/N}{G/N} = \frac{\comm{G}{G} N}{N} $.
\end{lemma}
\begin{proof}
    \begin{equation}
        \comm{g_1 N}{g_2 N} = \comm{g_1}{g_2} N
    \end{equation}
    If we now take a group generated by this commutator, we see that
    \begin{equation}
        \ev{\comm{g_1 N}{g_2 N}} = \ev{\comm{g_1}{g_2}} N = \comm{G}{G} N
    \end{equation}
\end{proof}

\begin{claim}
    If $ N \triangleleft G $, then $ G $ is solvable iff both $ N $ and $ G/N $ are.
\end{claim}

\begin{proof}
    Suppose $ G $ is solvable. $ N^{(i)} \subseteq G^{(i)} $ by induction on $ i $. We know that $ G^{(s)} = 1 $ for some $ s $, so $ N^{(s)} = 1 $ (it possibly has a shorter solvable length, but we don't care).

    Now what about $ G/N $? From our lemma, we know that
    \begin{equation}
        \left( \frac{G}{N} \right)^{(i)} = \frac{G^{(i)} N}{N} \implies \left( \frac{G}{N} \right)^{(s)} = \frac{N}{N} = 1_N
    \end{equation}


    In the converse direction, assume $ N $ and $ G/N $ are solvable. This means that $ \exists s $ such that $ \left( \frac{G}{N} \right)^{(s)} = 1 $ and $ \exists t $ such that $ N^{(t)} = 1 $. 

    \begin{equation}
        \frac{G^{(s)} N}{N} = \left( \frac{G}{N} \right)^{(s)} = 1_N \implies G^{(s)} \leq N
    \end{equation}
    \begin{equation}
        \implies G^{(s+i)} \leq N^{(i)} \quad\forall i
    \end{equation}
    \begin{equation}
        \implies G^{(s+t)} \leq N^{t} = 1
    \end{equation}
\end{proof}

\begin{claim}
    A group $ G $ is solvable iff there is a series of subgroups $ 1 = H_0 \triangleleft H_1 \triangleleft \cdots \triangleleft H_s = G $ and $ \frac{H_{i+1}}{H_i} $ are abelian for all $ i $. 
\end{claim}
\begin{proof}
    First, an observation. For $ H \triangleleft G $, if $ G/H $ is abelian,
    \begin{equation}
        \frac{\comm{G}{G} H}{H} = \comm{\frac{G}{H}}{\frac{G}{H}} = 1
    \end{equation}
    and so $ \comm{G}{G} \subseteq H $.

    Now suppose $ H_0, H_1, \cdots, H_s $ exist. We claim that $ G^{(i)} \subseteq H_{s-i} $ by induction. For $ i = 0 $ this is trivially true.
    \begin{equation}
        G^{(i+1)} = \comm{G^{(i)}}{G^{(i)}} \subseteq \comm{H_{s-i}}{H_{s-i}} \subseteq H_{s-i-1}
    \end{equation}

    For the converse, set $ H_i = G^{(s-i)} $. $ \frac{H_{i+1}}{H_{i}} $ is abelian.
\end{proof}

\section{Ring Morphisms}
\label{sec:ring_morphisms}

Rings are associative but not necessarily with $ 1 $.
\begin{definition}
    A \textit{ring morphism} is a morphism $ \varphi \colon R \to S $ such that $ \varphi(a+b) = \varphi(a) + \varphi(b) $ and $ \varphi(ab) = \varphi(a) \varphi(b) $.
\end{definition}

\begin{definition}
    The \textit{ideal} is the kernel of a morphism.
\end{definition}

\end{document}
