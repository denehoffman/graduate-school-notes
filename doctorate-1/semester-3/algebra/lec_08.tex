\documentclass[a4paper,twoside,master.tex]{subfiles}
\begin{document}
\lecture{8}{Friday, September 18, 2020}{}

\begin{definition}
    $ a \neq 0 $ is a \textit{zero divisor} if $ \exists b $ such that $ ab = 0 $ or $ b a = 0 $.
\end{definition}

\begin{definition}
    A ring $ R $ is an \textit{integral domain} if $ R $ is commutative and has no zero divisors.
\end{definition}

\begin{definition}
    A \textit{field} is a commutative ring where all nonzero elements have an inverse.
\end{definition}

\section{Polynomial Rings}
\label{sec:polynomial_rings}

$ R $ is a commutative ring with $ 1 $. $ X $ is a symbol. $ R[X] $ consists of all formal expressions of the form
\begin{equation}
    a_0 + a_1 x + \cdots + a_n x^n
\end{equation}
with $ n \in \Z_{\geq 0} $ and $ n_i \in R $ with the usual addition and multiplication in the ring.
\begin{equation}
    (a_0 + a_1 x + \cdots + a xu n x^n)(b_0 + b_1 x + \cdots + b_m x^m) = \sum_{t=0}^{\infty} x^t \sum_{i+j=t} a_i b_j
\end{equation}

Formally, $ R[X] $ consists of functions $ f\colon \Z_{\geq 0} \to R $ with finite support ($ f(i) \neq 0 $ only at finitely many points).

\subsection{Power Series and Laurent Series}
\label{sub:power_series_and_laurent_series}

A power series is an expression of the form $ a_0 + a_1 x + \cdots + a_n x^n + \cdots $, usually denoted $ R[ [X]] $ (no finite support).

A Laurent series a power series that allows for negative powers:
\begin{equation}
    a_{-m} x^{-m} + a_{-m+1} x^{-m+1} + \cdots + a_0 + a_1 x + \cdots
\end{equation}

\begin{note}{Notation}
    \begin{equation}
        (R[X])[Y] \equiv R[X,Y] \cong R[Y,X]
    \end{equation}
    and if
    \begin{equation}
        X = (X_1, X_2, \cdots, X_n)
    \end{equation}
    \begin{equation}
        R[X] = R[X_1, X_2, \cdots, X_n]
    \end{equation}
\end{note}

\begin{note}{Fun Aside}
    We can construct ordered rings by adding some ordering operation $ (R, <) $ with the usual rules. For example, $ \R[X] $ with the ordering $ p<q \iff LC(q-p) > 0 $ where $ LC(p) $ is the leading coefficient of $ p $.

    This ordering is a way to define the statement
    \begin{equation}
        \lim_{x \to \infty} (q-p)(x) > 0
    \end{equation}

    Another example is $ q>p $ if the least significant nonzero coefficient of $ q - p $ is $ >0 $. Instead of $ \R[X] $, we can think of instead $ \R[\epsilon] $ where $ \epsilon $ is ``infinitesimal''
\end{note}

\section{Group Rings}
\label{sec:group_rings}

\begin{definition}
    $ G $ is a group, $ R $ is a commutative ring with $ 1 $. The ring $ R[G] $ consists of formal expressions of the form $ a_1 g_1 + a_2 g_2 + \cdots + a_n g_n $ where $ n \in \Z_{\geq 0} $, $ a_i \in R $, and $ g_i \in G $.
\end{definition}

For example, if we look at $ R[\Z] $, this is almost the same as $ R[X] $. If we have $ a_1 n_1 + a_2 n_2 + \cdots + a_m n_m $, this is equivalent to writing $ a_1 x^{n_1} + a_2 x^{n_2} + \cdots + a_m x^{n_m} $.

Formally, we define group rings as functions $ f\colon G \to R $ with finite support and
\begin{equation}
    (f+g)(x) = f(x) + g(x)\qquad (fg)(x) = \sum_{y,z = x} f(y) g(z)
\end{equation}

\begin{note}{Polynomial Functions}
    $ f \in R[X] $ induces a function $ R \to R $: $ x \in R \mapsto a_0 + a_1 x + \cdots + a_n x^n $.
\end{note}

\begin{ex}
    \begin{equation}
        R = \Z/2\Z
    \end{equation}
    \begin{equation}
        x \in \Z/2\Z[X]
    \end{equation}
    and
    \begin{equation}
        x^2 \in \Z/2\Z[X]
    \end{equation}
    induce the same functions, since $ 0^2 = 0 $ and $ 1^2 = 1 $.
\end{ex}

\section{Ideals}
\label{sec:ideals}

Recall $ I \subseteq R $ is an ideal if $ I $ is a subring of $ R $ and $ a b \in I $ if either $ a $ or $ b $ is in $ I $. This is equivalent to saying $ I = \text{ker} \varphi $ for some morphism $ \varphi $: $ \varphi \colon R \to R/I $.

\begin{ex}
    The main example is $ n\Z \subseteq \Z \quad\forall n $.
\end{ex}

\subsection{Generation}
\label{sub:generation}

For a ring $ R $, if $ A \subseteq R $, $ (A) $ is the ideal generated by $ A $: 
\begin{equation}
    (A)= \Cap_{I\supseteq A} I
\end{equation}
where $ I $ is ideal.

If $ R \subseteq S $ and $ A \subseteq S $, the notation $ R[A] $ is the subring of $ S $ generated by $ R \cup A $. 

For example, $ \Z[\sqrt{2}] = \{a + b \sqrt{2} \colon a,b \in \Z\} $.

\begin{note}{Notation}
    Say $ F $ is a field and $ F \subset G $, another field. $ A \subseteq G $ is a set, and $ F(A) $ is the field generated by $ F \cup A $.
\end{note}

\begin{definition}
    Ideal $ M \subset R $ is \textit{maximal} if $ M \neq R $ and there is no ideal $ I $ such that $ M \subset I \subset R $.
\end{definition}

\begin{definition}
    An ideal $ P \subset R $ is \textit{prime} if $ \forall a, b $, $ a b \in P $ then either $ a \in P $ or $ b \in P $. 
\end{definition}

\begin{ex}
    $ n\Z $ is prime in $ \Z $ iff $ n $ is prime or negative prime.
\end{ex}

Properties of commutative $ R $:
\begin{itemize}
    \item $ M \subseteq R $ is maximal iff $ R/M $ is a field
    \item $ P \subset R $ is prime iff $ R/P $ is an integral domain (no zero divisors)
\end{itemize}
\begin{proof}
    Fun observation: if $ I \subseteq R $ is ideal, then $ I = R \iff 1 \in I $.
\end{proof}



\end{document}
