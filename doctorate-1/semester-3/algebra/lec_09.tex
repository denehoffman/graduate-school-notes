\documentclass[a4paper,twoside,master.tex]{subfiles}
\begin{document}
\lecture{9}{Monday, September 21, 2020}{}


\section{Factorization}
\label{sec:factorization}

\begin{definition}
    Suppose $ R $ is an integral domain. Then $ N\colon R \to \Z_{\geq 0} $ is a \textit{norm} that makes $ R $ into a \textit{Euclidean domain} if $ \forall a \in R $, $ \forall q \neq 0 $, $ \exists b,r $ such that $ a = qb + r $ such that $ N(r) \geq N(q) $.
\end{definition}

\begin{ex}
    In $ \Z $, $ N(a) = \abs{a} $.
\end{ex}
\begin{ex}
    In a field $\mathbb{F}$, $ f \in \mathbb{F}[x] $, $ N(f) = \text{deg}(f) + 1 $ and $ N(0) = 0 $.
\end{ex}


\begin{definition}
    A \textit{principal ideal domain} is an integral domain all of whose ideals are principal (they are generated by a single element).
\end{definition}

\begin{claim}
    Euclidean domains are principal ideal domains.
\end{claim}
\begin{proof}
    Consider ideal $ I \subset R $. Say that $ m = \min \{N(a) \colon a \in I-0\} $ and pick $ a \in I $ such that $ N(a) = m $. We claim that $ I = (a) $ ($ I $ is generated by $ a $). $ (a)\subseteq I $, so for $ b \in I $, $ \exists c,r $ such that $ b = ac + r $. $ N(r) < N(a) = m $, and $ r \in (a,b) \subseteq I $.
\end{proof}

Recall that a prime ideal $ P \subseteq R $ is an ideal such that $ ab\in P $ implies $ a \in P $ or $ b \in P $.

\begin{definition}
    A \textit{prime element} is a $ p \in R $ such that $ (p) $ is prime and $ a \in (p) \iff \exists b, a = b p $.
\end{definition}

\begin{definition}
    An \textit{irreducible} element $ a $ is an element such that when $ a = b c $ then either $ b $ or $ c $ is a unit invertible.
\end{definition}

\begin{claim}
    If $ p \in R $ is prime, then $ p $ is irreducible.
\end{claim}
\begin{proof}
    We want to show that if $ p = a b $ then either $ p\mid a $ or $ p\mid b $. Say $ a = c p $. Then $ p = (c p)b \implies c b \text{ is irreducible} $.
\end{proof}

The reverse is not true (not all irreducibles are prime):
\begin{ex}
    \begin{equation}
        \Z[\sqrt{-5}] = \{a + b \sqrt{-5}\colon a,b \in \Z\}
    \end{equation}
    Now $ 6 = 2 \cdot 3 \in \Z[\sqrt{-5}] $, and $ 2 $ and $ 3 $ are irreducible. Consider the mapping $ x\mapsto \abs{x}^2 $, so $ N(a + b \sqrt{-5}) = a^2 + 5 b^2 $. There are other ways to factor $ 6 $:
    \begin{equation}
        6 = (1+ \sqrt{-5})(1 - \sqrt{-5})
    \end{equation}
    $ 2 $ and $ 3 $ are irreducible but neither divides $ 1 \pm \sqrt{-5} $.
\end{ex}


\subsection{Uniqueness of Factorization into Ideals}
\label{sub:uniqueness_of_factorization_into_ideals}

This is true for many rings, and for example, consider the ring we just used, $ \Z[\sqrt{-5}] $.
\begin{equation}
    (2) = (2,1+ \sqrt{-5})(2, 1- \sqrt{-5}) = (4, 2\pm 2 \sqrt{-5},6) = (2,2\pm 2 \sqrt{-5})
\end{equation}
which is ideal.

\begin{definition}
    An integral domain $ R $ is a \textit{unique factorization domain} (UFD) if every $ x \in R $ can be written $ x = u r_1 r_2 \cdots r_s $ where $ r_i $ are irreducible, $ u $ is a unit, and this is unique up to permuting $ r $'s and replacing them by associates (i.e. replace $ r \to r u $). 
\end{definition}

\begin{claim}
    Prime ideal domains are unique factorization domains (converse is not true).
\end{claim}
\begin{proof}
    Let $ S $ be the set of all elements in PID $ R $ such that they cannot be factored into irreducibles (we want to show this set is empty). Consider chains of ideals
    \begin{equation}
        (a_1) \subset (a_2) \subset \cdots
    \end{equation}
    where $ a_i \in S $.

    \begin{equation}
        I = \bigcup_{i=1}^{\infty} (a_i)
    \end{equation}
    is an ideal. A PID implies that $ \exists a \in R $ such that $ I = (a) $. $ a \in I \implies a \in (a_i) $ for some $ i $. This implies $ (a_i) \subseteq I = (a) \subseteq (a_i) $, so the chain must be finite. Let the last element be $ (a_n) $. $ a_n $ is not irreducible since it is in $ S $, therefore it can be factored:
    \begin{equation}
        \exists b,c\qquad b c = a_n
    \end{equation}
    where $ b, c $ are not units. In this case, we can extend the chain: If both $ b $ and $ c $ are not in $ S $, then they have factorizations, so multiplying those factorizations gives a factorization of $ a_n $. If $ b \in S $, $ (a_n) \subset (b) $. Hence $ S = \emptyset $. 

    Proving uniqueness is easier. Suppose $ x = u r_1 r_2 \cdots r_m $. We claim that in a PID, irreducibles are prime. For some irreducible $ p $, $ p\mid a b $, and we want to show it divides either $ a $ or $ b $. Suppose $ p $ does not divide $ a $. Consider $ (p,a) $. This is an ideal inside the PID. That means there is a way to generate it with a single element: $ \exists c \in R $ such that $ (p,a) = (c) $. This means that $ c\mid p $, and since $ p $ is irreducible, $ c $ is a unit or $ c $ is $ p $ times some unit. If the second was the case, then $ (c) = (p) $ which would imply $ p\mid a $, a contradiction. Therefore, $ c $ is a unit, so $ (c) = (1) $.

    Therefore, $ \exists x,y \in R $ such that $ 1 = x p + y a $. Multiplying both sides by $ b $, we see that $ b = b x p + y a b $. $ p \mid a b $, so $ p\mid b $.

    Now we know that the irreducibles in a PID are prime. $ x = u r_1 \cdots r_n = u' y_1 \cdots y_m $. $ r_1 $ is prime implies $ r_1 \mid y_i $ for some $ i $. Say $ r_1\mid y_1 $, so $ y_1 = w r_1 $. Then $ u r_2 \cdots r_n = u' w y_2 \cdots y_m $.
\end{proof}




\end{document}
