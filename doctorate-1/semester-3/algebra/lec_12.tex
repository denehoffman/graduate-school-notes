\documentclass[a4paper,twoside,master.tex]{subfiles}
\begin{document}
\lecture{12}{Monday, September 28, 2020}{}

\begin{proof}
    Proof continued from last lecture:

    \begin{lemma}
        If $ R $ is a UFD and $ F $ is its field of functions, then $ f \in R[x] $ is irreducible in $ R[x] $ iff it is irreducible in $ F[x] $.
    \end{lemma}
    \begin{proof}
        Without loss of generality, let $ \text{content}(f) = 1 $, $ f = a b $ in $ F[x] $. Define $ a' = \frac{a}{\text{content}(a)} $ and $ b' = \frac{b}{\text{content}(b)} $, so $ f = a'b' \in R[x] $. 
    \end{proof}
    Finally, we can prove the theorem. If $ f \in R[x] $. By definition, $ f = \text{content}(f) f' $ for some $ f' $. We can factor $ \text{content}(f) $ in $ R $.

    Without loss of generality, suppose $ \text{content}(f) = 1 $. Then we can factor $ f $ inside $ F[x] $ as $ f = a_1 a_2 \cdots a_n $, where $ a_i $ are irreducible in $ F[x] $.

    Define $ a_i' = \frac{a_i}{\text{content}(a_i)} \in R[x]  $, since we have a mapping $ \pi \text{content}(a_i) = \text{content}(f) = 1 $. Therefore, $ f = a_1' a_2' \ldots a_n' $, and by the most recent lemma, $ a_i' $ are irreducible in $ R[x] $. The proof of uniqueness is similar.
\end{proof}

\subsection{Irreducibility Criteria}
\label{sub:irreducibility_criteria}

\begin{itemize}
    \item[1.] $ x - a \mid f(x) $, $ f \in F[x] $ $ \iff f(a) = 0 $
    \item[2.] If $ R $ is an integral domain, $ f \in R[x] $, and $ P \subset R $ is a proper prime ideal, then let $ \bar{f} $ denote its image in $ (R/P)[x] $. If $ \bar{f} $ cannot be factored into polynomials of degree smaller than $ \text{deg} f $, then $ f $ is irreducible in $ R[x] $. (Proof: $ f = a b \implies \bar{f} = \bar{a} \bar{b} $)
    \item[3.] Eisensteins Criterion:
        \begin{equation}
            f = x^n + a_{n-1} x^{n-1} + \cdots + a_1 x + a_0
        \end{equation}
        Prime ideal $ P $ with $ \forall i,\ a_i \in P $ and $ a_0 \notin P^2 $, then $ f $ is irreducible.
        \begin{proof}
            $ (R/P)[x] $, $ f = b c $, $ b = \sum_{i=0}^{s} b_i x^i $ and $ c = \sum_{i=0}^{t} c_i x^i $. Note we cannot have $ b_0, c_0 \in P $ because that would imply $ a_0 = b_0 c_0 \in P^2 $.

            Say $ c_0 \notin P $. Let $ i $ be the least such that $ b_i \notin P $. Then $ a_i \in P = b_i c_0 + b_{i-1} c_1 + b_0 c_i $, where the last two terms are in $ P $ but the first is not.
        \end{proof}
    \item[4.] $ F[x_1, \cdots, x_n = F[x] $. As notation, for $ \alpha \in \Z^n_{\geq 0} $, write $ x^{\alpha} = x_1^{\alpha_1} \cdots x_n^{\alpha_n} $.

            We can write $ f = \sum_{\alpha \in S} C_{\alpha} x^{\alpha} $. 

            \begin{definition}
                The \textit{Newton polytope} for $ f $ is the set $ \text{conv}(S) $ which is the convex hull of $ S $ denoted by $ N(f) $.
            \end{definition}
            \begin{claim}
                \begin{equation}
                    N(fg) = N(f) + N(g)
                \end{equation}
            \end{claim}
\end{itemize}

\section{Noetherian Rings}
\label{sec:noetherian_rings}

\begin{definition}
    A ring $ R $ is called \textit{Noetherian} if $ R $ is commutative with $ 1 $ and every ideal is finitely generated (this is a generalized PID).
\end{definition}

\begin{theorem}
    If $ R $ is Noetherian, then so is $ R[x] $.
\end{theorem}
\begin{corollary}
    If $ F $ is a field, then $ F[x_1, \cdots, x_n] $ is Noetherian.
\end{corollary}

\subsection{Systems of Polynomial Equations over Fields}
\label{sub:systems_of_polynomial_equations_over_fields}

Take a field $ F[X] = F[x_1, \cdots, x_n] $. We can look at a system of equations $ f_1(X) = f_2(X) = \cdots = f_n(X) = 0 $. A consequence of this system is that for some $ a_i $, $ a_1 f_1 + a_2 f_2 + \cdots + a_m f_m = 0 $.

The set $ \{a_1 f_1 + \cdots + a_m f_m\colon a_i \in F[x]\} $ is the ideal generated by the polynomials: $ (f_1, \cdots, f_m) $. Therefore, we can think of a system of equations as an ideal.

If we have an infinite system of equations, we can do the same thing: $ f_1 = f_2 = \cdots = 0 $ can be defined as the ideal $ (f_1, \cdots) $, but the consequence of Noetherian rings is that this can be written as some finitely generated set $ (g_1, \cdots, g_t) $.

\begin{note}{Warning}
    $ x^2 - y = y = 0 $. The ideal $ (x^2 - y, y)= (x^2, y) \not\ni x $
\end{note}
\begin{definition}
    The \textit{radical} of an ideal is $ \text{rad}(I) = \{f\colon \exists m,\ f^m \in I\} $.
\end{definition}

\end{document}
