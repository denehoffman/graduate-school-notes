\documentclass[a4paper,twoside,master.tex]{subfiles}
\begin{document}
\lecture{16}{Wednesday, October 07, 2020}{Modules}

\section{Modules}\label{sec:modules}

\begin{definition}
    Given a ring $ R $ which is not necessarily commutative (and not necessarily with $ 1 $), a left $ R $-module is an abelian group $ M $ with an action of $ R $ on $ M $. This is a map $ R \times M \to M $. We will denote this map $ (r,m) \mapsto \varphi(r,m) $ as $ rm = \varphi(r,m) $.

    \begin{align}
        r(m + m') &= rm + rm'\\
        (rr')m &= r(r'm)\\
        (r+r')m &= rm+r'm
    \end{align}
    If $ 1 \in R $, then $ 1m = m $.
\end{definition}


\begin{ex}
    If $ F $ is a field, $ F $-module is the same as a vector space over $ F $.
\end{ex}
\begin{ex}
    Take $ R $ to be any ring and $ M = R $ with action being the ring multiplication. In this example, $ R $-submodules of $ M $ are left ideals.
\end{ex}
\begin{ex}
    If $ R = \Z $ then $ R $-modules are abelian groups.

    $ r m = \overbrace{1+\cdots+1}^r m = \overbrace{m+\cdots+m}^r $.
\end{ex}
\begin{ex}
    Let $ F $ be a field. Consider the ring of polynomials over $ F $ as $ F[x] $-modules $ M $. $ M $ is an $ F $-module, since $ F \subset F[x] $. Multiplication by $ x $ takes $ m \mapsto xm $ and is a linear map\textemdash{} call it $ T $. We now have a vector space $ V $ together with a linear map $ T\colon V \to V $.

    Say we have a polynomial $ p(x) = \sum_{i=0}^{n} a_i x^i \in F[x] $.
    \begin{equation}
        p(x) V \equiv (a_0 + a_1 T + \cdots + a_n T^n)V
    \end{equation}
    An $ F[x] $-submodule is a subspace invariant of V under $ T $.
\end{ex}
\begin{ex}
    $ F[x,g] $-modules would have two such linear maps, say $ T $ and $ S $ such that $ TS = ST $.
\end{ex}
\begin{ex}
    Take a group $ G $ and field $ F $. The group ring $ FG $ is $ \{a_1 g + \cdots + a_n g_n \mid n \in \Z^+, a_i \in F\} $. The $ FG $-module is given by an $ F $-vector space together with linear maps $ Tg\colon V \to V\ \forall g \in G $ such that $ T_g T_{g'} = T_{gg'} $.
\end{ex}

\subsection{Module morphisms}\label{sub:module_morphisms}

Module morphisms are maps $ \varphi \colon M \to N $ (where $ M,N $ are left $ R $-modules) such that
\begin{align}
    \varphi(r,m) &= r \varphi(m) \\
    \varphi(m+m') = \varphi(m) + \varphi(m')
\end{align}

\begin{equation}
    M \times N = \{(m,n)\colon m \in M, n \in N\}
\end{equation}
$ r(m,n) = (rm,rn) $. Similarly, $ \prod_{i \in I} M_i $ and $ \bigoplus_{i \in I} M_i $ both operate as expected.

\begin{equation}
    \text{Hom}_R(M,N) = \{\text{homomorphisms from } M \to N \}
\end{equation}
$ \text{Hom}_R(M,N) $ is an $ R $-module:
\begin{align}
    \varphi, \psi &\in \text{Hom}_R(M,N) \\
    (\varphi + \psi)(m) &= \varphi(m) + \psi(m) \\
    (r \varphi)(m) &= r(\varphi(m))
\end{align}
If $ A $ is a subset of an $ R $-module $ M $, $ RA = \{r_1 a_1 + \cdots + r_n a_n \colon n \in \Z^+, a_i \in A, r_i \in R\} $.

\section{Tensor Products}\label{sec:tensor_products}

\begin{note}{Digression on Universal Properties}
    Free groups on a set $ S $ of generators are defined as $ F(S) = \{s_1^{\epsilon_1} s_2^{\epsilon_2} \cdots s_m^{epxilon_m}\} $.

    The universal property is satisfied as follows. If $ \varphi \colon G \to S $ (for group $ G $), then there exists a unique way to extend it as a mapping to the free group generated by $ S $: $ \exists! \Phi \colon G \to F(S) $ and conversely, any $ \Phi $ gives a unique $ \varphi $ such that the diagram commutes.

    Now take the free abelian group on $ S $, $ FA(S) = F(S)/ \{x_1 x_2 x_1^{-1} x_2^{-1} \colon x_1, x_2 \in F(S)\} = F(S)/[F(S),F(S)] $. This satisfies a similar universal property.

\end{note}


Consider vector spaces $ V,W,U $ over $ F $. The map $ T\colon V \times W \to U $ is bilinear. $ T(v, \cdot) \colon W \to U $ is linear $ \forall V $ and $ T(\cdot, w) \colon V \to U $ is linear $ \forall w \in W $.


\end{document}
