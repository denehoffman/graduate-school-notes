\documentclass[a4paper,twoside,master.tex]{subfiles}
\begin{document}
\lecture{17}{Friday, October 09, 2020}{Tensor Products}

\begin{definition}
    \textit{Tensor products} are universal objects for bilinear maps. If we can define left $ R $-modules $ rm $ and right $ R $-modules $ mr $, we can also define $ (R,S) $-bimodules with the form $ rms $.

    In particular, we want to take a right $ R $-module $ M $ and left $ R $-module $ N $ and construct $ mrn $ through some map $ \varphi \colon M \times N \to T $. Such a map is \textit{R-balanced} if $ \varphi(mr,n) = \varphi(m,rn) $ and is linear for each $ M $ and $ N $.
\end{definition}

Recall that if we have abelian groups $ A $, $ B $, and $ C $ and we look at maps $ A/B \to C $, and these are the same as maps $ \varphi \colon A \to C $ such that $ \varphi(B) = 0 $.

We can similarly construct $ R $-balanced maps from $ M \times N $ as maps from $ M \otimes_R N $. Consider $ \text{FA}(M \times N) $, the free abelian group on $ M \times N $. Let $ B \subset A $ be generated $ (mr,n)-(m,rn) $ and $ (m,n_1+n_2)-(m,n_1)-(m,n_2) $ and so on. $ A $ is an abelian group, and we can define $ M\otimes_R N \equiv A/B $.

If $ M $ is and $ (S,R) $-bimodule, $ M\otimes_R N $ is a left $ S $-module. Similarly, if $ N $ is not a left $ R $-module, it can be an $ (R,S) $-bimodule.

\begin{note}{Notation}
    The equivalence class of $ (m,n) \in M \times N $ in $ M\otimes_R N $ is denoted $ m\otimes n $.
\end{note}

\begin{ex}
    $ \Z/2\Z \otimes_{\Z} \Z/3\Z \cong 0 $. Let's look at some element $ a\otimes b $ (this is not the most general element, the general elements are linear combinations of such elements).
    \begin{equation}
        a\otimes b = 3a\otimes b = a \otimes 3b = a \otimes 0 = 0
    \end{equation}
    since $ x \otimes 0 = x\otimes 0 \cdot 0 = 0x \otimes 0 = 0 \otimes 0 = 0 $. 
\end{ex}
\begin{ex}
    $ \Z/2\Z \otimes_{\Z} \Z/2\Z \cong \Z/2\Z $. $ T $ is generated by $ 1\otimes 1 $, since other products are trivial. Consider a bilinear map on $ \Z/2\Z \times \Z/2\Z $:
    \begin{equation}
        \varphi(a,b) = ab
    \end{equation}
    \begin{equation}
        1 \otimes 1 + 1 \otimes 1 = 2 \otimes 1 = 0 \otimes 1 = 0
    \end{equation}
\end{ex}
\begin{ex}
    In general, $ \Z/m\Z \otimes_{\Z} \Z/n\Z \cong \Z/ \gcd(m ,n) \Z $
\end{ex}
\begin{ex}
    $ \Q \otimes_{\Z} \Q/\Z \cong 0 $:
    \begin{equation}
        \frac{a}{b} \otimes \frac{c}{d} = \frac{1}{d} \frac{a}{b} \otimes d \frac{c}{d} = \frac{a}{bd} \otimes c = \frac{a}{bd} \otimes 0 = 0
    \end{equation}
\end{ex}
\begin{ex}
    Say $ V,U $ are $ F $-vector spaces. $ U \otimes_F V $. Suppose $ b_1, \cdots, b_n $ is a basis for $ U $ and $ c_1, \cdots, c_n $ is a basis for $ V $. Then the set $ G = \{b_i \otimes c_j\} $ spans $ U \otimes_F V $.
    
    $ G $ is linearly independent. We want to show that $ \sum a_{ij} b_i \otimes c_j = 0 $, all $ a_{ij} $ must be $ 0 $. This is to say that for every bilinear $ T\colon U \times V $, $ \sum a_{ij} T(b_i, c_j) = 0 $.

    Define $ T_{ij} \colon U \times V \to F $ by $ T_{ij}(\sum_k \beta_k b_k, \sum_l \gamma_l c_l) \equiv \beta_i \gamma_j $. This map shows that $ a_{ij} = 0 $. 
\end{ex}

\begin{claim}
    $ M \otimes_R(N \oplus N) cong (M \otimes_R N) \otimes (M \otimes_R N) $
\end{claim}
\begin{proof}
    $ (m,(n,n')) \mapsto (m \otimes n, m \otimes n') $. The inverse mapping is defined as $(m \otimes n, m' \otimes n') \mapsto m \otimes (n,0)+ m' \otimes (0,n') $. 
\end{proof}
\begin{ex}
    $ R $ is a subring of $ S $. Think of $ S $ as an $ (S,R) $-bimodule. Say $ M $ is a left $ R $-module. Then $ S \otimes_R M $ is a left $ S $-module.
\end{ex}
\begin{ex}
    $ S = \C $, $ R = \R $. For a vector space $ V $ over $ \R $, $ V' = \C \otimes_{\R} V $ is a $ \C $-vector space. If $ v \in V $, then $ 1\otimes v \in V' $.
\end{ex}

\subsection{Tensors of Homomorphisms}\label{sub:tensors_of_homomorphisms}

$ \varphi \colon M \to M' $ is a hom of right $ R $-modules and $ \psi \colon N \to N' $ is a hom of left $ R $-modules. Define $ \varphi \otimes \psi \colon M \otimes_{R} N \to M' \otimes_R N' $ by $ (\varphi \otimes \psi)(m \otimes n) = \varphi(m) \otimes \psi(n) $.

If $ R $ is a field, the matrix of $ \varphi $ is $ \mqty(a_{11} & a_{12} & \cdots\\a_{21} &\ddots&\cdots\\\vdots&\vdots&\ddots) $, and the matrix of $ \psi $ is $ A $, matrix representations of $ \varphi \otimes \psi $ are a block matrix $ \mqty(a_{11}A & a_{12}A & \cdots\\a_{21}A &\ddots&\cdots\\\vdots&\vdots&\ddots) $. 

\end{document}
