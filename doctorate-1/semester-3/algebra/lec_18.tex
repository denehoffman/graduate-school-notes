\documentclass[a4paper,twoside,master.tex]{subfiles}
\begin{document}
\lecture{18}{Monday, October 12, 2020}{Tensor Products, cont.}

Tensor products are \textit{associative}. If we have three modules, $ M $, $ N $, and $ L $ which are right $ R $-modules, $ (R,S) $-bimodules, and left $ S $-modules respectively, then
\begin{equation}
    (M \otimes_R N) \otimes_S L \cong M \otimes_R (N \otimes_S L)
\end{equation}
These are isomorphic because both the left and right-hand sides are in bijection with triply balanced maps $ \varphi \colon M \times N \times L \to D $ such that $ \varphi $ is trilinear and $ \varphi(mr,n,l) = \varphi(m,rn,l) $ and $ \varphi(r,ns,l) = \varphi(r,n,sl) $.

Given such $\varphi$, we can define the map $ (M \otimes_R N) \times L \to D $ by $ (m \otimes n, l) \to \varphi(m,n,l) $. This map is bilinear. Linearity in the second coordinate follows from linearity of $ \varphi $ in the third coordinate. Linearity in the first coordinate follows from biliniarity of $ \varphi $ in the first and second coordinates.

\paragraph{Main Case:} $ R $ commutative $ (R,R) $-bimodule where left and right actions coincide. We then talk about trilinear maps where
\begin{equation}
    \varphi(rm,n,l) = \varphi(m,rn,l) = \varphi(m,n,rl)
\end{equation}

Write $ M \otimes_R N \otimes_R L $.
\begin{definition}
    For $ R $ commutative with $ 1 $, $ M $ is an $ R $-algebra if $ M $ is a ring together with a map $ \varphi \colon R \to Z(M) $. In other words, $ M $ is an $ (R,R) $-bimodule ($ R $-module) given by $ rm = \varphi(r) m $.
\end{definition}

\begin{ex}
    $ \R $, $ \C $, quaternions, etc.
\end{ex}
If $ A $ and $ B $ are $ R $-algebras, then $ A \otimes_R B $ is also an $ R $-algebra under multiplication given by $ (a \otimes b)(a' \otimes b')\mapsto aa' \otimes bb' $.

Think of $ A \times B \times A \times B $ under the mapping $ (a,b,a',b')\mapsto aa' \otimes bb' $. This is quadrilinear and introduces a map $ (A \otimes_R B) \otimes_R (A \otimes_R B) \to A \otimes_R B $.

\section{Tensor Algebras}\label{sec:tensor_algebras}

Given an $ R $-module $ V $ (when we say $ R $-module you should really think ``vector space''), we can look at the tensor product of $ V $ with itself:
\begin{equation}
    V^{\otimes m} \equiv \underbrace{V \otimes V \otimes \cdots \otimes V}_{m}
\end{equation}
If we now have some element $ \alpha \in V^{\otimes i} $ and $ \beta \in V^{\otimes j} $, then $ \alpha \otimes \beta \in V^{\otimes i} \otimes V^{\otimes j} $.

\begin{definition}
    $ T(V) = \bigoplus_{m=0}^{\infty} V^{\otimes m} $ is a \textit{tensor algebra}.

    Given $ \alpha, \beta \in T(V) $, their product in $ T(V) $ is given by the tensor product. We can write any $ \alpha $ as $ \alpha = \alpha_0 \oplus \alpha_1 \oplus \cdots $ and $ \beta = \beta_0 \oplus \beta_1 \oplus \cdots $ such that $ \alpha \beta \equiv \alpha_0 \beta_0 \oplus (\alpha_0 \otimes \beta_1 + \alpha_1 \otimes \beta_0) \oplus (\alpha_0 \otimes \beta_2 + \alpha_1 \otimes \beta_1 + \alpha_2 \otimes \beta_0) \oplus \cdots $.
\end{definition}

\paragraph{Universal Property:} For every $ R $-module homomorphism $ \varphi \colon V \to A $ there exists a unique $ R $-algebra homomorphism $ \Phi\colon T(M) \to A $ where $ A $ is an $ R $-algebra such that $ \eval{\Phi}_{V} = \varphi $.

$ V^{\otimes i} \to A $ is given by $ v_1 \otimes \cdots \otimes v_i \mapsto \varphi(v_1) \varphi(v_2) \cdots \varphi(v_i) $. 


\begin{definition}
    A bilinear map $ T\colon V \times V \to D $. $ T $ is \textit{symmetric} if $ T(v,v') = T(v',v) $. There is an associated symmetric algebra $ S(V) = T(V) / I(V) $ where $ I(V) $ is an ideal generated by $ (v \otimes v' - v' \otimes v\colon v,v' \in V) $.
\end{definition}

\begin{definition}
    A map $ T\colon V \times V \to D $ is \textit{alternating} if $ T(v,v) = 0 \quad\forall v \in V $.
\end{definition}

\begin{definition}
    The \textit{exterior algebra} is $ \Lambda(V) = T(V) / A(V) $ where $ A(V) $ is the ideal generated by $ (v \otimes v\colon v \in V) $. 

    $ A(V) $ contains, for example, $ w \otimes v \otimes v \otimes u $ for some $ w $ and $ u $. 
\end{definition}

In $ \Lambda(V) $, $ (a + b) \otimes (a + b) $ is $ 0 $. By distributivity, the is $ a \otimes a + a \otimes b + b \otimes a + b \otimes b = a \otimes b + b \otimes a $ because of the $ / A(V) $. Therefore, $ a \otimes b = - b \otimes a $ in general, and we call this behavior \textit{antisymmetric}. This is not entirely equivalent to being alternating. An alternating map is antisymmetric, but the converse is not necessarily true. Suppose $ T(a,b) = -T(b,a) \quad\forall a, b $. This implies that $ T(a,a) + T(a,a) = 0 $. By bilinearity, this implies that $ T(2a,a) = 0 $. If $ R $ is a field such that $ 2 \neq 0 $, then $ T(a,a) = 0\forall a $, so it is alternating. If this is not the case (say $ R = \Z/2\Z) $, then $ (a,b)\mapsto ab $ is antisymmetric but not alternating ($ 1 \cdot 1 = 1 \neq 0 $).

We can also write the exterior algebra as a direct sum:
\begin{equation}
    \Lambda(M) = R \oplus M \oplus \Lambda^2 M \oplus \cdots
\end{equation}
where $ \Lambda^k M = M^{\otimes k} / (A(M) \cap M^{\otimes k}) $. This is an example of a graded algebra. Another example is
\begin{equation}
    R[x] = R \oplus Rx \oplus Rx^2 \oplus \cdots
\end{equation}


\end{document}
