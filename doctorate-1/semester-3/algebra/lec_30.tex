\documentclass[a4paper,twoside,master.tex]{subfiles}
\begin{document}
\lecture{30}{Friday, November 13, 2020}{}

\section{Separable Polynomials}\label{sec:separable_polynomials}

\begin{definition}
    For a field $ K $, we say that $ f \in K[x] $ is \textit{separable} if $ f $ has distinct roots in $ K $ (or equivalently in the splitting field of $ K $).
\end{definition}

The derivative of a polynomial $ f = \sum_n a_n x^n \in K[x] $ is
\begin{equation}
    \D f = \sum_n n a_n x^{n-1}
\end{equation}

It's easy to show that $ \D (fg) = f\D g + (\D f)g $.

\begin{claim}
    $ f \in K[x] $ is separable iff $ \gcd(f ,\D f) = 1 $.
\end{claim}
\begin{proof}
    If $ f $ is not separable, then $ f = (x-a)^2 g $ in $ K[x] $, so
    \begin{equation}
        \D f = 2(x-a)g + (x-a)^2 \D g
    \end{equation}
    so $ x-a \mid f, \D f $.

    Conversely, suppose $ x-a \mid f, \D f $. Then we could write
    \begin{equation}
        f = (x-a)g
    \end{equation}
    so
    \begin{equation}
        \D f = g + (x-a)\D g
    \end{equation}
    Now we know that $ x-a \mid \D f $, so therefore $ x-a\mid g $. Substituting this back into the equation, we say that $ g = (x-a)h $ so $ f = (x-a)^2 h $.
\end{proof}

\begin{claim}
    If $ f \in K[x] $ is irreducible and is not separable, then $ \text{char} K = p \neq 0 $ and $ f $ is of the form $ f(x) = g(x^p) $.
\end{claim}
\begin{proof}
    $ \deg \D f < \deg f $, and $ \gcd(f ,\D f) \neq 1 $ implies $ \exists g \mid f, \D f $. Since $ f $ is irreducible and UFD, $ \D f = 0 $. Therefore, $ n a_n x^n = 0 $ for all $ n $, so if $ a_n \neq 0 $ then $ n = 0 $ in $ K $, so $ \text{char} K \mid n $ $ \forall n $ such that $ a_n \neq 0 $.
\end{proof}

\begin{claim}
    If $ K $ is a field and $ \text{char} K = p \neq 0 $, then the map $ y \mapsto y^p $ is a homomorphism from $ K \to K $.
\end{claim}
\begin{proof}
    \begin{equation}
        (yz)^p = y^p z^p
    \end{equation}
    \begin{equation}
        (y+z)^p = \sum_{i+j=p} \binom{p}{i} y^i z^j
    \end{equation}
    We know that $ \binom{p}{i} = \frac{p(p-1)\cdots(p-i+1)}{i!} $ if $ i \geq 1 $, and if $ i<p $ then $ p\mid \binom{n}{i} $, so $ (a+b)^p = a^p + b^p $.
\end{proof}
\begin{corollary}
    If $ k = \Z/p\Z $, the $ g \in K[x] $, so $ g(x)^p = g(x^p) = (ax^s + bx^t)^p = a^p x^{sp} + b^p x^{pt} = ax^{sp} + bx^{pt} $.
\end{corollary}

\section{Cyclotomic Extensions}\label{sec:cyclotomic_extensions}

Consider $ x^n - 1 $ over $ \Q $. Its roots are $ n $th roots of unity $ w_n = \{\tau \in \Q\mid \tau^n = 1\} $.

\begin{definition}
    The element of
    \begin{equation}
        w_n \setminus \bigcup_{d\mid n, d<n} w_d 
    \end{equation}
    is called a \textit{primitive} root of unity of order $ n $.
\end{definition}

Let's define
\begin{equation}
    \Phi_n(x) = \prod_{\tau^n = 1, \tau \text{ is primitive with order } n} (x - \tau) \in \Q[x]
\end{equation}
\begin{claim}
    $ \Phi \in \Z[x] $.
\end{claim}
\begin{proof}
    \begin{align}
        x^n - 1 &= \prod_{d\mid n} \Phi_d\\
                &= \Phi_n \prod_{d\mid n, d<n} \Phi_d
    \end{align}
    By induction on $ n $, we take the base case where $ \Phi_1 = x-1 $. The equation we just showed tells us that $ x^n - 1 = \Phi_n f $ where $ f \in \Z[x] $ by the division algorithm. $ \Phi_n \in \Q[x] $, so by Gauss's lemma, $ \text{content}(x^n - 1) = 1 $ and $ \text{content}(f) = 1 $ because $ x^n - 1 $ and $ f $ are monic. Therefore, $ \text{content}(\Phi_n) = 1 $.
\end{proof}

Note that $ \deg \Phi_n $ is the number of the elements of the set $ \#\{m \leq n \mid \gcd(m ,n) = 1\} = \varphi(n) $ (Euler's totient function).

\begin{theorem}
    $ \Phi_n $ is irreducible over $ \Q $. 
\end{theorem}
\begin{proof}
    By Gauss's lemma, $ \Phi_n $ is irreducible over $ \Z $. Suppose it is reducible:
    \begin{equation}
        \Phi_n(x) = f(x) g(x)
    \end{equation}
    \begin{equation}
        \tau \in w_n \setminus \bigcup_{d\mid n, d<n} w_d
    \end{equation}
    Let $ p $ be any prime not dividing $ n $ and consider $ \tau^p $. Since $ p \nmid n $, $ \tau^p $ is also a primitive root of order $ n $, so $ \Phi_n(\tau^p) = 0 $.

    There are two cases. First, $ g(\tau^p) = 0 $, so $ \tau $ is a root of $ g(x^p) $. Therefore, $ f(x) \mid g(x^p) $. Say $ g(x^p) = f(x) h(x) $.

    Let $ \bar{f} $, $ \bar{g} $, $ \bar{h} $ be the images of $ f,g,h \in \Z[x] $ in $ \Z/p\Z[x] $ under the projection map.

    Then $ \bar{g}(x^p) = \bar{f}(x) \bar{h}(x) $, so $ g(x)^p = \bar{f}(x) \bar{h}(x) $. Hence, the right-hand side is not separable.

    Then, $ \gcd(\bar{g}, \bar{f}) \neq 1 $, hence $ \bar{\Phi}_n = \bar{f} \bar{g} $ is non-separable.

    On the other hand, $ \Phi_n \mid x^n - 1 $ in $ \Z[x] $ and so in $ \Z/p\Z[x] $, then $ \bar{\Phi}_n $ non-separable implies $ x^n - 1 $ is non-separable. However, it's easy to check that $ x^n - 1 $ is separable, since $ \D (x^n - 1) = nx^{n-1} $ and $ \gcd(n x^{n-1} ,x^n - 1) = 1 $ if $ p \nmid n $, so we have a contradiction. Therefore, the first case leads to a contradiction.

    The second case is where $ f(\tau^p) = 0 $ wherever $ \tau $ is a prime of order $ n $ and $ (p ,n) = 1 $.

    Then $ \tau^{p_1 p_2 \cdots p_l} $ is a root of $ f $ whenever primes $ p_1, \cdots, p_l $ do not divide $ n, \cdots, l $. Then $ f(\tau^m) = 0 $ $ \forall m $ such that $ (m,n) = 1 $, so $ f $ vanishes on all primitive roots of order $ n $. Therefore $ \Phi_n \mid f \implies \Phi_n = f $ is irreducible.
\end{proof}

\end{document}
