\documentclass[a4paper,twoside,master.tex]{subfiles}
\begin{document}
\lecture{32}{Wednesday, November 18, 2020}{Primitive Element Theorem}

Recall that a finite extension $ F/K $ is Galois if $ F/K $ is normal and separable. Recall an extension is separable if a minimal polynomial of every $ \alpha \in F $ is separable. Finite extensions are separable if they are generated by some roots of a separable polynomial.

Take $ F/K $ Galois and finite.
\begin{claim}
    If $ G = \text{Gal}(F/K) $ then $ \text{Fix}(G) = K $.
\end{claim}
\begin{proof}
    $ \text{Fix}(G) \supseteq K $. Conversely, if $ \alpha \in \text{Fix}(G) $, then if $ \alpha \notin K $, we can take any $ F(\alpha) $ and $ \beta $ where $ \beta $ is another root of the same irreducible polynomial as $\alpha$. Then $ \exists \sigma \colon F(\alpha) \to F(\beta) $, and we can extend $ \sigma $ to an automorphism of $ F $ so $ \alpha \notin \text{Fix}(G) $, a contradiction.
\end{proof}

\begin{claim}
    If $ F/E $ is an intermediate extension of Galois $ F/K $ ($ E/K $ is an extension), then $ F/E $ is Galois and the map $ E \mapsto \text{Gal}(F/E) $ is injective as a map.
\end{claim}
\begin{proof}
    Since $ F/K $ is Galois, $ F/K $ is normal, so $ F/E $ is normal. Next, $ F/K $ is separable and finite, so $ F $ is generated by separable polynomials over $ K $, so $ F $ is generated by the elements of $ E $, so $ F/E $ is separable.
\end{proof}

Suppose $ H = \text{Gal}(F/E) $ and $ H' = \text{Gal}(F'/E) $. If $ H=H' $, then $ F = \text{Fix}(H) $ and $ F' = \text{Fix}(H') $, so $ F = F' $.

\begin{corollary}
    If $ F/K $ is finite and Galois, there are only finitely many intermediate extensions.
\end{corollary}
\begin{proof}
    $ \text{Gal}(F/K) $ is finite.
\end{proof}

\begin{theorem}[Primitive Element Theorem]
    If $ F/K $ has only finitely many intermediate extensions, then $ \exists alpah \in F $ such that $ F = K(\alpha) $.
\end{theorem}
\begin{proof}
    There are a few cases depending on whether $ K $ is finite. If $ K $ is infinite, then $ K \subseteq K(\alpha_1) \subseteq K(\alpha_1, \alpha_2) \subseteq \cdots \subseteq F $.

    Therefore, we just need to prove that if $ F = K(\alpha, \beta) $,there exists $ \delta $ such that $ F = K(\delta) $.

    Consider $ K(\alpha + c \beta) $ where $ c \in K $. By the pigeonhole principle, $ \exists c,c' \in K $ distinct such that $ K(\alpha + c \beta) = K(\alpha + c' \beta) = E $. Then $ \alpha + c \beta, \alpha + c' \beta \in E $, so $ (c-c') \beta \in E $. $ c-c' \neq 0 $, so $ \beta \in E $ and $ \alpha \in E $. Then $ K(\alpha, \beta) \subseteq E $, so $ K(\alpha, \beta) = K(\gamma) $ where $ \gamma = \alpha + c \beta $.
\end{proof}

\begin{lemma}
    If $ F/K $ is Galois and $ [K(\alpha) \colon K] \leq n $, $ \forall \alpha \in F $, then $ [F\colon K] \leq n $.
\end{lemma}
\begin{proof}
    $ F = K(\alpha) $ by the primitive element theorem.
\end{proof}
\begin{theorem}
    $ K $ is a field and $ G $ is a finite subgroup of $ \text{Aut}(F) $, then $ K = \text{Fix}(G) $, the extension $ F/K $ is Galois, and $ \text{Gal}(F/K) = G $.
\end{theorem}
\begin{proof}
    Pick $ \alpha \in F $. $ G_{\alpha} $ is the orbit of $ \alpha $ under $ G $. $ G_{\alpha} = \{\sigma_1 \alpha, \sigma_2 \alpha, \cdots, \sigma_n \alpha\} $. Note $ G $ acts by permutations on $ G_{\alpha} $. Consider $ f(x) = \prod_{i=1}^{n} (x - \sigma_i \alpha) $. Note that $ \tau \in G $ implies $ \tau f(x) = \prod_{i=1}^{n}(x - \tau \sigma_i \alpha) = f(x) $. $ f $ is fixed by $ \tau $ so every coefficient of $ f $ is fixed by $ \tau $. This means $ f(x) \in K[x] $.

    $ f $ is separable, so $ F/K $ is separable. $ \deg f \leq \abs{G} $, so by the lemma, $ [F\colon K] \leq \abs{G} $. On the other hand, $ F/K $ is normal because all roots of $ f $ are in $ F $. $ \text{Gal}(F/K) \supseteq G $, and we know $ [F\colon K] \geq \abs{\text{Aut}(F/K)} \geq \abs{G} $, so $ [F\colon K] = \abs{G} $ and $ G = \text{Gal}(F/K) $.
\end{proof}
\begin{corollary}
    If $ F/K $ is Galois, the map of sets of intermediate extensions $ F/E $, $ E/K $ to $ \text{Gal}(F/E) $ is surjective.
\end{corollary}


\end{document}
