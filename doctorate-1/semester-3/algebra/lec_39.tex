\documentclass[a4paper,twoside,master.tex]{subfiles}
\begin{document}
\lecture{39}{Wednesday, December 09, 2020}{Intersections}


Recall that a variety is a set $ V(I) = \{c \in k^n \colon f(c) = 0,\quad V_f \in I\} $ where $ I \subset k[x_1, \cdots, x_n] $ 
\begin{claim}
    If $ U $ and $ W $ are varieties, then so are $ U \cup W $ and $ U \cap W $.
\end{claim}
\begin{proof}
    Suppose $ U = V(I) $ and $ W = V(J) $. Then
    \begin{equation}
        U \cap W = V(I + J)
    \end{equation}
    where $ I + J $ is the ideal generated by $ I \cup J $.
    \begin{equation}
        U \cup W = V(IJ)
    \end{equation}
    This is a bit less obvious. First note that $ U \subseteq V(IJ) $ because $ IJ \subseteq I $, and similar for $ W $, so $ U \cup W \subseteq V(IJ) $. If $ c \not\in U \cup W $, then $ \exists f \in I $ with $ f(c) \neq 0 $ and $ \exists g \in G $ such that $ g(c) \neq 0 $. Then $ fg \in IJ $ implies $ (fg)(c) \neq 0 $ so $ c \not\in V(IJ) $.
\end{proof}

\begin{definition}
    A variety $ U $ is irreducible if whenever $ U = U_1 \cup U_2 $, then either $ U = U_1 $ or $ U = U_2 $.
\end{definition}

\begin{theorem}
    Every variety can be written as a union of finitely many irreducible varieties.
\end{theorem}
\begin{lemma}
    If $ V_1 \subseteq V_2 \subseteq V_3 \subseteq \cdots $ is an infinite chain of varieties, then $ \exists n $ such that $ V_n = V_{n+1} = \cdots $.
\end{lemma}
\begin{proof}
    Recall than $ (I(V_i)) = V_i $, so $ I(V_1) \subseteq I(V_2) \subseteq \cdots $. Since $ k[X] $ is Noetherian, any sequence of ideals must stabilize, which implies that the sequence of varieties must stabilize.
\end{proof}
We can now prove the theorem:
\begin{proof}
    Say $ V_1 $ is not a union of finitely many irreducibles. $ V_1 $ is therefore not irreducible, so $ V_1 = V_2 \cup V_2' $ and $ V_1 \neq V_2 $ and $ V_1 \neq V_2' $. Since $ V_1 $ is ``bad'' then at least one of these $ V_2 $ or $ V_2' $ must be ``bad'' as well, because if it wasn't, then we could write them both as finite unions of irreducibles. Without loss of generality, suppose $ V_2 $ is ``bad''. Write $ V_2 = V_3 \cup V_3' $. This process must continue, so
    \begin{equation}
        V_1 \supsetneq V_2 \supsetneq V_3 \supsetneq \cdots
    \end{equation}
    Because of the lemma we just proved, this is a contradiction.
\end{proof}

\begin{note}{Remark}
    $ V(f) $ is irreducible iff $ f $ is irreducible.
\end{note}

\begin{theorem}[Bezout's Theorem in the Plane]
    The number of intersections of $ f = 0 $ and $ g = 0 $ is equal to $ (\text{deg}(f)) \times (\text{deg}(g)) $ if $ \gcd(f, g) = 1 $.
\end{theorem}
Before we prove this, let's do an example to convince ourselves it's even right.
\begin{ex}
    Suppose
    \begin{equation}
        f = (x - a_1)(x - a_2)\cdots(x - a_n)
    \end{equation}
    and
    \begin{equation}
        g = (y - b_1)(y - b_2)\cdots(y - b_m)
    \end{equation}
    If you look at the points $ (a_i, b_i) $ which make these functions zero, there are clearly no more than $ n \times m $ of them. However, there is the possibility that one of the roots has multiplicity greater than one.
\end{ex}

The other problem is if the curves $ f $ and $ g $ are in the same variable, like $ f = x $ and $ g = x - 1 $.

\section{Projective Geometry}\label{sec:projective_geometry}

Consider $ \R^2 \subset \R^3 $. If we take a plane $ z=1 $ in $ \R^3 $, we can imagine a $ 1 $-to-$ 1 $ correspondence between points in the plane and lines through the origin meeting the plane. This correspondence ensures that a line in the $ z=1 $ plane corresponds to a plane meeting $ z=1 $ in a line. Suppose there are now two parallel lines in the $ z=1 $ plane, $ l_1, l_2 $. Then $ l_1 \leftrightarrow \pi_1 $ and $ l_2 \leftrightarrow \pi_2 $ where $ \pi_1 $ and $ \pi_2 $ are planes in $ \R^3 $. $ \pi_1 \cap \pi_2 $ is a line parallel to the $ z=1 $ plane.

We denote the real projective plane by $ \R\mathbb{P}^2 $ or $ \mathbb{P}^2(\R) $. Points of this set are lines through the origin in $ \R^3 $ and lines in this set are planes through the origin in $ \R^3 $.

A point in $ \mathbb{P}^2(\R) $ is an equivalence of triples $ (x,y,z) \neq 0 $ and $ (cx, cy, cz) $. More generally, we can extend this to $ \mathbb{P}^d(\R) $ with
\begin{equation}
    (x_0, \cdots, x_d) \sim (cx_0, \cdots, cx_d)
\end{equation}

If $ f $ is a homogeneous polynomial, then $ f(cx) = c^{\text{deg}(f)} f(x) $, so $ f(cx) = 0 \iff f(x) = 0 $. 

\begin{lemma}
    Let $ k $ be a field and $ f,g \in k[x,y] $ have no common factors. Then $ \text{Res}(f,g,x) \neq 0 $.
\end{lemma}
\begin{proof}
    If $ \text{Res}(f,g,x) = 0 $, then $ f,g \in k(y)[x] $ have a common factor in $ k(y)[x] $.

    By Gauss's lemma, since $ k[y] $ is a UFD, this implies that $ f $ and $ g $ have a common factor in $ k[y][x] = k[x,y] $.
\end{proof}
\end{document}
