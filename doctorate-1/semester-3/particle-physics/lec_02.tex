\documentclass[a4paper,twoside,master.tex]{subfiles}
\begin{document}
\lecture{2}{Wednesday, September 02, 2020}{}

The photon couples only to charge, which is a linear combination of $ T_3 = \frac{1}{2} Y_W = q $. The photon will not couple to the Higgs boson.

Spontaneous symmetry breaking results in massless Goldstone bosons, but $ m_W > 80\giga\electronvolt $ and $ m_{Z^0} > 91 \giga\electronvolt $. The Higgs field was zero at the beginning of the universe, but the field has a sombrero potential, so when it left the top of the potential, it entered a state where there were a finite number of Higgs bosons, giving the $ W^{\pm} $ and $ Z^0 $ their mass while $ m_{\gamma} $ remains zero.

$ W $ bosons can change quark flavors.

\section{The Strong Force}
\label{sec:the_strong_force}

The gluon (g) couples to the ``color'' charge, the QCD charge carried by quarks and gluons. The nuclear force is a residual interaction between colorless hadrons because they are made of colored objects. The Van der Waals force is to the electromagnetic force as the nuclear force is to the strong force. The nuclear force is responsible for the energy of fusion, fission, and radioactive decay.

\section{Dark Matter and Dark Energy}
\label{sec:dark_matter_and_dark_energy}

We don't know.

\section{Quarks}
\label{sec:quarks}

Quarks are never seen alone. The strong charge seems to be confined (although we can't prove it), so only colorless objects can exist. Colorless means $ rgb $ combinations or $ r \bar{r} $, $ g \bar{g} $, or $ b \bar{b} $ (the anti-colors are often called cyan ($ \bar{r} $), magenta ($ \bar{g} $), and yellow ($ \bar{b} $)).

The first ``generation'' of these particles are the up and down quarks, the electron, and electron neutrino.

\begin{quote}
    ``Who ordered that?''
    
    \textemdash Rabi, on the muon
\end{quote}

The second generation of particles includes the charm and strange quarks and the muon and its corresponding neutrino.

The third generation includes the top and bottom quarks, the tau, and the tau neutrino.

The $ Z^0 $ decay width has been precisely measured by LEP and can be explained by calculable partial widths of all known quark-antiquark, lepton-antilepton, gamma-gamma, and neutrino-antineutrino decays. Of course, the neutrinos were originally thought to be massless, but just adding the other contributions leaves some extra stuff which must be accounted for by the neutrinos.

There's no pattern in the distribution of the quark masses (probably).
\begin{note}{Pro Tip}
    Find a pattern here for a free $ \text{Nobel Prize}^{\text{TM}} $.
\end{note}

Besides the usual baryons and mesons, there are other possible hadrons with more quarks (pentaquarks and so on), as well as hybrid states (states with ``active glue'') and glueballs (states made of only gluons)\textemdash plus a sea of virtual gluons and virtual quark pairs (these states just describe the valence quarks).

There are then a ton of different particles with different quark contents (and even some with the same quark contents but different internal structure i.e. wave functions).

The Standard Model includes QCD, which works well in perturbative expansions at high energy (the tree-level diagrams have the highest contribution). Unfortunately, on the other end, predictions of hadron properties require calculations of quark interactions at low energy, where the coupling constant $ \sim 1 $. This means perturbative expansions don't work well because they don't necessarily (and usually don't at all) converge.

\end{document}
