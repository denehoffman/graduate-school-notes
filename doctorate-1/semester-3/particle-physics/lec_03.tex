\documentclass[a4paper,twoside,master.tex]{subfiles}
\begin{document}
\lecture{3}{Friday, September 04, 2020}{}

Simulations support early qualitative arguments that widely-separated quark pairs are connected by a gluonic ``flux tube''. The force between the quarks is constant and the energy stored increases apparently without end as you stretch the two quarks apart until there's enough energy to make another quark-antiquark pair out of the vacuum. If a quark is hit hard, a jet of hadrons is observed.

\section{Weak Interaction of Constituent Fermions}
\label{sec:weak_interaction_of_constituent_fermions}

$ \beta $-decay: $ n \to p + e^- + \bar{\nu}_e $ or (in nucleus) $ p \to n + e^+ + \nu_e $.

In 1956, Lee and Yang realized that parity conservation hadn't been tested for the Weak force but had been demonstrated for the Strong and EM forces. Chien-Shiung and Wu discovered that there was an observable asymmetry consistent with $ \nu_e $ being all left-handed, meaning $ e^+ $ was mostly right-handed. The same correlation was found for the antiparticles, so this first fact violated parity conservation while the second violated charge conjugation. Furthermore, it was thought that $ CP $ was conserved until Conin and Fitch showed in 1964 that $ K^0 $ decays can violate $ CP $. Later there was $ CP $ violation in $ B $ decays and in neutrinos.

$ W $-exchange flips particles within weak-isospin doublets (leptons and their neutrinos, quarks within generations).

For leptons and quarks, $ Z^0 $ exchange gives the same particle. Most importantly, there are no flavor-changing neutral currents. For leptons, the $ W $ bosons still cant change flavor, only charge.

We can construct a unitary matrix called the CKM matrix which predicts the probabilities of flavor change, but the values are empirical. The matrix really converts u, d, s states to u', d', s' states, which are linear combinations of the former. Notably $ s \to u $ is suprressed by a factor of $ \abs{V_{us}}^2 \sim 0.05 $, which explains why $ s $ quarks decay so slowly.

\section{Neutrinos}
\label{sec:neutrinos}

\begin{quote}
    ``Neutrinos have no mass so we're done''
    \textemdash Quinn
\end{quote}

Ray Davis, in the 60\textendash 80's, did some experiments to measure chlorine decays (to argon, by getting hit by neutrinos). He kept seeing less argon than was predicted. Additional experiments showed that non-electron-neutrinos coming were coming from the sun, and other flavored neutrinos had conflicting fluxes on either side of the earth. Eventually, the idea that neutrinos can oscillate was born. Based on these results, a unitary matrix can be written to describe a change of base to mass eigenstates rather than the weak flavor eigenstates of neutrinos. It can be shown that the neutrino mass-squared differences can be calculated, but measurements are not precise enough to measure the mass of an individual state.

\end{document}
