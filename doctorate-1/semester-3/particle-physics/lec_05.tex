\documentclass[a4paper,twoside,master.tex]{subfiles}
\begin{document}
\lecture{5}{Friday, September 11, 2020}{}

\subsection{Electromagnetic Cascades}
If a high energy photon, electron, positron, (or really high energy muon) enters a thick material, it may initiate an cascade of radiation (or shower).

\subsection{Cherenkov Radiation}
This is emitted if a particle travels faster than the speed of light in a material (this is perfectly allowed and predicted by even Classical EM). A handy way to remember the angle of the radiation is:
\begin{equation}
    \cos(\theta_c) = \frac{c/n}{\beta c} = \frac{1}{n \beta}
\end{equation}

Additionally, the number of photons per unit distance is given by
\begin{equation}
    \pdv{N}{E}{x} = \frac{\alpha Z_R^2}{\hbar c} \sin[2](\theta_c) \propto 1 - \frac{1}{\beta^2 n^2(E)}
\end{equation}
or
\begin{equation}
    \pdv{N}{\lambda}{x} = \frac{2 \pi \alpha Z^2_R}{\lambda^2} \left( 1 - \frac{1}{\beta^2 n^2(\lambda)} \right)
\end{equation}
Because of this dependence, Cherenkov radiation tends to be UV or blue.

\subsection{Transition Radiation}
\label{sub:transition_radiation}

Charged particles crossing a transition between materials with different indices of refraction will radiate. For very high $ \gamma $ and many layers, the emissions can be x-rays.

\subsection{Hadronic Showers}
\label{sub:hadronic_showers}

If a hadron enters a thick material, it may initiate a hadronic shower. The scale for these interactions is set by the nuclear interaction length. The probability for a hadron to travel a distance $ x $ without a nuclear interaction is
\begin{equation}
    \Pr(x) = e^{-(x/ \lambda_{\text{nuc}} )}
\end{equation}
where
\begin{equation}
    \lambda_{\text{nuc}} = \frac{1}{\rho_A \sigma_{\text{nuc}}} = \frac{A}{N_A \rho \sigma_{\text{nuc}}}
\end{equation}
where $ \rho_A $ is the atomic density.

\section{Experiment Triggers}
\label{sec:experiment_triggers}

Experiments require triggers to select events of interest from the huge amount of interactions.
\begin{itemize}
    \item First-level triggers use fast detectors, hardware discriminators, and fast hardware logic
    \item Second-level triggers may add information from slower devices and use digitized results from fast-readout electronics to make more sophisticated decisions
    \item Higher-level triggers that decide what data gets written to the disk
\end{itemize}

\section{Detectors}
\label{sec:detectors}

\begin{itemize}
    \item[(A)] Scintillators collect ionization energy and convert it into light through some intrinsic property of a material or a material mixed with fluorescent particles.
        \subitem Light collection usually uses total internal reflection to lead light into a photo-detector.
\end{itemize}


\end{document}
