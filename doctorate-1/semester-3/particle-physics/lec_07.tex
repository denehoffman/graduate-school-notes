\documentclass[a4paper,twoside,master.tex]{subfiles}
\begin{document}
\lecture{7}{Wednesday, September 16, 2020}{The Effect of a Magnetic Field}

A particle of charge $ q $ and momentum $ \va{p} $ in a uniform magnetic field $ \va{B} $ moves in an arc of radius $ R = \abs{\frac{p_{\perp}}{Bq}} $.

Magnetic ``optics'' are possible because of magnetic quadrupoles. A doublet of these quadrupoles can be designed to focus a beam in both directions.

\end{document}
