\documentclass[a4paper,twoside,master.tex]{subfiles}
\begin{document}
\lecture{20}{Wednesday, October 21, 2020}{Nuclear Physics, cont.}

From our previous lecture, we said that the mass of the nucleus could be written
\begin{equation}
    M_n = N m_n + Z m_p - B
\end{equation}
This is based on the ``liquid drop model'', where we treat nuclear matter as an incompressible liquid ($ R \propto A^{1/3} $).

Strong (nuclear) interactions are short-ranged, so they saturate: $ B_1 = a_v A $. However, nucleons on the surface are not as tightly bound (since they are not surrounded by nucleons): $ B_2 = a_v A - a_s A^{2/3} $.

\subsection{Coulomb Repulsion}\label{sub:coulomb_repulsion}

$ B $ is reduced by repulsion of two protons from $ Z - 1 $ other protons with distance $ R \propto A^{1/3} $:
\begin{equation}
    \Delta B \propto \sum_{i=1}^{Z} \frac{e(Z-1)e}{R} \propto \frac{Z(Z-1)}{A^{1/3}}
\end{equation}
so
\begin{equation}
    B_3 = a_v A - a_s A^{2/3} - a_c \frac{Z(Z-1)}{A^{1/3}}
\end{equation}

\subsection{Asymmetry Term}\label{sub:asymmetry_term}

We expect $ N > Z $ for heavy nuclei. The advantage is reduced as $ N $ grows by some asymmetry term (otherwise $ \nuclide[238]{H} $ would be common). Treating the nucleus as two degenerate Fermi gasses (one of neutrons and another of protons), the Pauli exclusion principle says that if $ N >> Z $, extra neutrons added are at a higher energy than extra protons (the Fermi energy for the neutrons will be higher than that of the protons if there are more neutrons than protons).

The Fermi pressure for particles on a lattice is $ p = \frac{h}{L} \sqrt{n_x^2 + n_y^2 + n_z^2} $. $ Z = \frac{4}{3} \pi n^3_{\mathcal{F}_p} $ so $ n_{\mathcal{F}_p} \propto Z^{1/3} $ and $ n_{\mathcal{F}_n} \propto N^{1/3} $.

Therefore, the Fermi energy will be
\begin{equation}
    E_{\mathcal{F}_p} = \frac{p^2_{\mathcal{F}_p}}{2m} = \frac{1}{2m} \left( \frac{h}{L} \right)^2 \left( \frac{3}{4 \pi} Z \right)^{2/3}
\end{equation}
so
\begin{equation}
    E_{\text{tot}_p} = 4 \pi \int_0^{n_{\mathcal{F}_p}} E(n) n^2 \dd{n} = 4 \pi \int \frac{1}{2m} \left( \frac{h}{L} n \right) n^2 \dd{n} \propto \frac{1}{L^2} n^5_{\mathcal{F}_p} \propto \frac{Z^{5/3}}{A^{2/3}}
\end{equation}
For $ N $ neutrons and $ Z $ protons vs. $ \frac{A}{2} $ neutrons and $ \frac{A}{2} $ protons,
\begin{equation}
    \Delta E_{\text{tot}} \propto \frac{1}{A^{2/3}} N^{5/3} + \frac{1}{A^{2/3}} Z^{5/3} - 2 \frac{1}{A^{2/3}} \left( \frac{A}{2} \right)^{5/3}
\end{equation}
Let $ N = \frac{A}{2} (1 + \lambda) $ and $ Z = \frac{A}{2} (1 - \lambda) $ ($ \lambda \equiv \frac{N - Z}{A} $). Then
\begin{equation}
    \Delta E_{\text{tot}} \propto \frac{1}{A^{2/3}} \left( \frac{A}{2} \right)^{5/3} \left( (1 + \lambda)^{5/3} + (1 - \lambda)^{5/3} - 2 \right)
\end{equation}
If $ \lambda \ll 1 $,
\begin{align}
    \Delta E &\propto A \left[ \left( 1 + \frac{5}{3} \lambda + \frac{5}{3} \left( \frac{5}{3} - 1 \right) \lambda^2 \right) + \left( 1 - \frac{5}{3} \lambda + \frac{5}{3} \left( \frac{5}{3} - 1 \right) \lambda^2 \right) - 2 \right] \\
    &\propto A \lambda^2 = A \left( \frac{N - Z}{A} \right)^2 = \frac{(N - Z)^2}{A}
\end{align}
so
\begin{equation}
    B_4 = B_3 - a_{\text{sym}} \frac{(N - Z)^2}{A}
\end{equation}

Finally, beyond the liquid-drop model plus the Fermi gas model, there are pairing correlations. Nuclei are more tightly bound if $ N $ is even and $ Z $ is even. They are less bound if one of them is odd, and even less if both are:
\begin{equation}
    B_5 = a_v A - a_s A^{2/3} - a_c \frac{Z(Z-1)}{A^{1/3}} - a_{\text{sym}} \frac{(N - Z)^2}{A} - a_{\text{pairity}} \delta(N,Z)
\end{equation}
where
\begin{equation}
    \delta = \begin{cases} 1& \qif N \qodd Z \qodd \\ 0&\qif A\qood\\ -1&\qif N\qeven Z\qeven \end{cases}
\end{equation}
An empirical fit (for $ A > 20 $) yields (in $ \mega\electronvolt $):

\begin{tabular}{@{}ccccc@{}}
        \toprule
        $ a_v $ & $ a_s $ & $ a_c $ & $ a_{\text{sym}} $ & $ a_{\text{pairity}} $ \\
        \midrule 
        $ 15.76 $ & $ 17.8 $ & $ 0.711 $ & $ 23.7 $ & $ 11.18/ \sqrt{A} $ \\
        \bottomrule
\end{tabular}
This predicts the curve of binding energy roughly and also predicts the $ \beta $-stability curve and proton/neutron drip lines. The liquid drop model also qualitatively explains fusion.


\end{document}
