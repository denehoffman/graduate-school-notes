\documentclass[a4paper,twoside,master.tex]{subfiles}
\begin{document}
\lecture{21}{Monday, October 26, 2020}{Nuclear Shapes}

\section{Nuclear Shapes}\label{sec:nuclear_shapes}

Nuclei are spherical for low-$ J $ ground states. Nuclei in high-$ J $ ground states are prolate (stretched-sphere) or oblate (flattened-sphere), which modifies the Coulomb energy. The liquid drop model is also applied to high-excitation, short-lived collective excitations:
\begin{equation}
    r = \sum_{l,m} a_{l,m} Y_{lm}(\Omega)
\end{equation}
We call the first two excitations giant monopole (``Breathing''): $ J^{\pi} = 0^+ $, $ E \sim 80\mega\electronvolt / A^{1/3} $ and giant dipole: $ J^{\pi} = 1^- $ with $ E \sim 77\mega\electronvolt / A^{1/3} $. Beyond this, there are quadrupole moments with $ J^{\pi} = 2^+ $ discrete excitations around $ 1 - 2\mega\electronvolt $. 

\section{The Shell Model}\label{sec:the_shell_model}

For hydrogenic atoms, one unit of angular orbital momentum $ l $ has the same energy as one unit of radial excitation, which leads to the approximate degeneracy of $ 2s $ and $ 1p $ states in a non-perturbative model. Compare this to a 3D harmonic oscillator, where two units of $ l $ requires the same energy as one unit of radial excitation. The actual potential due to the distribution of nucleons is more like a ``rounded'' square well. Nucleons move like ``free'' particles in this potential because they can't scatter. The Pauli principle blocks changes in wave function. However, nucleon scattering experiments reveal strong $ \va{L} \cdot \va{S} $ coupling. This is not due to magnetic effects, but rather due to coupling to exchanged mesons. This splits each $ l $-level into two $ j $-levels with $ j = l\pm \frac{1}{2} $ where the larger $ j $ is pushed to lower energy. This splitting is the reason for some of the ``magic numbers'' we've been seeing with certain values of $ A $ and $ Z $, such as $ 2 $ and $ 82 $. This complicated mixing leads to what are called bands or shells of levels with gaps in between. Shells are filled for $ Z $ (or $ N $) equal to $ 2,8,20,28,50, \sim 82, \sim 126 $. These numbers correspond to the elements (in order) helium, oxygen, calcium (two isotopes), nickel (two isotopes), and lead (82 protons and 126 neutrons). The further you move away from these magic numbers, the more difficult it is to predict properties about energy levels.

Single-particle excitations of highest shell-model nucleons explain many observed states, such as excitations of one particle beyond a magic number or two-body couplings of excited nucleon and unexcited nucleon states. These single particle excitations de-excite by $ \gamma $-emission when possible or by electron conversion ($\epsilon$-rays) if no radiative transition is possible.

\end{document}
