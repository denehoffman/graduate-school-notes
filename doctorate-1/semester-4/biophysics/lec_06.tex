\documentclass[a4paper,twoside,master.tex]{subfiles}
\begin{document}
\lecture{6}{Friday, February 19, 2021}{Two-State Biological Systems II}

\begin{ex}
    Phosphorylation can be described as a two-state system as proteins without a phosphate group can be thought of as being in an ``off'' state and proteins with phosphate are ``on''. This phosphate comes from ATP (adenosine triphosphate) which changes to ADP (adenosine diphosphate) through the process of protein kinase. A protein which is unphosphorylated exists in an ``active'' and ``inactive'' state with energies $ \epsilon $ and $ 0 $ respectively (such that the energy to be in the active state is higher). In the phosphorylated state, the free energy landscape changes such that the active energy is $ \epsilon - I_1 $ and the inactive state has a higher energy $ - I_2 $. We can then define four states, (in)active (un)phosphorylated, which we can describe using two state variables. We will say $ \sigma_P = 0,1 $ if the protein isn't/is phosphorylated. Additionally $ \sigma_S = 0,1 $ if the state is inactive/active.

    \begin{tabular}{@{}ccccc@{}}
        State & $ \sigma_P $ & $ \sigma_S $ & Energy & Weight \\
            \toprule
        Inactive Unphosphorylated & 0 & 0 & 0 & 1\\
        Active Unphosphorylated & 0 & 1 & $ \epsilon $ & $ e^{- \beta \epsilon} $ \\
        Inactive Phosphorylated & 1 & 0 & $ - I_2 $ & $ e^{\beta I_2} $ \\
        Active Phosphorylated & 1 & 1 & $ \epsilon - I_1 $ & $ e^{- \beta (\epsilon - I_1)} $ \\
            \bottomrule
    \end{tabular}

    We can describe the energy as
    \begin{equation}
        G(\sigma_P, \sigma_S) = \epsilon \sigma_S - I_2 \sigma_P + (I_2 - I_1) \sigma_S \sigma_P
    \end{equation}
    Then
    \begin{equation}
        p_{\text{active}} = \frac{e^{- \beta G(0,1)}}{\sum_{\sigma_S} e^{- \beta G(0, \sigma_S)}} = \frac{e^{- \beta \epsilon}}{1 + e^{- \beta \epsilon}}
    \end{equation}
    We can also ask the probability of being in the active state when phosphorylated:
    \begin{equation}
        p^*_{\text{active}} = \frac{e^{- \beta G(1,1)}}{\sum_{\sigma_S} e^{- \beta G(1, \sigma_S)}} = \frac{e^{- \beta (\epsilon - I_1)}}{e^{- \beta I_2} + e^{- \beta (\epsilon - I_1)}}
    \end{equation}
    Then
    \begin{equation}
        \frac{p^*_{\text{active}}}{p_{\text{active}}} = \frac{1 + e^{\beta \epsilon}}{1 + e^{\beta (\epsilon + I_2 - I_1)}}
    \end{equation}
    If $ \epsilon \sim 5 k_B T $ and $ I_2 - I_1 \sim - 10 k_B T $, then
    \begin{equation}
        \frac{p^*_{\text{active}}}{p_{\text{active}}} \sim 150
    \end{equation}
    For enzymes found in cells,
    \begin{equation}
        \frac{p^*_{\text{active}}}{p_{\text{active}}} \sim 2\text{ to }1000
    \end{equation}
\end{ex}


\begin{ex}
    Haemoglobin: A case study in cooperative binding

    Haemoglobin is a receptor molecule which has four binding sites for oxygen molecules. consider four sites described by $ \sigma_i $ where $ i = 1,2,3,4 $ and $ \sigma_i = 0 $ for unbound and $ 1 $ for bound.
    
    These binding sites are arranged in a square, and the entire state can be described by a string of numbers, like $ \{0,0,0,0\} $ or $ \{0,1,1,0\} $. There are $ 2^4 = 16 $ states in this configuration, but not all are unique. We can imagine that the nearby binding energies are different than the diagonal binding energies. Consider a simpler model which we will call ``dimoglobin''; haemoglobin but with only two states.

    If one of the sites is occupied, it will have some binding energy $ \epsilon $. If both are occupied, there will also be some interaction energy $ J $. In terms of state variables, we can write
    \begin{equation}
        E(\sigma_1, \sigma_2) = \epsilon(\sigma_1 + \sigma_2) + J \sigma_1 \sigma_2
    \end{equation}
    Additionally, $ J < 0 $ because we want cooperative binding, where the energy to bind more oxygen is lower with each additional bound site. We can now construct the partition function:
    \begin{equation}
        Z = 1 + 2e^{- \beta (\epsilon - \mu)} + e^{- \beta (2 \epsilon + J - 2 \mu)}
    \end{equation}
    where $ \mu $ is the chemical potential for adding an oxygen molecule. We would now like to calculate the average occupancy, $ \ev{N} $:
    \begin{equation}
        \ev{N} = \frac{1}{Z} \left[ 0 + 1 e^{- \beta (\epsilon - \mu) + 1 e^{- \beta (\epsilon - \mu)}} + 2 e^{- \beta (2 \epsilon + J - 2 \mu)} \right]
    \end{equation}
    Additionally,
    \begin{equation}
        \mu = \mu_0 + k_B T \ln(c/c_0)
    \end{equation}
    where $ c $ is the concentration of oxygen, so
    \begin{equation}
        \ev{N} = \frac{2(c/c_0)e^{- \beta \Delta \epsilon} + 2(c/c_0)^2 e^{- \beta (2 \Delta \epsilon + J)}}{1 + 2(c/c_0)e^{- \beta \Delta \epsilon} + 2(c/c_0)^2 e^{- \beta (2 \Delta \epsilon + J)}}
    \end{equation}
    where $ \Delta \epsilon = \epsilon - \mu_0 $.


    Now back to the haemoglobin. How many non-degenerate microstates exist? There should be $ 5 $ if we consider all the interaction energies to be the same: one state for each occupancy number. In order of increasing number occupancy, we cna write the partition function as
    \begin{equation}
        Z = 1 + 4 e^{- \beta (\epsilon - \mu)} + 6 e^{- \beta (2 \epsilon - 2 \mu + J)} + 4 e^{- \beta (3 \epsilon - 3 \mu + 3 J)} + e^{- \beta (4 \epsilon - 4 \mu + 6 J)}
    \end{equation}
    If we were to arrange this similarly to the dimoglobin model, we would see a faster phase transition to the completely occupied state with oxygen concentration in haemoglobin than with dimoglobin.
\end{ex}

\chapter{Structure of Macromolecules}
\begin{note}{Note}
    We had extra time at the end of this lecture, so we have started next week's lecture early.
\end{note}

\section{Structure of DNA}\label{sec:structure_of_dna}

We can characterize a strand of DNA by a curve in space, $ \va{r}(x,y,z) $. Each random configuration defines a microstate, and if we imagine the curve to be continuous in space, there will be an infinite number of microstates, so we can't just count them like we did before.

One way we can represent the structure is a construction called a Kuhn segment. We can approximate the DNA as small, rigid, rod-like sections. The length of those segments depends on the physical properties of the molecule or material (not the length of the material). For example, the Kuhn segment of a rigid object, like a pen, is infinite in length. Spaghetti has a smaller Kuhn segment than copper wire. For DNA, the scale is around $ 50 $ to $ 100\nano\meter $.

We can think of the structure as the path of a random walker. Each segment points in a direction independent of the previous one.

\subsection{Random Walks}\label{sub:random_walks}

In 1D, if we want to know the average displacement (the mean-square displacement or variance), we find that
\begin{equation}
    \ev{R^2} = \ev{\sum_{i=1}^{N} \sum_{j=1}^{N} x_i x_j} = \sum_{i=1}^{N} \underbrace{\ev{x_i^2}}_{a^2} + \sum_{i \neq j=1}^{N} \underbrace{\ev{x_i x_j}}_{0}
\end{equation}
so
\begin{equation}
    \ev{R^2} = N a^2
\end{equation}
or $ \sqrt{\ev{R^2}} = a \sqrt{N} $.

\end{document}
