\documentclass[a4paper,twoside,master.tex]{subfiles}
\begin{document}
\lecture{17}{Wednesday, April 14, 2021}{Molecular Motors II}

Let's continue looking at the two-state motor model. $ p_1(n,t) $ is the probability that the system is in state $ 1 $ at position $ n $ at time $ t $, as mentioned in the previous class. For simplicity, we will drop any positional dependence.

We can consider transitions from $ n \to n + 1 $ which change state, or transitions from $\ket{0} \to\ket{1} $ which don't change position. There are multiple ways for the motor to end up in state $\ket{0} $ or $\ket{1} $ for example:
\begin{center}
    \begin{tabular}{@{}ccc@{}}
        \toprule
        Position & Internal State & Rate \\
        \midrule
        $ n - 1 \to n $ & $ 1 \to 0 $ & $ k_A^+ $ \\
        $ n \to n $ & $ 1 \to 0 $ & $ k_B^- $ \\
        $ n \to n - 1 $ & $ 0 \to 1 $ & $ k_A^- $ \\
        $ n \to n $ & $ 0 \to 1 $ & $ k_B^+ $ \\
        \bottomrule
    \end{tabular}
\end{center}
so
\begin{equation}
    \dv{p_0}{t} = k_A^+ p_1 + k_B^- p_1 - k_A^- p_0 - k_B^+ p_0
\end{equation}

We can also have
\begin{center}
    \begin{tabular}{@{}ccc@{}}
        \toprule
        Position & Internal State & Rate \\
        \midrule
        $ n + 1 \to n $ & $ 0 \to 1 $ & $ k_A^- $ \\
        $ n \to n $ & $ 0 \to 1 $ & $ k_B^+ $ \\
        $ n \to n $ & $ 1 \to 0 $ & $ k_A^+ $ \\
        $ n \to n + 1 $ & $ 1 \to 0 $ & $ k_B^- $ \\
        \bottomrule
    \end{tabular}
\end{center}
which gives us
\begin{equation}
    \dv{p_1}{t} = k_A^- p_0 + k_B^+ p_0 - k_A^+ p_1 - k_B^- p_1
\end{equation}

We can now look at what happens at steady state, where $ \dv{p_{0,1}}{t} = 0 $. From this, we get
\begin{equation}
    (k_A^+ + k_B^-)p_1 = (k_A^- + k_B^+)p_0
\end{equation}
We also have the normalization condition that $ p_0 + p_1 = 1 $, so using this we can uniquely calculate
\begin{equation}
    p_0 = \frac{k_A^+ + k_B^-}{k_A^+ + k_A^- + k_B^+ + k_B^-} \qquad p_1 = \frac{k_A^- + k_B^+}{k_A^+ + k_A^- + k_B^+ + k_B^-}
\end{equation}

Consider that when the motor changes state from $\ket{0} \to\ket{1} $, it moves a physical distance $ \delta $, but when it jumps from $ n \to n + 1 $ it moves a distance $ a - \delta $. Then
\begin{equation}
    V = \delta (p_0 k_B^+ - p_1 k_B^-) + (a - \delta)(p_1 k_A^+ - p_0 k_A^-)
\end{equation}
Using $ p_0 $ and $ p_1 $ from above, we get the steady-state velocity
\begin{equation}
    \ev{V} = a \frac{k_A^+ k_B^+ - k_A^- k_B^-}{k_A^+ + k_A^- + k_B^+ + k_B^-}
\end{equation}

\section{Polymerisation as Motor Action}\label{sec:polymerisation_as_motor_action}

Growing filaments can also generate forces. How does this happen? Suppose we have a cell membrane, and immediately below it we have some actin filament which is growing. To grow, it adds monomers primarily on the barbed end (+). To attach a monomer near the cell membrane, there needs to be some amount of space between the existing actin and the membrane. That space is generated by fluctuation in the cell membrane. This process is called a polymerization ratchet (or Brownian ratchet).

The probability that a monomer attaches, $ \Delta p(\text{on}) $
\begin{equation}
    p(\text{allowed}) \times k_{\text{on}} m \Delta t
\end{equation}
where $ p(\text{allowed}) $ is the probability that the attachment is allowed ($p(x > \delta) $), $ m $ is the concentration of monomers, and $ k_{\text{on}} $ is the rate of attachment.

\begin{equation}
    p(x > \delta) = \int_{\delta}^{\infty} p(x) \dd{x}
\end{equation}
where
\begin{equation}
    p(x) = \beta F e^{- \beta F x}
\end{equation}
is the probability that the gap opens a distance $ x $. 

We can then consider the speed at which the monomer is moving as
\begin{equation}
    v = \delta [k_{\text{on}} m e^{- \beta F \delta} - k_{\text{off}}]
\end{equation}
In equilibrium, $ v = 0 $ and $ k_{\text{on}} m e^{- \beta F \delta} = k_{\text{off}} $. Then we can see that
\begin{equation}
    m = \left( \frac{k_{\text{off}}}{k_{\text{on}}} \right) e^{\beta F \delta}
\end{equation}
if we consider $ m^* = \frac{k_{\text{off}}}{k_{\text{on}}} $, then
\begin{equation}
    m(F) = m^* e^{\beta F \delta}
\end{equation}
or
\begin{equation}
    F = \frac{1}{\beta \delta} \ln(\frac{m(F)}{m^*})
\end{equation}

The presence of force makes the transition less favorable by an amount $ F \delta $, so the difference in free energy would be like $ \Delta G = \Delta G_0 - F \delta $. 

We can find $ m \sim 20 \micro\molar $ and $ m^* \sim 0.2 \micro\molar $ with $ \delta \sim 2.7\nano\meter $, and this makes $ F \sim 7 \pico\newton $ for F-actin. For microtubules, $ m = 100m^* $ and $ \delta \sim 0.6\nano\meter $, so $ F \sim 30\pico\newton $. Is this enough force to buckle the microtubule? The buckling force depends on the length of the polymer, and by taking these forces above and knowing the persistence length, we can estimate the critical length of the polymers which would allow this polymerization process to cause buckling:
\begin{equation}
    F_{\text{crit}} = 12 \frac{\xi_p}{\beta L_c}
\end{equation}
For F-actin, $ L_c \sim 0.1 \micro\meter $ and for microtubules, $ L_c \sim 0.6\micro\meter $. For lengths larger than $ L_c $, the polymer will buckle upon adding new monomers.

\end{document}
