\documentclass[a4paper,twoside,master.tex]{subfiles}
\begin{document}
\chapter{Neural Signalling and Action Potentials}
\lecture{18}{Wednesday, April 21, 2021}{Neural Signalling and Action Potentials I}

\section{Charge State of the Cell}\label{sec:charge_state_of_the_cell}

Neural signalling works on the idea that certain ions can diffuse across the cell membrane, carrying an electric charge gradient which carries a signal down the cell. Take a cell with potassium ions on the inside and outside. If there are more ions on one side of the membrane, there is an electric gradient created by the fact that these chemicals are ions, but there is also a chemical gradient formed due to diffusion and leak channels which allow for certain ions to move in only one direction. In cells, sodium ions are pumped into the cell and potassium ions are pumped out. In addition to the leak channels, there is a sodium-potassium pump which uses ATP hydrolysis to pump sodium out and potassium in, restoring equilibrium. When the cell is at equilibrium, $ V_{\text{mem}} \sim -70 \milli\volt $, the leak channels are closed.

Now let's discuss how potentials are generated by the flow of chemical species. Let's take a system with two subsystems, $ 1 $ and $ 2 $, connected by a channel. There is some charge $ c_{1/2} $ and some potential $ V_{1/2} $ in each subsystem. We want to figure out the equilibrium. The probability that an ion is in one of the subsystems is
\begin{equation}
    p_1 = \frac{1}{Z} e^{- \beta q V_1} \qquad p_2 = \frac{1}{Z} e^{- \beta q V_2}
\end{equation}
where $ q = ze $ is the total charge. Then the charge concentration in each subsystem will be
\begin{equation}
    c_1 = \frac{N p_1}{\mathcal{V}} \qquad c_2 = \frac{N p_2}{\mathcal{V}}
\end{equation}
and the ratio of concentrations is
\begin{equation}
    \frac{c_1}{c_2} = \frac{p_1}{p_2} = e^{-ze(V_1 - V_2)/k_B T}
\end{equation}
or
\begin{equation}
    V_1 - V_2 = \frac{k_B T}{ze} \ln(\frac{c_2}{c_1})
\end{equation}
or
\begin{equation}
    V_1 - V_2 = \Delta V = \frac{k_B T}{Q} \ln(\frac{c_1}{c_2}) \tag{Nernst Potential}
\end{equation}

There are regular fluctuations around the $ -70\milli\volt $ equilibrium due to leak channels, but once the fluctuations pass $ -55\milli\volt $, voltage-gated (sodium) channels are opened. From here, sodium ions (positive) will leak into the cell and the cell will become positively charged.

\section{Two-State Model of Voltage Gating}\label{sec:two-state_model_of_voltage_gating}

We can imagine the voltage-gated channels as a spring. In the open state, we can say that the spring is in a relaxed state, and in the closed state, the spring is compressed. Then the closed state has a higher potential because it takes energy to compress the spring. For a two-state system, we know that the probability for the open state is given by
\begin{equation}
    p_{\text{open}} = \frac{e^{- \beta \Delta \epsilon}}{1 + e^{- \beta \Delta \epsilon}}
\end{equation}
In this case, $ \Delta \epsilon = \epsilon_{\text{open}} - \epsilon_{\text{closed}} $. In the open state, the spring is relaxed, so we can say
\begin{equation}
    \epsilon_{\text{open}} = E_{\text{relaxed}} - Q V_{\text{membrane}}
\end{equation}
where $ V_{\text{membrane}} $ is the potential difference across the membrane. It comes with a negative sign because if the difference is lower, it will favor an open channel. Similarly, we can say
\begin{equation}
    \epsilon_{\text{closed}} = \left( E_{\text{relaxed}} + \frac{1}{2} k x^2 \right) - Q (1-f) V_{\text{membrane}}
\end{equation}
Then
\begin{align}
    \Delta \epsilon &= - \frac{1}{2} k x^2 - Q f V_{\text{membrane}} \\
                    &= \Delta \epsilon_{\text{conf}} - Q f V_{\text{mem}}
\end{align}
We can then say
\begin{align}
    p_{\text{open}} &= \frac{1}{1 + e^{\beta (\Delta \epsilon_{\text{conf}} - Q f V_{\text{mem}})}} \\
                    &= \frac{1}{1 + e^{a(V^* - V_{\text{mem}})}}
\end{align}
where $ a = \beta Qf $ and $ V^* = \frac{\Delta \epsilon_{\text{conf}}}{Qf} $. $ p_{\text{open}} \sim 0 $ for low voltages, but at $ V \sim V^* $, $ p_{\text{open}} \to 1 $ for values above $ V^* $ very quickly.


This model only discusses the opening probability, but doesn't mention the transition back into the closed state. In reality, there is another threshold at $ +30\milli\volt $ which opens potassium channels that release potassium ions from the cell, causing the potential to decrease again (depolarization to repolarization).


\section{Current Across Membrane}\label{sec:current_across_membrane}

We can figure out the total difference in chemical potential across the membrane:
\begin{equation}
    \Delta \mu = \left( \mu_0 + k_B T \ln(\frac{c_{\text{in}}}{c_0} + z e V_{\text{in}}) \right) - \left( \mu_0 + k_B T \ln(\frac{c_{\text{out}}}{c_0} + z e V_{\text{out}}) \right)
\end{equation}

Then $ I = g \Delta \mu / ze $ gives us $ I = g(V_{\text{mem}} - V_{\text{Nernst}}) $ where
\begin{equation}
    V_{\text{Nernst}} = - \frac{k_B Tt}{ze} \ln(\frac{c_{\text{in}}}{c_{\text{out}}})
\end{equation}
and
\begin{equation}
    V_{\text{mem}} = V_{\text{in}} - V_{\text{out}}
\end{equation}

We can measure the actual current $ I $, and experimentally, we see that $ g $ is likely some nonlinear function of $ V_{\text{mem}} $ for the sodium voltage and a mostly linear function of $ V_{\text{mem}} $ for the potassium voltage.

We want to figure out what causes the nonlinearity for sodium channels. We actually know that it is due to the opening of channels, so we can suppose that the total function is composed out of the summation of many channels with a probability distribution:
\begin{equation}
    g_{\text{Na}} = N g_1 p_{\text{open}} = N g_1 \frac{1}{1 + e^{a(V^* - V_{\text{mem}})}}
\end{equation}
$ g $ now has a sigmoid-like shape as a function of $ V_{\text{mem}} $.

If we imagine $ g $ is constant with $ V_{\text{mem}} $, then by Ohm's law, $ I = g (V_{\text{mem}} - V_{\text{Nernst}}) $ gives a linear function with an zero at $ V_{\text{Nernst}} $.

If $ g $ is a step function which switches at $ V^* $, then we have two lines with two different slopes that are disconnected around $ V^* $. When we approach the sigmoid function, we get something similar to the nonlinear response function in the current.

\end{document}
