\documentclass[a4paper,twoside,master.tex]{subfiles}
\begin{document}
\lecture{17}{Tuesday, March 23, 2021}{Scattering Cross Sections}

Suppose a beam of incident particles has a density $ n_a $ particles per unit volume and are incident on a target with relative velocity $ v $. The the number of particles incident upon the target in a time $ \Delta t $ is
\begin{equation}
    \Delta N_a = n_a v \Delta t \Delta A
\end{equation}
where $ v \Delta t \Delta A $ is the volume of a cylinder of area $ \Delta A $ and height $ v \Delta t $. Then we can define the incident flux of particles as
\begin{equation}
    \mathcal{F} \equiv n_a v = \frac{\Delta N_a}{\Delta A \Delta t}
\end{equation}
The total number of reactions $ a + b \to c + d $ per unit time is defined as
\begin{equation}
    \frac{\Delta N_{ab \to cd}}{\Delta t} = \underbrace{\mathcal{F}}_{n_a v} \times \sigma
\end{equation}
where $ \sigma $ is the total cross section. We call this the reaction rate, i.e. $ \Gamma_{ab \to cd} \equiv \dv{N_{ab \to cd}}{t} \equiv n_a v \sigma $.

We can think of $ \sigma $ as the effective cross-sectional area that the target presents to the incident beam (note that it has dimensions of area).

In a statistical ensemble of particles, we find that
\begin{equation}
    \Gamma_{ab \to cd} \equiv \ev{\sigma \abs{\va{v}}} n_a
\end{equation}

\subsection{Examples of Cross Sections}\label{sub:examples_of_cross_sections}

Let's examine this in the case of Thompson scattering, $ e \gamma \to e \gamma $. With $ P_i^{\mu} \equiv p_i^{\mu} + k_i^{\mu} $, then we define
\begin{equation}
    v = \frac{P_i^{\mu} P_{i \mu} - m_e^2}{P^{\mu}_i p_{i \mu} + m_e^2}
\end{equation}
so
\begin{equation}
    \sigma \equiv \frac{\pi \alpha^2}{m_e^2 v} (1 - v)\left[ \frac{4 v}{1 + v} + (v^2 + 2v - 2) \ln(\frac{1 + v}{1 - v}) - 2 v^3 \frac{(1 + 2v)}{(1 + v^2)} \right]\tag{Klein-Nishina Cross Section}
\end{equation}

For small photon energy $ \abs{k_i} \ll m_e $, $ P_i^{\mu} \sim p_i^{\mu} \implies v \to 0 $, so we get the Thompson scattering limit
\begin{equation}
    \sigma_{\text{Th}} \equiv \frac{8 \pi \alpha^2}{3 m_e^2} = \frac{8 \pi}{3} r_0^2 = 6.65 \times 10^{-25} \centi\meter^2
\end{equation}
where $ r_0^2 = \frac{\alpha^2}{m_e^2} $ is the ``classical radius of the electron''.

In the high energy limit with $ \abs{k_i} \gg m_e $, $ P_i^{\mu} P_{i \mu} \gg m_e^2 \implies v \to 1 $, so
\begin{equation}
    \sigma \to \frac{\alpha^2}{P_i^{\mu} P_{i \mu}} \underbrace{\to \sim\frac{\alpha^2}{m_e^2}}_{\text{low energy limit}} \qor \underbrace{\to \sim \frac{\alpha^2}{P^2}}_{\text{high energy limit}}
\end{equation}

The moral of the story is that when the typical energy is much greater than the mass of the intermediate state,
\begin{equation}
    \sigma \sim \frac{\alpha^2}{(\text{typical energy})^2}
\end{equation}
and when $ E \ll m $, we get
\begin{equation}
    \sigma \sim \frac{\alpha^2}{m^2}
\end{equation}
so $ \sigma $ has dimension of $ [L]^2 \sim \frac{1}{[E]^2} $. 


We can extend this to other processes. In general, we look at the Feynman diagram for the process, count the powers of $ \alpha $ in the transition probability, which is the absolute square of the transition amplitude, and look at the intermediate state particle and use $ \frac{1}{E^2} $ or $ \frac{1}{m^2} $ depending on the energy regime. For weak interactions at low energies, Fermi theory tells us the exchange of a weak boson has an interaction strength like $ g^2 / M_{W,Z}^2 = G_F $, so the probability is proportional to $ G_F^2 \sim \frac{10^{-10}}{(\giga\electronvolt)^4} $. We then need two powers of energy in the numerator to get a cross section, so
\begin{equation}
    \sigma_{WI} \sim G_F^2 E^2 \qfor E \ll M_{W,Z}
\end{equation}

\section{Mean Free Path}\label{sec:mean_free_path}

The collision rate, or number of collisions per unit time is given by
\begin{equation}
    \Gamma_{ab \to cd} = \sigma \abs{v} n_a \qquad (=\ev{\sigma \abs{v}}n_a)
\end{equation}
In a time $ \Delta t $, the total number of collisions is $ \sigma \abs{v} n_a \Delta t $. In this time, the particle travels a distance $ L = \abs{V} \Delta t $, so the average distance between collisions can be defined as
\begin{equation}
    \lambda \equiv \frac{L}{\text{total number of collisions}} = \frac{\abs{v} \Delta t}{\sigma \abs{v} n_a \Delta t} = \frac{1}{\sigma n_a}
\end{equation}

\section{Thermal Equilibrium}\label{sec:thermal_equilibrium}

In an expanding cosmology, thermal equilibrium is established if the average time between collisions is shorter than the expansion time scale $ 1/H $ (the collision rate is much greater than $ H $). We can think of the average time between collisions as a relaxation time $ \tau = \frac{1}{\Gamma} $ such that thermal equilibrium occurs when $ \tau \ll \frac{1}{H} $. If $ \Gamma < H $, the particle is \textit{decoupled} from other species (it does not interact in an expansion time scale) and its distribution function ``freezes out'' and no longer adjusts to thermal equilibrium.

Why might this freeze-out happen? $ \Gamma = \sigma \abs{v} n $ where $ n $ is the particle density, so if the density is diluting upon expansion either as $ \frac{1}{a^3} $ for matter or $ \frac{1}{a?} $ (he left this blank in the notes, I'm not sure what it is) for radiation, $ \Gamma \lt H $ at some point. Similarly, $ \sigma $ could depend on energy which could diminish (the weak force, for example) by the redshift of the energy. The full evolution of a distribution function is obtained through a Boltzmann equation:
\begin{equation}
    \dv{f(P_{\text{phys}}(t) ; t)}{t} \equiv \mathbb{C}[f]
\end{equation}
where $ \mathbb{C} $ is some function of $ f $ and $ P_{\text{phys}} $ is the physical momentum of a particle, $ \frac{2 \pi}{\lambda_{\text{phys}}} = \frac{p_c}{a(t)} $. We can write the total time derivative as
\begin{equation}
    \dv{f}{t} = \underbrace{\pdv{f(P_{\text{phys} (t)} ; t)}{t}}_{derivative on explicit time dependence} + \dv{P_{\text{phys}}(t)}{t} \pdv{f}{P_{\text{phys}}}
\end{equation}
with $ P_{\text{phys}} = \frac{p_c}{a(t)} $. $ p_c $ is comoving and time-independent, so $ \dv{P_{\text{phys}}}{t} = - \frac{p_c \dot{a}}{a^2} = - P_{\text{phys}}(t) H(t) $. Therefore, the Boltzmann equation in an expanding cosmology is
\begin{equation}
    \pdv{f(P_{\text{phys}} ; t)}{t} - H(t) P_{\text{phys}}(t) \pdv{f(P_{\text{phys}} ; t)}{P_{\text{phys}}} \equiv \mathbb{C}[f]
\end{equation}
where $ \mathbb{C}[f] $ is proportional to $ \Gamma $, so when $ \Gamma \ll H $, we can neglect this and set $ \mathbb{C} = 0 $. This is a collision-less ``free-streaming'' of particles.

We can do a back of the envelope estimate for reaction rates of ultra-relativistic particles with $ T \gg m $. Assume near thermal equilibrium such that
\begin{equation}
    n \sim g \int \frac{\dd[3]{k}}{(2 \pi)^3} \frac{1}{e^{k/T} \pm 1}
\end{equation}
Taking $ k/T = x $, we can estimate this to be
\begin{equation}
    n \sim \frac{g T^3}{2 \pi^2} \underbrace{\int_0^{\infty} \dd{x} \frac{x^2}{e^{x}\pm 1}}_{\order{1} \implies \text{a number}}
\end{equation}
Then $ n \sim T^3 \mathbb{C} $ where $ \mathbb{C} $ depends on $ g $ and the Bose/Fermi statistics of the particle. The typical energy of an ultra-relativistic particle (with $ k_B = \hbar = c $) is $ k \sim T $. For strong or electromagnetic interactions (or weak interactions with $ T\gg M_{W,Z} $),
\begin{equation}
    \sigma \sim \frac{\alpha^2}{E^2} \sim \frac{\alpha^2}{T^2}
\end{equation}
with $ v \sim 1 $, and
\begin{equation}
    n \sim T^3 \mathbb{C} \implies \Gamma = \sigma n v \simeq \frac{\alpha^2 T^3}{T^2} \mathbb{C} \implies \Gamma \sim \alpha^2 T \mathbb{C}
\end{equation}
and
\begin{equation}
    \lambda \sim \frac{1}{\Gamma}
\end{equation}
since $ v \sim 1 $.

For weak interactions with $ T\ll M_{W,Z} $ but $ T \gg m_{e, \mu, \nu, \ldots} $, we have $ E \sim T $, so $ \sigma \sim G_F^2 T^2 $ and $ \Gamma \sim G_F^2 T^5 \sim 10^{-10} T \left( \frac{T}{\giga\electronvolt} \right)^4 $.

\section{Equilibrium Thermodynamics}\label{sec:equilibrium_thermodynamics}

Now let's consider cases where $ \Gamma \gg H $. For a species with $ g $ internal degrees of freedom, we have (in natural units)
\begin{equation}
    n = \frac{g}{2 \pi^2} \int_0^{\infty} \dd{p} p^2 f(p)
\end{equation}
\begin{equation}
    \rho = \frac{g}{2 \pi^2} \int_0^{\infty} \dd{p} p^2 E(p) f(p)
\end{equation}
and
\begin{equation}
    P = \frac{1}{3} \frac{g}{2 \pi} \int_0^{\infty} \dd{p} p^2 \frac{p^2}{E(p)} f(p)
\end{equation}
where $ v(p) = \frac{p}{E(p)} = \dv{E(p)}{p} $ and $ f(p) = (e^{\beta (E(p) - \mu)} \pm 1)^{-1} $ with $ + $ for Fermions and $ - $ for Bosons and $ \beta = \frac{1}{T} $, $ E(p) = \sqrt{p^2 + m^2} $. It is usually convenient to change variables to $ E $ with $ p \dd{p} = E \dd{E} $ and $ p = \sqrt{E^2 - m^2} $ and rescale $ \frac{E}{T} = x $, $ \frac{m}{T} = y $, and $ \frac{\mu}{T} = \xi $. For species $ i $ (and for antiparticles, take $ \xi \to - \xi $), we have
\begin{equation}
    n_i = \frac{g_i T_i^3}{2 \pi^2} \int_{y_i}^{\infty} \dd{x} \frac{(x^2 - y_i^2)^{1/2} x}{e^{x- \xi_i} \pm 1}
\end{equation}
\begin{equation}
    \rho_i = \frac{g_i T_i^4}{2 \pi^2} \int_{y_i}^{\infty} \frac{x^2 (x^2 - y_i^2)^{1/2} \dd{x}}{e^{x - \xi_i} \pm 1}
\end{equation}
and
\begin{equation}
    P_i = \frac{1}{3} \frac{g_i T_i^4}{2 \pi^2} \int_{y_i}^{\infty} \dd{x} \frac{(x^2 - y_i^2)^{3/2}}{e^{x - \xi_i} \pm 1}
\end{equation}
In general these integrals must be performed numerically, but they simplify for ultra-relativistic particles with $ T \gg m_i \implies y_i \to 0 $ and also for the non-relativistic limit $ T \ll m_i \implies y_i \to \infty $.

For a given species, the asymmetry between particles and antiparticles (for fermions) can be written as
\begin{equation}
    n_i - \bar{n}_i = \frac{g_i T_i^3}{2 \pi^2} \int_{y_i}^{\infty} \dd{x} x (x_i - y_i)^{1/2} \left( \frac{1}{e^{x - \xi_i} + 1} - \frac{1}{e^{x + \xi_i} + 1} \right)
\end{equation}

For photons, $ g = 2 $ and there are no antiparticles, so
\begin{equation}
    n_{\gamma} = \frac{2 T_{\gamma}^3}{2 \pi^2} \underbrace{\int_0^{\infty} \frac{x^2 \dd{x}}{e^x - 1}}_{2 \zeta(3)} = \frac{2}{\pi^2} T_{\gamma}^3 \zeta(3)
\end{equation}
where $ \zeta(3) = 1.20206 $ is the Riemann zeta-function.

For ultra-relativistic fermionic species, $ y_i \to 0 $, and the particle-antiparticle asymmetry becomes
\begin{equation}
    n_i - \bar{n}_i = \frac{g_i T_i^3}{6 \pi^2} \left[ \xi_i^3 + \pi^2 \xi_i \right]
\end{equation}
and the asymmetry per photon is
\begin{equation}
    \eta_i \equiv \frac{n_i - \bar{n}_i}{n_{\gamma}} = \frac{g_i \left( \frac{T_i}{T_{\gamma}} \right)^3}{12 \zeta(3)} \left[ \xi_i^3 + \pi^2 \xi_i \right]
\end{equation}

Observations tell us that the baryon (and charged lepton) asymmetry per photon is $ \eta \simeq 10^{-9} $, so the universe has much more matter than antimatter (but is electrically neutral, more on this later).

A species $ i $ may reach thermal equilibrium with members of its own species but might not with other species $ j $, in which case the temperature $ T_i \neq T_j $. The ultra-relativistic limit is important here. $ T_i \gg m_i \implies y_i \to 0 $, for which (with $ \xi_i \to 0 $),
\begin{equation}
    \rho = g_i T_i^4 \frac{\pi^2}{30} \times \begin{cases} 1 & \text{Bose} \\ \frac{7}{8} & \text{Fermi} \end{cases}
\end{equation}
\begin{equation}
    n_i = \bar{n}_i = g_i T_i^3 \frac{\zeta(3)}{\pi^2} \times \begin{cases} 1 & \text{Bose} \\ \frac{3}{4} & \text{Fermi} \end{cases}
\end{equation}
and $ P_i = \rho_i / 3 $, $ S_i = \frac{P_i + \rho_i}{T_i} = \frac{4}{3} \frac{\rho_i}{T_i} $. In the ultra-relativistic limit, each species acts like radiation with $ \frac{P}{\rho} \equiv \frac{1}{3} $.


In the non-relativistic limit, $ \frac{m_i}{T_i} \gg 1 $, $ E_i = m_i + \frac{p_i^2}{2m} $ gives us the Maxwell-Boltzmann limit:
\begin{equation}
    \frac{1}{e^{\beta (E - \mu)} \pm 1} \to e^{- \frac{(m_i - \mu_i)}{T_i}} e^{- \frac{p_i^2}{2m_i T_i}}
\end{equation}
If we write $ \mu_i - m_i \equiv \bar{\mu}_i $ as the non-relativistic chemical potential, or the energy per particle measured from the rest energy, we have the classical result
\begin{equation}
    n_i = g_i \left( \frac{m_i T_i}{2 \pi} \right)^{3/2} e^{- \bar{\mu}_i / T_i}
\end{equation}
\begin{equation}
    \rho_i \simeq m_i n_i \qquad P_i \simeq n_i T_i \left( \text{from } P = \frac{N}{V} T \right)
\end{equation}
so $ \frac{P}{\rho} \sim \frac{T_i}{m_i}\ll 1 $, so non-relativistic species behave as matter, and we can neglect pressure. Since $ \eta \sim 10^{-9} $, let's take $ \mu = 0 $. Now accounting for all ultra-relativistic species of particles and antiparticles, we have
\begin{equation}
    \rho_R = T^4 \sum_i \left( \frac{\rho_i}{T} \right)^4 \equiv \frac{\pi^2}{30} g_{*} T^4
\end{equation}
where
\begin{equation}
    g_{*} = \sum_{\text{Bosons}} g_i \left( \frac{T_i}{T} \right)^4 + \frac{7}{8} \sum_{\text{Fermions}} g_i \left( \frac{T_i}{T} \right)^4
\end{equation}
and
\begin{equation}
    P_R = \frac{1}{3} \rho_R = \frac{\pi^2}{90} g_* T^4
\end{equation}
\begin{equation}
    S_R = \frac{4}{3} g_*^{(S)} T^3 \frac{\pi^2}{30}
\end{equation}
where
\begin{equation}
    g_{*}^{(S)} = \sum_{\text{Bosons}} g_i \left( \frac{T_i}{T} \right)^3 + \frac{7}{8} \sum_{\text{Fermions}} g_i \left( \frac{T_i}{T} \right)^3
\end{equation}
(difference is in the power of temperature).

Typically, $ T \equiv T_{\gamma} $ if all species are in thermal equilibrium at the same $ T_i \equiv T $, so we can neglect the $ T $-dependent parts of $ g $ and $ g_*^{(S)} = g_* $.

In the case of matter-radiation equality, $ \frac{\Gamma_{0,R}}{a^4} = \frac{\Gamma_{0,M}}{a^3} $ with $ \Omega_{0,R} \sim 10^{-5} $, $ \Omega_{0,M} \sim 0.25 $, we have $ z_{\text{eq}} \sim 4000 $ and $ T_{\text{eq}} = T_{\text{CMB}} \times 4000 \sim 1\electronvolt $. For $ T\gg 1\electronvolt $, the universe was \textit{radiation dominated} and
\begin{equation}
    H^2 = \frac{8 \pi G}{3} \rho_R \equiv \frac{8 \pi G}{3} g_* \left( \frac{\pi^2}{30} \right) G T^4
\end{equation}
We can introduce the Planck mass,
\begin{equation}
    M_{\text{pl}} = \frac{1}{\sqrt{G}} = 1.22 \times 10^{19} \giga\electronvolt 
\end{equation}
such that
\begin{equation}
    H(T) \simeq 1.66 g_*^{1/2} \left( \frac{T^2}{M_{pl}} \right)
\end{equation}
Since, for the radiation-dominated universe, $ a(t) \sim t^{1/2} \implies H = \frac{1}{2t} $, then
\begin{equation}
    t \sim \frac{0.3}{\sqrt{g_*}} \left( \frac{M_{pl}}{T^2} \right) \simeq 1\second \left[ \frac{T}{\mega\electronvolt} \right]^{-2} \frac{1}{\sqrt{g_*}}
\end{equation}
and
\begin{equation}
    S_R = \frac{2 \pi^2}{45} g_*^{(S)} T^3
\end{equation}
The effective degrees of freedom, $ g_* $ and $ g_*^{(S)} $ depend on the temperature even if $ T_i = T $ for all species $ i $. This is because only the ultra-relativistic degrees of freedom contribute to the expansion, namely species for which $ T \gg m_i $.

\end{document}
