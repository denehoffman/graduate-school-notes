\documentclass[a4paper,twoside,master.tex]{subfiles}
\begin{document}
\lecture{22}{Thursday, April 08, 2021}{Primordial Nucleosynthesis, Cont.}

After the deuteron is formed, the other reactions happen quickly with large neutron capture cross-sections, turning all neutrons into $ \nuclide[4]{He} $ nuclei. Since each of those has four baryons, the mass fraction is
\begin{equation}
    Y_{\text{He}} = \frac{4 (n_n / 2)}{n_n + n_p} = \frac{2 \left( \frac{n_n}{n_p} \right)}{1 + \left( \frac{n_n}{n_p} \right)} \sim \frac{2 \left( 1/7 \right)}{1 + \left( 1/7 \right)} \sim \frac{2}{8} \sim 1/4 \sim 0.25
\end{equation}

Cosmological measurements in planetary nebula, globular clusters, interstellar gas, and other sources yield $ Y_{\text{He}} \simeq 0.238\pm 0.006 $.

\subsection{Heavier Nuclei}\label{sub:heavier_nuclei}

Once the deuteron is formed, a series of nuclear reactions lead to the production of lithium:
\begin{align}
    \nuclide[4]{He} + \nuclide[3]{H} &\leftrightarrow \nuclide[7]{Li} + \gamma \\
    \nuclide[4]{He} + \nuclide[3]{He} &\leftrightarrow \nuclide[7]{Be} + \gamma \\
    \nuclide[7]{Be} + n &\leftrightarrow \nuclide[7]{Li} + p
\end{align}

Production of heavier nuclei is hindered by low density, the Coulomb barrier of heavier nuclei, and the lack of stable elements with $ A = 5, 8 $. Nucleosynthesis effectively stops at $ t \sim 3\minute $ with
\begin{align}
    Y_{\text{He}} &= \frac{\nuclide[4]{He}}{\nuclide[4]{He} + \nuclide{H}} \simeq 0.24 \\
    \frac{\nuclide{Li}}{\nuclide{H}} &\simeq 1.23 \times 10^{-10} \\
    \frac{\nuclide{D}}{\nuclide{H}} &\simeq 2.6 \times 10^{-5} \\
    \frac{\nuclide[3]{He}}{\nuclide{H}} &\sim 10^{-6} 
\end{align}

While the abundance of $ \nuclide[4]{He} $ is rather insensitive to $ \eta $, the production of $ D $ critically depends on $ \eta $ and is considered a ``baryometer''. The abundance of light elements in the primordial universe as predicted by Big-Bang-Nucleosynthesis are in remarkable agreement with observations that pinpoint $ 3 \lesssim \eta_{10} \lesssim 10 $ ($ \eta_{10} \equiv 10^{10} \eta $), leading to $ \Omega_B h^2 \simeq 0.02 $, an independent configuration of dark matter since $ \Omega_B \ll \Omega_M $. 

\section{Photon Reheating, the $\text{C} \nu \text{B}$, and Dark Radiation}\label{sec:photon_reheating,_the_cnb,_and_dark_radiation}

At the scale $ T \sim 1 \mega\electronvolt $, another important process occurs:
\begin{equation}
    e^+ e^- \to 2 \gamma
\end{equation}
as well as the inverse process. When this happens, the effective number of UR degrees of freedom changes. $ g_*(T) $ jumps when species disappear (or when they become non-relativistic) via annihilations. At $ T \gtrsim 1 \mega\electronvolt $, only $ e^+ $, $ e^- $ (2 dof each), $ \gamma $ (2 dof), and $ \nu $ (3 particles with 1 dof each) are ultrarelativistic. However, there are way more photons than baryons (and electrons) and $ \gamma \gamma \to e^+ e^- $ continues below $ 1 \mega\electronvolt $. A more detailed Boltzmann calculation shows that $ e^+ e^- \to 2 \gamma $ at $ T \leq 0.3 \mega\electronvolt $, at which the neutrinos decouple from other species. The entropy is dominated by the ultrarelativistic $ \nu $ and $ \gamma $, whereas the entropy in baryons is negligible. For NR baryons,
\begin{equation}
    S = N \left( \ln(\frac{V}{N} (mT)^{3/2}) + \frac{5}{2} \right) \tag{Sackur-Tetrode Equation}
\end{equation}
The entropy density is
\begin{equation}
    \mathcal{S} = n_B \left\{ \ln(\frac{1}{n_B} (mT)^{3/2}) + \frac{5}{2} \right\}
\end{equation}
but the entropy of an ultrarelativistic photon is
\begin{equation}
    \mathcal{S}_{\text{UR}} \propto T^3 \propto n_{\gamma}
\end{equation}
and $ n_B / n_{\gamma} \sim 10^{-10} $, so 
\begin{equation}
    \frac{\mathcal{S}_B}{\mathcal{S}_{\text{UR}}} \sim 10^{-9}
\end{equation}

At the temperature at which $ e^+ e^- \to 2 \gamma $ ($ 0.3 \mega\electronvolt $), the neutrinos are already decoupled. They were in equilibrium with all species at $ T_{\nu} = T_{\gamma} $ until $ T_D \sim 0.8 \mega\electronvolt $ and decoupled at this temperature where $ T_D = T_{\gamma} $. However, in this reaction, the entropy of the electron-positron pairs is given off to photons, since the total entropy remains constant (adiabatic expansion requires this). Since
\begin{equation}
    S = \text{const.} = \frac{4}{3} g_{\text{eff}} \frac{\pi^2}{30} T^3(t) \underbrace{V(t)}_{a^3(t) V_0}
\end{equation}
we require
\begin{equation}
    \eval{g_{\text{eff}}(T) \times (T a)^3}_{\text{before annihilation}} = \eval{g_{\text{eff}}(T) \times (T a)^3}_{\text{after annihilation}}
\end{equation}

If we assume instantaneous annihilation so that $ a_{\text{before}} = a_{\text{after}} $, and because the entropy of neutrinos is constant as they are decoupled from all other species (and cannot give or gain entropy), the only relevant entropies are those of $ e^+ $, $ e^- $, and $ \gamma $:
\begin{equation}
    \eval{g_{\text{eff}}}_{\text{before}} = 2 + \underbrace{4 \times \frac{7}{8}}_{\text{electrons + positrons}} = \frac{11}{2} \qquad \eval{g_{\text{eff}}}_{\text{after}} = 2
\end{equation}

Since the entropy is constant, we must then have
\begin{equation}
    \left( \frac{11}{2} \right) \eval{T_{\gamma}^3}_{\text{before}} = 2 \eval{T_{\gamma}^3}_{\text{after}}
\end{equation}
or
\begin{equation}
    \frac{\eval{T_{\gamma}}_{\text{after}}}{\eval{T_{\gamma}}_{\text{before}}} = \left( \frac{11}{4} \right)^{1/3}
\end{equation}

Therefore, the photon gas \textit{reheats} after electron-positron annihilation, but the neutrino gas decoupled with $ T_{\nu} = \eval{T_{\gamma}}_{\text{before}} $ and redshifted along with the photons down to when the electrons and positrons annihilate, at which point the gas reheats by the factor above, so
\begin{equation}
    \frac{T_{\nu}}{\eval{T_{\gamma}}_{\text{after}}} = \left( \frac{4}{11} \right)^{1/3}
\end{equation}

The outcome of this is that today there is a cosmic neutrino background at $ T_{\nu} \sim \left( \frac{4}{1} \right)^{1/3} T_{\gamma} = 1.95\kelvin $.

\subsection{Dark Radiation}\label{sub:dark_radiation}

After photon reheating, the total energy density in radiation and ultrarelativistic species is
\begin{equation}
    \rho_{\text{rel}} = g_* \frac{\pi^2}{30} T_{\gamma}^{4} 
\end{equation}
where $ g_* = 2 + \frac{7}{8} \times 2 \times N_{\nu} \left( \frac{T_{\nu}}{T_{\gamma}} \right)^4 $, where $ N_{\nu} $ is the number of relativistic neutrino species. During the radiation-dominated period, $ N_{\nu} $ effects the expansion history for $ T < 0.3 \mega\electronvolt $. This means $ g_{*} $ can be used to determine the number of ultrarelativistic neutrinos:
\begin{equation}
    g_* = 2 + 2 \frac{7}{8} N_{\text{eff}} \left( \frac{4}{11} \right)^{4/3}
\end{equation}
If there are only 3 species of ultrarelativistic neutrinos \textit{and} the instantaneous annihilation assumption is correct, then $ N_{\text{eff}} = 3 $. However, using a detailed Boltzmann analysis, we can show that the instantaneous annihilation assumption is in fact not exact but close. If $ N_{\nu} = 3 $, then it turns out $ N_{\text{eff}} $ as defined is $ N_{\text{eff}} = 3.046 $. There is a very small contribution to $ \nu $-reheating through $ e^+ e^- \to \nu \bar{\nu} $ via charged and neutral currents, but it is very small. From particle physics, the tighter constraint on the number of neutrinos emerges from the ``invisible'' width of the $ Z^0 $. The contributions from electron, muon, and quark pair productions are ``visible'', but $ Z_0 \to \nu \bar{\nu} $ is ``invisible'' and causes $ N_{\nu} = 2.96 \pm 0.04 $.

Early measurements from WMAP suggested $ N_{\text{eff}} > 4 $, and also Big Bang Nucleosynthesis gave hints for a new species of neutrinos that contributes to radiation called ``dark radiation''. More recent analysis by WMAP, PLANCK, SPT, and ACT are consistent with $ N_{\text{eff}} = 3.045 $ (they find $ N_{\text{eff}} = 3.3 \pm 0.27 $). The analysis above also predicts the existence of a $ \text{C} \nu \text{B} $ at $ T = 1.95 \kelvin $, and proposals of large volume detectors to measure it have been advanced.
\end{document}
