\documentclass[a4paper,twoside,master.tex]{subfiles}
\begin{document}
\lecture{27}{Tuesday, April 27, 2021}{Structure Formation: Jeans Instability}

\section{Jeans Instability}\label{sec:jeans_instability}

In the derivation of $ T^{\mu \nu} $ we arrived at the two main equations which describe fluids in the Newtonian non-relativistic limit:
\begin{equation}
    \pdv{\rho}{t} + \grad{(\rho \va{v})} = 0 \tag{Continuity}
\end{equation}
\begin{equation}
    \pdv{\va{v}}{t} + (\va{v} \vdot \grad)\va{v} = - \frac{1}{\rho} \grad{P} - \grad{\phi} \tag{Euler}
\end{equation}
along with Poisson's equation for $ \phi $:
\begin{equation}
    \laplacian{\phi} = 4 \pi G \rho
\end{equation}

We will use this to understand the growth of structures under gravitational instabilities during the matter-dominated era. We'll later learn why this is the era when structure formation begins. We can study the evolution of perturbations in the matter/velocity distributions for non-relativistic fluids, since the speed of propagation of these perturbations is $ c_s^2 \sim P/\rho \ll c^2 $.

We will study the evolution of perturbations in stages:
\begin{itemize}
    \item Minkowski Spacetime
        \subitem Perfect Newtonian fluids without gravity
        \subitem Adding viscosity
        \subitem Adding gravity but removing velocity (this leads to Jeans Instability)
    \item Ideal self-gravitating fluids in an expanding (matter-dominated) cosmology
\end{itemize}

\section{Minkowski Spacetime}\label{sec:minkowski_spacetime}

\subsection{Newtonian Fluids in Absence of Gravity}\label{sub:newtonian_fluids_in_absence_of_gravity}

In the absence of gravity,
\begin{equation}
    \partial_{\mu} T^{\mu \nu} = 0
\end{equation}
implies the continuity equation above and the Euler equation without $ - \grad{\phi} $.

If we consider a closed system and adiabatic perturbations, we have
\begin{equation}
    \dd{Q} = \dd{U} + P \dd{V} = 0
\end{equation}
For a fluid at rest ($ \va{v} = 0 $), $ \rho(x,t) = \rho_0 $ and $ P(x,t) = P_0 $, so the continuity and Euler equations are clearly satisfied. Consider a small perturbation
\begin{equation}
    \va{v} = \underbrace{0}_{\va{v}_0} + \va{v}_1 \qquad \rho(x,t) = \rho_0 + \rho_1(x,t) \qquad P(x,t) = P_0 + P_1(x,t)
\end{equation}
with $ \rho_1 \ll \rho_0 $ and $ P_1 \ll P_0 $. Then the continuity and Euler equations (to linear order in this perturbation) read
\begin{equation}
    \pdv{\rho_1}{t} + \grad{(\rho_0 \va{v}_1)} = \pdv{\rho_1}{t} + \rho_0 \grad{\va{v}_1} = 0
\end{equation}
\begin{equation}
    \pdv{\va{v}_1}{t} = - \frac{1}{\rho_0} \grad{P_1}
\end{equation}
Unfortunately, this gives us three unknowns (the perturbed values) but only two equations, so we need an equation of state, or some equation relating $ P_1 $ as a function of $ \rho_1 $ (like $ P_1(\rho_1) $). From $ \dd{Q} = 0 = \dd{U} + P \dd{V} $, we can take the ideal (non-relativistic) gas equation $ P = \frac{N}{V} k_B T $ and $ U = \frac{3}{2} PV $ from statistical mechanics to get
\begin{equation}
    \dd{U} = \frac{3}{2} \left( \dd{P} V + P \dd{V} \right) = - P \dd{V}
\end{equation}
so
\begin{equation}
    \pdv{\va{v}_1}{t} = - \frac{c_s^2}{\rho_0} \grad{\rho_1}
\end{equation}
Taking the time-derivative of the continuity equation, we get
\begin{equation}
    \pdv[2]{\rho_1}{t} + \rho_0 \grad{\left( \pdv{\va{v}_1}{t} \right)} = 0
\end{equation}
Inserting this into the previous equation, we get a wave equation:
\begin{equation}
    \pdv[2]{\rho_1}{t} - c_s^2 \laplacian{\rho_1} = 0
\end{equation}
or
\begin{equation}
    \frac{1}{c_s^2} \pdv[2]{\rho_1}{t} - \laplacian{\rho_1} = 0
\end{equation}
which describes waves with a propagation velocity $ c_s $. We can get solutions via a Fourier transform:
\begin{equation}
    \rho_1(x) = \tilde{\rho}_1(k) e^{- \imath \omega t} e^{\imath \va{k} \vdot \va{x}}
\end{equation}
so
\begin{equation}
    \omega^2 - c_s^2 \va{k}^2 = 0 \implies \omega(k) = \pm c_s \abs{\va{k}}
\end{equation}
This is the dispersion relation of sound waves. Then
\begin{equation}
    P_1(x, t) = c_s^2 \rho_1(x,t) = c_s^2 \tilde{\rho}(k) e^{- \imath \omega t} e^{\imath \va{k} \vdot \va{x}}
\end{equation}
and from
\begin{equation}
    \frac{3}{2} V \dd{P} = - \frac{5}{2} P \dd{V}
\end{equation}
we get
\begin{equation}
    \frac{\dd{P}}{P} = - \frac{5}{3} \frac{\dd{V}}{V}
\end{equation}
Density is related to the volume: $ \rho = \frac{Nm}{V} \implies \dd{\rho} = - \frac{Nm}{V^2} \dd{V} $, so
\begin{equation}
    \frac{\dd{\rho}}{\rho} = - \frac{\dd{V}}{V} \implies \frac{\dd{P}}{P} = \frac{5}{3} \frac{\dd{\rho}}{\rho} \implies P = P_0 \left( \frac{\rho}{\rho_0} \right)^{5/3}
\end{equation}
With $ P_0, \rho_0 $ constant, we get $ \dd{P} = P_1 $ and $ \dd{\rho} = \rho_1 $, so
\begin{equation}
    \frac{P_0}{\rho_0} = \frac{k_B T}{m} \implies P_1 = c_s^2 \rho_1 \qquad c_s^2 = \frac{5}{3} \frac{k_B T}{m}
\end{equation}

Note that from the classical equipartition theorem, we have
\begin{equation}
    \frac{1}{2} m \ev{v^2} = \frac{3}{2} k_B T \implies \ev{v^2} = \frac{3 k_B T}{m} \implies c_s^2 = \frac{5}{9} \ev{v^2}
\end{equation}

We can propose plane wave solutions to Euler's equation:
\begin{equation}
    \va{v}_1(x,t) = \va{\tilde{v}}(k) e^{- \imath \omega t} e^{\imath \va{k} \vdot \va{x}} \implies - \imath \omega \va{\tilde{v}}(k) = - \imath \frac{c_s^2}{\rho_0} \va{k} \tilde{\rho}_1(k)
\end{equation}
or $ \va{\tilde{v}}(k) = \frac{c_s}{\rho_0} \hat{k} \tilde{\rho}_1(k) $ is parallel to $ \va{k} $, so this is a longitudinal perturbation.

\subsection{Non-Ideal Fluids in Absence of Gravity}\label{sub:non-ideal_fluids_in_absence_of_gravity}

We can now add viscosity. Without proof, the energy-momentum tensor of viscous fluids is
\begin{equation}
    T^{00} = \rho c^2 \qquad T^{0i} = T^{i0} = \rho c v^i \qquad T^{ij} = P \delta^{ij} + \rho v^i v^j - \eta \left[ \partial_j v^i + \partial_i v^j - \frac{2}{3} \delta^{ij} \div{\va{v}} \right] - \xi \delta^{ij} \div{\va{v}}
\end{equation}
where $ \eta $ is the sheer viscosity and $ \xi $ is the bulk viscosity, both of which are linearly proportional to the mean free path. The continuity equation is unchanged, but the Euler equation becomes
\begin{equation}
    \pdv{(\rho v^i)}{t}+ \pdv{(\rho v^i v^j)}{x^j} = - \pdv{P}{x^i} + \eta \laplacian{v^i} + \left( \xi + \frac{\eta}{3} \right) \pdv{v^j}{x^i}{x^j}
\end{equation}

With
\begin{equation}
    \pdv{(\rho v^i)}{t} = \rho \pdv{v^i}{t} + v^i \pdv{\rho}{t}
\end{equation}
we can combine this with the continuity equation to get
\begin{equation}
    \pdv{\va{v}}{t} + (\va{v} \vdot \grad) \va{v} = - \frac{\grad{P}}{\rho} + \frac{\eta}{\rho} \laplacian{\va{v}} + \frac{\left( \xi + \frac{\eta}{3} \right)}{\rho} \grad{(\div{\va{v}})}
\end{equation}
If we neglect thermal conductivity, we use an adiabatic equation of state:
\begin{equation}
    \delta P = \frac{5}{3} \left( \frac{P}{\rho} \right) \delta \rho = c_s^2 \delta \rho
\end{equation}
with $ c_s^2 = \frac{5}{3} \frac{k_B T}{m} $. If we expand in small perturbations and linearize as before, we can propose Fourier transforms for $ P_1 $, $ \rho_1 $, and $ \va{v}_1 $ as before with $ \tilde{P}_1 = c_s^2 \tilde{\rho}_1 $. From the continuity equation, we get
\begin{equation}
    - \imath \omega \tilde{\rho}_1 + \imath \rho_0 k \tilde{v}_{\parallel} = 0
\end{equation}
where $ \va{\tilde{v}}_1 = \hat{k} \tilde{v}_{\parallel} + \tilde{v}_{\perp} $, so
\begin{equation}
    \tilde{\rho}_1 = \rho_0 \frac{k}{\omega} \tilde{v}_{\parallel}
\end{equation}
From the Euler equation, we get
\begin{equation}
    \imath \omega \tilde{v}_{\perp} = - k^2 \frac{\eta}{\rho_0} \tilde{v}_{\perp}
\end{equation}
along the perpendicular projection and
\begin{equation}
    \omega^2 - c_s^2 k^2 + \imath \omega \gamma_k = 0
\end{equation}
for the parallel projection, where $ \gamma_k = \frac{k^2}{\rho_0} \left( \xi + \frac{4}{3} \eta \right) $. this gives us a dispersion relation for the parallel component:
\begin{equation}
    \omega_{\pm} = - \imath \frac{\gamma_k}{2} \pm \sqrt{c_s^2 k^2 - \left( \frac{\gamma_k}{2} \right)^2}
\end{equation}
Note that $\gamma_k$, the damping factor related to viscosity, vanishes in the long wavelength limit ($ k \to 0 $), which is the hallmark of hydrodynamic modes, long-lived in the long-wavelength limit.

\subsection{Ideal Fluids with Gravity}\label{sub:ideal_fluids_with_gravity}

Consider a spherical shell of mass $ \Delta m $ outside a uniform mass distribution of (fixed) radius $ r_0 $ and total mass $ M_0 $. The acceleration of the shell towards the inner mass by gravitational force is called the ``gravitational collapse'' onto the inner mass. This happens on a characteristic time scale, $ t_{ff} $, the free-fall time. This is determined by energy conservation:
\begin{equation}
    \frac{1}{2} \left( \dv{r}{t} \right)^2 = \frac{GM_0}{r} - \frac{G M_0}{r_i}
\end{equation}
so that the shell begins to collapse with $ v = 0 $ at $ r_i \gg r_0 $ ($ r_i $ is the initial radius). Then
\begin{equation}
    t_{ff} = - \int_{r_i}^{r_0} \dv{t}{r} \dd{r} = - \int_{r_i}^{r_0} \frac{\dd{r}}{\sqrt{\frac{2 G M_0}{r} - \frac{2 G M_0}{r_i}}}
\end{equation}
Using $ x = r/r_i $ and taking $ r_0 \ll r_i $, we get
\begin{equation}
    t_{ff} = \left[ \frac{r_i^3}{2GM_0} \right]^{1/2} \int_0^1 \left[ \frac{x}{1-x} \right]^{1/2} \dd{x}
\end{equation}
Using $ x = \sin[2](\theta) $, we get
\begin{equation}
    t_{ff} \approx \frac{\pi}{2} \left[ \frac{r_i^3}{2 GM_0} \right]
\end{equation}
If we consider the average density
\begin{equation}
    \bar{\rho} = \frac{M_0}{\frac{4 \pi}{3} r_1^3}
\end{equation}
inside the initial ball, we get
\begin{equation}
    t_{ff} \approx \sqrt{\frac{3 \pi}{32}} \frac{1}{\sqrt{G \bar{\rho}}} \simeq \frac{1}{\sqrt{G \bar{\rho}}} 
\end{equation}

This is the time scale in absence of restoring forces. Note that an important feature is revealed in an expanding cosmology. If $ H^2 = \frac{8 \pi G}{3} \rho $, then $ H \sim \sqrt{G \rho} $ or $ t_{ff} \sim \frac{1}{H} $, which is the Hubble time scale.

\subsection{Jeans Instability}\label{sub:jeans_instability}

Using the equations from the beginning of this lecture, along with the adiabatic equation of state $ \dd{U} = - P \dd{V} \implies \delta P = c_s^2 \delta \rho $, consider a perturbation around a homogeneous state (same as before but adding $ \phi(x,t) = \phi_0 + \phi_1(x,t) $).

``Jeans Swindle'': Technically, there is no homogeneous state because if $ \rho = \rho_0 $ then $ \laplacian{\phi_0} = 4 \pi G \rho_0 $ implies $ \phi_0 $ cannot be homogeneous, since it would imply $ \phi_0 \sim \rho_0 \va{x}^2 $. Then with $ \va{v} = 0 $, we get $ \frac{\grad{P}}{\rho_0} = - \grad{\phi_0} \neq 0 $, so $ P $ can't be homogeneous either. We will see later how this changes with expansion.

Let's argue that the gravitational forces will balance out and linearize around a homogeneous state:
\begin{equation}
    \pdv{\rho_1}{t} + \rho_0 \div{\va{v}_1} = 0
\end{equation}
\begin{equation}
    \pdv{\va{v}_1}{t} = - \frac{\grad{P_1}}{\rho_0} - \grad{\phi_1}
\end{equation}
\begin{equation}
    \laplacian{\phi_1} = 4 \pi G \rho_1
\end{equation}
\begin{equation}
    P_1 = c_s^2 \rho_1
\end{equation}
Taking the time derivative of the first equation and using the second on the right-hand side, we can use the fourth equation to get
\begin{equation}
    \pdv[2]{\rho_1}{t} - c_s^2 \laplacian{\rho_1} - 4 \pi G \rho_0 \rho_1 = 0
\end{equation}
If we suppose the usual Fourier transform on $ \rho_1 $, we get a dispersion relation
\begin{equation}
    \omega(k) = \pm \sqrt{c_s^2 k^2 - 4 \pi G \rho_0}
\end{equation}
For $ c_s^2 k^2 > 4 \pi G \rho_0 $, $ \omega(k) $ is real, so these will be oscillations like sound waves. Otherwise, $ \omega $ is imaginary, so we have a constantly growing and a constantly decaying solution from $ e^{\pm \abs{\omega} t} $. For the growing solution, small density perturbations are \textit{unstable} and grow, becoming large amplitude perturbations. This is the definition of Jeans instability and it leads to gravitational collapse.

In the next class, we'll look at the physics behind this, or what happens when the gravitational term $ 4 \pi G \rho_0 $ overpowers the $ c_s^2 k^2 $ term.


\end{document}
